%%%%%%%%%%%%%%%%%%%%%%%%%%%%%%%%%%%%%%%%%%%%%%%%%%%%%%%%%%%%%%%%%
%%%%%%%%%%%%%%%%%%%%%%%%%%%%%%%%%%%%%%%%%%%%%%%%%%%%%%%%%%%%%%%%%
\setcounter{chapter}{4}
\newcommand{\graphicscompanion}{\emph{The \LaTeX{} Graphics Companion}~\cite{graphicscompanion}} 
\newcommand{\hobby}{\emph{A User's Manual for MetaPost}~\cite{metapost}}
\newcommand{\hoenig}{\emph{\TeX{} Unbound}~\cite{unbound}}
\newcommand{\graphicsinlatex}{\emph{Graphics in \LaTeXe{}}~\cite{ursoswald}}

\chapter{Produire des graphiques mathématiques}

\begin{intro}
De nombreuses personnes utilisent \LaTeX{} pour mettre en page leur
texte. Mais si l'approche \og orientée structure \fg{} est si
commode, \LaTeX{} fournit également des possibilités de production
graphique sur la base de descriptions textuelles quelque peu
limitées. De nombreuses extensions ont été créées pour surmonter ces
limitations. Vous découvrirez quelques-une d'entre elles dans cette
section.
\end{intro}

\section{Vue d'ensemble}

L'environnement \ei{picture} permet la programmation directe d'images
en \LaTeX. Vous pourrez en trouver une description détaillée dans le
\manuel. D'un côté, cet environnement a des contraintes sérieuses,
comme les pentes des segments ou les rayons des cercles restreints à
un choix limité de valeurs. D'un autre côté, l'environnement
\ei{picture} de \LaTeXe{} amène la commande \ci{qbezier}, avec \og
\texttt{q} \fg{} pour \og quadratique \fg{}. Des courbes fréquemment
utilisées comme des cercles, des ellipses ou des caténaires peuvent
être approchés de manière satisfaisantes par des courbes de B\'ezier,
bien que cela requiert quelques efforts mathématiques. Si en plus un
langage de programmation comme Java est utilisé pour générer des blocs
\ci{qbezier} en fichiers d'entrée, l'environnement \ei{picture}
devient soudain plus puissant.

Bien que programmer des images directement en \LaTeX{} est limité et
souvent fatiguant, il reste des raisons pour continuer à le faire : en
effet les documents produits ainsi sont de petite taille (en octets)
et il n'y a pas de fichier graphique à adjoindre.

Des environnements comme \pai{epic} et \pai{eepic} (décrits dans
\companion, par exemple), ou \pai{pstricks} aident à éliminer les
restrictions qui entravent l'environnement \ei{picture} originel. Ils
renforcent ainsi les capacités graphiques de \LaTeX.

Alors que les deux premières extensions précitées se contentent
d'améliorer \ei{picture}, l'environnement \pai{pstricks} fournit son
propre environnement de dessin, \ei{pspicture}. La puissance de
\pai{pstricks} provient de son usage intensif des capacités de
\texttt{PostScript}. De plus, de nombreux environnements ont été
écrits avec des objectifs précis en tête. L'un d'entre eux est
\texorpdfstring{\Xy}{Xy}-pic que nous présentons à la fin de ce
chapitre. \graphicscompanion{} décrit en détail une grande variété de
ces environnements (à ne pas confondre avec \companion).

Le plus puissant outil graphique lié à \LaTeX{} est \texttt{MetaPost},
le frère jumeau de \texttt{METAFONT} par Donald
E. Knuth. \texttt{MetaPost} possède le langage mathématique puissant
et sophistiqué de \texttt{METAFONT}. À la différence de
\texttt{METAFONT} qui génère des bitmaps, \texttt{MetaPost} génère du
\texttt{PostScript} encapsulé, qui peut être importé dans \LaTeX. Pour
une présentation, voyez \hobby ou le tutoriel de \cite{ursoswald}.

Vous pourrez trouver une discussion très détaillé des stratégies
\LaTeX{} et \TeX{} pour les images (et les polices) dans \hoenig.


\section{L'extension \texttt{picture}}
\secby{Urs Oswald}{osurs@bluewin.ch}

\subsection{Commandes de base}

L'une des deux commandes suivantes crée un environnement
\ei{picture}~:
\begin{lscommand}
\ci{begin}\verb|{picture}(|$x,y$\verb|)|\ldots\ci{end}\verb|{picture}|
\end{lscommand}
\noindent ou
\begin{lscommand}
\ci{begin}\verb|{picture}(|$x,y$\verb|)(|$x_0,y_0$\verb|)|\ldots\ci{end}\verb|{picture}|
\end{lscommand}
Les nombres $x,\,y,\,x_0,\,y_0$ se rapportent à une unité de longueur
\ci{unitlength} qui peut être réinitialisée à tout moment (sauf à
l'intérieur d'un environnement \ei{picture}) avec une commande telle
que
\begin{lscommand}
\ci{setlength}\verb|{|\ci{unitlength}\verb|}{1.2cm}|
\end{lscommand}
La valeur par défaut d'\ci{unitlength} est \texttt{1pt}. La première
paire $(x,y)$ réserve un espace rectangulaire pour l'image dans le
document. La deuxième paire optionnelle associe des coordonnées
arbitraires au coin inférieur gauche du rectangle réservé.

La plupart des commandes de dessin suivent l'une des deux formes
suivantes :
\begin{lscommand}
\ci{put}\verb|(|$x,y$\verb|){|\emph{objet}\verb|}|
\end{lscommand}
\noindent ou
\begin{lscommand}
\ci{multiput}\verb|(|$x,y$\verb|)(|$\Delta x,\Delta y$\verb|){|$n$\verb|}{|\emph{objet}\verb|}|
\end{lscommand}
Les courbes de B\'ezier font exception : elles sont dessinées avec la
commande
\begin{lscommand}
\ci{qbezier}\verb|(|$x_1,y_1$\verb|)(|$x_2,y_2$\verb|)(|$x_3,y_3$\verb|)|
\end{lscommand}

\subsection{Segments}

\begin{example}
\setlength{\unitlength}{5cm}
\begin{picture}(1,1)
  \put(0,0){\line(0,1){1}}
  \put(0,0){\line(1,0){1}}  
  \put(0,0){\line(1,1){1}}  
  \put(0,0){\line(1,2){.5}}
  \put(0,0){\line(1,3){.3333}}
  \put(0,0){\line(1,4){.25}}  
  \put(0,0){\line(1,5){.2}}
  \put(0,0){\line(1,6){.1667}}
  \put(0,0){\line(2,1){1}}
  \put(0,0){\line(2,3){.6667}}
  \put(0,0){\line(2,5){.4}}
  \put(0,0){\line(3,1){1}}  
  \put(0,0){\line(3,2){1}}
  \put(0,0){\line(3,4){.75}}
  \put(0,0){\line(3,5){.6}}
  \put(0,0){\line(4,1){1}}
  \put(0,0){\line(4,3){1}}  
  \put(0,0){\line(4,5){.8}}
  \put(0,0){\line(5,1){1}}
  \put(0,0){\line(5,2){1}}
  \put(0,0){\line(5,3){1}}
  \put(0,0){\line(5,4){1}}
  \put(0,0){\line(5,6){.8333}}
  \put(0,0){\line(6,1){1}}
  \put(0,0){\line(6,5){1}}
\end{picture}
\end{example}
Les segments sont dessinés via
\begin{lscommand}
\ci{put}\verb|(|$x,y$\verb|){|\ci{line}\verb|(|$x_1,y_1$\verb|){|$longueur$\verb|}}|
\end{lscommand}
La commande \ci{line} a deux arguments :
\begin{enumerate}
  \item Un vecteur de direction ;
  \item Une longueur.
\end{enumerate}
Les composants du vecteur de direction sont restreints aux entiers
\[
  -6,\,-5,\,\ldots,\,5,\,6,
\]
et doivent être premiers entre eux (pas de diviseur en commun sauf
1). La figure ci-dessus illustre les 25 valeurs possibles de pentes
pour le premier quadrant. La longueur est relative à
\ci{unitlength}. L'argument de longueur est la coordonnée verticale
dans le cas d'un segment vertical, horizontale dans tous les autres
cas.

\subsection{Flèches}

\begin{example}
\setlength{\unitlength}{1mm}
\begin{picture}(60,40)
  \put(30,20){\vector(1,0){30}}
  \put(30,20){\vector(4,1){20}}
  \put(30,20){\vector(3,1){25}}
  \put(30,20){\vector(2,1){30}}
  \put(30,20){\vector(1,2){10}}
  \thicklines
  \put(30,20){\vector(-4,1){30}}
  \put(30,20){\vector(-1,4){5}}
  \thinlines
  \put(30,20){\vector(-1,-1){5}}
  \put(30,20){\vector(-1,-4){5}}
\end{picture}
\end{example}
Les flèches sont dessinées via
\begin{lscommand}
\ci{put}\verb|(|$x,y$\verb|){|\ci{vector}\verb|(|$x_1,y_1$\verb|){|$longueur$\verb|}}|
\end{lscommand}
Pour les flèches, les composants du vecteur de direction sont encore
plus restreints que pour les segments, aux entiers
\[
  -4,\,-3,\,\ldots,\,3,\,4.
\]
qui doivent aussi être premiers entre eux. Remarquez l'effet de la
commande \ci{thicklines} sur les flèches qui pointent vers le coin
supérieur gauche.

\subsection{Cercles}

\begin{example}
\setlength{\unitlength}{1mm}
\begin{picture}(60, 40)
  \put(20,30){\circle{1}}
  \put(20,30){\circle{2}}
  \put(20,30){\circle{4}}
  \put(20,30){\circle{8}}
  \put(20,30){\circle{16}}
  \put(20,30){\circle{32}}
  
  \put(40,30){\circle{1}}
  \put(40,30){\circle{2}}
  \put(40,30){\circle{3}}
  \put(40,30){\circle{4}}
  \put(40,30){\circle{5}}
  \put(40,30){\circle{6}}
  \put(40,30){\circle{7}}
  \put(40,30){\circle{8}}
  \put(40,30){\circle{9}}
  \put(40,30){\circle{10}}
  \put(40,30){\circle{11}}
  \put(40,30){\circle{12}}
  \put(40,30){\circle{13}}
  \put(40,30){\circle{14}}
  
  \put(15,10){\circle*{1}}
  \put(20,10){\circle*{2}}
  \put(25,10){\circle*{3}}
  \put(30,10){\circle*{4}}
  \put(35,10){\circle*{5}}
\end{picture}
\end{example}
La commande
\begin{lscommand}
  \ci{put}\verb|(|$x,y$\verb|){|\ci{circle}\verb|{|\emph{diamètre}\verb|}}|
\end{lscommand}
\noindent dessine un cercle de centre $(x,y)$ et de diamètre
\emph{diamètre} (pas le rayon). L'environnement \ei{picture} admet
seulement des cercles de moins de 14\,mm, et même en-dessous de cette
limite tous les diamètres ne sont pas autorisés. La commande
\ci{circle*} produit quant à elle des disques (i.e. des cercles
remplis).

Comme pour les segments, il peut devenir nécessaire de recourir à des
extensions supplémentaires comme \pai{eepic} ou \pai{pstricks}. Pour
une description détaillée de ces extensions, voyez \graphicscompanion.

Il existe aussi une possibilité dans l'environnement \ei{picture}. À
condition de ne pas être effrayé par les calculs requis (ou en les
laissant à un programme dédié), des cercles et des ellipses de tailles
arbitraires peuvent être construits à base de courbes de B\'ezier,
quadratiques. Voyez \graphicsinlatex{} pour des exemples et des
fichiers source Java.


\subsection{Texte et formules}

\begin{example}
\setlength{\unitlength}{1cm}
\begin{picture}(6,5)
  \thicklines
  \put(1,0.5){\line(2,1){3}}
  \put(4,2){\line(-2,1){2}}
  \put(2,3){\line(-2,-5){1}}
  \put(0.7,0.3){$A$}
  \put(4.05,1.9){$B$}
  \put(1.7,2.95){$C$}
  \put(3.1,2.5){$a$}
  \put(1.3,1.7){$b$}
  \put(2.5,1.05){$c$}
  \put(0.3,4){$F=
    \sqrt{s(s-a)(s-b)(s-c)}$}  
  \put(3.5,0.4){$\displaystyle
    s:=\frac{a+b+c}{2}$}
\end{picture}
\end{example}
Comme le montre l'exemple ci-dessus, la commande \ci{put} permet
d'insérer du texte et des formules dans un environnement \ei{picture}
comme à l'accoutumée.

\subsection{Les commandes \ci{multiput} et \ci{linethickness}}

\begin{example}
\setlength{\unitlength}{2mm}
\begin{picture}(30,20)
  \linethickness{0.075mm}
  \multiput(0,0)(1,0){31}%
    {\line(0,1){20}}
  \multiput(0,0)(0,1){21}%
    {\line(1,0){30}}
  \linethickness{0.15mm}    
  \multiput(0,0)(5,0){7}%
    {\line(0,1){20}}
  \multiput(0,0)(0,5){5}%
    {\line(1,0){30}}
  \linethickness{0.3mm}    
  \multiput(5,0)(10,0){3}%
    {\line(0,1){20}}
  \multiput(0,5)(0,10){2}%
    {\line(1,0){30}}
\end{picture}
\end{example}
La commande
\begin{lscommand}
  \ci{multiput}\verb|(|$x,y$\verb|)(|$\Delta x,\Delta y$\verb|){|$n$\verb|}{|\emph{objet}\verb|}|
\end{lscommand}
\noindent possède 4 paramètres : le point de départ, le vecteur de
translation d'un objet à l'autre, le nombre d'objets et l'objet à
dessiner. La commande \ci{linethickness} s'applique aux segments
horizontaux et verticaux mais pas aux segments obliques ni aux
cercles. Elle s'applique cependant aux courbes de B\'ezier.

\subsection{Ovales. Les commandes \ci{thinlines} et \ci{thicklines}}

\begin{example}
\setlength{\unitlength}{1cm}
\begin{picture}(6,4)
  \linethickness{0.075mm}
  \multiput(0,0)(1,0){7}%
    {\line(0,1){4}}
  \multiput(0,0)(0,1){5}%
    {\line(1,0){6}}
  \thicklines
  \put(2,3){\oval(3,1.8)} 
  \thinlines
  \put(3,2){\oval(3,1.8)} 
  \thicklines
  \put(2,1){\oval(3,1.8)[tl]} 
  \put(4,1){\oval(3,1.8)[b]} 
  \put(4,3){\oval(3,1.8)[r]} 
  \put(3,1.5){\oval(1.8,0.4)}     
\end{picture}
\end{example}
La commande
\begin{lscommand}
  \ci{put}\verb|(|$x,y$\verb|){|\ci{oval}\verb|(|$w,h$\verb|)}|
\end{lscommand}
\noindent or
\begin{lscommand}
  \ci{put}\verb|(|$x,y$\verb|){|\ci{oval}\verb|(|$w,h$\verb|)[|\emph{position}\verb|]}|
\end{lscommand}
\noindent produit un ovale centré en $(x,y)$, de largeur $w$ et
d'hauteur $h$. Les paramètres optionnels de \emph{position}
\texttt{b}, \texttt{t}, \texttt{l} et \texttt{r} se rapportent à \og
haut \fg{}, \og bas \fg{},\og gauche \fg{} et \og droite \fg{}
respectivement. Ils peuvent être combinés comme l'illustre l'exemple
plus haut.

L'épaisseur de trait peut être contrôlées par deux sortes de
commandes : \\
\ci{linethickness}\verb|{|\emph{longueur}\verb|}|
d'un côté, \ci{thinlines} et \ci{thicklines} de l'autre. Alors que
\ci{linethickness}\verb|{|\emph{longueur}\verb|}| ne s'applique qu'aux
lignes horizontales, verticales et aux courbes de B\'ezier,
\ci{thinlines} et \ci{thicklines} s'appliquent aux segments obliques
ainsi qu'aux cercles et aux ovales.


\subsection{Usage multiple d'images prédéfinies}

\begin{example}
\setlength{\unitlength}{0.5mm}
\begin{picture}(120,168)
\newsavebox{\foldera}% declaration
\savebox{\foldera}
  (40,32)[l]{%         definition 
  \multiput(0,0)(0,28){2}
    {\line(1,0){40}}
  \multiput(0,0)(40,0){2}
    {\line(0,1){28}}
  \put(1,28){\oval(2,2)[tl]}
  \put(1,29){\line(1,0){5}}
  \put(9,29){\oval(6,6)[tl]}
  \put(9,32){\line(1,0){8}}
  \put(17,29){\oval(6,6)[tr]}
  \put(20,29){\line(1,0){19}}
  \put(39,28){\oval(2,2)[tr]}  
}
\newsavebox{\folderb}% declaration
\savebox{\folderb}
  (40,32)[l]{%         definition 
  \put(0,14){\line(1,0){8}}
  \put(8,0){\usebox{\foldera}}
}
\put(34,26){\line(0,1){102}} 
\put(14,128){\usebox{\foldera}}
\multiput(34,86)(0,-37){3}
  {\usebox{\folderb}} 
\end{picture}
\end{example}
Une boîte d'image peut être \emph{déclarée} par la commande
\begin{lscommand}
  \ci{newsavebox}\verb|{|\emph{nom}\verb|}|
\end{lscommand}
\noindent puis \emph{définie} par
\begin{lscommand}
  \ci{savebox}\verb|{|\emph{nom}\verb|}(|\emph{largeur,hauteur}\verb|)[|\emph{position}\verb|]{|\emph{contenu}\verb|}|
\end{lscommand}
\noindent et enfin dessinée arbitrairement souvent via
\begin{lscommand}
  \ci{put}\verb|(|$x,y$\verb|)|\ci{usebox}\verb|{|\emph{nom}\verb|}|
\end{lscommand}
Le paramètre optionnel de \emph{position} \texttt{l} (pour \og gauche
\fg{}, i.e. \og left \fg{}) a pour effet de positionner le coin
(inférieur) gauche de l'image emboîtée au niveau de l'argument de
\ci{put}. Le paramètre \emph{nom} se rapporte à un registre de
stockage \LaTeX{}: il s'agit donc d'une commande (ce qui explique
l'usage d'antislashs dans l'exemple). Les images emboîtées peuvent
être imbriquées : \ci{foldera} est utilisée à l'intérieur de la
définition de \ci{folderb} dans l'exemple.

La commande \ci{oval} a dû être utilisée car la commande \ci{line} ne
fonctionne pas avec des segments dont la longueur est inférieure à
3\,mm.


\subsection{Courbes de B\'ezier}

\begin{example}
\setlength{\unitlength}{1cm}
\begin{picture}(6,4)
  \linethickness{0.075mm}
  \multiput(0,0)(1,0){7}
    {\line(0,1){4}}
  \multiput(0,0)(0,1){5}
    {\line(1,0){6}}
  \thicklines
  \put(0.5,0.5){\line(1,5){0.5}}    
  \put(1,3){\line(4,1){2}} 
  \qbezier(0.5,0.5)(1,3)(3,3.5)
  \thinlines   
  \put(2.5,2){\line(2,-1){3}}
  \put(5.5,0.5){\line(-1,5){0.5}}
  \linethickness{1mm}
  \qbezier(2.5,2)(5.5,0.5)(5,3)
  \thinlines
  \qbezier(4,2)(4,3)(3,3)
  \qbezier(3,3)(2,3)(2,2)
  \qbezier(2,2)(2,1)(3,1)
  \qbezier(3,1)(4,1)(4,2)
\end{picture}
\end{example}
Cet exemple illustre l'inadéquation du découpage d'un cercle en 4
courbes de B\'ezier. Au moins 8 d'entre elles sont requises. La figure
montre à nouveau l'effet de la commande \ci{linethickness} sur les
lignes horizontales et verticales, ainsi que des commandes
\ci{thinlines} et \ci{thicklines} sur les segments obliques. Elle
montre enfin l'effet des deux sortes de commandes sur les courbes de
B\'ezier, chaque commande supplantant toutes les précédentes.

Soient $P_1=(x_1,\,y_1),\,P_2=(x_2,\,y_2)$ les points de début et fin
et $m_1,\,m_2$ leurs pentes respectives d'une courbe de B\'ezier. Le
point de contrôle intermédiaire est donné par les équations
suivantes~:
\begin{equation} \label{zwischenpunkt}
  \left\{
    \begin{array}{rcl}
      x & = & \displaystyle \frac{m_2 x_2-m_1x_1-(y_2-y_1}{m_2-m_1}, \\
      y & = & y_i+m_i(x-x_i)\qquad (i=1,\,2).
    \end{array}
  \right.
\end{equation}
\noindent Référez-vous à \graphicsinlatex{} pour un programme Java qui
génère les lignes de commandes \ci{qbezier} nécessaires.

\subsection{Caténaire}

\begin{example}
\setlength{\unitlength}{1.3cm}
\begin{picture}(4.3,3.6)(-2.5,-0.25)
  \put(-2,0){\vector(1,0){4.4}}
  \put(2.45,-.05){$x$}
  \put(0,0){\vector(0,1){3.2}}
  \put(0,3.35){\makebox(0,0){$y$}}
  \qbezier(0.0,0.0)(1.2384,0.0)
    (2.0,2.7622) 
  \qbezier(0.0,0.0)(-1.2384,0.0)
    (-2.0,2.7622)
  \linethickness{.075mm}
  \multiput(-2,0)(1,0){5}
    {\line(0,1){3}}
  \multiput(-2,0)(0,1){4}
    {\line(1,0){4}}
  \linethickness{.2mm}
  \put( .3,.12763){\line(1,0){.4}}
  \put(.5,-.07237){\line(0,1){.4}}
  \put(-.7,.12763){\line(1,0){.4}}
  \put(-.5,-.07237){\line(0,1){.4}}
  \put(.8,.54308){\line(1,0){.4}}
  \put(1,.34308){\line(0,1){.4}}
  \put(-1.2,.54308){\line(1,0){.4}}
  \put(-1,.34308){\line(0,1){.4}}
  \put(1.3,1.35241){\line(1,0){.4}}
  \put(1.5,1.15241){\line(0,1){.4}}
  \put(-1.7,1.35241){\line(1,0){.4}}
  \put(-1.5,1.15241){\line(0,1){.4}}
  \put(-2.5,-0.25){\circle*{0.2}}
\end{picture}
\end{example}

Dans cette figure, chaque moitié symmétrique du caténaire $y=\cosh x
-1$ est approchée par une courbe de B\'ezier. La moitié droite
s'achève au point \((2,\,2.7622)\) avec une pente \(m=3.6269\). À
l'aide de l'équation (\ref{zwischenpunkt}) nous pouvons calculer les
points de contrôle intermédiaires. Ils s'avèrent être $(1.2384,\,0)$
et $(-1.2384,\,0)$. Les croix indiquent les points du caténaire
\emph{réel}. L'erreur est négligeable, de moins d'un pourcent.

Cet exemple signale l'usage du paramètre optionel de la commande
\verb|\begin{picture}|. L'image est définie selon des coordonnées \og
  mathématiques \fg{} commodes, tandis que par la commande
\begin{lscommand} 
  \ci{begin}\verb|{picture}(4.3,3.6)(-2.5,-0.25)|
\end{lscommand}
\noindent son coin inférieur gauche (marqué par une disque noir) est
associé aux coordonnées $(-2.5,-0.25)$.

%SC: not a physicist here, hope I got this one right :-P
\subsection{La rapidité dans la théorie de la relativité restreinte}

\begin{example}
\setlength{\unitlength}{1cm}
\begin{picture}(6,4)(-3,-2)
  \put(-2.5,0){\vector(1,0){5}}
  \put(2.7,-0.1){$\chi$}
  \put(0,-1.5){\vector(0,1){3}}
  \multiput(-2.5,1)(0.4,0){13}
    {\line(1,0){0.2}}
  \multiput(-2.5,-1)(0.4,0){13}
    {\line(1,0){0.2}}
  \put(0.2,1.4)
    {$\beta=v/c=\tanh\chi$}
  \qbezier(0,0)(0.8853,0.8853)
    (2,0.9640)
  \qbezier(0,0)(-0.8853,-0.8853)
    (-2,-0.9640)
  \put(-3,-2){\circle*{0.2}}
\end{picture}
\end{example}
Les points de contrôle des deux courbes de B\'ezier ont été calculées
grâce aux formules (\ref{zwischenpunkt}). La branche positive est
déterminée par $P_1=(0,\,0),\,m_1=1$ and $P_2=(2,\,\tanh
2),\,m_2=1/\cosh^2 2$. L'image est à nouveau définie selon des
coordonnées mathématiques commodes et le coin inférieur gauche est
associé aux coordonnées mathématiques $(-3,-2)$ (le disque noir).

\section{\texorpdfstring{\Xy}{Xy}-pic}
\secby{Alberto Manuel Brand\~ao Sim\~oes}{albie@alfarrabio.di.uminho.pt}
\pai{xy} est une extension dédiée au dessin de diagrammes. Elle est
invoquée par l'ajout dans votre préambule de la ligne :
\begin{lscommand}
\verb|\usepackage[|\emph{options}\verb|]{xy}|
\end{lscommand}
\emph{options} est une liste des fonctions que vous voulez voir
\Xy-pic charger. Ces options sont surtout utiles pour déboguer
l'extension. Je vous suggère d'utiliser l'option \verb!all! afin de
faire charger toutes les commandes \Xy{} par \LaTeX{}.

Les diagrammes \Xy-pic sont dessinés sur une trame basée sur un
matrice, où chaque élément de dessin correspond à un élément de la
matrice :
\begin{example}
\begin{displaymath}
\xymatrix{A & B \\
          C & D }
\end{displaymath}
\end{example}
La commande \ci{xymatrix} requiert le mode mathématique. Ici nous
avons spécifié deux lignes et deux colonnes. Pour faire de cette
matrice un diagramme, nous ajoutons des flèches via la commande
\ci{ar}.
\begin{example}
\begin{displaymath}
\xymatrix{ A \ar[r] & B \ar[d] \\
           D \ar[u] & C \ar[l] }
\end{displaymath}
\end{example}
La commande de flèche est placée sur la cellule d'origine de la
flèche. Les arguments correspondent à sa direction (\texttt{u}p vers
le haut, \texttt{d}own vers le bas, \texttt{r}ight vers la droite et
\texttt{l}eft vers la gauche).


\begin{example}
\begin{displaymath}
\xymatrix{
  A \ar[d] \ar[dr] \ar[r] & B \\
  D                       & C }
\end{displaymath}
\end{example}
Les flèches en diagonales sont obtenus en utilisant plus d'une
direction. Vous pouvez aussi répéter les directions pour avoir de plus
grandes flèches.
\begin{example}
\begin{displaymath}
\xymatrix{
  A \ar[d] \ar[dr] \ar[drr] & & \\
  B                      & C & D }
\end{displaymath}
\end{example}

Nous pouvons dessiner des diagrammes plus intéressants en ajoutant des
étiquettes aux flèches grâce aux opérateurs d'indice et d'exposant.
\begin{example}
\begin{displaymath}
\xymatrix{
  A \ar[r]^f \ar[d]_g &
             B \ar[d]^{g'} \\
  D \ar[r]_{f'}       & C }
\end{displaymath}
\end{example}

Comme l'indique l'exemple, vous utilisez ces opérateurs en mode
mathématique. La seule différence est que l'exposant signifie \og
au-dessus de la flèche \fg{} et l'indice \og sous la flèche \fg{}. il
existe un troisième opérateur, la barre verticale: \verb+|+. Elle fait
se placer le texte \emph{dans} la flèche.
\begin{example}
\begin{displaymath}
\xymatrix{
  A \ar[r]|f \ar[d]|g &
             B \ar[d]|{g'} \\
  D \ar[r]|{f'}       & C }
\end{displaymath}
\end{example}

Pour dessiner une flèche trouée, utilisez \verb!\ar[...]|\hole!.

Dans certaines circonstances, il peut être important de distinguer les
différents types de flèches. Ceci peut être obtenu en leur assignant
des étiquettes ou en changeant leur apparence :

\begin{example}
\begin{displaymath}
\xymatrix{
 \bullet\ar@{->}[rr] && \bullet\\
 \bullet\ar@{.<}[rr] && \bullet\\
 \bullet\ar@{~)}[rr] && \bullet\\
 \bullet\ar@{=(}[rr] && \bullet\\
 \bullet\ar@{~/}[rr]  && \bullet\\
 \bullet\ar@{=+}[rr]   && \bullet
}
\end{displaymath}
\end{example}

Notez la différence entre les deux diagrammes suivants:

\begin{example}
\begin{displaymath}
\xymatrix{
 \bullet \ar[r] 
         \ar@{.>}[r] & 
 \bullet
}
\end{displaymath}
\end{example}

\begin{example}
\begin{displaymath}
\xymatrix{
 \bullet \ar@/^/[r] 
         \ar@/_/@{.>}[r] &
 \bullet
}
\end{displaymath}
\end{example}

Les modifieurs entre les slashs définissent la manière dont les
courbes sont dessinées. \Xy-pic offre de nombreuses façons
d'influencer le dessin des courbes. Pour plus d'informations, veuillez
consultez la documentation d'\Xy-pic.

% \begin{example}
% \begin{lscommand}
% \ci{dum}
% \end{lscommand}
% \end{example}

