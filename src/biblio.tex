%%%%%%%%%%%%%%%%%%%%%%%%%%%%%%%%%%%%%%%%%%%%%%%%%%%%%%%%%%%%%%%%%
% Contents: The Bibliography
% File: biblio.tex (lshort2e.tex)
% $Id: biblio.tex,v 1.1.1.1 2002/02/26 10:04:20 oetiker Exp $
%%%%%%%%%%%%%%%%%%%%%%%%%%%%%%%%%%%%%%%%%%%%%%%%%%%%%%%%%%%%%%%%%
\begin{thebibliography}{99}
\addcontentsline{toc}{chapter}{\bibname} 
\bibitem{manual} Leslie Lamport.  \newblock \emph{{\LaTeX:} A Document
    Preparation System}.  \newblock Addison-Wesley, Reading,
  Massachusetts, second edition, 1994, ISBN~0-201-52983-1.
  
\bibitem{texbook} Donald~E. Knuth.  \newblock \textit{The \TeX{}book,}
  Volume~A of \textit{Computers and Typesetting}, Addison-Wesley,
  Reading, Massachusetts, second edition, 1984, ISBN~0-201-13448-9.

\bibitem{companion} Michel Goossens, Frank Mittelbach and Alexander
  Samarin.  \newblock \emph{The {\LaTeX} Companion}.  \newblock
  Addison-Wesley, Reading, Massachusetts, 1994, ISBN~0-201-54199-8.

\bibitem{graphicscompanion} Michel Goossens, Sebastian Rahtz and Frank
  Mittelbach.  \newblock \emph{The {\LaTeX} Graphics Companion}.  \newblock
  Addison-Wesley, Reading, Massachusetts, 1997, ISBN~0-201-85469-4.
 
\bibitem{local} Each \LaTeX{} installation should provide a so-called
  \emph{\LaTeX{} Local Guide}, which explains the things that are
  special to the local system.  It should be contained in a file called
  \texttt{local.tex}. Unfortunately, some lazy sysops do not provide such a
  document. In this case, go and ask your local \LaTeX{} guru for help.
 
\bibitem{usrguide} \LaTeX3 Project Team.  \newblock \emph{\LaTeXe~for
    authors}.  \newblock Comes with the \LaTeXe{} distribution as
  \texttt{usrguide.tex}.

\bibitem{clsguide} \LaTeX3 Project Team.  \newblock \emph{\LaTeXe~for
    Class and Package writers}.  \newblock Comes with the \LaTeXe{}
  distribution as \texttt{clsguide.tex}.

\bibitem{fntguide} \LaTeX3 Project Team.  \newblock \emph{\LaTeXe~Font
    selection}.  \newblock Comes with the \LaTeXe{} distribution as
  \texttt{fntguide.tex}.

\bibitem{graphics} D.~P.~Carlisle.  \newblock \emph{Packages in the
    `graphics' bundle}.  \newblock Comes with the `graphics' bundle as
  \texttt{grfguide.tex}, available from the same source your \LaTeX{}
  distribution came from.

\bibitem{verbatim} Rainer~Sch\"opf, Bernd~Raichle, Chris~Rowley.  
\newblock \emph{A New Implementation of \LaTeX's verbatim
  Environments}.
 \newblock Comes with the `tools' bundle as
  \texttt{verbatim.dtx}, available from the same source your \LaTeX{}
  distribution came from. 

\bibitem{catalogue} Graham~Williams.  \newblock \emph{The TeX
    Catalogue} is a very complete listing of many \TeX{} and \LaTeX{}
    related packages.
  \newblock Available online from \CTAN|help/Catalogue/catalogue.html|
  
\bibitem{eps} Keith~Reckdahl.  \newblock \emph{Using EPS Graphics in
    \LaTeXe{} Documents}, which explains everything and much more than
  you ever wanted to know about EPS files and their use in \LaTeX{}
  documents.  \newblock Available online from
  \CTAN|info/epslatex.ps|

\bibitem{xy-pic} Kristoffer H. Rose.
  \newblock \emph{\Xy-pic User's Guide}.  \newblock
  Downloadable from CTAN with \Xy-pic distribution 
  
\bibitem{metapost} John D. Hobby.
  \newblock \emph{A User's Manual for MetaPost}. \newblock
  Downloadable from \url{http://cm.bell-labs.com/who/hobby/} 
  
\bibitem{unbound} Alan Hoenig.
  \newblock \emph{\TeX{} Unbound}. \newblock Oxford University Press, 1998,
    ISBN 0-19-509685-1; 0-19-509686-X (pbk.) 
  
\bibitem{ursoswald} Urs Oswald.  
    \newblock \emph{Graphics in \LaTeXe{}}, containing some Java source files for 
    generating arbitrary circles and ellipses within the \texttt{picture} environment,
    and \emph{MetaPost - A Tutorial}.
  \newblock Both downloadable from \url{http://www.ursoswald.ch}
  
  

\end{thebibliography}


%

% Local Variables:
% TeX-master: "lshort2e"
% mode: latex
% mode: flyspell
% End:
