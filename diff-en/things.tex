3c3
< % $Id: things.tex,v 1.4 1998/09/29 08:05:09 oetiker Exp oetiker $
---
> % $Id: things.tex,v 1.2 2003/03/19 20:57:47 oetiker Exp $
8,10c8,10
< In the first part of this chapter, you will get a short 
< overview about the philosophy and history of \LaTeXe. The second part
< of the chapter focuses on the basic structures of a \LaTeX{} document. 
---
> The first part of this chapter presents a short 
> overview of the philosophy and history of \LaTeXe. The second part
> focuses on the basic structures of a \LaTeX{} document. 
12,13c12,13
< of how \LaTeX{} works. When reading on, this will help you to integrate
< all the new information into the big picture.  
---
> of how \LaTeX{} works, which you will need to understand the rest
> of this book.  
31c31
< converging to $\pi$ and is now at $3.14159$.
---
> converging to $\pi$ and is now at $3.141592$.
34c34,44
< ``Ach'' or in the Scottish ``Loch.'' In an ASCII environment, \TeX{}
---
> ``Ach''\footnote{In german there are actually two pronounciations for ``ch''
> and one might assume that the soft ``ch'' sound from ``Pech'' would be a
> more appropriate. Asked about this, Knuth wrote in the German Wikipedia:
> \emph{I do not get angry when people pronounce \TeX{} in their favorite way
> \ldots{} and in Germany many use a soft ch because the X follows the vowel
> e, not the harder ch that follows the vowel a. In Russia, `tex' is a very
> common word, pronounced `tyekh'. But I believe the most proper pronunciation
> is heard in Greece, where you have the harsher ch of ach and Loch.}}
> or in the Scottish ``Loch.'' The ``ch'' originates from the Greek
> alphabet where X is the letter ``ch'' or ``chi''. \TeX{} is also the first syllable
> of the Greek word texnologia (technology). In an ASCII environment, \TeX{}
39c49
< \LaTeX{} is a macro package which enables authors to typeset and print
---
> \LaTeX{} is a macro package that enables authors to typeset and print
43c53,54
< \TeX{} formatter as its typesetting engine.
---
> \TeX{} formatter as its typesetting engine. These days \LaTeX{} is maintained
> by \index{Mittelbach, Frank}Frank Mittelbach.
45,51c56,64
< In 1994 the \LaTeX{} package was updated by the \index{LaTeX3@\LaTeX
<   3}\LaTeX 3 team, led by \index{Mittelbach, Frank}Frank Mittelbach,
< to include some long-requested improvements, and to re\-unify all the
< patched versions which had cropped up since the release of
< \index{LaTeX 2.09@\LaTeX{} 2.09}\LaTeX{} 2.09 some years earlier. To
< distinguish the new version from the old, it is called \index{LaTeX
<   2e@\LaTeXe}\LaTeXe. This documentation deals with \LaTeXe.
---
> %In 1994 the \LaTeX{} package was updated by the \index{LaTeX3@\LaTeX
> %  3}\LaTeX 3 team, led by \index{Mittelbach, Frank}Frank Mittelbach,
> %to include some long-requested improvements, and to re\-unify all the
> %patched versions which had cropped up since the release of
> %\index{LaTeX 2.09@\LaTeX{} 2.09}\LaTeX{} 2.09 some years earlier. To
> %distinguish the new version from the old, it is called \index{LaTeX
> %2e@\LaTeXe}\LaTeXe. This documentation deals with \LaTeXe. These days you
> %might be hard pressed to find the venerable \LaTeX{} 2.09 installed
> %anywhere.
57,59c70,72
< Figure~\ref{components} above % on page \pageref{components}
< shows how \TeX{} and \LaTeXe{} work together. This figure is taken from
< \texttt{wots.tex} by Kees van der Laan.
---
> %Figure~\ref{components} above % on page \pageref{components}
> %shows how \TeX{} and \LaTeXe{} work together. This figure is taken from
> %\texttt{wots.tex} by Kees van der Laan.
61,68c74,81
< \begin{figure}[btp]
< \begin{lined}{0.8\textwidth}
< \begin{center}
< \input{kees.fig}
< \end{center}
< \end{lined}
< \caption{Components of a \TeX{} System.} \label{components}
< \end{figure}
---
> %\begin{figure}[btp]
> %\begin{lined}{0.8\textwidth}
> %\begin{center}
> %\input{kees.fig}
> %\end{center}
> %\end{lined}
> %\caption{Components of a \TeX{} System.} \label{components}
> %\end{figure}
89c102
< additional information which describes the logical structure of his
---
> additional information to describe the logical structure of his
94,95c107,108
<   what you get.} approach which most modern word processors such as
< \emph{MS Word} or \emph{Corel WordPerfect} take. With these
---
>   what you get.} approach that most modern word processors, such as
> \emph{MS Word} or \emph{Corel WordPerfect}, take. With these
97c110
< typing text into the computer. All along the way, they can see on the
---
> typing text into the computer. They can see on the
100,101c113,114
< When using \LaTeX{} it is normally not possible to see the final output
< while typing the text. But the final output can be previewed on the
---
> When using \LaTeX{} it is not normally possible to see the final output
> while typing the text, but the final output can be previewed on the
111,112c124,125
< up in a picture gallery, the readability and understandability is of
< much greater importance than the beautiful look of it.
---
> up in a picture gallery, the readability and understandability is 
> much more important than the beautiful look of it.
117c130
< \item The line length has to be short enough to not strain
---
> \item The line length has to be short enough not to strain
130c143
< When People from the \wi{WYSIWYG} world meet people who use \LaTeX{},
---
> When people from the \wi{WYSIWYG} world meet people who use \LaTeX{},
145,146c158,159
< \item The user only needs to learn a few easy-to-understand commands
<   which specify the logical structure of a document. They almost never
---
> \item Users only need to learn a few easy-to-understand commands
>   that specify the logical structure of a document. They almost never
152c165
<   available to include \textsc{PostScript} graphics or to typeset
---
>   available to include \PSi{} graphics or to typeset
178c191
<     key elements which will be addressed in the upcoming \LaTeX 3
---
>     key elements that will be addressed in the upcoming \LaTeX 3
188,189c201,202
< with any text editor. It contains the text of the document as well as
< the commands which tell \LaTeX{} how to typeset the text.
---
> with any text editor. It contains the text of the document, as well as
> the commands that tell \LaTeX{} how to typeset the text.
193c206
< ``Whitespace'' characters such as blank or tab are
---
> ``Whitespace'' characters, such as blank or tab, are
196,197c209,210
< ``space''.  Whitespace at the start of a line is generally ignored, and
< a single linebreak is treated as ``whitespace''.
---
> ``space.''  Whitespace at the start of a line is generally ignored, and
> a single line break is treated as ``whitespace.''
235,236c248,249
< in front of it (\verb|\\|), this sequence is used for
< linebreaking.\footnote{Try the \texttt{\$}\ci{backslash}\texttt{\$} command instead. It
---
> in front of it (\verb|\\|); this sequence is used for
> line breaking.\footnote{Try the \texttt{\$}\ci{backslash}\texttt{\$} command instead. It
241c254
< \LaTeX{} \wi{commands} are case sensitive and take one of the following
---
> \LaTeX{} \wi{commands} are case sensitive, and take one of the following
247,249c260,261
<  space, a number or any other `non-letter'.
< \item They consist of a backslash and exactly one % numerical or
<  special character.
---
>  space, a number or any other `non-letter.'
> \item They consist of a backslash and exactly one non-letter.
274c286
< Some commands need a \wi{parameter} which has to be given between
---
> Some commands need a \wi{parameter}, which has to be given between
276c288
< \wi{optional parameters} which are added after the command name in
---
> \wi{optional parameters}, which are added after the command name in
278c290
< commands. Don't worry about them, they will be explained later.
---
> commands. Don't worry about them; they will be explained later.
293c305
< it ignores the rest of the present line, the linebreak, and all
---
> it ignores the rest of the present line, the line break, and all
308c320
< whitespace or linebreaks are allowed.
---
> whitespace or line breaks are allowed.
310,313c322,325
< For longer comments you should use the \ei{comment} environment
< provided by the \pai{verbatim} package. This means, to use the
< \ei{comment} environment you have to add the commend
< \verb|\usepackage{verbatim}| to the preamble of your document.
---
> For longer comments you could use the \ei{comment} environment
> provided by the \pai{verbatim} package. This means, that you have to add the
> line \verb|\usepackage{verbatim}| to the preamble of your document as
> explained below before you can use this command.
325c337
< Note that this won't work inside complex environments like math for example.
---
> Note that this won't work inside complex environments, like math for example.
336,337c348,349
< you can include commands which influence the style of the whole
< document, or you can load \wi{package}s which add new
---
> you can include commands that influence the style of the whole
> document, or you can load \wi{package}s that add new
346c358
<     begin$\mathtt{\{}$document$\mathtt{\}}$} is called
---
>     begin$\mathtt{\{}$document$\mathtt{\}}$} is called the
359c371
< command, which tells \LaTeX{} to call it a day. Anything which
---
> command, which tells \LaTeX{} to call it a day. Anything that
390c402
< \section{Start}
---
> \section{Some Interesting Words}
392c404
< \section{End}
---
> \section{Good Bye World}
397c409,412
< \caption{Example of a Realistic Journal Article.} \label{document}
---
> \caption[Example of a Realistic Journal Article.]{Example of a Realistic
> Journal Article. Note that all the commands you see in this example will be
> explained later in the introduction.} \label{document}
> 
400c415
< \section{A Typical Commandline Session}
---
> \section{A Typical Command Line Session}
405,411c420,427
< fancy buttons to press. It is just a program which crunches away
< on your input file. Some \LaTeX{} installations feature a graphical
< front end where you can click \LaTeX{} into compiling your input file.
< But Real Men don't Click, so here is how to coax \LaTeX{} into
< compiling your input file on a text based system. Please note, this
< description assumes that a working \LaTeX{} installation already sitts
< on your computer.
---
> fancy buttons to press. It is just a program that crunches away at your
> input file. Some \LaTeX{} installations feature a graphical front-end where
> you can click \LaTeX{} into compiling your input file. On other systems
> there might be some typing involved, so here is how to coax \LaTeX{} into
> compiling your input file on a text based system. Please note: this
> description assumes that a working \LaTeX{} installation already sits on
> your computer.\footnote{This is the case with most well groomed Unix
> Systems, and \ldots{} Real Men use Unix, so \ldots{} \texttt{;-)}}
417c433
<   text.  On Unix all the editors will create just that. On windows you
---
>   text.  On Unix all the editors will create just that. On Windows you
420c436
<   sure it bears the extention \texttt{.tex}.
---
>   sure it bears the extension \eei{.tex}.
423,427c439,447
< Run \LaTeX{} on your input file. If successful you will end up
< with a \texttt{.dvi} file.
< \begin{verbatim}
< latex foo.tex
< \end{verbatim}
---
> 
> Run \LaTeX{} on your input file. If successful you will end up with a
> \texttt{.dvi} file. It may be necessary to run \LaTeX{} several times to get
> the table of contents and all internal references right. When your input
> file has a bug \LaTeX{} will tell you about it and stop processing your
> input file. Type \texttt{ctrl-D} to get back to the command line.
> \begin{lscommand}
> \verb+latex foo.tex+
> \end{lscommand}
430,444c450,465
< Now you may view the DVI file.
< \begin{verbatim}
< xdvi foo.dvi
< \end{verbatim}
< or
< convert it to PS
< \begin{verbatim}
< dvips -Pcmz foo.dvi -o foo.ps
< \end{verbatim}
< \texttt{\wi{xdvi}} and \texttt{\wi{dvips}} are open-source
< tools for handling \texttt{.dvi} files. The first displays them on
< screen within the X11 environment and the other
< creates a PostScript file for printing. If you are not working on
< a Unix system, other means for handling the \texttt{.dvi} files may be
< provided. 
---
> Now you may view the DVI file. There are several ways to do that. You can show the file on screen with
> \begin{lscommand}
> \verb+xdvi foo.dvi &+
> \end{lscommand}
> This only works on Unix with X11. If you are on Windows you might want to try \texttt{yap} (yet another previewer).
> 
> You can also convert the dvi file to \PSi{} for printing or viewing with Ghostscript.
> \begin{lscommand}
> \verb+dvips -Pcmz foo.dvi -o foo.ps+
> \end{lscommand}
> 
> If you are lucky your \LaTeX{} system even comes with the \texttt{dvipdf} tool, which allows
> you to convert your \texttt{.dvi} files straight into pdf.
> \begin{lscommand}
> \verb+dvipdf foo.dvi+
> \end{lscommand}
471c492
< \begin{lined}{12cm}
---
> \begin{lined}{\textwidth}
476a498,503
> \item [\normalfont\texttt{proc}] a class for proceedings based on the article class.
>   \index{proc class}
> \item [\normalfont\texttt{minimal}] is as small as it can get.
> It only sets a page size and a base font. It is mainly used for debugging
> purposes.
>   \index{minimal class}
482c509
<         \texttt{CTAN:/tex-archive/macros/latex/contrib/supported/foiltex}} instead.
---
>         \CTANref|macros/latex/contrib/supported/foiltex|} instead.
490c517
< \begin{lined}{12cm}
---
> \begin{lined}{\textwidth}
516c543
<   document in \wi{one column}\wi{two column}s.
---
>   document in \wi{one column} or \wi{two column}s.
525c552
< 
---
> \item[\normalfont\texttt{landscape}] \quad Changes the layout of the document to print in landscape mode.
557,558c585,586
< \noindent command where \emph{package} is the name of the package and
< \emph{options} is a list of keywords which trigger special features in
---
> \noindent command, where \emph{package} is the name of the package and
> \emph{options} is a list of keywords that trigger special features in
563c591
< It contains descriptions of hundreds of packages along with
---
> It contains descriptions on hundreds of packages, along with
566c594,598
< \begin{table}[!hbp]
---
> Modern \TeX{} distributions come with a large number of packages
> preinstalled. If you are working on a Unix system, use the command
> \texttt{texdoc} for accessing package documentation.
> 
> \begin{table}[btp]
568c600
< \begin{lined}{11cm}
---
> \begin{lined}{\textwidth}
607,662d638
< \section{Files you might encounter}
< 
< When you work with \LaTeX{} you will soon find yourself in a maze of
< files with various \wi{extension}s and probably no clue. Below there is a
< list telling about the various \wi{file types} you might encounter when
< working with \TeX{}. Please note that this table does not claim to be
< a complete list of extensions, but if you find one missing which you
< think is important, please drop a line.
< 
< \begin{description}
<   
< \item[\wi{.tex}] \LaTeX{} or \TeX{} input file. Can be compiled with
<   \texttt{latex}.
< \item[\wi{.sty}] \LaTeX{} Macro package. This is a file you can load
<   into your \LaTeX{} document using the \ci{usepackage} command.
< \item[\wi{.dtx}] Documented \TeX{}. This is the main distribution
<   format for \LaTeX{} style files. If you process a .dtx file you get
<   documented macro code of the \LaTeX{} package contained in the .dtx
<   file.
< \item[\wi{.ins}] Is the installer for the files contained in the
<   matching .dtx file. If you download a \LaTeX{} package from the net,
<   you will normally get a .dtx and a .ins file. Run \LaTeX{} on the
<   .ins file to unpack the .dtx file.
< \item[\wi{.cls}] Class files define what your document looks
<   like. They are selected with the \ci{documentclass} command.
< \end{description}
< The following files are generated when you run \LaTeX{} on your input
< file:
< 
< \begin{description}
< \item[\wi{.dvi}] Device Independent file. This is the main result of a \LaTeX{}
<   compile run. You can look at its content with a DVI previewer
<   program or you can send it to a printer with \texttt{dvips} or a
<   similar application.
< \item[\wi{.log}] Gives a detailed account of what happened during the
<   last compiler run.
< \item[\wi{.toc}] Stores all your section headers. It gets read in for the
<   next compiler run and is used to produce the table of content.
< \item[\wi{.lof}] This is like .toc but for the list of figures.
< \item[\wi{.lot}] And again the same for the list of tables.
< \item[\wi{.aux}] Another file which transports information from one
<   compiler run to the next. Among other things, the .aux file is used
<   to store information associated with crossreferences.
< \item[\wi{.idx}] If your document contains an index. \LaTeX{} stores all
<   the words which go into the index in this file. Process this file with
<   \texttt{makeindex}. Refer to section \ref{sec:indexing} on
<   page \pageref{sec:indexing} for more information on indexing.
< \item[\wi{.ind}] Is the processed .idx file, ready for inclusion into your
<   document on the next compile cycle.
< \item[\wi{.ilg}] Logfile telling about what \texttt{makeindex} did.
< \end{description}
< 
< 
< % Package Info pointer
< %
< %
680c656
< \begin{lined}{12cm}
---
> \begin{lined}{\textwidth}
707a684,742
> \section{Files You Might Encounter}
> 
> When you work with \LaTeX{} you will soon find yourself in a maze of
> files with various \wi{extension}s and probably no clue. The following
> list explains the various \wi{file types} you might encounter when
> working with \TeX{}. Please note that this table does not claim to be
> a complete list of extensions, but if you find one missing that you
> think is important, please drop me a line.
> 
> \begin{description}
>   
> \item[\eei{.tex}] \LaTeX{} or \TeX{} input file. Can be compiled with
>   \texttt{latex}.
> \item[\eei{.sty}] \LaTeX{} Macro package. This is a file you can load
>   into your \LaTeX{} document using the \ci{usepackage} command.
> \item[\eei{.dtx}] Documented \TeX{}. This is the main distribution
>   format for \LaTeX{} style files. If you process a .dtx file you get
>   documented macro code of the \LaTeX{} package contained in the .dtx
>   file.
> \item[\eei{.ins}] The installer for the files contained in the
>   matching .dtx file. If you download a \LaTeX{} package from the net,
>   you will normally get a .dtx and a .ins file. Run \LaTeX{} on the
>   .ins file to unpack the .dtx file.
> \item[\eei{.cls}] Class files define what your document looks
>   like. They are selected with the \ci{documentclass} command.
> \item[\eei{.fd}] Font description file telling  \LaTeX{} about new fonts.
> \end{description}
> The following files are generated when you run \LaTeX{} on your input
> file:
> 
> \begin{description}
> \item[\eei{.dvi}] Device Independent File. This is the main result of a \LaTeX{}
>   compile run. You can look at its content with a DVI previewer
>   program or you can send it to a printer with \texttt{dvips} or a
>   similar application.
> \item[\eei{.log}] Gives a detailed account of what happened during the
>   last compiler run.
> \item[\eei{.toc}] Stores all your section headers. It gets read in for the
>   next compiler run and is used to produce the table of content.
> \item[\eei{.lof}] This is like .toc but for the list of figures.
> \item[\eei{.lot}] And again the same for the list of tables.
> \item[\eei{.aux}] Another file that transports information from one
>   compiler run to the next. Among other things, the .aux file is used
>   to store information associated with cross-references.
> \item[\eei{.idx}] If your document contains an index. \LaTeX{} stores all
>   the words that go into the index in this file. Process this file with
>   \texttt{makeindex}. Refer to section \ref{sec:indexing} on
>   page \pageref{sec:indexing} for more information on indexing.
> \item[\eei{.ind}] The processed .idx file, ready for inclusion into your
>   document on the next compile cycle.
> \item[\eei{.ilg}] Logfile telling what \texttt{makeindex} did.
> \end{description}
> 
> 
> % Package Info pointer
> %
> %
> 
> 
715c750
< into several parts. \LaTeX{} has two commands which help you to do
---
> into several parts. \LaTeX{} has two commands that help you to do
721c756
< \noindent you can use this command in the document body to insert the
---
> \noindent You can use this command in the document body to insert the
733c768
< \ci{include} commands for the filenames which are listed in the
---
> \ci{include} commands for the filenames that are listed in the
739c774
< pagebreaks will not move, even when some included files are omitted.
---
> page breaks will not move, even when some included files are omitted.
763,766c797,803
< %%% Local Variables: 
< %%% mode: latex
< %%% TeX-master: "lshort"
< %%% End: 
---
> %
> 
> % Local Variables:
> % TeX-master: "lshort2e"
> % mode: latex
> % mode: flyspell
> % End:
