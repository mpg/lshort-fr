3c3
< % $Id: custom.tex,v 1.2 1998/09/29 08:05:09 oetiker Exp oetiker $
---
> % $Id: custom.tex,v 1.2 2003/03/19 20:57:45 oetiker Exp $
8c8
< Documents produced by using the commands you have learned up to this
---
> Documents produced with the commands you have learned up to this
10c10
< looking fancy, they obey all the established rules of good
---
> fancy-looking, they obey all the established rules of good
13,14c13,14
< However there are situations where \LaTeX{} does not provide a
< command or environment which matches your needs, or the output
---
> However, there are situations where \LaTeX{} does not provide a
> command or environment that matches your needs, or the output
19c19
< which looks different than what is provided by default.
---
> that looks different from what is provided by default.
37,40c37,40
< In this example, I am using both a new environment called
< \ei{lscommand} which is responsible for drawing the box around the
< command and a new command named \ci{ci} which typesets the command
< name and also makes a corresponding entry in the index. You can check
---
> In this example, I am using both a new environment called\\
> \ei{lscommand}, which is responsible for drawing the box around the
> command, and a new command named \ci{ci}, which typesets the command
> name and makes a corresponding entry in the index. You can check
69c69
< short for ``The Not So Short Introduction to \LaTeXe''. Such a command
---
> short for ``The Not So Short Introduction to \LaTeXe.'' Such a command
82c82
< command which takes one argument.
---
> command that takes one argument.
98c98
< \LaTeX{} will not allow you to create a new command which would
---
> \LaTeX{} will not allow you to create a new command that would
112c112
< Similar to the \verb|\newcommand| command, there is also a command
---
> Just as with the \verb|\newcommand| command, there is a command
122,123c122,123
< Like the \verb|\newcommand| command, you can use \ci{newenvironment}
< with an optional argument or without. The material specified
---
> Again \ci{newenvironment} can have
> an optional argument. The material specified
144c144
< an environment which already exists. If you ever want to change an
---
> an environment that already exists. If you ever want to change an
148c148
< The commands used in this example will be explained later: For the
---
> The commands used in this example will be explained later. For the
153c153,212
< \subsection{Your own Package}
---
> \subsection{Extra Space}
> 
> When creating a new environment you may easily get bitten by extra spaces
> creeping in, which can potentially have fatal effects. For example when you
> want to create a title environment which supresses its own indentation as
> well as the one on the following paragraph. The \ci{ignorespaces} command in
> the begin block of the environment will make it ignore any space after
> executing the begin block. The end block is a bit more tricky as special
> processing occurs at the end of an environment. With the
> \ci{ignorespacesafterend} \LaTeX{} will issue an \ci{ignorespaces} after the
> special `end' processing has occured.
> 
> \begin{example}
> \newenvironment{simple}%
>  {\noindent}%
>  {\par\noindent}
> 
> \begin{simple}
> See the space\\to the left.
> \end{simple}
> Same\\here.
> \end{example}
> 
> \begin{example}
> \newenvironment{correct}%
>  {\noindent\ignorespaces}%
>  {\par\noindent%
>    \ignorespacesafterend}
> 
> \begin{correct}
> No space\\to the left.
> \end{correct}
> Same\\here.
> \end{example}
> 
> \subsection{Commandline \LaTeX}
> 
> If you work on a Unix like OS, you might be using Makefiles to build your
> \LaTeX{} projects. In that connection it might be interesting to produce
> different versions of the same document by calling \LaTeX{} with commandline
> parameters. If you add the following structure to your document:
> 
> \begin{verbatim}
> \usepackage{ifthen}
> \ifthenelse{\equal{\blackandwhite}{true}}{
>   % "black and white" mode; do something..
> }{
>   % "color" mode; do something different..
> }
> \end{verbatim}
> 
> Now you can call \LaTeX{} like this:
> \begin{verbatim}
> latex '\newcommand{\blackandwhite}{true}\input{test.tex}'
> \end{verbatim}
> 
> First the command \verb|\blackandwhite| gets defined and then the actual file is read with input.
> By setting \verb|\blackandwhite| to false the color version of the document would be produced.
> 
> \subsection{Your Own Package}
166c225,226
< \newcommand{\tnss}{The not so Short Introduction to \LaTeXe}
---
> \newcommand{\tnss}{The not so Short Introduction 
>                    to \LaTeXe}
175c235
< Writing a package consists  basically in copying the contents of
---
> Writing a package basically consists of copying the contents of
185c245
< package which contains the commands defined in the examples above.
---
> package that contains the commands defined in the examples above.
189c249
< \subsection{Font changing Commands}
---
> \subsection{Font Changing Commands}
206,208c266,268
< One important feature of \LaTeXe{} is, that the font attributes are
< independent. This means, that you can issue size or even font
< changing commands and still keep the bold or slant attribute set
---
> One important feature of \LaTeXe{} is that the font attributes are
> independent. This means that you can issue size or even font
> changing commands, and still keep the bold or slant attribute set
213,214c273,274
< switch to another font for math typesetting there exists another
< special set of commands. Refer to Table~\ref{mathfonts}.
---
> switch to another font for math typesetting you need another
> special set of commands; refer to Table~\ref{mathfonts}.
224,232c284,292
< \ci{textrm}\verb|{...}|        &      \textrm{\wi{roman}}&
< \ci{textsf}\verb|{...}|        &      \textsf{\wi{sans serif}}\\
< \ci{texttt}\verb|{...}|        &      \texttt{typewriter}\\[6pt]
< \ci{textmd}\verb|{...}|        &      \textmd{medium}&
< \ci{textbf}\verb|{...}|        &      \textbf{\wi{bold face}}\\[6pt]
< \ci{textup}\verb|{...}|        &       \textup{\wi{upright}}&
< \ci{textit}\verb|{...}|        &       \textit{\wi{italic}}\\
< \ci{textsl}\verb|{...}|        &       \textsl{\wi{slanted}}&
< \ci{textsc}\verb|{...}|        &       \textsc{\wi{small caps}}\\[6pt]
---
> \fni{textrm}\verb|{...}|        &      \textrm{\wi{roman}}&
> \fni{textsf}\verb|{...}|        &      \textsf{\wi{sans serif}}\\
> \fni{texttt}\verb|{...}|        &      \texttt{typewriter}\\[6pt]
> \fni{textmd}\verb|{...}|        &      \textmd{medium}&
> \fni{textbf}\verb|{...}|        &      \textbf{\wi{bold face}}\\[6pt]
> \fni{textup}\verb|{...}|        &       \textup{\wi{upright}}&
> \fni{textit}\verb|{...}|        &       \textit{\wi{italic}}\\
> \fni{textsl}\verb|{...}|        &       \textsl{\wi{slanted}}&
> \fni{textsc}\verb|{...}|        &       \textsc{\wi{Small Caps}}\\[6pt]
234c294
< \ci{textnormal}\verb|{...}|    &    \textnormal{document} font
---
> \fni{textnormal}\verb|{...}|    &    \textnormal{document} font
247,252c307,312
< \ci{tiny}      & \tiny        tiny font \\
< \ci{scriptsize}   & \scriptsize  very small font\\
< \ci{footnotesize} & \footnotesize  quite small font \\
< \ci{small}        &  \small            small font \\
< \ci{normalsize}   &  \normalsize  normal font \\
< \ci{large}        &  \large       large font
---
> \fni{tiny}      & \tiny        tiny font \\
> \fni{scriptsize}   & \scriptsize  very small font\\
> \fni{footnotesize} & \footnotesize  quite small font \\
> \fni{small}        &  \small            small font \\
> \fni{normalsize}   &  \normalsize  normal font \\
> \fni{large}        &  \large       large font
255,258c315,318
< \ci{Large}        &  \Large       larger font \\[5pt]
< \ci{LARGE}        &  \LARGE       very large font \\[5pt]
< \ci{huge}         &  \huge        huge \\[5pt]
< \ci{Huge}         &  \Huge        largest
---
> \fni{Large}        &  \Large       larger font \\[5pt]
> \fni{LARGE}        &  \LARGE       very large font \\[5pt]
> \fni{huge}         &  \huge        huge \\[5pt]
> \fni{Huge}         &  \Huge        largest
293,303c353,361
< \begin{lined}{\textwidth}
< \begin{tabular}{@{}lll@{}}
< \textit{Command}&\textit{Example}&    \textit{Output}\\[6pt]
< \ci{mathcal}\verb|{...}|&    \verb|$\mathcal{B}=c$|&     $\mathcal{B}=c$\\
< \ci{mathrm}\verb|{...}|&     \verb|$\mathrm{K}_2$|&      $\mathrm{K}_2$\\
< \ci{mathbf}\verb|{...}|&     \verb|$\sum x=\mathbf{v}$|& $\sum x=\mathbf{v}$\\
< \ci{mathsf}\verb|{...}|&     \verb|$\mathsf{G\times R}$|&        $\mathsf{G\times R}$\\
< \ci{mathtt}\verb|{...}|&     \verb|$\mathtt{L}(b,c)$|&   $\mathtt{L}(b,c)$\\
< \ci{mathnormal}\verb|{...}|& \verb|$\mathnormal{R_{19}}\neq R_{19}$|&
< $\mathnormal{R_{19}}\neq R_{19}$\\
< \ci{mathit}\verb|{...}|&     \verb|$\mathit{ffi}\neq ffi$|& $\mathit{ffi}\neq ffi$
---
> \begin{lined}{0.7\textwidth}
> \begin{tabular}{@{}ll@{}}
> \fni{mathrm}\verb|{...}|&     $\mathrm{Roman\ Font}$\\
> \fni{mathbf}\verb|{...}|&     $\mathbf{Boldface\ Font}$\\
> \fni{mathsf}\verb|{...}|&     $\mathsf{Sans\ Serif\ Font}$\\
> \fni{mathtt}\verb|{...}|&     $\mathtt{Typewriter\ Font}$\\
> \fni{mathit}\verb|{...}|&     $\mathit{Italic\ Font}$\\
> \fni{mathcal}\verb|{...}|&    $\mathcal{CALLIGRAPHIC\ FONT}$\\
> \fni{mathnormal}\verb|{...}|& $\mathnormal{Normal\ Font}$\\
305a364,376
> %\begin{tabular}{@{}lll@{}}
> %\textit{Command}&\textit{Example}&    \textit{Output}\\[6pt]
> %\fni{mathcal}\verb|{...}|&    \verb|$\mathcal{B}=c$|&     $\mathcal{B}=c$\\
> %\fni{mathscr}\verb|{...}|&    \verb|$\mathscr{B}=c$|&     $\mathscr{B}=c$\\
> %\fni{mathrm}\verb|{...}|&     \verb|$\mathrm{K}_2$|&      $\mathrm{K}_2$\\
> %\fni{mathbf}\verb|{...}|&     \verb|$\sum x=\mathbf{v}$|& $\sum x=\mathbf{v}$\\
> %\fni{mathsf}\verb|{...}|&     \verb|$\mathsf{G\times R}$|&        $\mathsf{G\times R}$\\
> %\fni{mathtt}\verb|{...}|&     \verb|$\mathtt{L}(b,c)$|&   $\mathtt{L}(b,c)$\\
> %\fni{mathnormal}\verb|{...}|& \verb|$\mathnormal{R_{19}}\neq R_{19}$|&
> %$\mathnormal{R_{19}}\neq R_{19}$\\
> %\fni{mathit}\verb|{...}|&     \verb|$\mathit{ffi}\neq ffi$|& $\mathit{ffi}\neq ffi$
> %\end{tabular}
> 
327,328c398,400
< {\Large Don't read this! It is not
< true. You can believe me!\par}
---
> {\Large Don't read this! 
>  It is not true.
>  You can believe me!\par}
362c434,435
< \newcommand{\oops}[1]{\textbf{#1}}
---
> \newcommand{\oops}[1]{%
>  \textbf{#1}}
364c437
< it's occupied by a \oops{machine}
---
> it's occupied by \oops{machines}
369,370c442,443
< stage whether you want to use some other visual representation of danger
< than \verb|\textbf| without having to wade through your document,
---
> stage that you want to use some visual representation of danger other
> than \verb|\textbf|, without having to wade through your document,
401c474
< the lines are not spread, therefore the default line spread factor
---
> the lines are not spread, so the default line spread factor
403a477,497
> Note that the effect of the \ci{linespread} command is rather drastic and     
> not appropriate for published work. So if you have a good reason for
> changing the line spacing you might want to use the command:
> \begin{lscommand}
> \verb|\setlength{\baselineskip}{1.5\baselineskip}|
> \end{lscommand}
> 
> \begin{example}
> {\setlength{\baselineskip}%
>            {1.5\baselineskip}
> This paragraph is typeset with
> the baseline skip set to 1.5 of
> what it was before. Note the par
> command at the end of the
> paragraph.\par}
> 
> This paragraph has a clear
> purpose, it shows that after the
> curly brace has been closed,
> everything is back to normal.
> \end{example}
418,419c512,513
< \TeX{}, that it can compress and expand the inter paragraph skip by the
< amount specified if this is necessary to properly fit the paragraphs
---
> \TeX{} that it can compress and expand the inter paragraph skip by the
> amount specified, if this is necessary to properly fit the paragraphs
427c521
< place after the \verb|\tableofcontents| or to not use them at all,
---
> place below the command \verb|\tableofcontents| or to not use them at all,
432c526
< If you want to indent a paragraph which is not indented, you can use 
---
> If you want to indent a paragraph that is not indented, you can use 
458c552
< \emph{length} in the simplest case just is a number plus a unit.  The
---
> \emph{length} in the simplest case is just a number plus a unit.  The
489,491c583,587
< remaining space on a line is filled up. If two
< \verb|\hspace{\stretch{|\emph{n}\verb|}}| commands are issued on the
< same line, they grow according to the stretch factor.
---
> remaining space on a line is filled up. If multiple
> \verb|\hspace{\stretch{|\emph{n}\verb|}}| commands are issued on the same
> line, they occupy all available space in proportion of their respective
> stretch factors.
> 
497a594,603
> When using horizontal space together with text, it may make sense to make
> the space adjust its size relative to the size of the current font.
> This can be done by using the text-relative units \texttt{em} and
> \texttt{ex}:
> 
> \begin{example}
> {\Large{}big\hspace{1em}y}\\
> {\tiny{}tin\hspace{1em}y}
> \end{example}
>  
508c614
< the starred version of the command \verb|\vspace*| instead of \verb|\vspace|.
---
> the starred version of the command, \verb|\vspace*|, instead of \verb|\vspace|.
511c617
< The \verb|\stretch| command in connection with \verb|\pagebreak| can
---
> The \verb|\stretch| command, in connection with \verb|\pagebreak|, can
538c644
< \makeatletter\@layout\makeatother
---
> \makeatletter\@mylayout\makeatother
542a649,660
> \cih{footskip}
> \cih{headheight}
> \cih{headsep}
> \cih{marginparpush}
> \cih{marginparsep}
> \cih{marginparwidth}
> \cih{oddsidemargin}
> \cih{paperheight}
> \cih{paperwidth}
> \cih{textheight}
> \cih{textwidth}
> \cih{topmargin}
548c665
< text \wi{margins}. But sometimes you may not be happy with 
---
> text \wi{margins}, but sometimes you may not be happy with 
552,554c669,671
< Figure~\ref{fig:layout} shows all the parameters which can be changed.
< The figure was produced with the \pai{layout} package from the tools bundle%
< \footnote{\texttt{CTAN:/tex-archive/macros/latex/required/tools}}.
---
> Figure~\ref{fig:layout} shows all the parameters that can be changed.
> The figure was produced with the \pai{layout} package from the tools bundle.%
> \footnote{\CTANref|macros/latex/required/tools|}
570c687
< This is also the reason why newspapers are typeset in multiple columns.
---
> This is also why newspapers are typeset in multiple columns.
584c701
< The second command adds a length to any of the parameters. 
---
> The second command adds a length to any of the parameters:
598,599c715,716
< In this context, you might want to look at the \pai{calc} package,
< it allows you to use arithmetic operations in the argument of setlength
---
> In this context, you might want to look at the \pai{calc} package.
> It allows you to use arithmetic operations in the argument of \ci{setlength}
603c720
< \section{More fun with lengths}
---
> \section{More Fun With Lengths}
614,616c731,733
< \ci{settoheight}\verb|{|\emph{lscommand}\verb|}{|\emph{text}\verb|}|\\
< \ci{settodepth}\verb|{|\emph{lscommand}\verb|}{|\emph{text}\verb|}|\\
< \ci{settowidth}\verb|{|\emph{lscommand}\verb|}{|\emph{text}\verb|}|
---
> \ci{settoheight}\verb|{|\emph{variable}\verb|}{|\emph{text}\verb|}|\\
> \ci{settodepth}\verb|{|\emph{variable}\verb|}{|\emph{text}\verb|}|\\
> \ci{settowidth}\verb|{|\emph{variable}\verb|}{|\emph{text}\verb|}|
632c749
< $b$ -- are adjunct to the right 
---
> $b$ -- are adjoin to the right 
651,653c768,770
< the point is that \TeX{} operates on glue and boxes. Not only a letter
< can be a box. You can put virtually everything into a box including
< other boxes. Each box will then be handled by \LaTeX{} as if it was a
---
> the point is that \TeX{} operates on glue and boxes. Letters are not the only things that
> can be boxes. You can put virtually everything into a box, including
> other boxes. Each box will then be handled by \LaTeX{} as if it were a
680,681c797,798
< between a \ei{minipage} and a \ei{parbox} is that you cannot use all commands
< and environments inside a \ei{parbox} while almost anything is possible in
---
> between a \ei{minipage} and a \ci{parbox} is that you cannot use all commands
> and environments inside a \ei{parbox}, while almost anything is possible in
685,687c802,804
< everything, there is also a class of boxing commands which operates
< only on horizontally aligned material. We already know one of them.
< It's called \ci{mbox}, it simply packs up a series of boxes into
---
> everything, there is also a class of boxing commands that operates
> only on horizontally aligned material. We already know one of them;
> it's called \ci{mbox}. It simply packs up a series of boxes into
698,706c815,822
<   material inside the box. You can even set the
<   width to 0pt so that the text inside the box will be typeset without
<   influencing the surrounding boxes.}  Apart from the length
<   expressions you can also use \ci{width}, \ci{height}, \ci{depth} and
<   \ci{totalheight} in the width parameter. They are set from values
<   obtained by measuring the typeset \emph{text}. The \emph{pos} parameter takes
< a one letter value: \textbf{c}enter, \textbf{l}eft flush,
< \textbf{r}ight flush or \textbf{s} which spreads the text inside the
< box to fill it.
---
> material inside the box. You can even set the
> width to 0pt so that the text inside the box will be typeset without
> influencing the surrounding boxes.}  Besides the length
> expressions, you can also use \ci{width}, \ci{height}, \ci{depth}, and
> \ci{totalheight} in the width parameter. They are set from values
> obtained by measuring the typeset \emph{text}. The \emph{pos} parameter takes
> a one letter value: \textbf{c}enter, flush\textbf{l}eft,
> flush\textbf{r}ight, or \textbf{s}pread the text to fill the box.
735c851
< \ci{raisebox}\verb|{|\emph{lift}\verb|}[|\emph{depth}\verb|][|\emph{height}\verb|]{|\emph{text}\verb|}|
---
> \ci{raisebox}\verb|{|\emph{lift}\verb|}[|\emph{extend-above-baseline}\verb|][|\emph{extend-below-baseline}\verb|]{|\emph{text}\verb|}|
739c855
< box. You can use \ci{width}, \ci{height}, \ci{depth} and
---
> box. You can use \ci{width}, \ci{height}, \ci{depth}, and
766c882
< \newpage
---
> 
776c892
< lines. The line on the title page for example, has been created with a
---
> lines. The line on the title page, for example, has been created with a
795,798c911,914
< %%% Local Variables: 
< %%% mode: latex
< %%% TeX-master: "lshort2e"
< %%% End: 
---
> \bigskip
> {\flushright The End.\par}
> 
> %
799a916,920
> % Local Variables:
> % TeX-master: "lshort2e"
> % mode: latex
> % mode: flyspell
> % End:
