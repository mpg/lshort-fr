3c3
< % $Id: math.tex,v 1.4 1998/09/29 08:05:09 oetiker Exp oetiker $
---
> % $Id: math.tex,v 1.2 2003/03/19 20:57:46 oetiker Exp $
14,16c14
<   that your problem is addressed in \AmS-\LaTeX{}%
<   \footnote{\texttt{CTAN:/tex-archive/macros/latex/required/amslatex}}
<   or some other package.
---
>   that your problem is addressed in \AmS-\LaTeX{}.
19d16
< \section{General}
21,25c18,41
< \LaTeX{} has a special mode for typesetting \wi{mathematics}.
< Mathematical text within a paragraph is entered between \ci{(}
< and \ci{)}, \index{$@\texttt{\$}} %$
< between \texttt{\$} and \texttt{\$} or between %}
< \verb|\begin{|\ei{math}\verb|}| and \verb|\end{math}|.\index{formulae}
---
> 
> \section{The \texorpdfstring{\AmS}{AMS}-\LaTeX{} bundle}
> 
> If you want to typeset (advanced) \wi{mathematics}, you should
> use \AmS-\LaTeX{}. The \AmS-\LaTeX{} bundle is a collection of packages and classes for
> mathematical typesetting. We will mostly deal with the \pai{amsmath} package
> which is a part of the bundle. \AmS-\LaTeX{} is produced by The \emph{\wi{American Mathematical Society}} 
> and it is used extensively for mathematical typesetting. \LaTeX{} itself does provide
> some basic features and environments for mathematics, but they are limited (or
> maybe it's the other way around: \AmS-\LaTeX{} is \emph{unlimited}!) and
> in some cases inconsistent. 
> 
> \AmS-\LaTeX{} is a part of the required distribution and is provided
> with all recent \LaTeX{} distributions.\footnote{If yours is missing it, go to
>   \texttt{CTAN:macros/latex/required/amslatex}.} In this chapter, we assume
>   \pai{amsmath} is loaded in the preamble; \verb|\usepackage{amsmath}|.
> 
> \section{Single Equations}
>   
> There two ways to typeset mathematical \wi{formulae}: in-line within a paragraph
> (\emph{\wi{text style}}), or the paragraph can be broken to 
> typeset it separately (\textit{\wi{display style}}). Mathematical \wi{equation}s 
> \emph{within} a paragraph is entered between \index{$@\texttt{\$}} %$
> between \texttt{\$} and \texttt{\$}:
30c46
< $c^{2}=a^{2}+b^{2}$
---
> $a^2 + b^2 = c^2$
34,35c50,51
< $\tau\epsilon\chi$.\\[6pt]
< 100~m$^{3}$ of water\\[6pt]
---
> $\tau\epsilon\chi$\\[5pt]
> 100~m$^{3}$ of water\\[5pt]
39,45c55,64
< It is preferable to \emph{display} larger mathematical equations or formulae,
< rather than to typeset them on separate lines. This means you enclose them
< in \ci{[} and \ci{]} or between
< \verb|\begin{|\ei{displaymath}\verb|}| and
<   \verb|\end{displaymath}|.  This produces formulae which are not
< numbered. If you want \LaTeX{} to number them, you can use the
< \ei{equation} environment.
---
> If you want your larger equations to be set apart
> from the rest of the paragraph, it is preferable to \emph{display} them
> rather than to break the paragraph apart.
> To do this, you enclose them between \verb|\begin{|\ei{equation}\verb|}| and
> \verb|\end{equation}|.\footnote{This is an \textsf{amsmath} command. If you don't
> have access to the package for some obscure reason, you can use \LaTeX's own
> \ei{displaymath} environment instead.} You can then \ci{label} an equation number and refer to
> it somewhere else in the text by using the \ci{eqref} command. If you want to
> name the equation something specific, you \ci{tag} it instead. You can't use
> \ci{eqref} with \ci{tag}.
49,53c68,81
< a more mathematical approach:
< \begin{displaymath}
< c^{2}=a^{2}+b^{2}
< \end{displaymath}
< And just one more line.
---
> a more mathematical approach
>  \begin{equation}
>    a^2 + b^2 = c^2
>  \end{equation}
> Einstein says
>  \begin{equation}
>    E = mc^2 \label{clever}
>  \end{equation}
> He didn't say
>  \begin{equation}
>   1 + 1 = 3 \tag{dumb}
>  \end{equation}
> This is a reference to 
> \eqref{clever}. 
56c84,93
< You can reference an equation with \ci{label} and \ci{ref}
---
> If you don't want \LaTeX{} to number the equations, use the starred
> version of \texttt{equation} using an asterisk, \ei{equation*}, or even easier, enclose the
> equation in \ci{[} and \ci{]}:\footnote{\index{equation!\textsf{amsmath}}
>   \index{equation!\LaTeX{}}This is again from \textsf{amsmath}. If you 
> didn't load the package, use \LaTeX{}'s own \texttt{equation} environment
> instead. The naming of the \textsf{amsmath}/\LaTeX{} commands may seem a bit
> confusing, but it's really not a problem since everybody uses \textsf{amsmath} anyway.
> In general, it is best to load the package from the beginning because you might use it
> later on, and then \LaTeX's unnumbered \texttt{equation} clashes with
> \AmS-\LaTeX's numbered \texttt{equation}.}
58,62c95,103
< \begin{equation} \label{eq:eps}
< \epsilon > 0
< \end{equation}
< From (\ref{eq:eps}), we gather 
< \ldots
---
> Add $a$ squared and $b$ squared
> to get $c$ squared. Or, using
> a more mathematical approach
>  \begin{equation*}
>    a^2 + b^2 = c^2
>  \end{equation*}
> or you can type less for the
> same effect:
>  \[ a^2 + b^2 = c^2 \]
65c106,108
< Note that expressions will be typeset in a different style if displayed:
---
> 
> Note the difference in typesetting style between \wi{text style} and \wi{display style}
> equations: 
66a110
> This is text style: 
68,69c112,119
< \sum_{k=1}^n \frac{1}{k^2} 
< = \frac{\pi^2}{6}$
---
>  \sum_{k=1}^n \frac{1}{k^2} 
>  = \frac{\pi^2}{6}$.
> And this is display style:
>  \begin{equation}
>   \lim_{n \to \infty} 
>   \sum_{k=1}^n \frac{1}{k^2} 
>   = \frac{\pi^2}{6}
>  \end{equation}
70a121,125
> 
> In text style, enclose tall or deep math expressions or sub
> expressions in \ci{smash}. This makes \LaTeX{} ignore the height of
> these expressions. This keeps the line spacing even.
> 
72,76c127,133
< \begin{displaymath}
< \lim_{n \to \infty} 
< \sum_{k=1}^n \frac{1}{k^2} 
< = \frac{\pi^2}{6}
< \end{displaymath}
---
> A $d_{e_{e_p}}$ mathematical
> expression  followed by a
> $h^{i^{g^h}}$ expression. As
> opposed to a smashed 
> \smash{$d_{e_{e_p}}$} expression 
> followed by a
> \smash{$h^{i^{g^h}}$} expression.
78a136
> \subsection{Math Mode}
80,82c138,139
< 
< There are differences between \emph{math mode} and \emph{text mode}. For
< example in \emph{math mode}: 
---
> There are also differences between \emph{\wi{math mode}} and \emph{text mode}. For
> example, in \emph{math mode}: 
86,89c143,146
< \item Most spaces and linebreaks do not have any significance, as all spaces
< either are derived logically from the mathematical expressions or
< have to be specified using special commands such as \ci{,}, \ci{quad} or
< \ci{qquad}.
---
> \item \index{spacing!math mode} Most spaces and line breaks do not have any significance, as all spaces
> are either derived logically from the mathematical expressions, or
> have to be specified with special commands such as \ci{,}, \ci{quad} or
> \ci{qquad} (we'll get back to that later, see section~\ref{sec:math-spacing}).
96c153,155
< text using the \verb|\textrm{...}| commands.
---
> text using the \verb|\text{...}| command (see also section \ref{sec:fontsz} on
> page \pageref{sec:fontsz}).
> 
99,102c158,159
< \begin{equation}
< \forall x \in \mathbf{R}:
< \qquad x^{2} \geq 0
< \end{equation}
---
> $\forall x \in \mathbf{R}:
>  \qquad x^{2} \geq 0$
105,108c162,163
< \begin{equation}
< x^{2} \geq 0\qquad
< \textrm{for all }x\in\mathbf{R}
< \end{equation}
---
> $x^{2} \geq 0\qquad
>  \text{for all }x\in\mathbf{R}$
111,114d165
< 
< %
< % Add AMSSYB Package ... Blackboard bold .... R for realnumbers
< %
116c167
< it would be conventional here to use `\wi{blackboard bold}',
---
> it would be conventional here to use the `\wi{blackboard bold}' font,
118c169,172
< package \pai{amsfonts} or \pai{amssymb}.
---
> package \pai{amssymb}.\footnote{\pai{amssymb} is not a part
>   of the \AmS-\LaTeX{} bundle, but it is perhaps still a part of your \LaTeX{}
>   distribution. Check your distribution
>   or go to \texttt{CTAN:/fonts/amsfonts/latex/} to obtain it.}
122,125c176,178
< \begin{displaymath}
< x^{2} \geq 0\qquad
< \textrm{for all }x\in\mathbb{R}
< \end{displaymath}
---
> $x^{2} \geq 0\qquad
>  \text{for all } x 
>  \in \mathbb{R}$
127a181,182
> See Table~\ref{mathalpha} on \pageref{mathalpha} and
> Table~\ref{mathfonts} on \pageref{mathfonts} for more math fonts.
129d183
< \section{Grouping in Math Mode}
131,138d184
< Most math mode commands act only on the next character. So if you
< want a command to affect several characters, you have to group them
< together using curly braces: \verb|{...}|.
< \begin{example}
< \begin{equation}
< a^x+y \neq a^{x+y}
< \end{equation}
< \end{example}
142,145c188,191
< In this section, the most important commands used in mathematical
< typesetting will be described. Take a look at section~\ref{symbols} on
< page~\pageref{symbols} for a detailed list of commands for typesetting
< mathematical symbols.
---
> In this section, we describe the most important commands used in mathematical
> typesetting. Most of the commands in this section will not require
> \textsf{amsmath} (if they do, it will be stated clearly), but load it anyway.
> 
150,151c196,197
<   uppercase Alpha defined in \LaTeXe{} because it looks the same as a
<   normal roman A. Once the new math coding is done, things will
---
>   uppercase Alpha, Beta etc. defined in \LaTeXe{} because it looks the same as a 
>   normal roman A, B\ldots{} Once the new math coding is done, things will
152a199,201
> 
> Take a look at Table~\ref{greekletters} on page~\pageref{greekletters} for a
> list of Greek letters.
154c203,204
< $\lambda,\xi,\pi,\mu,\Phi,\Omega$
---
> $\lambda,\xi,\pi,\theta,
>  \mu,\Phi,\Omega,\Delta$
156,157c206
< \enlargethispage{\baselineskip}
< \pagebreak[4]
---
> 
160a210,216
> Most math mode commands act only on the next character, so if you
> want a command to affect several characters, you have to group them
> together using curly braces: \verb|{...}|.
> 
> Table~\ref{binaryrel} on page \pageref{binaryrel} lists a lot of other binary
> relations like $\subseteq$ and $\perp$.
> 
162,165c218,221
< $a_{1}$ \qquad $x^{2}$ \qquad
< $e^{-\alpha t}$ \qquad
< $a^{3}_{ij}$\\
< $e^{x^2} \neq {e^x}^2$
---
> $p^3_{ij} \qquad
>  m_\text{Knuth} \\[5pt]
>  a^x+y \neq a^{x+y}\qquad 
>  e^{x^2} \neq {e^x}^2$
168,169c224,226
< The \textbf{\wi{square root}} is entered as \ci{sqrt}, the
< $n^\mathrm{th}$ root is generated with \verb|\sqrt[|$n$\verb|]|. The size of
---
> 
> The \textbf{\wi{square root}} is entered as \ci{sqrt}; the
> $n^\text{th}$ root is generated with \verb|\sqrt[|$n$\verb|]|. The size of
171a229,231
> 
> See other kinds of arrows like $\hookrightarrow$ and $\rightleftharpoons$ on
> Table~\ref{tab:arrows} on page \pageref{tab:arrows}. 
173,176c233,255
< $\sqrt{x}$ \qquad 
< $\sqrt{ x^{2}+\sqrt{y} }$ 
< \qquad $\sqrt[3]{2}$\\[3pt]
< $\surd[x^2 + y^2]$
---
> $\sqrt{x} \Leftrightarrow x^{1/2}
>  \quad \sqrt[3]{2}
>  \quad \sqrt{x^{2} + \sqrt{y}}
>  \quad \surd[x^2 + y^2]$
> \end{example}
> 
> 
> \index{dots!three}
> \index{vertical!dots}
> \index{horizontal!dots}
> Usually you don't typeset an explicit \textbf{\wi{dot}} sign to indicate
> the multiplication operation when handling symbols; however sometimes it is written
> to help the reader's eyes in grouping a formula.
> You should use \ci{cdot} which typesets a single dot centered. \ci{cdots} is
> three centered \textbf{\wi{dots}} while \ci{ldots} sets the dots on the
> baseline. Besides that, there are \ci{vdots} for 
> vertical and \ci{ddots} for \wi{diagonal dots}. You can find another example in
> section~\ref{sec:arraymat}.
> \begin{example}
> $\Psi = v_1 \cdot v_2
>  \cdot \ldots \qquad 
>  n! = 1 \cdot 2 
>  \cdots (n-1) \cdot n$
180,181c259,260
< \textbf{horizontal lines} directly over or under an expression.
< \index{horizontal!line}
---
> \textbf{horizontal lines} directly over or under an expression:
> \index{horizontal!line} \index{line!horizontal}
183c262,263
< $\overline{m+n}$
---
> $0.\overline{3} = 
>  \underline{\underline{1/3}}$
187,188c267,268
< long \textbf{horizontal braces} over or under an expression.
< \index{horizontal!brace}
---
> long \textbf{horizontal braces} over or under an expression:
> \index{horizontal!brace} \index{brace!horizontal} 
190c270,272
< $\underbrace{ a+b+\cdots+z }_{26}$
---
> $\underbrace{\overbrace{a+b+c}^6 
>  \cdot \overbrace{d+e+f}^9}
>  _\text{meaning of life} = 42$
193,195c275,277
< \index{mathematical!accents} To add mathematical accents such as small
< arrows or \wi{tilde} signs to variables, you can use the commands
< given in Table~\ref{mathacc} on page \pageref{mathacc}.  Wide hats and
---
> \index{mathematical!accents} To add mathematical accents such as \textbf{small
> arrows} or \textbf{\wi{tilde}} signs to variables, the commands
> given in Table~\ref{mathacc} on page~\pageref{mathacc} might be useful.  Wide hats and
197,198c279,281
< and \ci{widehat}.  The \verb|'|\index{'@\verb"|'"|} symbol gives a
< \wi{prime}.
---
> and \ci{widehat}. Notice the difference between \ci{hat} and \ci{widehat} and the placement of
> \ci{bar} for a variable with subscript. The \wi{apostrophe} mark
> \verb|'|\index{'@\verb"|'"|} gives a \wi{prime}:
201,203c284,287
< \begin{displaymath}
< y=x^{2}\qquad y'=2x\qquad y''=2
< \end{displaymath}
---
> $f(x) = x^2 \qquad f'(x) 
>  = 2x \qquad f''(x) = 2\\[5pt]
>  \hat{XY} \quad \widehat{XY}
>  \quad \bar{x_0} \quad \bar{x}_0$
206c290,291
< \textbf{Vectors}\index{vectors} often are specified by adding small
---
> 
> \textbf{Vectors}\index{vectors} are often specified by adding small
209,219c294
< \ci{overleftarrow} are useful to denote the vector from $A$ to $B$.
< \begin{example}
< \begin{displaymath}
< \vec a\quad\overrightarrow{AB}
< \end{displaymath}
< \end{example}
< 
< Usually you don't typeset an explicit dot sign to indicate
< the multiplication operation. However sometimes it is written
< to help the reader's eyes in grouping a formula.
< Then you should use \ci{cdot}
---
> \ci{overleftarrow} are useful to denote the vector from $A$ to $B$:
221,224c296,298
< \begin{displaymath}
< v = {\sigma}_1 \cdot {\sigma}_2
<     {\tau}_1 \cdot {\tau}_2
< \end{displaymath}
---
> $\vec{a} \qquad
>  \vec{AB} \qquad
>  \overrightarrow{AB}$
229c303
< font and not in italic like variables. Therefore \LaTeX{} supplies the
---
> font, and not in italics as variables are, so \LaTeX{} supplies the
233,238c307,313
< \begin{tabular}{lllllll}
< \ci{arccos} &  \ci{cos}  &  \ci{csc} &  \ci{exp} &  \ci{ker}    & \ci{limsup} & \ci{min} \\
< \ci{arcsin} &  \ci{cosh} &  \ci{deg} &  \ci{gcd} &  \ci{lg}     & \ci{ln}     & \ci{Pr}  \\
< \ci{arctan} &  \ci{cot}  &  \ci{det} &  \ci{hom} &  \ci{lim}    & \ci{log}    & \ci{sec} \\
< \ci{arg}    &  \ci{coth} &  \ci{dim} &  \ci{inf} &  \ci{liminf} & \ci{max}    & \ci{sin} \\
< \ci{sinh} & \ci{sup} & \ci{tan} & \ci{tanh}\\
---
> \begin{tabular}{llllll}
> \ci{arccos} &  \ci{cos}  &  \ci{csc} &  \ci{exp} &  \ci{ker}    & \ci{limsup} \\
> \ci{arcsin} &  \ci{cosh} &  \ci{deg} &  \ci{gcd} &  \ci{lg}     & \ci{ln}     \\
> \ci{arctan} &  \ci{cot}  &  \ci{det} &  \ci{hom} &  \ci{lim}    & \ci{log}    \\
> \ci{arg}    &  \ci{coth} &  \ci{dim} &  \ci{inf} &  \ci{liminf} & \ci{max}    \\
> \ci{sinh}   & \ci{sup}   &  \ci{tan}  & \ci{tanh}&  \ci{min}    & \ci{Pr}     \\
> \ci{sec}    & \ci{sin} \\
243c318,329
< \frac{\sin x}{x}=1\]
---
>  \frac{\sin x}{x}=1\]
> \end{example}
> 
> For functions missing from the list, use the \ci{DeclareMathOperator}
> command. There is even a starred version for functions with limits.
> This command works only in the preamble so the commented lines in the
> example below must be put into the preamble.
> 
> \begin{example}
> %\DeclareMathOperator{\argh}{argh}
> %\DeclareMathOperator*{\nut}{Nut}
> \[3\argh = 2\nut_{x=1}\]
249c335,339
< such as ``$x\equiv a \pmod{b}$.''
---
> such as ``$x\equiv a \pmod{b}$:''
> \begin{example}
> $a\bmod b \\
>  x\equiv a \pmod{b}$
> \end{example}
252c342,344
< \ci{frac}\verb|{...}{...}| command.
---
> \ci{frac}\verb|{...}{...}| command. In in-line equations, the fraction is shrunk to
> fit the line. This style is obtainable in display style with \ci{tfrac}. The
> reverse, i.e.\ display style fraction in text, is made with \ci{dfrac}.
254c346
< for small amounts of `fraction material.'
---
> for small amounts of `fraction material:'
256,261c348,350
< $1\frac{1}{2}$~hours
< \begin{displaymath}
< \frac{ x^{2} }{ k+1 }\qquad
< x^{ \frac{2}{k+1} }\qquad
< x^{ 1/2 }
< \end{displaymath}
---
> In display style:
> \[3/8 \qquad \frac{3}{8} 
>  \qquad \tfrac{3}{8} \]
264,272c353,357
< To typeset binomial coefficients or similar structures, you can use
< either the command \verb|{... |\ci{choose}\verb| ...}| or 
< \verb|{... |\ci{atop}\verb| ...}|. The second command produces the
< same output as the first one, but without braces.
< \footnote{Note that the usage of these old-style commands is expressly forbidden
< by the \pai{amsmath} package. They are replaced by
< \ci{binom} and \ci{genfrac}. The latter is a superset of all related
< construct, e.g. you may get a similar construct to \ci{atop}
< by \texttt{$\backslash$newcommand\{$\backslash$newatop\}[2]\{$\backslash$genfrac\{\}\{\}\{0pt\}\{1\}\{\#1\}\{\#2\}\}}.}
---
> \begin{example}
> In text style:
> $1\frac{1}{2}$~hours \qquad
> $1\dfrac{1}{2}$~hours
> \end{example}
273a359
> Here the \ci{partial} command for \wi{partial derivative}s is used:
275,277c361,364
< \begin{displaymath}
< {n \choose k}\qquad {x \atop y+2}
< \end{displaymath}
---
> \[\sqrt{\frac{x^2}{k+1}}\qquad
>   x^\frac{2}{k+1}\qquad
>   \frac{\partial^2f}
>   {\partial x^2} \]
280,282c367,379
< For binary relations it may be useful to stack symbols over each other.
< \ci{stackrel} puts the symbol given
< in the first argument in superscript-like size over the second which
---
> To typeset \wi{binomial coefficient}s or similar structures, use
> the command \ci{binom} from \pai{amsmath}:
> \begin{example}
> Pascal's rule is
> \begin{equation*}
>  \binom{n}{k} =\binom{n-1}{k}
>  + \binom{n-1}{k-1}
> \end{equation*}
> \end{example}
> 
> For \wi{binary relations} it may be useful to stack symbols over each other.
> \ci{stackrel}\verb|{#1}{#2}| puts the symbol given
> in \verb|#1| in superscript-like size over \verb|#2| which
285,287c382,384
< \begin{displaymath}
< \int f_N(x) \stackrel{!}{=} 1
< \end{displaymath}
---
> \begin{equation*}
>  f_n(x) \stackrel{*}{\approx} 1
> \end{equation*}
291c388
< \textbf{\wi{sum operator}} with \ci{sum} and the \textbf{\wi{product operator}}
---
> \textbf{\wi{sum operator}} with \ci{sum}, and the \textbf{\wi{product operator}}
293,294c390
< and~\verb|_| like subscripts and superscripts.
< \footnote{\AmS-\LaTeX{} in addition has multiline super-/subscripts}
---
> and~\verb|_| like subscripts and superscripts:
296,298c392,394
< \begin{displaymath}
< \sum_{i=1}^{n} \qquad
< \int_{0}^{\frac{\pi}{2}} \qquad
---
> \begin{equation*}
> \sum_{i=1}^n \qquad
> \int_0^{\frac{\pi}{2}} \qquad
300c396
< \end{displaymath}
---
> \end{equation*}
303,309c399,400
< For \textbf{\wi{braces}} and other \wi{delimiters}, there exist all
< types of symbols in \TeX{} (e.g.~$[\;\langle\;\|\;\updownarrow$).
< Round and square braces can be entered with the corresponding keys,
< curly braces with \verb|\{|, all other delimiters are generated with
< special commands (e.g.~\verb|\updownarrow|). For a list of all
< delimiters available, check table~\ref{tab:delimiters} on page
< \pageref{tab:delimiters}.
---
> To get more control over the placement of indices in complex
> expressions, \pai{amsmath} provides the \ci{substack} command:
311,313c402,406
< \begin{displaymath}
< {a,b,c}\neq\{a,b,c\}
< \end{displaymath}
---
> \begin{equation*}
> \sum^n_{\substack{0<i<n \\ 
>         j\subseteq i}}
>    P(i,j) = Q(i,j)
> \end{equation*}
316,317c409,423
< If you put the command \ci{left} in front of an opening delimiter or
< \ci{right} in front of a closing delimiter, \TeX{} will automatically
---
> 
> 
> \LaTeX{} provides all sorts of symbols for \textbf{\wi{braces}} and other
> \textbf{\wi{delimiters}} (e.g.~$[\;\langle\;\|\;\updownarrow$).
> Round and square braces can be entered with the corresponding keys and
> curly braces with \verb|\{|, but all other delimiters are generated with
> special commands (e.g.~\verb|\updownarrow|).
> \begin{example}
> \begin{equation*}
> {a,b,c} \neq \{a,b,c\}
> \end{equation*}
> \end{example}
> 
> If you put \ci{left} in front of an opening delimiter and
> \ci{right} in front of a closing delimiter, \LaTeX{} will automatically
319,321c425,426
< every \ci{left} with a corresponding \ci{right}, and that the size is
< determined correctly only if both are typeset on the same line. If you
< don't want anything on the right, use the invisible `\ci{right.}'!
---
> every \ci{left} with a corresponding \ci{right}. If you
> don't want anything on the right, use the invisible ``\ci{right.}'':
323,326c428,432
< \begin{displaymath}
< 1 + \left( \frac{1}{ 1-x^{2} }
<     \right) ^3
< \end{displaymath}
---
> \begin{equation*}
> 1 + \left(\frac{1}{1-x^{2}}
>     \right)^3 \qquad 
> \left. \ddagger \frac{~}{~}\right)
> \end{equation*}
329d434
< \pagebreak[4]
333,337c438
< \ci{Bigg} as prefixes to most delimiter commands.\footnote{These
<   commands do not work as expected if a size changing command has been
<   used, or the \texttt{11pt} or \texttt{12pt} option has been
<   specified.  Use the \pai{exscale} or \pai{amsmath} packages to
<   correct this behaviour.}
---
> \ci{Bigg} as prefixes to most delimiter commands:
339,342c440,445
< $\Big( (x+1) (x-1) \Big) ^{2}$\\
< $\big(\Big(\bigg(\Bigg($\quad
< $\big\}\Big\}\bigg\}\Bigg\}$\quad
< $\big\|\Big\|\bigg\|\Bigg\|$
---
> $\Big((x+1)(x-1)\Big)^{2}$\\
> $\big( \Big( \bigg( \Bigg( \quad
> \big\} \Big\} \bigg\} \Bigg\} \quad
> \big\| \Big\| \bigg\| \Bigg\| \quad
> \big\Downarrow \Big\Downarrow 
> \bigg\Downarrow \Bigg\Downarrow$
343a447,448
>  For a list of all delimiters available, see Table~\ref{tab:delimiters} on page
> \pageref{tab:delimiters}. 
345,355c450
< To enter \textbf{\wi{three dots}} into a formula, you can use several
< commands. \ci{ldots} typesets the dots on the baseline, \ci{cdots}
< sets them centred. Besides that, there are the commands \ci{vdots} for
< vertical and \ci{ddots} for \wi{diagonal dots}.\index{vertical
<   dots}\index{horizontal!dots} You can find another example in section~\ref{sec:vert}.
< \begin{example}
< \begin{displaymath}
< x_{1},\ldots,x_{n} \qquad
< x_{1}+\cdots+x_{n}
< \end{displaymath}
< \end{example}
---
> \section{Vertically Aligned Material} 
357c452,453
< \section{Math Spacing}
---
> \subsection{Multiple Equations}
> \index{equation!multiple}
359,369c455,466
< \index{math spacing} If the spaces within formulae chosen by \TeX{}
< are not satisfactory, they can be adjusted by inserting special
< spacing commands. There are some commands for small spaces: \ci{,} for
< $\frac{3}{18}\:\textrm{quad}$ (\demowidth{0.166em}), \ci{:} for $\frac{4}{18}\:
< \textrm{quad}$ (\demowidth{0.222em}) and \ci{;} for $\frac{5}{18}\:
< \textrm{quad}$ (\demowidth{0.277em}).  The escaped space character
< \verb*.\ . generates a medium sized space and \ci{quad}
< (\demowidth{1em}) and \ci{qquad} (\demowidth{2em}) produce large
< spaces. The size of a \ci{quad} corresponds to the width of the
< character `M' of the current font.  The \verb|\!|\cih{"!} command produces a
< negative space of $-\frac{3}{18}\:\textrm{quad}$ (\demowidth{0.166em}).
---
> For formulae running over several lines or for \wi{equation system}s,
> you can use the environments \ei{align} and \verb|align*|
> instead of \texttt{equation} and \texttt{equation*}.\footnote{The \ei{align}
>   environment is from \textsf{amsmath}. A similar environment without \textsf{amsmath}
>   from \LaTeX{} is \ei{eqnarray}, but it is generally not advised to use that
> because of spacing and label inconsistencies.} With \ei{align} each line gets an
> equation number. The \verb|align*| does not number anything.
> 
> The \ei{align} environments center the single equation around the \verb|&|
> sign. The \verb|\\| command breaks the lines. If you only want to enumerate some
> of equations, use \ci{nonumber} to remove the number. It has to be placed
> \emph{before} \verb|\\|:
371,379c468,473
< \newcommand{\ud}{\mathrm{d}}
< \begin{displaymath}
< \int\!\!\!\int_{D} g(x,y)
<   \, \ud x\, \ud y 
< \end{displaymath}
< instead of 
< \begin{displaymath}
< \int\int_{D} g(x,y)\ud x \ud y
< \end{displaymath}
---
> \begin{align}
> f(x) &= (a+b)(a-b) \label{1}\\
>      &= a^2-ab+ba-b^2  \\ 
>      &= a^2+b^2 \tag{wrong}
> \end{align}
> This is a reference to \eqref{1}.
381d474
< Note that `d' in the differential is conventionally set in roman.
383,387c476,478
< \AmS-\LaTeX{} provides another way for finetuning
< the spacing between multiple integral signs,
< namely the \ci{iint}, \ci{iiint}, \ci{iiiint}, and \ci{idotsint} commands.
< With the \pai{amsmath} package loaded, the above example can be
< typeset this way:
---
> \index{long equations} \textbf{Long equations} will not be
> automatically divided into neat bits.  The author has to specify
> where to break them and correct the indent:
389,392c480,485
< \newcommand{\ud}{\mathrm{d}}
< \begin{displaymath}
< \iint_{D} \, \ud x \, \ud y
< \end{displaymath}
---
> \begin{align}
> f(x) &= 3x^5 + x^4 + 2x^3 
>                 \nonumber \\
>      &\qquad + 9x^2 + 12x + 23 \\
>      &= g(x) - h(x)
> \end{align}
393a487,489
> The \pai{amsmath} package provides a couple of other useful environments: \verb|flalign|,
> \verb|gather|, \verb|multline| and \verb|split|. See the
> documentation for the package for a wide range of commands, environments and more.
395,399c491
< See the electronic document testmath.tex (distributed with
< \AmS-\LaTeX) or Chapter 8 of ``The LaTeX Companion'' for further details.
< 
< \section{Vertically Aligned Material}
< \label{sec:vert}
---
> \subsection{Arrays and Matrices} \label{sec:arraymat}
403c495
< used to break the lines.
---
> used to break the lines:
405,412c497,504
< \begin{displaymath}
< \mathbf{X} =
< \left( \begin{array}{ccc}
< x_{11} & x_{12} & \ldots \\
< x_{21} & x_{22} & \ldots \\
< \vdots & \vdots & \ddots
< \end{array} \right)
< \end{displaymath}
---
> \begin{equation*}
>  \mathbf{X} = \left( 
>   \begin{array}{ccc}
>    x_1 & x_2 & \ldots \\
>    x_3 & x_4 & \ldots \\
>    \vdots & \vdots & \ddots
>   \end{array} \right)
> \end{equation*}
415,417c507,511
< The \ei{array} environment can also be used to typeset expressions which have one
< big delimiter by using a ``\verb|.|'' as an invisible \ci{right} 
< delimiter:
---
> 
> The \ei{array} environment can also be used to typeset \wi{piecewise function}s by
> using a ``\verb|.|'' as an invisible \ci{right} delimiter:\footnote{If you want
>   to typeset a lot of constructions like these, the \ei{cases} environment from
>   \textsf{amsmath} simplifies the syntax, so it is worth a look.}  
419,423c513,518
< \begin{displaymath}
< y = \left\{ \begin{array}{ll}
<  a & \textrm{if $d>c$}\\
<  b+x & \textrm{in the morning}\\
<  l & \textrm{all day long}
---
> \begin{equation*}
> |x| = \left\{
>  \begin{array}{rl}
>   -x & \text{if } x < 0\\
>    0 & \text{if } x = 0\\
>    x & \text{if } x > 0
425c520
< \end{displaymath}
---
> \end{equation*}
428,430c523,531
< As within the \verb|tabular| environment you can also
< draw lines in the \ei{array} environent, e.g. separating the entries of
< a matrix:
---
> 
> 
> \ei{array} can be used to typeset matrices\index{matrix} as well, but
> \pai{amsmath} provides a better solution using the different \ei{matrix}
> environments. There are six versions with different delimiters: \ei{matrix}
> (none), \ei{pmatrix} $($, \ei{bmatrix} $[$, \ei{Bmatrix} $\{$, \ei{vmatrix} $\vert$ and
> \ei{Vmatrix} $\Vert$. You don't have to specify the number of columns as with
> \ei{array}. The maximum number is 10, but it is customisable (though it is not
> very often you need 10 columns!):
432,433c533,534
< \begin{displaymath}
< \left(\begin{array}{c|c}
---
> \begin{equation*}
>  \begin{matrix} 
435,438c536,543
< \hline
< 3 & 4
< \end{array}\right)
< \end{displaymath}
---
>    3 & 4 
>  \end{matrix} \qquad
>  \begin{bmatrix} 
>    1 & 2 & 3 \\
>    4 & 5 & 6 \\ 
>    7 & 8 & 9
>  \end{bmatrix}
> \end{equation*}
443,446c548
< For formulae running over several lines or for \wi{equation system}s,
< you can use the environments \ei{eqnarray}, and \verb|eqnarray*|
< instead of \texttt{equation}. In \texttt{eqnarray} each line gets an
< equation number. The \verb|eqnarray*| does not number anything.
---
> \section{Spacing in Math Mode} \label{sec:math-spacing}
448,462c550,560
< The \texttt{eqnarray} and the \verb|eqnarray*| environments work like
< a 3-column table of the form \verb|{rcl}|, where the middle column can
< be used for the equal sign or the not-equal sign. Or any other sign
< you see fit. The \verb|\\| command breaks the lines.
< \begin{example}
< \begin{eqnarray}
< f(x) & = & \cos x     \\
< f'(x) & = & -\sin x   \\
< \int_{0}^{x} f(y)dy &
<  = & \sin x
< \end{eqnarray}
< \end{example}
< Notice that the space on either side of the 
< the equal signs is rather large. It can be reduced by setting
< \verb|\setlength\arraycolsep{2pt}|, as in the next example.
---
> \index{math spacing} If the spacing within formulae chosen by \LaTeX{}
> is not satisfactory, it can be adjusted by inserting special
> spacing commands: \ci{,} for
> $\frac{3}{18}\:\textrm{quad}$ (\demowidth{0.166em}), \ci{:} for $\frac{4}{18}\:
> \textrm{quad}$ (\demowidth{0.222em}) and \ci{;} for $\frac{5}{18}\:
> \textrm{quad}$ (\demowidth{0.277em}).  The escaped space character
> \verb*|\ | generates a medium sized space comparable to the interword spacing and \ci{quad}
> (\demowidth{1em}) and \ci{qquad} (\demowidth{2em}) produce large
> spaces. The size of a \ci{quad} corresponds to the width of the
> character `M' of the current font. \verb|\!|\cih{"!} produces a
> negative space of $-\frac{3}{18}\:\textrm{quad}$ ($-$\demowidth{0.166em}).
464,476c562
< \index{long equations} \textbf{Long equations} will not be
< automatically divided into neat bits.  The author has to specify
< where to break them and how much to indent. The following two methods
< are the most common ones used to achieve this.
< \begin{example}
< {\setlength\arraycolsep{2pt}
< \begin{eqnarray}
< \sin x & = & x -\frac{x^{3}}{3!}
<      +\frac{x^{5}}{5!}-{}
<                     \nonumber\\
<  & & {}-\frac{x^{7}}{7!}+{}\cdots
< \end{eqnarray}}
< \end{example}
---
> Note that `d' in the differential is conventionally set in roman:
478,484c564,567
< \begin{eqnarray}
< \lefteqn{ \cos x = 1
<      -\frac{x^{2}}{2!} +{} }
<                     \nonumber\\
<  & & {}+\frac{x^{4}}{4!}
<      -\frac{x^{6}}{6!}+{}\cdots
< \end{eqnarray}
---
> \begin{equation*}
>  \int_1^2 \ln x \mathrm{d}x \qquad
>  \int_1^2 \ln x \,\mathrm{d}x
> \end{equation*}
487,489d569
< %\enlargethispage{\baselineskip}
< \noindent The \ci{nonumber} command causes \LaTeX{} to not generate a number for
< this equation.
491,493c571,579
< It can be difficult to get vertically aligned equations to look right
< with these methods; the package \pai{amsmath} provides a more
< powerful set of alternatives. (see \verb|split| and \verb|align| environments).
---
> In the next example, we define a new command \ci{ud} which produces
> ``$\,\mathrm{d}$'' (notice the spacing \demowidth{0.166em} before the
> $\text{d}$), so we don't have to write it every time. The \ci{newcommand} is
> placed in the preamble. %  More on
> % \ci{newcommand} in section~\ref{} on page \pageref{}. To Do: Add label and
> % reference to "Customising LaTeX" -> "New Commands, Environments and Packages"
> % -> "New Commands".
> \begin{example}
> \newcommand{\ud}{\,\mathrm{d}}
495c581,584
< \section{Phantom}
---
> \begin{equation*}
>  \int_a^b f(x)\ud x 
> \end{equation*}
> \end{example}
497,499c586,590
< We can't see phantoms, but they still occupy some space in the minds of a
< lot of people. \LaTeX{} is no different. We can use this for
< some interesting spacing tricks.
---
> If you want to typeset multiple integrals, you'll discover that the spacing
> between the integrals is too wide. You can correct it using \ci{!}, but
> \pai{amsmath} provides an easier way for fine-tuning
> the spacing, namely the \ci{iint}, \ci{iiint}, \ci{iiiint}, and \ci{idotsint}
> commands.
501,504d591
< When vertically aligning text using \verb|^| and \verb|_| \LaTeX{} sometimes
< is just a little bit too helpful. Using the \ci{phantom} command you can
< reserve space for characters which do not show up in the final output. Best
< is to look at the following examples.
506,510c593,599
< \begin{displaymath}
< {}^{12}_{\phantom{1}6}\textrm{C}
< \qquad \textrm{versus} \qquad
< {}^{12}_{6}\textrm{C}
< \end{displaymath}
---
> \newcommand{\ud}{\,\mathrm{d}}
> 
> \[ \int\int f(x)g(y) 
>                   \ud x \ud y \]
> \[ \int\!\!\!\int 
>          f(x)g(y) \ud x \ud y \]
> \[ \iint f(x)g(y) \ud x \ud y \]
511a601,610
> 
> See the electronic document \texttt{testmath.tex} (distributed with
> \AmS-\LaTeX) or Chapter 8 of \companion{} for further details.
> 
> \subsection{Phantoms}
> 
> When vertically aligning text using \verb|^| and \verb|_| \LaTeX{} is sometimes
> just a little too helpful. Using the \ci{phantom} command you can
> reserve space for characters that do not show up in the final output.
> The easiest way to understand this is to look at an example:
513,517c612,616
< \begin{displaymath} 
< \Gamma_{ij}^{\phantom{ij}k}
< \qquad \textrm{versus} \qquad
< \Gamma_{ij}^{k}
< \end{displaymath}  
---
> \begin{equation*}
> {}^{14}_{6}\text{C}
> \qquad \text{versus} \qquad
> {}^{14}_{\phantom{1}6}\text{C}
> \end{equation*}
518a618,619
> If you want to typeset a lot of isotopes as in the example, the \pai{mhchem}
> package is very useful for typesetting isotopes and chemical formulae too.
520d620
< \section{Math Font Size}
522,531c622,624
< \index{math font size} In math mode, \TeX{} selects the font size
< according to the context. Superscripts, for example, get typeset in a
< smaller font. If you want to typeset part of an equation in roman,
< don't use the \verb|\textrm| command, because the font size switching
< mechanism will not work, as \verb|\textrm| temporarily escapes to text
< mode. Use \verb|\mathrm| instead to keep the size switching mechanism
< active. But pay attention, \ci{mathrm} will only work well on short
< items. Spaces are still not active and accented characters do not
< work.\footnote{The \AmS-\LaTeX{} package makes the \ci{textrm} command
<   work with size changing.}
---
> \section{Fiddling with the Math Fonts}\label{sec:fontsz}
> Different math fonts are listed on Table~\ref{mathalpha} on page
> \pageref{mathalpha}.
533,536c626,629
< \begin{equation}
< 2^{\textrm{nd}} \quad 
< 2^{\mathrm{nd}}
< \end{equation}
---
>  $\Re \qquad
>   \mathcal{R} \qquad
>   \mathfrak{R} \qquad
>   \mathbb{R} \qquad $  
537a631
> The last two require \pai{amssymb} or \pai{amsfonts}.
539,540c633,634
< Nevertheless, sometimes you need to tell \LaTeX{} the correct font
< size. In math mode, the fontsize is set with the four commands:
---
> Sometimes you need to tell \LaTeX{} the correct font
> size. In math mode, this is set with the following four commands:
548c642,643
< Changing styles also affects the way limits are displayed.
---
> If $\sum$ is placed in a fraction, it'll be typeset in text style unless you tell
> \LaTeX{} otherwise:
550,559c645,653
< \begin{displaymath}
< \mathop{\mathrm{corr}}(X,Y)= 
<  \frac{\displaystyle 
<    \sum_{i=1}^n(x_i-\overline x)
<    (y_i-\overline y)} 
<   {\displaystyle\biggl[
<  \sum_{i=1}^n(x_i-\overline x)^2
< \sum_{i=1}^n(y_i-\overline y)^2
< \biggr]^{1/2}}
< \end{displaymath}    
---
> \begin{equation*}
>  P = \frac{\displaystyle{ 
>    \sum_{i=1}^n (x_i- x)
>    (y_i- y)}} 
>    {\displaystyle{\left[
>    \sum_{i=1}^n(x_i-x)^2
>    \sum_{i=1}^n(y_i- y)^2
>    \right]^{1/2}}}
> \end{equation*}    
560a655,656
> Changing styles generally affects the way big operators and limits are displayed.
> 
564,565d659
< \noindent This is one of those examples in which we need larger
< brackets than the standard \verb|\left[  \right]| provides.
566a661,662
> \subsection{Bold Symbols}
> \index{bold symbols}
568c664,687
< \section{Theorems, Laws, \ldots}
---
> It is quite difficult to get bold symbols in \LaTeX{}; this is
> probably intentional as amateur typesetters tend to overuse them.  The
> font change command \verb|\mathbf| gives bold letters, but these are
> roman (upright) whereas mathematical symbols are normally italic, and
> furthermore it doesn't work on lower case Greek letters.
> There is a \ci{boldmath} command, but \emph{this can only be used
> outside math mode}. It works for symbols too, though:
> \begin{example}
> $\mu, M \qquad 
> \mathbf{\mu}, \mathbf{M}$
> \qquad \boldmath{$\mu, M$}
> \end{example}
> 
> The package \pai{amsbsy} (included by \pai{amsmath}) as well as the
> \pai{bm} from the \texttt{tools} bundle make this much easier as they include
> a \ci{boldsymbol} command:
> 
> \begin{example}
> $\mu, M \qquad
> \boldsymbol{\mu}, \boldsymbol{M}$
> \end{example}
> 
> 
> \section{Theorems, Lemmas, \ldots}
572c691
< structures. \LaTeX{} supports this with the command
---
> structures.
577,579c696,698
< The \emph{name} argument, is a short keyword used to identify the
< ``theorem''. With the \emph{text} argument, you define the actual name
< of the ``theorem'' which will be printed in the final document.
---
> The \emph{name} argument is a short keyword used to identify the
> ``theorem''. With the \emph{text} argument you define the actual name
> of the ``theorem'', which will be printed in the final document.
582,583c701,702
< specify the numbering used on the ``theorem''. With the \emph{counter}
< argument you can specify the \emph{name} of a previously declared
---
> specify the numbering used on the ``theorem''. Use  the \emph{counter}
> argument to specify the \emph{name} of a previously declared
586c705
< sectional unit within which you want your ``theorem'' to be numbered.
---
> sectional unit within which the ``theorem'' should get its numbers.
596,597c715,723
< This should be enough theory. The following examples will hopefully
< remove the final remains of doubt and make it clear that the
---
> The \pai{amsthm} package (part of \AmS-\LaTeX) provides the 
> \ci{theoremstyle}\verb|{|\emph{style}\verb|}|
> command which lets you define what the theorem is all about by picking
> from three predefined styles: \texttt{definition} (fat title, roman body),
> \texttt{plain} (fat title, italic body) or \texttt{remark} (italic
> title, roman body).
> 
> This should be enough theory. The following examples should
> remove any remaining doubt, and make it clear that the
598a725,738
> 
> % actually define things
> \theoremstyle{definition} \newtheorem{law}{Law}
> \theoremstyle{plain}      \newtheorem{jury}[law]{Jury}
> \theoremstyle{remark}     \newtheorem*{marg}{Margaret}
> 
> First define the theorems:
> 
> \begin{verbatim}
> \theoremstyle{definition} \newtheorem{law}{Law}
> \theoremstyle{plain}      \newtheorem{jury}[law]{Jury}
> \theoremstyle{remark}     \newtheorem*{marg}{Margaret}
> \end{verbatim}
> 
600,604d739
< % definitions for the document
< % preamble
< \newtheorem{law}{Law}
< \newtheorem{jury}[law]{Jury}
< %in the document
610,611c745,746
< see law~\ref{law:box}\end{jury}
< \begin{law}No, No, No\end{law}
---
> see law~\ref{law:box}.\end{jury}
> \begin{marg}No, No, No\end{marg}
615c750
< theorem. Therefore it gets a number which is in sequence with
---
> theorem, so it gets a number that is in sequence with
619d753
< \flushleft
621,626c755,760
< \begin{mur}
< If there are two or more 
< ways to do something, and 
< one of those ways can result 
< in a catastrophe, then 
< someone will do it.\end{mur}
---
> 
> \begin{mur} If there are two or 
> more ways to do something, and 
> one of those ways can result in
> a catastrophe, then someone 
> will do it.\end{mur}
629c763
< The ``Murphy'' theorem gets a number which is linked to the number of
---
> The ``Murphy'' theorem gets a number that is linked to the number of
633,634c767
< \section{Bold symbols}
< \index{bold symbols}
---
> The \pai{amsthm} package also provides the \ei{proof} environment.
636,641d768
< It is quite difficult to get bold symbols in \LaTeX{}; this is
< probably intentional as amateur typesetters tend to overuse them.  The
< font change command \verb|\mathbf| gives bold letters, but these are
< roman (upright) whereas mathematical symbols are normally italic.
< There is a \ci{boldmath} command, but \emph{this can only be used
< outside mathematics mode}. It works for symbols too.
643,646c770,773
< \begin{displaymath}
< \mu, M \qquad \mathbf{M} \qquad
< \mbox{\boldmath $\mu, M$}
< \end{displaymath}
---
> \begin{proof}
>  Trivial, use
> \[E=mc^2\]
> \end{proof}
649,650c776,777
< \noindent
< Notice that the comma is bold too, which may not be what is required.
---
> With the command \ci{qedhere} you can move the `end of proof' symbol
> around for situations where it would end up alone on a line.
652,655d778
< The package \pai{amsbsy} (included by \pai{amsmath}) as well as the
< \pai{bm} from the tools bundle make this much easier as they include
< a \ci{boldsymbol} command.
< \ifx\boldsymbol\undefined\else
657,660c780,783
< \begin{displaymath}
< \mu, M \qquad
< \boldsymbol{\mu}, \boldsymbol{M}
< \end{displaymath}
---
> \begin{proof}
>  Trivial, use
> \[E=mc^2 \qedhere\]
> \end{proof}
662d784
< \fi
663a786,787
> If you want to customize your theorems down to the last dot, the
> \pai{ntheorem} package offers a plethora of options.
665,668d788
< %%% Local Variables: 
< %%% mode: latex
< %%% TeX-master: "lshort"
< %%% End: 
669a790
> %
670a792,796
> % Local Variables:
> % TeX-master: "lshort"
> % mode: latex
> % mode: flyspell
> % End:
