%%%%%%%%%%%%%%%%%%%%%%%%%%%%%%%%%%%%%%%%%%%%%%%%%%%%%%%%%%%%%%%%
% Contents: Math typesetting with LaTeX
% $Id: math.tex,v 1.4 1998/09/29 08:05:09 oetiker Exp oetiker $
%%%%%%%%%%%%%%%%%%%%%%%%%%%%%%%%%%%%%%%%%%%%%%%%%%%%%%%%%%%%%%%%%
 
\chapter{Typesetting Mathematical Formulae}

\begin{intro}
  Now you are ready! In this chapter, we will attack the main strength
  of \TeX{}: mathematical typesetting. But be warned, this chapter
  only scratches the surface. While the things explained here are
  sufficient for many people, don't despair if you can't find a
  solution to your mathematical typesetting needs here. It is highly likely
  that your problem is addressed in \AmS-\LaTeX{}%
  \footnote{\texttt{CTAN:/tex-archive/macros/latex/required/amslatex}}
  or some other package.
\end{intro}
  
\section{General}

\LaTeX{} has a special mode for typesetting \wi{mathematics}.
Mathematical text within a paragraph is entered between \ci{(}
and \ci{)}, \index{$@\texttt{\$}} %$
between \texttt{\$} and \texttt{\$} or between %}
\verb|\begin{|\ei{math}\verb|}| and \verb|\end{math}|.\index{formulae}
\begin{example}
Add $a$ squared and $b$ squared 
to get $c$ squared. Or, using 
a more mathematical approach:
$c^{2}=a^{2}+b^{2}$
\end{example}
\begin{example}
\TeX{} is pronounced as 
$\tau\epsilon\chi$.\\[6pt]
100~m$^{3}$ of water\\[6pt]
This comes from my $\heartsuit$
\end{example}

It is preferable to \emph{display} larger mathematical equations or formulae,
rather than to typeset them on separate lines. This means you enclose them
in \ci{[} and \ci{]} or between
\verb|\begin{|\ei{displaymath}\verb|}| and
  \verb|\end{displaymath}|.  This produces formulae which are not
numbered. If you want \LaTeX{} to number them, you can use the
\ei{equation} environment.
\begin{example}
Add $a$ squared and $b$ squared 
to get $c$ squared. Or, using 
a more mathematical approach:
\begin{displaymath}
c^{2}=a^{2}+b^{2}
\end{displaymath}
And just one more line.
\end{example}

You can reference an equation with \ci{label} and \ci{ref}
\begin{example}
\begin{equation} \label{eq:eps}
\epsilon > 0
\end{equation}
From (\ref{eq:eps}), we gather 
\ldots
\end{example}

Note that expressions will be typeset in a different style if displayed:
\begin{example}
$\lim_{n \to \infty} 
\sum_{k=1}^n \frac{1}{k^2} 
= \frac{\pi^2}{6}$
\end{example}
\begin{example}
\begin{displaymath}
\lim_{n \to \infty} 
\sum_{k=1}^n \frac{1}{k^2} 
= \frac{\pi^2}{6}
\end{displaymath}
\end{example}



There are differences between \emph{math mode} and \emph{text mode}. For
example in \emph{math mode}: 

\begin{enumerate}

\item Most spaces and linebreaks do not have any significance, as all spaces
either are derived logically from the mathematical expressions or
have to be specified using special commands such as \ci{,}, \ci{quad} or
\ci{qquad}.
 
\item Empty lines are not allowed. Only one paragraph per formula.

\item Each letter is considered to be the name of a variable and will be
typeset as such. If you want to typeset normal text within a formula
(normal upright font and normal spacing) then you have to enter the
text using the \verb|\textrm{...}| commands.
\end{enumerate}
\begin{example}
\begin{equation}
\forall x \in \mathbf{R}:
\qquad x^{2} \geq 0
\end{equation}
\end{example}
\begin{example}
\begin{equation}
x^{2} \geq 0\qquad
\textrm{for all }x\in\mathbf{R}
\end{equation}
\end{example}
 

%
% Add AMSSYB Package ... Blackboard bold .... R for realnumbers
%
Mathematicians can be very fussy about which symbols are used:
it would be conventional here to use `\wi{blackboard bold}',
\index{bold symbols} which is obtained using \ci{mathbb} from the
package \pai{amsfonts} or \pai{amssymb}.
\ifx\mathbb\undefined\else
The last example becomes
\begin{example}
\begin{displaymath}
x^{2} \geq 0\qquad
\textrm{for all }x\in\mathbb{R}
\end{displaymath}
\end{example}
\fi

\section{Grouping in Math Mode}

Most math mode commands act only on the next character. So if you
want a command to affect several characters, you have to group them
together using curly braces: \verb|{...}|.
\begin{example}
\begin{equation}
a^x+y \neq a^{x+y}
\end{equation}
\end{example}
 
\section{Building Blocks of a Mathematical Formula}

In this section, the most important commands used in mathematical
typesetting will be described. Take a look at section~\ref{symbols} on
page~\pageref{symbols} for a detailed list of commands for typesetting
mathematical symbols.

\textbf{Lowercase \wi{Greek letters}} are entered as \verb|\alpha|,
 \verb|\beta|, \verb|\gamma|, \ldots, uppercase letters
are entered as \verb|\Gamma|, \verb|\Delta|, \ldots\footnote{There is no
  uppercase Alpha defined in \LaTeXe{} because it looks the same as a
  normal roman A. Once the new math coding is done, things will
  change.} 
\begin{example}
$\lambda,\xi,\pi,\mu,\Phi,\Omega$
\end{example}
\enlargethispage{\baselineskip}
\pagebreak[4]

\textbf{Exponents and Subscripts} can be specified using\index{exponent}\index{subscript}
the \verb|^|\index{^@\verb"|^"|} and the \verb|_|\index{_@\verb"|_"|} character.
\begin{example}
$a_{1}$ \qquad $x^{2}$ \qquad
$e^{-\alpha t}$ \qquad
$a^{3}_{ij}$\\
$e^{x^2} \neq {e^x}^2$
\end{example}

The \textbf{\wi{square root}} is entered as \ci{sqrt}, the
$n^\mathrm{th}$ root is generated with \verb|\sqrt[|$n$\verb|]|. The size of
the root sign is determined automatically by \LaTeX. If just the sign
is needed, use \verb|\surd|.
\begin{example}
$\sqrt{x}$ \qquad 
$\sqrt{ x^{2}+\sqrt{y} }$ 
\qquad $\sqrt[3]{2}$\\[3pt]
$\surd[x^2 + y^2]$
\end{example}

The commands \ci{overline} and \ci{underline} create
\textbf{horizontal lines} directly over or under an expression.
\index{horizontal!line}
\begin{example}
$\overline{m+n}$
\end{example}

The commands \ci{overbrace} and \ci{underbrace} create
long \textbf{horizontal braces} over or under an expression.
\index{horizontal!brace}
\begin{example}
$\underbrace{ a+b+\cdots+z }_{26}$
\end{example}

\index{mathematical!accents} To add mathematical accents such as small
arrows or \wi{tilde} signs to variables, you can use the commands
given in Table~\ref{mathacc} on page \pageref{mathacc}.  Wide hats and
tildes covering several characters are generated with \ci{widetilde}
and \ci{widehat}.  The \verb|'|\index{'@\verb"|'"|} symbol gives a
\wi{prime}.
% a dash is --
\begin{example}
\begin{displaymath}
y=x^{2}\qquad y'=2x\qquad y''=2
\end{displaymath}
\end{example}

\textbf{Vectors}\index{vectors} often are specified by adding small
\wi{arrow symbols} on top of a variable. This is done with the
\ci{vec} command. The two commands \ci{overrightarrow} and
\ci{overleftarrow} are useful to denote the vector from $A$ to $B$.
\begin{example}
\begin{displaymath}
\vec a\quad\overrightarrow{AB}
\end{displaymath}
\end{example}

Usually you don't typeset an explicit dot sign to indicate
the multiplication operation. However sometimes it is written
to help the reader's eyes in grouping a formula.
Then you should use \ci{cdot}
\begin{example}
\begin{displaymath}
v = {\sigma}_1 \cdot {\sigma}_2
    {\tau}_1 \cdot {\tau}_2
\end{displaymath}
\end{example}


Names of log-like functions are often typeset in an upright
font and not in italic like variables. Therefore \LaTeX{} supplies the
following commands to typeset the most important function names:
\index{mathematical!functions}

\begin{tabular}{lllllll}
\ci{arccos} &  \ci{cos}  &  \ci{csc} &  \ci{exp} &  \ci{ker}    & \ci{limsup} & \ci{min} \\
\ci{arcsin} &  \ci{cosh} &  \ci{deg} &  \ci{gcd} &  \ci{lg}     & \ci{ln}     & \ci{Pr}  \\
\ci{arctan} &  \ci{cot}  &  \ci{det} &  \ci{hom} &  \ci{lim}    & \ci{log}    & \ci{sec} \\
\ci{arg}    &  \ci{coth} &  \ci{dim} &  \ci{inf} &  \ci{liminf} & \ci{max}    & \ci{sin} \\
\ci{sinh} & \ci{sup} & \ci{tan} & \ci{tanh}\\
\end{tabular}

\begin{example}
\[\lim_{x \rightarrow 0}
\frac{\sin x}{x}=1\]
\end{example}

For the \wi{modulo function}, there are two commands: \ci{bmod} for the
binary operator ``$a \bmod b$'' and \ci{pmod}
for expressions
such as ``$x\equiv a \pmod{b}$.''

A built-up \textbf{\wi{fraction}} is typeset with the
\ci{frac}\verb|{...}{...}| command.
Often the slashed form $1/2$ is preferable, because it looks better
for small amounts of `fraction material.'
\begin{example}
$1\frac{1}{2}$~hours
\begin{displaymath}
\frac{ x^{2} }{ k+1 }\qquad
x^{ \frac{2}{k+1} }\qquad
x^{ 1/2 }
\end{displaymath}
\end{example}

To typeset binomial coefficients or similar structures, you can use
either the command \verb|{... |\ci{choose}\verb| ...}| or 
\verb|{... |\ci{atop}\verb| ...}|. The second command produces the
same output as the first one, but without braces.
\footnote{Note that the usage of these old-style commands is expressly forbidden
by the \pai{amsmath} package. They are replaced by
\ci{binom} and \ci{genfrac}. The latter is a superset of all related
construct, e.g. you may get a similar construct to \ci{atop}
by \texttt{$\backslash$newcommand\{$\backslash$newatop\}[2]\{$\backslash$genfrac\{\}\{\}\{0pt\}\{1\}\{\#1\}\{\#2\}\}}.}

\begin{example}
\begin{displaymath}
{n \choose k}\qquad {x \atop y+2}
\end{displaymath}
\end{example}

For binary relations it may be useful to stack symbols over each other.
\ci{stackrel} puts the symbol given
in the first argument in superscript-like size over the second which
is set in its usual position.
\begin{example}
\begin{displaymath}
\int f_N(x) \stackrel{!}{=} 1
\end{displaymath}
\end{example}

The \textbf{\wi{integral operator}} is generated with \ci{int}, the
\textbf{\wi{sum operator}} with \ci{sum} and the \textbf{\wi{product operator}}
with \ci{prod}. The upper and lower limits are specified with~\verb|^|
and~\verb|_| like subscripts and superscripts.
\footnote{\AmS-\LaTeX{} in addition has multiline super-/subscripts}
\begin{example}
\begin{displaymath}
\sum_{i=1}^{n} \qquad
\int_{0}^{\frac{\pi}{2}} \qquad
\prod_\epsilon
\end{displaymath}
\end{example}

For \textbf{\wi{braces}} and other \wi{delimiters}, there exist all
types of symbols in \TeX{} (e.g.~$[\;\langle\;\|\;\updownarrow$).
Round and square braces can be entered with the corresponding keys,
curly braces with \verb|\{|, all other delimiters are generated with
special commands (e.g.~\verb|\updownarrow|). For a list of all
delimiters available, check table~\ref{tab:delimiters} on page
\pageref{tab:delimiters}.
\begin{example}
\begin{displaymath}
{a,b,c}\neq\{a,b,c\}
\end{displaymath}
\end{example}

If you put the command \ci{left} in front of an opening delimiter or
\ci{right} in front of a closing delimiter, \TeX{} will automatically
determine the correct size of the delimiter. Note that you must close
every \ci{left} with a corresponding \ci{right}, and that the size is
determined correctly only if both are typeset on the same line. If you
don't want anything on the right, use the invisible `\ci{right.}'!
\begin{example}
\begin{displaymath}
1 + \left( \frac{1}{ 1-x^{2} }
    \right) ^3
\end{displaymath}
\end{example}

\pagebreak[4]
In some cases it is necessary to specify the correct size of a
mathematical delimiter\index{mathematical!delimiter} by hand,
which can be done using the commands \ci{big}, \ci{Big}, \ci{bigg} and
\ci{Bigg} as prefixes to most delimiter commands.\footnote{These
  commands do not work as expected if a size changing command has been
  used, or the \texttt{11pt} or \texttt{12pt} option has been
  specified.  Use the \pai{exscale} or \pai{amsmath} packages to
  correct this behaviour.}
\begin{example}
$\Big( (x+1) (x-1) \Big) ^{2}$\\
$\big(\Big(\bigg(\Bigg($\quad
$\big\}\Big\}\bigg\}\Bigg\}$\quad
$\big\|\Big\|\bigg\|\Bigg\|$
\end{example}

To enter \textbf{\wi{three dots}} into a formula, you can use several
commands. \ci{ldots} typesets the dots on the baseline, \ci{cdots}
sets them centred. Besides that, there are the commands \ci{vdots} for
vertical and \ci{ddots} for \wi{diagonal dots}.\index{vertical
  dots}\index{horizontal!dots} You can find another example in section~\ref{sec:vert}.
\begin{example}
\begin{displaymath}
x_{1},\ldots,x_{n} \qquad
x_{1}+\cdots+x_{n}
\end{displaymath}
\end{example}
 
\section{Math Spacing}

\index{math spacing} If the spaces within formulae chosen by \TeX{}
are not satisfactory, they can be adjusted by inserting special
spacing commands. There are some commands for small spaces: \ci{,} for
$\frac{3}{18}\:\textrm{quad}$ (\demowidth{0.166em}), \ci{:} for $\frac{4}{18}\:
\textrm{quad}$ (\demowidth{0.222em}) and \ci{;} for $\frac{5}{18}\:
\textrm{quad}$ (\demowidth{0.277em}).  The escaped space character
\verb*.\ . generates a medium sized space and \ci{quad}
(\demowidth{1em}) and \ci{qquad} (\demowidth{2em}) produce large
spaces. The size of a \ci{quad} corresponds to the width of the
character `M' of the current font.  The \verb|\!|\cih{"!} command produces a
negative space of $-\frac{3}{18}\:\textrm{quad}$ (\demowidth{0.166em}).
\begin{example}
\newcommand{\ud}{\mathrm{d}}
\begin{displaymath}
\int\!\!\!\int_{D} g(x,y)
  \, \ud x\, \ud y 
\end{displaymath}
instead of 
\begin{displaymath}
\int\int_{D} g(x,y)\ud x \ud y
\end{displaymath}
\end{example}
Note that `d' in the differential is conventionally set in roman.

\AmS-\LaTeX{} provides another way for finetuning
the spacing between multiple integral signs,
namely the \ci{iint}, \ci{iiint}, \ci{iiiint}, and \ci{idotsint} commands.
With the \pai{amsmath} package loaded, the above example can be
typeset this way:
\begin{example}
\newcommand{\ud}{\mathrm{d}}
\begin{displaymath}
\iint_{D} \, \ud x \, \ud y
\end{displaymath}
\end{example}

See the electronic document testmath.tex (distributed with
\AmS-\LaTeX) or Chapter 8 of ``The LaTeX Companion'' for further details.

\section{Vertically Aligned Material}
\label{sec:vert}

To typeset \textbf{arrays}, use the \ei{array} environment. It works
somewhat similar to the \texttt{tabular} environment. The \verb|\\| command is
used to break the lines.
\begin{example}
\begin{displaymath}
\mathbf{X} =
\left( \begin{array}{ccc}
x_{11} & x_{12} & \ldots \\
x_{21} & x_{22} & \ldots \\
\vdots & \vdots & \ddots
\end{array} \right)
\end{displaymath}
\end{example}

The \ei{array} environment can also be used to typeset expressions which have one
big delimiter by using a ``\verb|.|'' as an invisible \ci{right} 
delimiter:
\begin{example}
\begin{displaymath}
y = \left\{ \begin{array}{ll}
 a & \textrm{if $d>c$}\\
 b+x & \textrm{in the morning}\\
 l & \textrm{all day long}
  \end{array} \right.
\end{displaymath}
\end{example}

As within the \verb|tabular| environment you can also
draw lines in the \ei{array} environent, e.g. separating the entries of
a matrix:
\begin{example}
\begin{displaymath}
\left(\begin{array}{c|c}
 1 & 2 \\
\hline
3 & 4
\end{array}\right)
\end{displaymath}
\end{example}



For formulae running over several lines or for \wi{equation system}s,
you can use the environments \ei{eqnarray}, and \verb|eqnarray*|
instead of \texttt{equation}. In \texttt{eqnarray} each line gets an
equation number. The \verb|eqnarray*| does not number anything.

The \texttt{eqnarray} and the \verb|eqnarray*| environments work like
a 3-column table of the form \verb|{rcl}|, where the middle column can
be used for the equal sign or the not-equal sign. Or any other sign
you see fit. The \verb|\\| command breaks the lines.
\begin{example}
\begin{eqnarray}
f(x) & = & \cos x     \\
f'(x) & = & -\sin x   \\
\int_{0}^{x} f(y)dy &
 = & \sin x
\end{eqnarray}
\end{example}
Notice that the space on either side of the 
the equal signs is rather large. It can be reduced by setting
\verb|\setlength\arraycolsep{2pt}|, as in the next example.

\index{long equations} \textbf{Long equations} will not be
automatically divided into neat bits.  The author has to specify
where to break them and how much to indent. The following two methods
are the most common ones used to achieve this.
\begin{example}
{\setlength\arraycolsep{2pt}
\begin{eqnarray}
\sin x & = & x -\frac{x^{3}}{3!}
     +\frac{x^{5}}{5!}-{}
                    \nonumber\\
 & & {}-\frac{x^{7}}{7!}+{}\cdots
\end{eqnarray}}
\end{example}
\begin{example}
\begin{eqnarray}
\lefteqn{ \cos x = 1
     -\frac{x^{2}}{2!} +{} }
                    \nonumber\\
 & & {}+\frac{x^{4}}{4!}
     -\frac{x^{6}}{6!}+{}\cdots
\end{eqnarray}
\end{example}

%\enlargethispage{\baselineskip}
\noindent The \ci{nonumber} command causes \LaTeX{} to not generate a number for
this equation.

It can be difficult to get vertically aligned equations to look right
with these methods; the package \pai{amsmath} provides a more
powerful set of alternatives. (see \verb|split| and \verb|align| environments).

\section{Phantom}

We can't see phantoms, but they still occupy some space in the minds of a
lot of people. \LaTeX{} is no different. We can use this for
some interesting spacing tricks.

When vertically aligning text using \verb|^| and \verb|_| \LaTeX{} sometimes
is just a little bit too helpful. Using the \ci{phantom} command you can
reserve space for characters which do not show up in the final output. Best
is to look at the following examples.
\begin{example}
\begin{displaymath}
{}^{12}_{\phantom{1}6}\textrm{C}
\qquad \textrm{versus} \qquad
{}^{12}_{6}\textrm{C}
\end{displaymath}
\end{example}
\begin{example}
\begin{displaymath} 
\Gamma_{ij}^{\phantom{ij}k}
\qquad \textrm{versus} \qquad
\Gamma_{ij}^{k}
\end{displaymath}  
\end{example}

\section{Math Font Size}

\index{math font size} In math mode, \TeX{} selects the font size
according to the context. Superscripts, for example, get typeset in a
smaller font. If you want to typeset part of an equation in roman,
don't use the \verb|\textrm| command, because the font size switching
mechanism will not work, as \verb|\textrm| temporarily escapes to text
mode. Use \verb|\mathrm| instead to keep the size switching mechanism
active. But pay attention, \ci{mathrm} will only work well on short
items. Spaces are still not active and accented characters do not
work.\footnote{The \AmS-\LaTeX{} package makes the \ci{textrm} command
  work with size changing.}
\begin{example}
\begin{equation}
2^{\textrm{nd}} \quad 
2^{\mathrm{nd}}
\end{equation}
\end{example}

Nevertheless, sometimes you need to tell \LaTeX{} the correct font
size. In math mode, the fontsize is set with the four commands:
\begin{flushleft}
\ci{displaystyle}~($\displaystyle 123$),
 \ci{textstyle}~($\textstyle 123$), 
\ci{scriptstyle}~($\scriptstyle 123$) and
\ci{scriptscriptstyle}~($\scriptscriptstyle 123$).
\end{flushleft}

Changing styles also affects the way limits are displayed.
\begin{example}
\begin{displaymath}
\mathop{\mathrm{corr}}(X,Y)= 
 \frac{\displaystyle 
   \sum_{i=1}^n(x_i-\overline x)
   (y_i-\overline y)} 
  {\displaystyle\biggl[
 \sum_{i=1}^n(x_i-\overline x)^2
\sum_{i=1}^n(y_i-\overline y)^2
\biggr]^{1/2}}
\end{displaymath}    
\end{example}
% This is not a math accent, and no maths book would be set this way.
% mathop gets the spacing right.

\noindent This is one of those examples in which we need larger
brackets than the standard \verb|\left[  \right]| provides.


\section{Theorems, Laws, \ldots}

When writing mathematical documents, you probably need a way to
typeset ``Lemmas'', ``Definitions'', ``Axioms'' and similar
structures. \LaTeX{} supports this with the command
\begin{lscommand}
\ci{newtheorem}\verb|{|\emph{name}\verb|}[|\emph{counter}\verb|]{|%
         \emph{text}\verb|}[|\emph{section}\verb|]|
\end{lscommand}
The \emph{name} argument, is a short keyword used to identify the
``theorem''. With the \emph{text} argument, you define the actual name
of the ``theorem'' which will be printed in the final document.

The arguments in square brackets are optional. They are both used to
specify the numbering used on the ``theorem''. With the \emph{counter}
argument you can specify the \emph{name} of a previously declared
``theorem''. The new ``theorem'' will then be numbered in the same
sequence.  The \emph{section} argument allows you to specify the
sectional unit within which you want your ``theorem'' to be numbered.

After executing the \ci{newtheorem} command in the preamble of your
document, you can use the following command within the document.
\begin{code}
\verb|\begin{|\emph{name}\verb|}[|\emph{text}\verb|]|\\
This is my interesting theorem\\
\verb|\end{|\emph{name}\verb|}|     
\end{code}

This should be enough theory. The following examples will hopefully
remove the final remains of doubt and make it clear that the
\verb|\newtheorem| environment is way too complex to understand.
\begin{example}
% definitions for the document
% preamble
\newtheorem{law}{Law}
\newtheorem{jury}[law]{Jury}
%in the document
\begin{law} \label{law:box}
Don't hide in the witness box
\end{law}
\begin{jury}[The Twelve]
It could be you! So beware and
see law~\ref{law:box}\end{jury}
\begin{law}No, No, No\end{law}
\end{example}

The ``Jury'' theorem uses the same counter as the ``Law''
theorem. Therefore it gets a number which is in sequence with
the other ``Laws''. The argument in square brackets is used to specify 
a title or something similar for the theorem.
\begin{example}
\flushleft
\newtheorem{mur}{Murphy}[section]
\begin{mur}
If there are two or more 
ways to do something, and 
one of those ways can result 
in a catastrophe, then 
someone will do it.\end{mur}
\end{example}

The ``Murphy'' theorem gets a number which is linked to the number of
the current section. You could also use another unit, for example chapter or
subsection.

\section{Bold symbols}
\index{bold symbols}

It is quite difficult to get bold symbols in \LaTeX{}; this is
probably intentional as amateur typesetters tend to overuse them.  The
font change command \verb|\mathbf| gives bold letters, but these are
roman (upright) whereas mathematical symbols are normally italic.
There is a \ci{boldmath} command, but \emph{this can only be used
outside mathematics mode}. It works for symbols too.
\begin{example}
\begin{displaymath}
\mu, M \qquad \mathbf{M} \qquad
\mbox{\boldmath $\mu, M$}
\end{displaymath}
\end{example}

\noindent
Notice that the comma is bold too, which may not be what is required.

The package \pai{amsbsy} (included by \pai{amsmath}) as well as the
\pai{bm} from the tools bundle make this much easier as they include
a \ci{boldsymbol} command.
\ifx\boldsymbol\undefined\else
\begin{example}
\begin{displaymath}
\mu, M \qquad
\boldsymbol{\mu}, \boldsymbol{M}
\end{displaymath}
\end{example}
\fi


%%% Local Variables: 
%%% mode: latex
%%% TeX-master: "lshort"
%%% End: 


