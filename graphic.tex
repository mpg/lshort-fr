%%%%%%%%%%%%%%%%%%%%%%%%%%%%%%%%%%%%%%%%%%%%%%%%%%%%%%%%%%%%%%%%%
%%%%%%%%%%%%%%%%%%%%%%%%%%%%%%%%%%%%%%%%%%%%%%%%%%%%%%%%%%%%%%%%%
\setcounter{chapter}{4}
\newcommand{\graphicscompanion}{\emph{The \LaTeX{} Graphics Companion}~\cite{graphicscompanion}} 
\newcommand{\hobby}{\emph{A User's Manual for \MP{}}~\cite{metapost}}
\newcommand{\hoenig}{\emph{\TeX{} Unbound}~\cite{unbound}}
\newcommand{\graphicsinlatex}{\emph{Graphics in \LaTeXe{}}~\cite{ursoswald}}

\selectlanguage{english}
\chapter{Producing Mathematical Graphics}
\label{chap:graphics}

\begin{intro}
Most people use \LaTeX\ for typesetting their text. But as the non content and
structure oriented approach to authoring is so convenient, \LaTeX\ also offers a,
if somewhat restricted, possibility for producing graphical output from textual 
descriptions. Furthermore, quite a number of \LaTeX\ extensions have been created 
in order to overcome these restrictions. In this section, you will learn about a 
few of them.
\end{intro}

\section{Overview}

The \ei{picture} environment allows programming pictures directly in
\LaTeX. A detailed
description can be found in the \manual. On the one hand, there are rather
severe constraints, as the slopes of line segments as well as the radii of
circles are restricted to a narrow choice of values.  On the other hand, the
\ei{picture} environment of \LaTeXe\ brings with it the \ci{qbezier}
command, ``\texttt{q}'' meaning ``quadratic''.  Many frequently used curves
such as circles, ellipses, or catenaries can be satisfactorily approximated
by quadratic B\'ezier curves, although this may require some mathematical
toil. If, in addition, a programming language like Java is used to generate
\ci{qbezier} blocks of \LaTeX\ input files, the \ei{picture} environment
becomes quite powerful.

Although programming pictures directly in \LaTeX\ is severely restricted,
and often rather tiresome, there are still reasons for doing so. The documents
thus produced are ``small'' with respect to bytes, and there are no additional
graphics files to be dragged along.

Packages like \pai{epic} and \pai{eepic} (described, for instance, in \companion), or
\pai{pstricks} help to eliminate the restrictions hampering the original \ei{picture} 
environment, and greatly strengthen the graphical power of \LaTeX.

While the former two packages just enhance the \ei{picture} environment, the \pai{pstricks}
package has its own drawing environment, \ei{pspicture}. The power of \pai{pstricks} stems
from the fact that this package makes extensive use of \PSi{} possibilities.
In addition, numerous packages have been written for specific purposes. One of them is
\texorpdfstring{\Xy}{Xy}-pic, described at the end of this chapter. A wide variety of these
packages is described in detail in \graphicscompanion{} (not to be confused with \companion).

Perhaps the most powerful graphical tool related with \LaTeX\ is \MP, the twin of
Donald E. Knuth's \MF. \MP{} has the very powerful and 
mathematically sophisticated programming language of \MF. Contrary to \MF,
which generates bitmaps, \MP{} generates encapsulated \PSi{} files, 
which can be imported in \LaTeX. For an introduction, see \hobby, or the tutorial on \cite{ursoswald}.

A very thorough discussion of \LaTeX{} and \TeX{} strategies for graphics (and fonts) can 
be found in \hoenig.

\section{The \texttt{picture} Environment}
\secby{Urs Oswald}{osurs@bluewin.ch}

\subsection{Basic Commands}

A \ei{picture} environment\footnote{Believe it or not, the picture environment works out of the
box, with standard \LaTeXe{} no package loading necessary.} is created with one of the two commands
\begin{lscommand}
\ci{begin}\verb|{picture}(|$x,y$\verb|)|\ldots\ci{end}\verb|{picture}|
\end{lscommand}
\noindent or
\begin{lscommand}
\ci{begin}\verb|{picture}(|$x,y$\verb|)(|$x_0,y_0$\verb|)|\ldots\ci{end}\verb|{picture}|
\end{lscommand}
The numbers $x,\,y,\,x_0,\,y_0$ refer to \ci{unitlength}, which can be reset any time
(but not within a \ei{picture} environment) with a command such as
\begin{lscommand}
\ci{setlength}\verb|{|\ci{unitlength}\verb|}{1.2cm}|
\end{lscommand}
The default value of \ci{unitlength} is \texttt{1pt}. The first pair, $(x,y)$, effects
the reservation, within the document, of rectangular space for the picture. The optional
second pair, $(x_0,y_0)$, assigns arbitrary coordinates to the bottom left corner of the
reserved rectangle. 

Most drawing commands have one of the two forms
\begin{lscommand}
\ci{put}\verb|(|$x,y$\verb|){|\emph{object}\verb|}|
\end{lscommand}
\noindent or
\begin{lscommand}
\ci{multiput}\verb|(|$x,y$\verb|)(|$\Delta x,\Delta y$\verb|){|$n$\verb|}{|\emph{object}\verb|}|\end{lscommand}
B\'ezier curves are an exception. They are drawn with the command
\begin{lscommand}
\ci{qbezier}\verb|(|$x_1,y_1$\verb|)(|$x_2,y_2$\verb|)(|$x_3,y_3$\verb|)|
\end{lscommand}
\newpage

\subsection{Line Segments}
\begin{example}
\setlength{\unitlength}{5cm}
\begin{picture}(1,1)
  \put(0,0){\line(0,1){1}}
  \put(0,0){\line(1,0){1}}  
  \put(0,0){\line(1,1){1}}  
  \put(0,0){\line(1,2){.5}}
  \put(0,0){\line(1,3){.3333}}
  \put(0,0){\line(1,4){.25}}  
  \put(0,0){\line(1,5){.2}}
  \put(0,0){\line(1,6){.1667}}
  \put(0,0){\line(2,1){1}}
  \put(0,0){\line(2,3){.6667}}
  \put(0,0){\line(2,5){.4}}
  \put(0,0){\line(3,1){1}}  
  \put(0,0){\line(3,2){1}}
  \put(0,0){\line(3,4){.75}}
  \put(0,0){\line(3,5){.6}}
  \put(0,0){\line(4,1){1}}
  \put(0,0){\line(4,3){1}}  
  \put(0,0){\line(4,5){.8}}
  \put(0,0){\line(5,1){1}}
  \put(0,0){\line(5,2){1}}
  \put(0,0){\line(5,3){1}}
  \put(0,0){\line(5,4){1}}
  \put(0,0){\line(5,6){.8333}}
  \put(0,0){\line(6,1){1}}
  \put(0,0){\line(6,5){1}}
\end{picture}
\end{example}
Line segments are drawn with the command
\begin{lscommand}
\ci{put}\verb|(|$x,y$\verb|){|\ci{line}\verb|(|$x_1,y_1$\verb|){|$length$\verb|}}|
\end{lscommand}
The \ci{line} command has two arguments:
\begin{enumerate}
  \item a direction vector,
  \item a length.
\end{enumerate}
The components of the direction vector are restricted to the integers
\[
  -6,\,-5,\,\ldots,\,5,\,6,
\]
and they have to be coprime (no common divisor except 1). The figure illustrates all
25 possible slope values in the first quadrant. The length is relative to \ci{unitlength}.
The length argument is the vertical coordinate in the case of a vertical line segment, the
horizontal coordinate in all other cases.

\subsection{Arrows}

\begin{example}
\setlength{\unitlength}{0.75mm}
\begin{picture}(60,40)
  \put(30,20){\vector(1,0){30}}
  \put(30,20){\vector(4,1){20}}
  \put(30,20){\vector(3,1){25}}
  \put(30,20){\vector(2,1){30}}
  \put(30,20){\vector(1,2){10}}
  \thicklines
  \put(30,20){\vector(-4,1){30}}
  \put(30,20){\vector(-1,4){5}}
  \thinlines
  \put(30,20){\vector(-1,-1){5}}
  \put(30,20){\vector(-1,-4){5}}
\end{picture}
\end{example}
Arrows are drawn with the command
\begin{lscommand}
\ci{put}\verb|(|$x,y$\verb|){|\ci{vector}\verb|(|$x_1,y_1$\verb|){|$length$\verb|}}|
\end{lscommand}
For arrows, the components of the direction vector are even more narrowly restricted than
for line segments, namely to the integers
\[
  -4,\,-3,\,\ldots,\,3,\,4.
\]
Components also have to be coprime (no common divisor except 1). Notice the effect  of the
\ci{thicklines} command on the two arrows pointing to the upper left.

\subsection{Circles}

\begin{example}
\setlength{\unitlength}{1mm}
\begin{picture}(60, 40)
  \put(20,30){\circle{1}}
  \put(20,30){\circle{2}}
  \put(20,30){\circle{4}}
  \put(20,30){\circle{8}}
  \put(20,30){\circle{16}}
  \put(20,30){\circle{32}}
  
  \put(40,30){\circle{1}}
  \put(40,30){\circle{2}}
  \put(40,30){\circle{3}}
  \put(40,30){\circle{4}}
  \put(40,30){\circle{5}}
  \put(40,30){\circle{6}}
  \put(40,30){\circle{7}}
  \put(40,30){\circle{8}}
  \put(40,30){\circle{9}}
  \put(40,30){\circle{10}}
  \put(40,30){\circle{11}}
  \put(40,30){\circle{12}}
  \put(40,30){\circle{13}}
  \put(40,30){\circle{14}}
  
  \put(15,10){\circle*{1}}
  \put(20,10){\circle*{2}}
  \put(25,10){\circle*{3}}
  \put(30,10){\circle*{4}}
  \put(35,10){\circle*{5}}
\end{picture}
\end{example}
The command
\begin{lscommand}
  \ci{put}\verb|(|$x,y$\verb|){|\ci{circle}\verb|{|\emph{diameter}\verb|}}|
\end{lscommand}
\noindent draws a circle with center $(x,y)$ and diameter (not radius) \emph{diameter}.
The \ei{picture} environment only admits diameters up to approximately 14\,mm,
and even below this limit, not all diameters are possible. The \ci{circle*}
command produces disks (filled circles).

As in the case of line segments, one may have to resort to additional packages, 
such as \pai{eepic} or \pai{pstricks}. 
For a thorough description of these packages, see \graphicscompanion.

There is also a possibility within the
\ei{picture} environment. If one is not afraid of doing the necessary calculations
(or leaving them to a program), arbitrary circles and ellipses can be patched
together from quadratic B\'ezier curves. 
See \graphicsinlatex\ for examples and Java source files.

\subsection{Text and Formulas}

\begin{example}
\setlength{\unitlength}{0.8cm}
\begin{picture}(6,5)
  \thicklines
  \put(1,0.5){\line(2,1){3}}
  \put(4,2){\line(-2,1){2}}
  \put(2,3){\line(-2,-5){1}}
  \put(0.7,0.3){$A$}
  \put(4.05,1.9){$B$}
  \put(1.7,2.95){$C$}
  \put(3.1,2.5){$a$}
  \put(1.3,1.7){$b$}
  \put(2.5,1.05){$c$}
  \put(0.3,4){$F=
    \sqrt{s(s-a)(s-b)(s-c)}$}  
  \put(3.5,0.4){$\displaystyle
    s:=\frac{a+b+c}{2}$}
\end{picture}
\end{example}
As this example shows, text and formulas can be written into a \ei{picture} environment with
the \ci{put} command in the usual way.

\subsection{\ci{multiput} and \ci{linethickness}}

\begin{example}
\setlength{\unitlength}{2mm}
\begin{picture}(30,20)
  \linethickness{0.075mm}
  \multiput(0,0)(1,0){26}%
    {\line(0,1){20}}
  \multiput(0,0)(0,1){21}%
    {\line(1,0){25}}
  \linethickness{0.15mm}    
  \multiput(0,0)(5,0){6}%
    {\line(0,1){20}}
  \multiput(0,0)(0,5){5}%
    {\line(1,0){25}}
  \linethickness{0.3mm}    
  \multiput(5,0)(10,0){2}%
    {\line(0,1){20}}
  \multiput(0,5)(0,10){2}%
    {\line(1,0){25}}
\end{picture}
\end{example}
The command
\begin{lscommand}
  \ci{multiput}\verb|(|$x,y$\verb|)(|$\Delta x,\Delta y$\verb|){|$n$\verb|}{|\emph{object}\verb|}|
\end{lscommand}
\noindent has 4 arguments: the starting point, the translation vector from one object to the next, 
the number of objects, and the object to be drawn. The \ci{linethickness} command applies to 
horizontal and vertical line segments, but neither to oblique line segments, nor to circles. 
It does, however, apply to quadratic B\'ezier curves!

\subsection{Ovals}

\begin{example}
\setlength{\unitlength}{0.75cm}
\begin{picture}(6,4)
  \linethickness{0.075mm}
  \multiput(0,0)(1,0){7}%
    {\line(0,1){4}}
  \multiput(0,0)(0,1){5}%
    {\line(1,0){6}}
  \thicklines
  \put(2,3){\oval(3,1.8)} 
  \thinlines
  \put(3,2){\oval(3,1.8)} 
  \thicklines
  \put(2,1){\oval(3,1.8)[tl]} 
  \put(4,1){\oval(3,1.8)[b]} 
  \put(4,3){\oval(3,1.8)[r]} 
  \put(3,1.5){\oval(1.8,0.4)}     
\end{picture}
\end{example}
The command
\begin{lscommand}
  \ci{put}\verb|(|$x,y$\verb|){|\ci{oval}\verb|(|$w,h$\verb|)}|
\end{lscommand}
\noindent or
\begin{lscommand}
  \ci{put}\verb|(|$x,y$\verb|){|\ci{oval}\verb|(|$w,h$\verb|)[|\emph{position}\verb|]}|
\end{lscommand}
\noindent produces an oval centered at $(x,y)$ and having width $w$ and height $h$. The optional 
\emph{position} arguments \texttt{b}, \texttt{t}, \texttt{l}, \texttt{r} refer to 
``top'', ``bottom'', ``left'', ``right'', and can be combined, as the example illustrates. 

Line thickness can be controlled by two kinds of commands: \\ 
\ci{linethickness}\verb|{|\emph{length}\verb|}|
on the one hand, \ci{thinlines} and \ci{thicklines} on the other. While \ci{linethickness}\verb|{|\emph{length}\verb|}|
applies only to horizontal and vertical lines (and quadratic B\'ezier curves), \ci{thinlines} and \ci{thicklines}
apply to oblique line segments as well as to circles and ovals. 


\subsection{Multiple Use of Predefined Picture Boxes}

\begin{example}
\setlength{\unitlength}{0.5mm}
\begin{picture}(120,168)
\newsavebox{\foldera}
\savebox{\foldera}
  (40,32)[bl]{% definition 
  \multiput(0,0)(0,28){2}
    {\line(1,0){40}}
  \multiput(0,0)(40,0){2}
    {\line(0,1){28}}
  \put(1,28){\oval(2,2)[tl]}
  \put(1,29){\line(1,0){5}}
  \put(9,29){\oval(6,6)[tl]}
  \put(9,32){\line(1,0){8}}
  \put(17,29){\oval(6,6)[tr]}
  \put(20,29){\line(1,0){19}}
  \put(39,28){\oval(2,2)[tr]}  
}
\newsavebox{\folderb}
\savebox{\folderb}
  (40,32)[l]{%         definition 
  \put(0,14){\line(1,0){8}}
  \put(8,0){\usebox{\foldera}}
}
\put(34,26){\line(0,1){102}} 
\put(14,128){\usebox{\foldera}}
\multiput(34,86)(0,-37){3}
  {\usebox{\folderb}} 
\end{picture}
\end{example}
A picture box can be \emph{declared} by the command
\begin{lscommand}
  \ci{newsavebox}\verb|{|\emph{name}\verb|}|
\end{lscommand}
\noindent then \emph{defined} by  
\begin{lscommand}
  \ci{savebox}\verb|{|\emph{name}\verb|}(|\emph{width,height}\verb|)[|\emph{position}\verb|]{|\emph{content}\verb|}|
\end{lscommand}
\noindent and finally arbitrarily often be \emph{drawn} by
\begin{lscommand}
  \ci{put}\verb|(|$x,y$\verb|)|\ci{usebox}\verb|{|\emph{name}\verb|}|
\end{lscommand}

The optional \emph{position} parameter has the effect of defining the
`anchor point' of the savebox. In the example it is set to \texttt{bl} which
puts the anchor point into the bottom left corner of the savebox. The other
position specifiers are \texttt{t}op and \texttt{r}ight.

The \emph{name} argument refers to a \LaTeX{} storage bin and therefore is
of a command nature (which accounts for the backslashes in the current
example). Boxed pictures can be nested: In this example, \ci{foldera} is
used within the definition of \ci{folderb}.

The \ci{oval} command had to be used as the \ci{line} command does not work if the segment length is less than 
about 3\,mm.

\subsection{Quadratic B\'ezier Curves}

\begin{example}
\setlength{\unitlength}{0.8cm}
\begin{picture}(6,4)
  \linethickness{0.075mm}
  \multiput(0,0)(1,0){7}
    {\line(0,1){4}}
  \multiput(0,0)(0,1){5}
    {\line(1,0){6}}
  \thicklines
  \put(0.5,0.5){\line(1,5){0.5}}    
  \put(1,3){\line(4,1){2}} 
  \qbezier(0.5,0.5)(1,3)(3,3.5)
  \thinlines   
  \put(2.5,2){\line(2,-1){3}}
  \put(5.5,0.5){\line(-1,5){0.5}}
  \linethickness{1mm}
  \qbezier(2.5,2)(5.5,0.5)(5,3)
  \thinlines
  \qbezier(4,2)(4,3)(3,3)
  \qbezier(3,3)(2,3)(2,2)
  \qbezier(2,2)(2,1)(3,1)
  \qbezier(3,1)(4,1)(4,2)
\end{picture}
\end{example}
As this example illustrates, splitting up a circle into 4 quadratic B\'ezier curves
is not satisfactory. At least 8 are needed. The figure again shows the effect of
the \ci{linethickness} command on horizontal or vertical lines, and of the 
\ci{thinlines} and the \ci{thicklines} commands on oblique line segments. It also 
shows that both kinds of commands affect quadratic B\'ezier curves, each command
overriding all previous ones.

Let $P_1=(x_1,\,y_1),\,P_2=(x_2,\,y_2)$ denote the end points, and $m_1,\,m_2$ the
respective slopes, of a quadratic B\'ezier curve. The intermediate control point 
$S=(x,\,y)$ is then given by the equations
\begin{equation} \label{zwischenpunkt}
  \left\{
    \begin{array}{rcl}
      x & = & \displaystyle \frac{m_2 x_2-m_1x_1-(y_2-y_1)}{m_2-m_1}, \\
      y & = & y_i+m_i(x-x_i)\qquad (i=1,\,2).
    \end{array}
  \right.
\end{equation}
\noindent See \graphicsinlatex\ for a Java program which generates
the necessary \ci{qbezier} command line.

\subsection{Catenary}

\begin{example}
\setlength{\unitlength}{1cm}
\begin{picture}(4.3,3.6)(-2.5,-0.25)
\put(-2,0){\vector(1,0){4.4}}
\put(2.45,-.05){$x$}
\put(0,0){\vector(0,1){3.2}}
\put(0,3.35){\makebox(0,0){$y$}}
\qbezier(0.0,0.0)(1.2384,0.0)
  (2.0,2.7622) 
\qbezier(0.0,0.0)(-1.2384,0.0)
  (-2.0,2.7622)
\linethickness{.075mm}
\multiput(-2,0)(1,0){5}
  {\line(0,1){3}}
\multiput(-2,0)(0,1){4}
  {\line(1,0){4}}
\linethickness{.2mm}
\put( .3,.12763){\line(1,0){.4}}
\put(.5,-.07237){\line(0,1){.4}}
\put(-.7,.12763){\line(1,0){.4}}
\put(-.5,-.07237){\line(0,1){.4}}
\put(.8,.54308){\line(1,0){.4}}
\put(1,.34308){\line(0,1){.4}}
\put(-1.2,.54308){\line(1,0){.4}}
\put(-1,.34308){\line(0,1){.4}}
\put(1.3,1.35241){\line(1,0){.4}}
\put(1.5,1.15241){\line(0,1){.4}}
\put(-1.7,1.35241){\line(1,0){.4}}
\put(-1.5,1.15241){\line(0,1){.4}}
\put(-2.5,-0.25){\circle*{0.2}}
\end{picture}
\end{example}

In this figure, each symmetric half of the catenary $y=\cosh x -1$ is approximated by a quadratic
B\'ezier curve. The right half of the curve ends in the point \((2,\,2.7622)\), the slope there having the value 
\(m=3.6269\). Using again equation (\ref{zwischenpunkt}), we can 
calculate the intermediate control points. They turn out to be $(1.2384,\,0)$ and $(-1.2384,\,0)$. 
The crosses indicate points of the \emph{real} catenary. The error is barely noticeable, being less 
than one percent.

This example points out the use of the optional argument of the \\
\verb|\begin{picture}| command.
The picture is defined in convenient ``mathematical'' coordinates, whereas by the command
\begin{lscommand} 
  \ci{begin}\verb|{picture}(4.3,3.6)(-2.5,-0.25)|
\end{lscommand}
\noindent its lower left corner (marked by the black disk) is assigned the coordinates $(-2.5,-0.25)$. 

\subsection{Rapidity in the Special Theory of Relativity}

\begin{example}
\setlength{\unitlength}{0.8cm}
\begin{picture}(6,4)(-3,-2)
  \put(-2.5,0){\vector(1,0){5}}
  \put(2.7,-0.1){$\chi$}
  \put(0,-1.5){\vector(0,1){3}}
  \multiput(-2.5,1)(0.4,0){13}
    {\line(1,0){0.2}}
  \multiput(-2.5,-1)(0.4,0){13}
    {\line(1,0){0.2}}
  \put(0.2,1.4)
    {$\beta=v/c=\tanh\chi$}
  \qbezier(0,0)(0.8853,0.8853)
    (2,0.9640)
  \qbezier(0,0)(-0.8853,-0.8853)
    (-2,-0.9640)
  \put(-3,-2){\circle*{0.2}}
\end{picture}
\end{example}
The control points of the two B\'ezier curves were calculated with formulas (\ref{zwischenpunkt}).
The positive branch is determined by $P_1=(0,\,0),\,m_1=1$ and $P_2=(2,\,\tanh 2),\,m_2=1/\cosh^2 2$.
Again, the picture is defined in mathematically convenient coordinates, and the lower left corner
is assigned the mathematical coordinates $(-3,-2)$ (black disk).


\section{\texorpdfstring{\Xy}{Xy}-pic}
\secby{Alberto Manuel Brand\~ao Sim\~oes}{albie@alfarrabio.di.uminho.pt}
\pai{xy} is a special package for drawing diagrams. To use it,
simply add the following line to the preamble of your document:
\begin{lscommand}
\verb|\usepackage[|\emph{options}\verb|]{xy}|
\end{lscommand}
\emph{options} is a list of functions from \Xy-pic you want to
load. These options are primarily useful when debugging the package.  I recommend
you pass the \verb!all! option, making \LaTeX{} load all the \Xy{} commands.

\Xy-pic diagrams are drawn over a matrix-oriented canvas, where
each diagram element is placed in a matrix slot:
\begin{example}
\begin{displaymath}
\xymatrix{A & B \\
          C & D }
\end{displaymath}
\end{example}
The \ci{xymatrix} command must be used in math mode. Here, we
specified two lines and two columns. To make this matrix a diagram we
just add directed arrows using the \ci{ar} command.
\begin{example}
\begin{displaymath}
\xymatrix{ A \ar[r] & B \ar[d] \\
           D \ar[u] & C \ar[l] }
\end{displaymath}
\end{example}
The arrow command is placed on the origin cell for the arrow. The
arguments are the direction the arrow should point to (\texttt{u}p,
\texttt{d}own, \texttt{r}ight and \texttt{l}eft).

\begin{example}
\begin{displaymath}
\xymatrix{
  A \ar[d] \ar[dr] \ar[r] & B \\
  D                       & C }
\end{displaymath}
\end{example}
To make diagonals, just use more than one direction. In
fact, you can repeat directions to make bigger arrows.
\begin{example}
\begin{displaymath}
\xymatrix{
  A \ar[d] \ar[dr] \ar[drr] & & \\
  B                      & C & D }
\end{displaymath}
\end{example}

We can draw even more interesting diagrams by adding
labels to the arrows. To do this, we use the common superscript and
subscript operators.
\begin{example}
\begin{displaymath}
\xymatrix{
  A \ar[r]^f \ar[d]_g &
             B \ar[d]^{g'} \\
  D \ar[r]_{f'}       & C }
\end{displaymath}
\end{example}

As shown, you use these operators as in math mode. The only
difference is that that superscript means ``on top of the arrow,''
and subscript means ``under the arrow.'' There is a third operator, the vertical bar: \verb+|+
It causes text to be placed \emph{in} the arrow.
\begin{example}
\begin{displaymath}
\xymatrix{
  A \ar[r]|f \ar[d]|g &
             B \ar[d]|{g'} \\
  D \ar[r]|{f'}       & C }
\end{displaymath}
\end{example}

To draw an arrow with a hole in it, use \verb!\ar[...]|\hole!.

In some situations, it is important to distinguish between different types of
arrows. This can be done by putting labels on them, or changing their appearance:

\begin{example}
\shorthandoff{"}
\begin{displaymath}
\xymatrix{
\bullet\ar@{->}[rr] && \bullet\\
\bullet\ar@{.<}[rr] && \bullet\\
\bullet\ar@{~)}[rr] && \bullet\\
\bullet\ar@{=(}[rr] && \bullet\\
\bullet\ar@{~/}[rr] && \bullet\\
\bullet\ar@{^{(}->}[rr] &&
                       \bullet\\
\bullet\ar@2{->}[rr] && \bullet\\
\bullet\ar@3{->}[rr] && \bullet\\
\bullet\ar@{=+}[rr]  && \bullet
}
\end{displaymath}
\shorthandon{"}
\end{example}

Notice the difference between the following two diagrams:

\begin{example}
\begin{displaymath}
\xymatrix{
 \bullet \ar[r] 
         \ar@{.>}[r] & 
 \bullet
}
\end{displaymath}
\end{example}

\begin{example}
\begin{displaymath}
\xymatrix{
 \bullet \ar@/^/[r] 
         \ar@/_/@{.>}[r] &
 \bullet
}
\end{displaymath}
\end{example}

The modifiers between the slashes define how the curves are drawn.
\Xy-pic offers many ways to influence the drawing of curves;
for more information, check \Xy-pic documentation.

\selectlanguage{french}
