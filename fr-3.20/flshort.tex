%%%%%%%%%%%%%%%%%%%%%%%%%%%%%%%%%%%%%%%%%%%%%%%%%%%%%%%%%%%%%%%%%
% Contents: Main Input File of the LaTeX2e Introduction
% $Id: flshort.tex,v 1.6 2001/11/25 17:43:58 matthieu Exp $
%%%%%%%%%%%%%%%%%%%%%%%%%%%%%%%%%%%%%%%%%%%%%%%%%%%%%%%%%%%%%%%%%
% flshort.tex - Une courte (?) introduction a LaTeX2e
%                                                      by Tobias Oetiker
%                                                  tobias@ife.ee.ethz.ch
%                                                      oetiker@dmu.ac.uk
%
%                           based on LKURTZ.TEX Uni Graz & TU Wien,1987
%                           traduit en fran�ais par Matthieu Herrb,1996
%-----------------------------------------------------------------------
%
% To compile flshort, you need TeX 3.x, LaTeX2e with the babel frenchb
% extension  and makeindex
%
% The sources files of the Intro are:
%      flshort.tex (this file),
%      title.tex, contrib.tex, biblio.tex
%      things.tes, typeset.tex, math.tex, lssym.tex, spec.tex,
%      flshort.sty
%
% Further the  verbatim.sty and the layout.sty 
% from the LaTeX Tools distribution is
% required.
%
% To print the AMS symbols you need the AMS fonts and the packages
% amsfonts, eufrak and eucal from (AMS LaTeX 1.2)
%
% ---------------------------------------------------------------------
\newif\ifpdf 
\ifx\pdfoutput\undefined 
  \pdffalse 
\else \ifx \pdfoutput\relax 
    \pdffalse 
  \else 
    \pdftrue 
  \fi 
\fi 

\ifpdf
\documentclass[pdftex,11pt,a4paper,twoside]{book}
\usepackage{thumbpdf}
\pdfcompresslevel=9
\else
\documentclass[11pt,a4paper,twoside]{book}
\fi
\usepackage[T1]{fontenc}
\usepackage[latin1]{inputenc}
\usepackage{amsmath}
\usepackage{flshort}
\usepackage{shortvrb,latexsym}
\usepackage[french]{mylayout}
\usepackage{lettrine}
\usepackage{makeidx}

\ifpdf
\RequirePackage[colorlinks,hyperindex,plainpages=false]{hyperref}
\def\pdfBorderAttrs{/Border [0 0 0] } % No border arround Links
\else
\RequirePackage[colorlinks]{hyperref}
\fi

\usepackage[frenchb,english]{babel}
%
% Copyright \copyright{} 1998 Tobias Oetiker and all the Contributers
% to LShort.  All rights reserved.
% 
% This document is free; you can redistribute it and/or modify it
% under the terms of the GNU General Public License as published by
% the Free Software Foundation; either version 2 of the License, or
% (at your option) any later version.
% 
% This document is distributed in the hope that it will be useful, but
% WITHOUT ANY WARRANTY; without even the implied warranty of
% MERCHANTABILITY or FITNESS FOR A PARTICULAR PURPOSE.  See the GNU
% General Public License for more details.
% 
% You should have received a copy of the GNU General Public License
% along with this program; if not, write to the Free Software
% Foundation, Inc., 59 Temple Place, Suite 330, Boston, MA 02111-1307,
% USA.
%
% Original Copyright H.Partl, E.Schlegl, and I.Hyna (1987).
% English Version Copyright by Tobias Oetiker (1994,1995),
% 
% ---------------------------------------------------------------------
%
% Formats also with\textt{letterpaper} option, but the pagebreaks might not
% fall as nicely.
%
% To produce a A5 booklet, use a tool like  pstops or dvitodvi
% to  past them together in the right order. Most dvi printer drivers
% can shrink the resulting output to fit on a A4 sheet.
%
\makeindex
\typeout{Copyright T.Oetiker, H.Partl, E.Schlegl, I.Hyna}
%\includeonly{typeset,math} 
\begin{document}
\selectlanguage{frenchb}
\NoAutoSpaceBeforeFDP
\frontmatter
%%%%%%%%%%%%%%%%%%%%%%%%%%%%%%%%%%%%%%%%%%%%%%%%%%%%%%%%%%%%%%%%%
% Contents: The title page
% $Id: title.tex,v 1.1.1.1 2002/02/26 10:04:20 oetiker Exp $
%%%%%%%%%%%%%%%%%%%%%%%%%%%%%%%%%%%%%%%%%%%%%%%%%%%%%%%%%%%%%%%%%

\ifx\pdfoutput\undefined % We're not running pdftex
\else
\pdfbookmark{Title Page}{title}
\fi
\newlength{\centeroffset}
\setlength{\centeroffset}{-0.5\oddsidemargin}
\addtolength{\centeroffset}{0.5\evensidemargin}
%\addtolength{\textwidth}{-\centeroffset}
\thispagestyle{empty}
\vspace*{\stretch{1}}
\noindent\hspace*{\centeroffset}\makebox[0pt][l]{\begin{minipage}{\textwidth}
\flushright
{\Huge\bfseries The Not So Short\\ 
Introduction to \LaTeXe

}
\noindent\rule[-1ex]{\textwidth}{5pt}\\[2.5ex]
\hfill\emph{\Large Or \LaTeXe{} in \pageref{verylast} minutes}
\end{minipage}}

\vspace{\stretch{1}}
\noindent\hspace*{\centeroffset}\makebox[0pt][l]{\begin{minipage}{\textwidth}
\flushright
{\bfseries 
by Tobias Oetiker\\[1.5ex]
Hubert Partl, Irene Hyna and  Elisabeth Schlegl\\[3ex]} 
Version~3.21, 04 October, 2002
\end{minipage}}

%\addtolength{\textwidth}{\centeroffset}
\vspace{\stretch{2}}


\pagebreak
\begin{small} 
  Copyright \copyright 2000-2002 Tobias Oetiker and all the Contributers to
  LShort.  All rights reserved.
 
  This document is free; you can redistribute it and/or modify it
  under the terms of the GNU General Public License as published by
  the Free Software Foundation; either version 2 of the License, or
  (at your option) any later version.
  
  This document is distributed in the hope that it will be useful, but
  WITHOUT ANY WARRANTY; without even the implied warranty of
  MERCHANTABILITY or FITNESS FOR A PARTICULAR PURPOSE\@.  See the GNU
  General Public License for more details.
  
  You should have received a copy of the GNU General Public License
  along with this document; if not, write to the Free Software
  Foundation, Inc., 675 Mass Ave, Cambridge, MA 02139, USA.

\end{small}


\endinput

%%% Local Variables: 
%%% mode: latex
%%% TeX-master: "lshort2e"
%%% End: 

% Local Variables:
% mode: flyspell
% End:

%%%%%%%%%%%%%%%%%%%%%%%%%%%%%%%%%%%%%%%%%%%%%%%%%%%%%%%%%%%%%%%%%
% Contents: Who contributed to this Document
% $Id: contrib.tex,v 1.1.1.1 2002/02/26 10:04:20 oetiker Exp $
%%%%%%%%%%%%%%%%%%%%%%%%%%%%%%%%%%%%%%%%%%%%%%%%%%%%%%%%%%%%%%%%%
\chapter{Merci !}
\thispagestyle{plain}

\label{lettrine}
\lettrine{C}{e document}
est une traduction en fran�ais de �\,\emph{The
not so short introduction to LaTeX2e}\,� par Tobias Oetiker.

\noindent 
Une grande partie de ce document provient d'une introduction
autrichienne � \LaTeX\ 2.09, �crite en allemand par :
\begin{verse}
\contrib{Hubert Partl}{partl@mail.boku.ac.at}%
{Zentraler Informatikdienst der Universit\"at f\"ur Bodenkultur, Wien}
\contrib{Irene Hyna}{Irene.Hyna@bmwf.ac.at}%
   {Bundesministerium f\"ur Wissenschaft und Forschung, Wien}
\contrib{Elisabeth Schlegl}{no email}%
   {in Graz}
\end{verse}

La version courante en fran�ais est disponible sur :\\
\texttt{CTAN:/info/lshort/french/}% 
\,\footnote{Voir page~\pageref{CTAN} la liste des sites \texttt{CTAN}.}

Vous trouverez la version anglaise de Tobias Oetiker sur :\\
\texttt{CTAN:/info/lshort/english/}

Si vous �tes int�ress�s par la version allemande, vous trouverez une
version adapt�e � \LaTeXe{} par J\"org Knappen sur :\\
\texttt{CTAN:/info/lshort/german/}

\newpage
Pour la pr�paration de ce document, l'aide des lecteurs du forum 
\mbox{Usenet}
\texttt{comp.text.tex} a �t� sollicit�e. De nombreuses personnes ont
r�pondu et ont fourni des corrections, des suggestions et du texte pour
am�liorer ce document. Qu'ils en soient ici remerci�s
sinc�rement. Ajoutons que je suis responsable de toutes les erreurs
que vous pourriez trouver dans ce document.

Merci en particulier � :

\begin{quote}
\flushleft
Rosemary~Bailey,        %r.a.bailey@qmw.ac.uk 0.2
Marc~Bevand,            % <bevand_m@epita.fr>
Friedemann~Brauer,      %fbrauer@is.dal.ca 3.4
Jan~Busa,               % <busaj@ccsun.tuke.sk>
Markus~Br\"uhwiler,     % <m.br@switzerland.org>
Pietro~Braione,         % <braione@elet.polimi.it>
David~Carlisle,         %GONE carlisle@cs.man.ac.uk 1.0
Jos\'e~Carlos~Santos,   % <jcsantos@fc.up.pt>
Mike~Chapman,           %chapman@eeh.ee.ethz.ch 3.16
Pierre~Chardaire,       % <pc@sys.uea.ac.uk
Christopher~Chin,       %chris.chin@rmit.edu.au 3.1
Carl~Cerecke,           %cdc@cosc.canterbury.ac.nz>
Chris~McCormack,        %GONE chrismc@eecs.umich.edu 0.1
Wim~van~Dam,            %GONE wimvdam@cs.kun.nl 2.2
Jan~Dittberner,         %jan@jan-dittberner.de 3.15
Michael~John~Downes,    %<mjd@ams.org> 14 Oct 1999
Matthias~Dreier,        %dreier@ostium.ch
David~Dureisseix,       %dureisse@lmt.ens-cachan.fr 1.1
Elliot,                 %GONE enh-a@minster.york.ac.uk 1.1
Hans~Ehrbar,            %ehrbar@econ.utah.edu
Daniel~Flipo,           %Daniel.Flipo@univ-lille1.fr
David~Frey,             %david@eos.lugs.ch 2.2
Hans~Fugal,             %hans@fugal.net
Robin~Fairbairns,       %Robin.Fairbairns@cl.cam.ac.uk 0.2 1.0
J\"org~Fischer,        %j.fischer@xpoint.at 3.16
Erik~Frisk,             %frisk@isy.liu.se 3.4
Mic~Milic~Frederickx,   % <mic.milic@web.de>
Frank,                  %frank@freezone.co.uk 11 Feb 2000
Kasper~B.~Graversen,    % <kbg@dkik.dk>
Arlo~Griffiths,         % <A.Griffiths@let.leidenuniv.nl>
Alexandre~Guimond,      %guimond@IRO.UMontreal.CA 0.9
Cyril~Goutte,           %goutte@ei.dtu.dk 2.1 2.2
Greg~Gamble,            %gregg@maths.uwa.edu.au 2.2
Neil~Hammond,           %nfh@dmu.ac.uk 0.3
Rasmus~Borup~Hansen,    %GONE rbhfamos@math.ku.dk 0.2 0.9 0.91 0.92 1.9.9
Joseph~Hilferty,        % <hilferty@fil.ub.es>
Bj\"orn Hvittfeldt,     %bjorn@hvittfeldt.com 3.13
Martien~Hulsen,         %M.A.Hulsen@WbMt.TUDelft.NL 1.0 1.1
Werner~Icking,          %<Werner.Icking@gmd.de> 3.1
Jakob,                  %diness@get2net.dk
Eric~Jacoboni,          %GONE jacoboni@enseeiht.fr 0.1 0.9
Alan~Jeffrey,           %alanje@cogs.sussex.ac.uk 0.2
Byron~Jones,            %bj@dmu.ac.uk 1.1
David~Jones,            %GONE djones@CA.McMaster.dcss.insight 1.1
Johannes-Maria~Kaltenbach, %<kaltenbach@zeiss.de> 3.01
Michael~Koundouros,     % <mkoundouros@hotmail.com>
Andrzej~Kawalec,        %GONE akawalec@prz.rzeszow.pl 1.9.9
Alain~Kessi,            %ALAIN_KESSI@HOTMAIL.COM 2.2
Christian Kern,         %ck@unixen.hrz.uni-oldenburg.de 2.1
J\"org~Knappen,         %knappen@vkpmzd.kph.uni-mainz.de 0.1
Kjetil~Kjernsmo,        %<kjetil.kjernsmo@astro.uio.no> 3.2
Maik~Lehradt,           %greek@uni-paderborn.de 0.1
R\'emi~Letot,           % <r_letot@yahoo.com>
Johan~Lundberg,         %p99jlu@physto.se
Alexander~Mai,          %Alexander.Mai@physik.tu-darmstadt.de 3.8
Martin~Maechler,        %<maechler@stat.math.ethz.ch> 2.2
Aleksandar~S~Milosevic, % <aleksandar.milosevic@yale.edu>
Henrik~Mitsch,          % <Henrik.Mitsch@gmx.at>
Claus~Malten,           %GONE <ASI138%BITNET.DJUKFA11@BITNET.CEARN> 1.1
Kevin~Van~Maren,        % <vanmaren@fast.cs.utah.edu>  24 Nov 1999
Lenimar~Nunes~de~Andrade, % <lenimar@mat.ufpb.br> Fri, 12 Nov 1999
Demerson~Andre~Polli,   % polli@linux.ime.usp.br
Maksym~Polyakov         % <polyama@myrealbox.com>
Hubert~Partl,           %partl@mail.boku.ac.at 0.2 1.1
John~Refling,           %refling@sierra.lbl.gov 0.1 0.9
Mike~Ressler,           %ressler@cougar.jpl.nasa.gov 0.1 0.2 0.9 1.0 1.9.9
Brian~Ripley,           %ripley@stats.ox.ac.uk 2.1
Young~U.~Ryu,           %ryoung@utdallas.edu 2.1
Bernd~Rosenlecher,      %9rosenle@informatik.uni-hamburg.de 10 Feb 2000
Chris~Rowley,           %C.A.Rowley@open.ac.uk 0.91
Risto~Saarelma,         %risto.saarelma@cs.helsinki.fi
Hanspeter~Schmid,       %schmid@isi.ee.ethz.ch
Craig~Schlenter,        %cschle@lucy.ee.und.ac.za 0.1 0.2 0.9
Baron~Schwartz,         % <bps7j@cs.virginia.edu>      
Christopher~Sawtell,    %<csawtell@xtra.co.nz> 1 Sep 1999
Geoffrey~Swindale,      % <geofftswin@ntlworld.com>
Boris~Tobotras,         % <tobotras@jet.msk.su>
Josef~Tkadlec,          %tkadlec@math.feld.cvut.cz 2.0 2.2
Scott~Veirs,            %scottv@ocean.washington.edu
Didier~Verna,           %verna@inf.enst.fr 2.2
Fabian~Wernli,          %wernli@iap.fr 3.2
Carl-Gustav~Werner,     % <Carl-Gustav.Werner@math.lu.se> 11 Oct 1999,3.16
David~Woodhouse,        % <dwmw2@infradead.org> 3.16
Chris~York,             % <c.s.york@Cummins.com>  21 Nov 1999
Fritz~Zaucker,          %zaucker@ee.ethz.ch 3.0
Rick~Zaccone,           %zaccone@bucknell.edu 2.2
et Mikhail~Zotov.       %zotov@eas.npi.msu.su 3.1
\end{quote}

\vspace{2ex}
La version fran�aise a b�n�fici� des contributions des lecteurs du 
forum \texttt{fr.comp.text.tex} et en particulier de :
\begin{quote}
\flushleft
 Sebastien Blondeel,            % blondeel@clipper.ens.fr 
 Marie-Dominique Cabanne,       % Marie-Dominique.Cabanne@laas.fr
 Christophe Dousson,            % christophe.dousson@rd.francetelecom.com
 Olivier Dupuis,		% olivier.dupuis@incotec.fr
 Daniel Flipo,                  % Daniel.Flipo@univ-lille1.fr
 Paul Gaborit,                  % gaborit@enstimac.fr
 Thomas Ribo,			% thomas.ribo@free.fr
 Philippe Spiesser              % Philippe.Spiesser@laas.fr
et Vincent Zoonekynd.           % zoonek@math.jussieu.fr
\end{quote}

\vspace*{\stretch{1}}

\emph{Note du traducteur :} je tiens �galement � remercier
chaleureusement les auteurs de ce document de le rendre publiquement
utilisable et d'avoir ainsi rendu possible cette version fran�aise.

\pagebreak
\endinput
%

% Local Variables:
% TeX-master: "lshort2e"
% mode: latex
% mode: flyspell
% End:

%%%%%%%%%%%%%%%%%%%%%%%%%%%%%%%%%%%%%%%%%%%%%%%%%%%%%%%%%%%%%%%%%
% Contents: Who contributed to this Document
% $Id: overview.tex,v 1.2 2003/03/19 20:57:46 oetiker Exp $
%%%%%%%%%%%%%%%%%%%%%%%%%%%%%%%%%%%%%%%%%%%%%%%%%%%%%%%%%%%%%%%%%

% Because this introduction is the reader's first impression, I have
% edited very heavily to try to clarify and economize the language.
% I hope you do not mind! I always try to ask "is this word needed?"
% in my own writing but I don't want to impose my style on you... 
% but here I think it may be more important than the rest of the book.
% --baron

\chapter{Pr�face}
\thispagestyle{plain}

\LaTeX{}\cite{manual} est un logiciel de composition typographique
adapt� � la production de documents scientifiques et math�matiques de
grande qualit� typographique. Il permet �galement de produire
toutes sortes d'autres documents, qu'il s'agisse de simples lettres ou
de livres entiers. \LaTeX{} utilise \TeX\cite{texbook} comme outil de
mise en page. 

Cette introduction d�crit \LaTeXe{} et devrait se montrer suffisante
pour la plupart des applications de \LaTeX. Pour une description
compl�te du syst�me \LaTeX{}, reportez-vous
�~\cite{manual,companion}. 

\bigskip
\noindent Cette introduction est compos�e de six chapitres :
\begin{description}

\item[Le chapitre 1] pr�sente la structure �l�mentaire d'un document
  \LaTeXe{}. Il vous apprendra �galement quelques �l�ments sur
  l'histoire de \LaTeX{}. Apr�s avoir lu ce chapitre, vous devriez
  avoir une vue g�n�rale de ce qu'est \LaTeX{}.

\item[Le chapitre 2] entre dans les d�tails de la mise en page d'un
  document. Il explique les commandes et les environnements
  essentiels de \LaTeX{}. Apr�s avoir lu ce chapitre, vous serez
  capables de r�diger vos premiers documents.

\item[Le chapitre 3] explique comment coder des formules
  math�matiques avec \LaTeX{}. De nombreux exemples sont donn�s pour
  montrer comment utiliser cet atout majeur de \LaTeX{}. � la fin de ce
  chapitre, vous trouverez des tableaux qui listent tous les symboles
  math�matiques disponibles.

\item[Le chapitre 4] explique comment r�aliser un index, une liste de
  r�f�rences bibliographiques ou l'insertion de figures en PostScript
  encapsul�.  Il pr�sente la cr�ation de documents PDF avec
  pdf\LaTeX{} ainsi que quelques autres extensions utiles.

\item[Le chapitre 5] montre comment utiliser \LaTeX{} pour cr�er des
  figures. Au lieu de dessiner une image � l'aide d'un programme
  d'infographie donn�, la sauvegarder et l'inclure dans \LaTeX{}, vous
  d�crirez l'image et laisserez \LaTeX{} la dessiner pour vous.

\item[Le chapitre 6] contient des informations potentiellement
  dangeureuses. Il vous apprend � modifier la mise en page standard
  produite par \LaTeX{} et vous  permet de transformer 
  les pr�sentations plut�t r�ussies de \LaTeX{} en quelque chose
  de laid ou magnifique, selon votre habilet�.
  d'assez laid.
\end{description}

\bigskip
\noindent Il est important de lire ces chapitres dans l'ordre. Apr�s
tout, ce livre n'est pas si long.  L'�tude attentive des exemples
est indispensable � la compr�hension de l'ensemble car ils contiennent
une bonne partie de l'information que vous pourrez trouver ici.

\bigskip
\noindent \LaTeX{} est disponible pour une vaste gamme de syst�mes
informatiques, des PCs et Macs aux syst�mes UNIX
\footnote{UNIX est une marque d�pos�e de The Open Group}
 et VMS. Dans de
nombreuses universit�s, il est install� sur le r�seau informatique,
pr�t � �tre utilis�. L'information n�cessaire pour y acc�der devrait
�tre fournie dans le \guide. Si vous avez des difficult�s pour
commencer, demandez de l'aide � la personne qui vous a donn� cette
brochure.  Ce document \emph{n'est pas} un guide d'installation du
syst�me \LaTeX{}. Son but est de vous montrer comment �crire vos
documents afin qu'ils puissent �tre trait�s par \LaTeX{}.

\bigskip
\index{CTAN@\texttt{CTAN}}
Si vous avez besoin de r�cup�rer des fichiers relatifs � \LaTeX{},
utilisez les sites \texttt{CTAN}(\emph{Comprehensive
  \TeX{} Archive Network})\label{CTAN}.
Le site principal est sur \url{http://www.ctan.org} et toutes les
extensions peuvent �tre obtenues sur l'archive ftp
\url{ftp://www.ctan.org} ou l'un de ses nombreux miroirs. En France un
miroir se trouve sur \url{ftp\string://ftp.lip6.fr/pub/TeX/CTAN/}. Aux
�tats-Unis, il s'agit de \url{ftp\string://ctan.tug.org/}, en Allemagne de
\url{ftp\string://ftp.dante.de/} et au Royaume-Uni de
\url{ftp\string://ftp.tex.ac.uk/}. 
Si vous n'�tes pas dans l'un de ces pays, choisissez le site le plus
proche de chez vous.

Vous verrez plusieurs r�f�rences � CTAN au long de ce document, en
particulier des pointeurs vers des logiciels ou des documents. Au lieu
d'�crire des URL complets, nous avons simplement �crit \texttt{CTAN:}
suivi du chemin d'acc�s � partir de l'un des sites CTAN ci-dessus.

Si vous souhaitez installer \LaTeX{} sur votre ordinateur, vous
trouverez sans doute une version adapt�e � votre syst�me sur
sur \CTAN|systems|.

\vspace{\stretch{1}}
\noindent Si vous avez des suggestions concernant ce qui pourrait �tre
ajout�, supprim� ou modifi� dans ce document, contactez soit
directement l'auteur de la version originale, soit moi-m�me, le
traducteur.  Nous sommes particuli�rement int�ress�s par des retours
d'utilisateurs d�butants en \LaTeX{} au sujet des passages de ce livre
qui devraient �tre mieux expliqu�s.


\bigskip
\begin{verse}
\contrib{Tobias Oetiker}{oetiker@ee.ethz.ch}%
{Department of Information Technology and\\ Electrical Engineering,
Swiss Federal Institute of Technology, Z�rich.}

\contrib{Matthieu Herrb}{matthieu.herrb@laas.fr}%
{Laboratoire d'Analyse et d'Architecture des Syst�mes,\\
Centre National de la Recherche Scientifique, Toulouse.}

\contrib{Samuel Colin}{scolin@hivernal.org}%
{(� partir de la version 3.21fr)}
\end{verse}
\vspace{\stretch{1}}
\noindent La version courante de ce document est disponible sur
\CTAN|info/lshort|

\endinput

%

% Local Variables:
% TeX-master: "lshort2e"
% mode: latex
% mode: flyspell
% End:

\tableofcontents
\listoffigures
\listoftables
\mainmatter
%%%%%%%%%%%%%%%%%%%%%%%%%%%%%%%%%%%%%%%%%%%%%%%%%%%%%%%%%%%%%%%%%
% Contents: Things you need to know
% $Id: things.tex,v 1.2 2003/03/19 20:57:47 oetiker Exp $
%%%%%%%%%%%%%%%%%%%%%%%%%%%%%%%%%%%%%%%%%%%%%%%%%%%%%%%%%%%%%%%%%

\chapter{Ce qu'il faut savoir}
\thispagestyle{plain}

\begin{intro}
La premi�re partie de ce chapitre couvre rapidement la philosophie et
l'histoire de \LaTeXe{}. La
deuxi�me partie met l'accent sur les structures fondamentales d'un
document \LaTeX{}. Apr�s avoir lu ce chapitre, vous devriez avoir une
id�e d'ensemble du fonctionnement de \LaTeX{} qui vous permettra de
mieux comprendre les chapitres suivants.
\end{intro}

\section{Le nom de la b�te}
\subsection{\TeX}
 
\TeX{} est un programme �crit par \index{Knuth, Donald E.}Donald
E.~Knuth~\cite{texbook}.  Il est con�u pour la composition de textes
et d'�quations math�matiques.  

Knuth a commenc� le d�veloppement de \TeX{} en 1977 parce qu'il �tait
frustr� par la mani�re avec laquelle ses articles �taient publi�s par
l'American Mathematical Society. Il avait arr�t� de soumettre des
articles vers 1974 parce que \og le r�sultat final �tait trop p�nible �
regarder \fg{}. \TeX{}, tel que nous l'utilisons aujourd'hui, est sorti en
1982 et a �t� am�lior� au fil des ans. Ces derni�res ann�es \TeX{} a
atteint une grande stabilit�. Aujourd'hui Knuth affirme qu'il n'y a
virtuellement plus de \og bug \fg{}. Le num�ro de version de \TeX{} tend
vers $\pi$ et est actuellement $3.14159$.

\TeX{} se prononce \og Tech \fg{}, avec un \og ch \fg{} comme dans le mot
�cossais \og Loch \fg{}.
\,\footnote{Il est � noter que l'orthographe particuli�re de \TeX{}
  tend � laisser la prononciation suivre la fa�on dont les lettres qui
  le composent sont prononc�es dans le pays o� il est utilis�. Les
  allemands, plut�t que d'utiliser le \og ch \fg{} de \og Ach \fg{},
  pr�f�rent celui de \og Pech \fg{}, ce qui donnerait comme
  prononciation en fran�ais \og t�che \fg{}. � propos de ce point,
  Knuth a �crit dans le Wikipedia allemand : \emph{Je ne m'offusque
    pas que les gens prononcent \TeX{} de la mani�re qu'ils pr�f�rent
    \ldots{} et en allemand, nombreux sont ceux qui utilisent le \og
    ch \fg{} doux car le X suit la voyelle \og e \fg{}, pas comme le
    \og ch \fg{} qui suit le \og a \fg{}. En Russie, \og tex \fg{} est
    un mot commun qui se prononce \og tyekh \fg{}. Je crois cependant
    que la meilleure prononciation est la grecque, o� l'on a le \og ch
    \fg{} dur de \og ach \fg{} et \og Loch \fg{}.}}
En alphabet phon�tique cela donne
\textbf{[tex]}\dots Dans un environnement \texttt{ASCII}, \TeX{}
devient \texttt{TeX}.

\subsection{\LaTeX}
 
\LaTeX{} est un ensemble de macros qui permettent � un auteur de
mettre en page son travail avec la meilleure qualit� typographique en
utilisant un format professionnel pr�-d�fini. \LaTeX{} a �t� �crit par
\index{Lamport, Leslie}Leslie Lamport~\cite{manual}. Il utilise \TeX{}
comme outil de mise en page.

% En 1994, \LaTeX{} a �t� mis � jour par
% l'�quipe \index{LaTeX3@\LaTeX 3}\LaTeX 3, men�e par \index{Mittelbach,
% Frank}Frank Mittelbach, afin de r�aliser certaines am�liorations
% demand�es depuis longtemps et de fusionner toutes les variantes qui
% s'�taient d�velopp�es depuis la sortie de \index{LaTeX 2.09@\LaTeX{}
% 2.09}\LaTeX{} 2.09 quelques ann�es auparavent. Pour distinguer cette
% nouvelle version des pr�c�dentes, elle est appel�e \index{LaTeX
% 2e@\LaTeXe}\LaTeXe. Ce document est relatif � \LaTeXe.

\LaTeX{} se prononce \textbf{[latex]}. Si vous
voulez faire r�f�rence � \LaTeX{} dans un environnement
\texttt{ASCII}, utilisez \texttt{LaTeX}. \LaTeXe{} se prononce
\textbf{[latex d\o{}z\o{}]} et s'�crit
\texttt{LaTeX2e}.

En anglais, cela donne \textbf{[la\i{}tex]} et \textbf{[la\i{}tex tu:
i:]}. 

% La figure~\ref{components}, page~\pageref{components} montre
% l'interaction entre les diff�rents �l�ments d'un syst�me \TeX{}. Cette
% figure est extraite de \texttt{wots.tex} de Kees van der Laan.

% \begin{figure}[btp]
% \begin{lined}{0.8\textwidth}
% \begin{center}
% \input{kees.fig}
% \end{center}
% \end{lined}
% \caption{�l�ments d'un syst�me \TeX{}} \label{components}
% \end{figure}

\section{Les bases}
 
\subsection{Auteur, �diteur et typographe}

Pour publier un texte, un auteur confie son manuscrit  � une maison
d'�dition. L'�diteur d�cide alors de la mise en page du document
(largeur des colonnes, polices de caract�res, pr�sentation des
en-t�tes,\dots). L'�diteur note ses instructions sur le manuscrit et
le passe � un technicien typographe qui r�alise la mise en page
en suivant ces instructions.

Un �diteur humain essaye de comprendre ce que l'auteur veut mettre en
valeur et d�cide de la pr�sentation en fonction de son exp�rience
professionnelle et du contenu du manuscrit. 

Dans un environnement \LaTeX{}, celui-ci joue le r�le de l'�diteur et
utilise \TeX{} comme typographe pour la composition. Mais \LaTeX{}
n'est qu'un programme et a donc besoin de plus de directives. L'auteur
doit en particulier lui fournir la structure logique de son
document. Cette information est ins�r�e dans le texte sous la forme de
\og commandes \LaTeX{} \fg{}.

Cette approche est totalement diff�rente de l'approche
\wi{WYSIWYG}\,\footnote{What you see is what you get -- Ce que vous
voyez est ce qui sera imprim�.}  utilis�e par les traitements de texte
modernes tels que \emph{Microsoft Word} ou \emph{Corel
WordPerfect}. Avec ces programmes, l'auteur d�finit la mise en page du
document de mani�re interactive pendant la saisie du texte. Il voit �
l'�cran � quoi ressemblera le document final une fois imprim�.

Avec \LaTeX{}, il n'est normalement pas possible de voir le r�sultat
final durant la saisie du texte, mais celui-ci peut �tre
pr�-visualis� apr�s traitement du fichier par \LaTeX{}. Des corrections
peuvent alors �tre apport�es avant d'envoyer la version d�finitive
vers l'imprimante.

\subsection{Choix de la mise en page}

La typographie est un m�tier (un art ?). Les auteurs inexp�riment�s
font souvent de graves erreurs en consid�rant que la mise en page est
avant tout une question d'esth�tique : \og si un document est beau, il
est bien con�u \fg{}. Mais un document doit �tre lu et non accroch� dans
une galerie d'art. La lisibilit� et la compr�hensibilit� sont bien
plus importantes que le \og look \fg{}. Par exemple :
\begin{itemize}
\item la taille de la police et la num�rotation des en-t�tes doivent
      �tre choisies afin de mettre en �vidence la structure des
      chapitres et des sections ;
\item les lignes ne doivent pas �tre trop longues pour ne pas fatiguer
      la vue du lecteur, tout en remplissant la page de mani�re
      harmonieuse. 
\end{itemize}

Avec un logiciel \wi{WYSIWYG}, l'auteur produit g�n�ralement des
documents esth�tiquement plaisants (quoi que\dots) mais tr�s peu ou
mal structur�s. \LaTeX{} emp�che de telles erreurs de formatage en
for�ant l'auteur � d�crire la structure logique de son document et en
choisissant lui-m�me la mise en page la plus appropri�e.


\subsection{Avantages et inconv�nients}

Un sujet de discussion qui  revient souvent quand des gens du monde
\wi{WYSIWYG} rencontrent des utilisateurs de \LaTeX{} est le 
suivant : \og les \wi{avantages de \LaTeX{}} par rapport � un
traitement de texte classique \fg{} ou bien le contraire.
La meilleure chose � faire quand une telle discussion d�marre, est de
garder son calme, car souvent cela d�g�n�re. Mais parfois on ne peut y
�chapper\dots

\medskip Voici donc quelques arguments. Les principaux avantages de
\LaTeX{} par rapport � un traitement de texte traditionnel sont :

\begin{itemize}

\item mise en page professionnelle qui donne aux documents l'air de
      sortir de l'atelier d'un imprimeur ;
\item la composition des formules math�matiques se fait de mani�re
      pratique ;
\item il suffit de conna�tre quelques commandes de
      base pour d�crire la structure logique du document. 
      Il n'est pas n�cessaire de se pr�occuper de la mise en page ;
\item des structures complexes telles que des notes de bas de page,
      des renvois, la table des mati�res ou les r�f�rences
      bibliographiques sont produites facilement ;
\item pour la plupart des t�ches de la typographie qui ne sont pas
      directement g�r�es par \LaTeX{}, il existe des extensions
      gratuites. Par exemple pour inclure des figures
      \PSi{} ou pour formater une bibliographie selon un
      standard pr�cis. La majorit� de ces extensions sont d�crites dans
      \companion{} (en anglais) et dans \desgraupes{} (en fran�ais) ;
\item \LaTeX{} encourage les auteurs � �crire des documents bien
      structur�s, parce que c'est ainsi qu'il fonctionne (en
      d�crivant la structure) ;
\item \TeX{}, l'outil de formatage de \LaTeXe{}, est r�ellement
      portable et gratuit. Ainsi il est disponible sur quasiment
      toutes les machines existantes.
%
% Add examples ...
%
\end{itemize}

\medskip

\LaTeX{} a �galement quelques inconv�nients ; il est difficile
pour moi d'en trouver, mais d'autres vous en citeront des centaines  :

\begin{itemize}
\item \LaTeX{} ne fonctionne pas bien pour ceux qui ont vendu leur 
      �me ;
\item bien que quelques param�tres des mises en page pr�-d�finies
      puissent �tre personnalis�s, la mise au point d'une pr�sentation
      enti�rement nouvelle est difficile et demande beaucoup de
      temps\,\footnote{La rumeur dit que c'est un des points qui
      devrait �tre am�lior�s dans la future version \LaTeX
      3}\index{LaTeX3@\LaTeX 3} ;
\item �crire des documents mal organis�s et mal structur�s est tr�s
      difficile. 
\end{itemize}
 
\section{Fichiers source \LaTeX{}}

L'entr�e de \LaTeX{} est un fichier texte \texttt{ASCII}. Vous pouvez
le cr�er avec l'�diteur de texte de votre choix. Il contient le texte
de votre document ainsi que les commandes qui vont dire � \LaTeX{}
comment mettre en page le texte. On appelle ce fichier \emph{\wi{fichier
source}}. 

\subsection{Espaces}

Les caract�res d'espacement, tels que les blancs ou les tabulations,
sont trait�s de mani�re unique comme \og \wi{espace} \fg{} par
\LaTeX{}. Plusieurs \wi{blancs} \emph{cons�cutifs} sont consid�r�s
comme \emph{une seule} espace\,\footnote{En langage typographique,
    \emph{espace} est un mot f�minin. \emph{NdT.}}.  Les espaces en d�but
d'une ligne sont en g�n�ral ignor�es et un retour � la ligne unique est
trait� comme une espace.  \index{espace!en d�but de ligne}

Une ligne vide entre deux lignes de texte marque la fin d'un
paragraphe. \emph{Plusieurs} lignes vides sont consid�r�es comme
\emph{une seule} ligne vide. Le texte ci-dessous est un exemple. Sur
la gauche se trouve le contenu du fichier source et � droite le
r�sultat format�.

\begin{example}
Saisir un ou      plusieurs
espaces entre  les     mots
n'a pas d'importance.

Une ligne vide commence 
un nouveau paragraphe.
\end{example}
 
\subsection{Caract�res sp�ciaux}

Les symboles suivants sont des \wi{caract�res r�serv�s} qui, soit ont 
une signification sp�ciale dans \LaTeX{}, soit ne sont pas disponibles
dans toutes les polices. Si vous les saisissez directement dans votre
texte, ils ne seront pas imprim�s et forceront \LaTeX{} � faire des
choses que vous n'avez pas voulues.

\begin{code}
\verb.$ & % # _ { }  ~  ^  \ . %$
\end{code}

Comme vous le voyez, certains de ces caract�res peuvent �tre utilis�s
dans vos documents en les pr�fixant par un antislash :

\begin{example}
\$ \& \% \# \_ \{ \}
\end{example} 
%$

%%% XXX Trouver un terme meilleur qu'antislash

Les autres et bien d'autres encore peuvent �tre obtenus avec des
commandes sp�ciales � l'int�rieur de formules math�matiques ou comme
accents. L'antislash $\backslash$ ne peut pas �tre saisi en ajoutant
un second antislash (\verb|\\|) : cette s�quense est utilis�e pour
indiquer les coupures de ligne\,\footnote{Utilisez la commande
\texttt{\$}\ci{backslash}\texttt{\$}. Elle produit un $\backslash$.}.

\subsection{Commandes \LaTeX{}}

Les \wi{commandes} \LaTeX{} sont sensibles � la casse des caract�res
(majuscules ou minuscules) et utilisent l'un des deux formats
suivants :

\begin{itemize}
\item soit elles commencent par un \wi{antislash} \verb|\| et ont un
      nom compos� uniquement de lettres. Un nom de commande est
      termin� par une espace, un chiffre ou tout autre caract�re qui
      n'est pas une lettre ;
\index{backslash}
\item soit elles sont compos�es d'un antislash et d'un 
      caract�re sp�cial (non-lettre) exactement.
\end{itemize}

%
% \\* doesn't comply !
%

%
% Can \3 be a valid command ? (jacoboni)
%
\label{whitespace}
\LaTeX{} ignore les espaces apr�s les commandes. Si vous souhaitez
obtenir un blanc apr�s une commande\index{espace!apr�s une commande},
il faut ou bien ins�rer \verb|{}| suivi d'un blanc ou bien utiliser une
commande d'espacement sp�cifique de \LaTeX{}. La s�quence \verb|{}|
emp�che \LaTeX{} d'ignorer les blancs apr�s une commande.

\begin{example}
J'ai lu que Knuth classe les
gens qui utilisent \TeX{} en
\TeX{}niciens et en \TeX perts.\\
Aujourd'hui nous sommes le \today.
\end{example}

Certaines commandes sont suivies d'un \wi{param�tre} fourni entre
\wi{accolades} \verb|{ }|. Certaines commandes supportent des
\wi{param�tres optionnels} qui suivent le nom de la commande entre
\wi{crochets}~\verb|[ ]|. L'exemple suivant montre quelques commandes
\LaTeX{}. Ne vous tracassez pas pour les comprendre, elles seront
expliqu�es plus loin.

\begin{example}
\textsl{Penchez}-vous !
\end{example}
\begin{example}
S'il vous plait, passez \`a la 
ligne ici.\newline
Merci !
\end{example}

\subsection{Commentaires}
\index{commentaires}

Quand \LaTeX{} rencontre un caract�re \verb|%| dans le fichier
source, il ignore le reste de la ligne en cours, le retour � la ligne
et les espaces au d�but de la ligne suivante.

C'est utile pour ajouter des notes qui n'appara�tront pas dans la
version imprim�e.

\begin{example}
% Demonstration :
Ceci est un % mauvais
exemple: anticonstitu%
       tionnellement
\end{example}

Le caract�re \texttt{\%} peut �galement �tre utilis� pour couper des
lignes trop longues dans le fichier d'entr�e, lorsqu'aucun espace ou
retour � la ligne n'est autoris�.

Pour cr�er des commentaires plus longs, il vaut mieux utilier
l'environnement \ei{comment} fourni par l'extension \pai{verbatim}. 
Vous apprendrez plus loin � utiliser une extension. 

\begin{example}
Voici un autre exemple
\begin{comment}
Limit� mais demonstratif
\end{comment}
de commentaires. 
\end{example}

Notez cependant que cet environnement n'est pas utilisable �
l'int�rieur d'autres environnements complexes, tels que le mode
math�matique par exemple.

\section{Structure du fichier source}

Quand \LaTeX{} analyse un fichier source, il s'attend � y trouver une
certaine structure. C'est pourquoi chaque fichier source doit
commencer par la commande :
\begin{code}
\verb|\documentclass{...}|
\end{code}
Elle indique quel type de document vous voulez �crire. Apr�s cela vous
pouvez ins�rer des commandes qui vont influencer le style du document
ou vous pouvez charger des \wi{extension}s qui ajoutent de nouvelles
fonctionnalit�s au syst�me \LaTeX{}. Pour charger une extension,
utilisez la commande :
\begin{code}
\verb|\usepackage{...}|
\end{code}

Quand tout le travail de pr�paration est fait\,\footnote{La partie entre
\texttt{\bs{}documentclass} et
\texttt{\bs{}begin$\mathtt{\{}$document$\mathtt{\}}$} est appel�e le
\emph{\wi{pr�ambule}}.}, vous pouvez commencer le corps du texte avec
la commande :
\begin{code}
\verb|\begin{document}|
\end{code}

Maintenant vous pouvez saisir votre texte et y ins�rer des commandes
\LaTeX{}. � la fin de votre document, utilisez la commande
\begin{code}
\verb|\end{document}|
\end{code}
pour dire � \LaTeX{} qu'il en a fini. Tout ce qui suivra dans le
fichier source sera ignor�.

La figure~\ref{mini} montre le contenu d'un document \LaTeXe{}
minimum. Un fichier source plus complet est pr�sent� sur la
figure~\ref{document}. 

\begin{figure}[hbp]
\begin{lined}{6cm}
\begin{verbatim}
\documentclass{article}
\begin{document}
Small is beautiful.
\end{document}
\end{verbatim}
\end{lined}
\caption{Un fichier \LaTeX{} minimal} \label{mini}
\end{figure}
 
\begin{figure}[htbp]
\begin{lined}{10cm}
\begin{verbatim}
\documentclass[a4paper,11pt]{article}
\usepackage[T1]{fontenc}
\usepackage[english,francais]{babel}
\author{P.~Tar}
\title{Le Minimalisme}
\begin{document}
\maketitle
% ins�rer la table des mati�res
\tableofcontents
\section{Quelques mots descriptifs}
Et bien, ici commence mon \oe uvre.
\section{Au revoir, monde}
\ldots{} Et ainsi s'ach�ve mon ouvrage.
\end{document}
\end{verbatim}
\end{lined}
\caption{Exemple d'un article de revue plus r�aliste} \label{document}
\end{figure}
 
\section{Utilisation typique en ligne de commande}

Vous br�lez probablement d'envie d'essayer l'exemple pr�sent�
page~\pageref{mini}. Voici quelques informations : \LaTeX{} lui-m�me
ne propose pas d'interface graphique ni de jolis boutons � cliquer. Il
s'agit simplement d'un programme qui \og dig�re \fg{} votre fichier
source. Certaines installations de \LaTeX{} ajoutent une interface
graphique permettant de cliquer pour lancer la mise en page de votre
document. Sur d'autres syst�mes il faudra probablement taper quelques
lignes de commande, aussi voici comment convaincre \LaTeX{} de
compiler votre fichier d'entr�e sur un syst�me � interface
textuelle. Notez cependant que ces explications supposent que \LaTeX{}
soit d�j� install� sur votre ordinateur.\footnote{
C'est le cas de toute bonne d�clinaison d'Unix, et \ldots{} les
Puristes utilisent Unix, donc \ldots{} \texttt{;-)}}

\begin{enumerate}
\item Cr�ez/�ditez votre fichier source \LaTeX{}. Il s'agit d'un
      fichier texte pur. Sur les syst�mes Unix, tous les �diteurs
      cr�ent ce type de fichier. Sous Windows, assurez-vous que le
      fichier est sauvegard� en texte seul (ASCII ou encore \og plain
      text \fg{}). Choisissez pour votre fichier un nom avec le suffixe
      \eei{.tex}. 

\item 

Ex�cutez \LaTeX{} sur votre fichier. Si tout se passe bien, vous
obtiendrez un nouveau fichier avec le suffixe \texttt{.dvi}. Il peut
�tre n�cessaire d'ex�cuter \LaTeX{} plusieurs fois afin que la table
des mati�res et les r�f�rences internes soient � jour. S'il y a un
bogue dans votre fichier, \LaTeX{} vous le signalera et s'arr�tera.
Appuyez sur la combinaison \texttt{Ctrl-D} pour revenir � la ligne de
commande.
\begin{lscommand}
\verb+latex document.tex+
\end{lscommand}

\item � pr�sent, vous pouvez visualiser le r�sultat, le fichier
  DVI. Il y a plusieurs fa�ons de le faire. Vous pouvez par exemple
  afficher le fichier � l'�cran via
\begin{lscommand}
\verb+xdvi document.dvi+
\end{lscommand}
Cela ne fonctionne que sous un syst�me Unix avec X11. Si vous utilisez
Windows vous pouvez essayer via \texttt{yap} (\emph{yet another
  previewer} -- NdT. \emph{encore un autre pr�visualiseur}).

Vous pouvez �galement transformer le r�sultat en \PSi{} pour
impression ou affichage avec Ghostcript.
\begin{lscommand}
\verb+dvips -Pcmz -o document.ps document.dvi+
\end{lscommand}

Avec un peu de chance, votre installation \LaTeX{} contient l'outil
\texttt{\wi{dvipdf}}, qui permet de convertir les fichiers \eei{.dvi}
directement en \eei{.pdf}.
\begin{lscommand}
\verb+dvipdf document.dvi+
\end{lscommand}

\end{enumerate}

% \clearpage
\section{La mise en page du document}
 
\subsection {Classes de documents}\label{sec:documentclass}

La premi�re information dont \LaTeX{} a besoin en traitant un fichier
source est le type de document que son auteur est en train de
cr�er. Ce type est sp�cifi� par la commande \ci{documentclass}.
\begin{lscommand}
\ci{documentclass}\verb|[|\emph{options}\verb|]{|\emph{classe}\verb|}|
\end{lscommand}
Ici \emph{classe} indique le type de document � cr�er. Le
tableau~\ref{documentclasses} donne la liste des classes de documents
pr�sent�es dans cette introduction. \LaTeXe{} fournit d'autres classes
pour d'autres types de documents, notamment des lettres et des
transparents. Le param�tre \emph{\wi{option}s} permet de modifier le
comportement de la classe de document. Les options sont s�par�es par
des virgules. Les principales options disponibles sont pr�sent�es dans
le tableau~\ref{options}.


\begin{table}[!thbp]
\caption{Classes de documents} \label{documentclasses}
\begin{lined}{12cm}
\begin{description}
 
\item [\normalfont\texttt{article}] pour des articles dans des revues
      scientifiques, des pr�sentations, des rapports courts, des
      documentations, des invitations, etc.
  \index{article (classe)}
\item [\normalfont\texttt{report}] pour des rapports plus longs
      contenant plusieurs chapitres, des petits livres, des th�ses, etc.
  \index{report (classe)}
  \index{rapport}
\item [\normalfont\texttt{book}] pour des vrais livres.
  \index{book (classe)}
  \index{livre}
\item [\normalfont\texttt{slides}] pour des transparents. Cette classe
      utilise de grands caract�res sans serif. Voir �galement la
      classe Foil\TeX{}\,\footnote{%
        \CTANref|macros/latex/contrib/supported/foiltex|}
   \index{slides@\textsf{slides}}\index{foiltex@\textsf{foiltex}}
   \index{transparents}
\end{description}
\end{lined}
\end{table}

\begin{table}[!hbp]
\caption{Options de classes de document} \label{options}
\begin{lined}{12cm}
\begin{flushleft}
\begin{description}
\item[\normalfont\texttt{10pt}, \texttt{11pt}, \texttt{12pt}] \quad
d�finit la taille de la police principale du document. Si aucune
option n'est pr�sente, la taille par d�faut est de \texttt{10pt}.
 \index{taille!de la police par d�faut}
\item[\normalfont\texttt{a4paper}, \texttt{letterpaper}, \dots] \quad
d�finit la taille du papier. Le papier par d�faut est
\texttt{letterpaper}, le format standard am�ricain. Les autres valeurs
possibles sont : \texttt{a5paper}, \texttt{b5paper}, \texttt{executivepaper},
  et \texttt{legalpaper}. \index{legal (papier)}
  \index{taille!du papier}\index{A4 (papier)}\index{letter (papier)} \index{A5
    (papier)}\index{B5 (papier)}\index{executive (papier)}
\index{papier!taille du}
\index{papier!A4}
\index{papier!A5}
\index{papier!letter}

\item[\normalfont\texttt{fleqn}] \quad aligne les formules
  math�matiques � gauche au lieu de les centrer.
\index{fleqn@\texttt{fleqn}}

\item[\normalfont\texttt{leqno}] \quad place la num�rotation des
formules � gauche plut�t qu'� droite.
\index{leqno@\texttt{leqno}}

\item[\normalfont\texttt{titlepage}, \texttt{notitlepage}] \quad
indique si une nouvelle page doit �tre commenc�e apr�s le \wi{titre du
document} ou non. La classe \texttt{article} continue par d�faut sur
la m�me page contrairement aux classes \texttt{report} et
\texttt{book}.  \index{titlepage@\texttt{titlepage}}
\index{notitlepage@\texttt{notitlepage}}

\item[\normalfont\texttt{onecolumn}, \texttt{twocolumn}] \quad 
demandent � \LaTeX{} de formater le texte sur une seule colonne
(\wi{deux colonnes}, respectivement).
\index{onecolumn@\texttt{onecolumn}}
\index{twocolumn@\texttt{twocolumn}}

\item[\normalfont\texttt{twoside, oneside}] \quad indique si la sortie
se fera en \wi{recto-verso} ou en \wi{recto simple}. Par d�faut, les classes
\texttt{article} et \texttt{report} sont en \wi{simple face} 
alors que la classe \texttt{book} est en \wi{double-face}.
\index{twoside@\texttt{twoside}}
\index{oneside@\texttt{oneside}}

+\item[\normalfont\texttt{landscape}] \quad change la disposition du
mode portrait au mode paysage.
\index{landscape@\texttt{landscape}}

\item[\normalfont\texttt{openright, openany}] \quad fait commencer un
chapitre sur la page de droite ou sur la prochaine page. Cette option
n'a pas de sens avec la classe \texttt{article} qui ne conna�t pas la
notion de chapitre. Par d�faut, la classe \texttt{report} commence les
chapitres sur la prochaine page vierge alors que la classe
\texttt{book} les commence toujours sur une page de droite.
\index{openright@\texttt{openright}}
\index{openany@\texttt{openany}}

\end{description}
\end{flushleft}
\end{lined}
\end{table}

Exemple : un fichier source pour un document \LaTeX{} pourrait
commencer par la ligne
\begin{code}
\ci{documentclass}\verb|[11pt,twoside,a4paper]{article}|
\end{code}
elle informe \LaTeX{} qu'il doit composer ce document comme un
\emph{article} avec une taille de caract�re de base de \emph{onze
points} et qu'il devra produire une mise en page pour une impression
\emph{double face} sur du papier au format \emph{A4}%
\,\footnote{Sans l'option \texttt{a4paper}, le format de papier sera
am�ricain : 8,5~$\times$~11 pouces, soit 216~$\times$~280 mm.}.
\pagebreak[2]

\subsection{Extensions}
\index{extension} 
En r�digeant votre document, vous remarquerez peut-�tre qu'il y a des
domaines o� les commandes de base de \LaTeX{} ne permettent pas
d'exprimer ce que vous voudriez. Si vous voulez inclure des
graphiques, du texte en couleur ou du code d'un programme dans votre
document, il faut augmenter les possibilit�s de \LaTeX{} gr�ce � des
extensions. 
Une extension est charg�e par la commande
\begin{lscommand}
\ci{usepackage}\verb|[|\emph{options}\verb|]{|\emph{extension}\verb|}|
\end{lscommand}
\emph{extension} est le nom de l'extension et \emph{options} une liste
de mots-cl�s qui d�clenchent certaines fonctions de
l'extension. Certaines extensions font partie de la distribution
standard de \LaTeXe{} (reportez-vous au
tableau~\ref{extensions}). D'autres sont distribu�es � part. Le
\guide{} peut vous fournir plus d'informations sur les extensions
install�es sur votre site. \companion{} est la principale source
d'information au sujet de \LaTeXe{}. Ce livre contient la description
de centaines d'extensions ainsi que les informations n�cessaires pour
�crire vos propres extensions � \LaTeXe.

Les distributions \TeX{} modernes sont fournies avec un tr�s grand
nombre d'extensions pr�install�es. Si vous travaillez sur un syst�me
de type Unix, utilisez la commande \texttt{texdoc} pour acc�der � la
documentation d'une extension.

\begin{table}[!tbp]
\caption{Quelques extensions fournies avec \LaTeX} \label{extensions}
\begin{lined}{11cm}
\begin{description}
\item[\normalfont\texttt{doc}] permet de documenter des programmes
 pour  \LaTeX{}.\\
 D�crite dans \texttt{doc.dtx}\,\footnote{Ce fichier devrait �tre intall�
 sur votre syst�me et vous devriez �tre capable de le formater avec
 la commande \texttt{latex doc.dtx}. Il en est de m�me pour les autres
 fichiers cit�s dans ce tableau.} et dans \companion.

\item[\normalfont\pai{exscale}] fournit des versions de taille
  param�trable des polices math�matiques �tendues.\\ 
  D�crite dans \texttt{ltexscale.dtx}.

\item[\normalfont\pai{fontenc}] sp�cifie le \wi{codage} des polices
  de caract�re que \LaTeX{} va utiliser.\\
  D�crite dans \texttt{ltoutenc.dtx}.

\item[\normalfont\pai{ifthen}] fournit des commandes de la forme\\
  `if\dots then do\dots otherwise do\dots.'\\ 
  D�crite dans \texttt{ifthen.dtx}, dans \companion{} et dans
  \desgraupes{}. 

\item[\normalfont\pai{latexsym}] permet l'utilisation de la police des
  symboles \LaTeX{}.\\
  D�crite dans \texttt{latexsym.dtx}, dans \companion{} et dans
  \desgraupes{}. 
 
\item[\normalfont\pai{makeidx}] fournit des commandes pour r�aliser
  un index.\\
  D�crite dans ce document, section~\ref{sec:indexing}, dans
  \companion{} et dans \desgraupes{}.

\item[\normalfont\pai{syntonly}] analyse un document sans le
  formater.\\
  D�crite dans \texttt{syntonly.dtx} et dans \companion. Utile pour une
  v�rification rapide de la syntaxe.
  
\item[\normalfont\pai{inputenc}] permet de sp�cifier le codage des
  caract�res utilis� dans le fichier source, parmi ASCII, ISO Latin-1,
  ISO Latin-2, 437/850 IBM code pages,  Apple Macintosh, Next,
  ANSI-Windows ou un codage d�fini par l'utilisateur.\\
  D�crite dans \texttt{inputenc.dtx}. 
\end{description}
\end{lined}
\end{table}


\subsection{Styles de page}

\LaTeX{} propose trois combinaisons d'\wi{en-t�te}s et de \wi{pieds de
page}, appel�es styles de page et d�finies par le param�tre \emph{style} de la
commande :
\index{style de page!plain@\texttt{plain}}\index{plain@\texttt{plain}}
\index{style de page!headings@\texttt{headings}}
\index{headings@\texttt{headings}}
\index{style de page!empty@\texttt{empty}}\index{empty@\texttt{empty}}
\begin{lscommand}
\ci{pagestyle}\verb|{|\emph{style}\verb|}|
\end{lscommand}
Le tableau~\ref{pagestyle}
donne la liste des styles pr�d�finis.

\begin{table}[!hbp]
\caption{Les styles de page de \LaTeX} \label{pagestyle}
\begin{lined}{12cm}
\begin{description}

\item[\normalfont\texttt{plain}] imprime le num�ro de page au milieu
du pied de page. C'est le style par d�faut.

\item[\normalfont\texttt{headings}] imprime le titre du chapitre
courant et le num�ro de page dans l'en-t�te de chaque page et laisse le
pied de page vide. C'est � peu pr�s le style utilis� dans ce document.

\item[\normalfont\texttt{empty}] laisse l'en-t�te et le pied de page
vides. 
\end{description}
\end{lined}
\end{table}

On peut changer le style de la page en cours gr�ce � la commande
\begin{lscommand}
\ci{thispagestyle}\verb|{|\emph{style}\verb|}|
\end{lscommand}

Au chapitre~\ref{chap:spec}, page~\pageref{sec:fancyhdr}, vous apprendrez
comment cr�er vos propres en-t�tes et pieds de pages.


\section{Les fichiers manipul�s}

L'utilisateur de \LaTeX{} est amen� � cotoyer un grand nombre de
fichiers aux \wi{suffixe}s divers. Comme chaque suffixe renseigne sur
le \wi{type de fichier} dont il s'agit, il est utile d'en conna�tre la
signification, voici les suffixes les plus courants. Si vous pensez
qu'il en manque, n'h�sitez pas � contacter les auteurs:

\begin{description}
\item[\eei{.tex}] fichier source \TeX{} ou \LaTeX{} ;
\item[\eei{.sty}] fichier contenant des commandes, que l'on charge dans le
  pr�ambule d'un document gr�ce � une commande \ci{usepackage} ;
\item[\eei{.dtx}] fichier contenant du code \LaTeX{} (commandes) document�,
  le lancement de \LaTeX{} sur un fichier \texttt{.dtx} en extrait la
  documentation.
\item[\eei{.ins}] fichier permettant d'installer le contenu du
  fichier~\texttt{.dtx} de m�me nom. Une extension \LaTeX{}
  t�l�charg�e de l'Internet est compos�e d'un fichier \texttt{.dtx} et
  d'un \texttt{.ins}. Ex�cuter \LaTeX{} sur le fichier \texttt{.ins}
  pour extraire les fichiers � installer du \texttt{.dtx}.
\item[\eei{.cls}] d�signe un fichier de \emph{classe} contenant la
  description d'un type de document, charg� par la commande
  \ci{documentclass};
\item[\eei{.fd}] fichier contenant des d�finitions pour les polices de
  caract�res ;
\end{description}

Les fichiers suivants sont produits par \LaTeX{} � partir du fichier
source :

\begin{description}
\item[\eei{.dvi}] signifie \emph{DeVice Independent}, c'est le fichier
  que l'on visualise � l'�cran et qui servira � l'impression (par
  \texttt{dvips} par exemple) ;
\item[\eei{.log}] fichier contenant le compte-rendu d�taill� de la
  compilation du fichier source (avec les messages d'erreur
  �ventuels) ;
\item[\eei{.toc}] contient le mat�riel n�cessaire � la production de
  la   table des mati�res, si celle-ci a �t� demand�e. Ce fichier sera
  lu � la prochaine ex�cution de \LaTeX{} ;
\item[\eei{.lof}] contient la liste num�rot�e des figures, si elle a
  �t� demand�e ;
\item[\eei{.lot}] contient la liste num�rot�e des tableaux, si elle a
  �t� demand�e ;
\item[\eei{.aux}] contient diverses informations utiles � \LaTeX, en
  particulier ce qui est n�cessaire au fonctionnement des r�f�rences
  crois�es. Le fichier \texttt{.aux} produit lors d'une ex�cution de
  \LaTeX{} est lu lors de l'ex�cution suivante ;
\item[\eei{.idx}] fichier produit seulement si un index est demand�,
  il doit �tre trait� par \texttt{makeindex} (voir
  section~\ref{sec:indexing} page~\pageref{sec:indexing}). \LaTeX{} y
  stocke tous les mots qui iront en index ;
\item[\eei{.ind}] fichier produit par \texttt{makeindex} � partir
  du~\texttt{.idx}, il contient l'index pr�t � �tre inclus dans le
  document ;
\item[\eei{.ilg}] fichier contenant le compte-rendu du travail de
  \texttt{makeindex}.
\end{description}


% Package Info pointer
%
%



%
% Add Info on page-numbering, ...
% \pagenumbering

\section{Gros documents}

Lorsque l'on travaille sur de gros documents, il peut �tre
pratique de couper le fichier source en plusieurs morceaux. \LaTeX{} a
deux commandes qui vous permettent de g�rer plusieurs fichiers sources.

\begin{lscommand}
\ci{include}\verb|{|\emph{fichier}\verb|}|
\end{lscommand}
Vous pouvez utiliser cette commande dans le corps de votre document
pour ins�rer le contenu d'un autre fichier source. \LaTeX{} ajoute
automatiquement le suffixe \texttt{.tex} au nom sp�cifi�. Remarquez que
\LaTeX{} va sauter une page pour traiter le contenu de
\emph{fichier}\texttt{.tex}. 

La seconde commande peut �tre utilis�e dans le pr�ambule. Elle permet
de dire � \LaTeX{} de n'inclure que certains des fichiers d�sign�s par
les commandes \verb|\include|.
\begin{lscommand}
\ci{includeonly}\verb|{|\emph{fichier}\verb|,|\emph{fichier}%
\verb|,|\ldots\verb|}|
\end{lscommand}
Apr�s avoir rencontr� cette commande dans le pr�ambule d'un document,
seules les commandes \ci{include} dont les fichiers sont cit�s en
param�tre de la commande \ci{includeonly} seront ex�cut�es. Attention,
il ne doit pas y avoir d'espace entre les noms de fichiers et les
virgules. 

La commande \ci{include} saute une page avant de commencer le
formatage du texte inclus. Ceci est utile lorsqu'on utilise
\ci{includeonly}, parce qu'ainsi les sauts de pages ne bougeront pas,
m�me si certains morceaux ne sont pas inclus. Parfois ce comportement
n'est pas souhaitable. Dans ce cas, vous pouvez utiliser la commande :
\begin{lscommand}
\ci{input}\verb|{|\emph{fichier.tex}\verb|}|
\end{lscommand}
\noindent qui ins�re simplement le fichier indiqu� sans aucun traitement
sophistiqu�. 

\enlargethispage{\baselineskip}

Il est possible de demander � \LaTeX{} de simplement v�rifier la syntaxe d'un
document, sans produire de fichier~\texttt{.dvi} pour gagner du temps, 
en  utilisant l'extension~\texttt{syntonly} :
\begin{verbatim}
\usepackage{syntonly}
\syntaxonly
\end{verbatim}
La v�rification termin�e, il suffit de mettre ces deux lignes
(ou simplement la seconde) en commentaire en pla�ant un~\texttt{\%}
en t�te de ligne.

% Cette remarque est-elle utile ?
% Attention : certaines extensions red�finissent parfois certains
% caract�res sp�ciaux\footnote{Par exemple le caract�re \og soulign� \fg{} :
% \texttt{\_}.}. Ils ne peuvent plus alors �tre utilis�s dans les
% \emph{noms de fichiers}.

\endinput

%

% Local Variables:
% TeX-master: "lshort2e"
% mode: latex
% mode: flyspell
% End:

%%%%%%%%%%%%%%%%%%%%%%%%%%%%%%%%%%%%%%%%%%%%%%%%%%%%%%%%%%%%%%%%%
% Contents: Typesetting Part of LaTeX2e Introduction
% $Id: typeset.tex 169 2008-09-24 07:32:13Z oetiker $
%%%%%%%%%%%%%%%%%%%%%%%%%%%%%%%%%%%%%%%%%%%%%%%%%%%%%%%%%%%%%%%%%
\chapter{Mise en page}
\thispagestyle{plain}

\begin{intro}
  Apr�s la lecture du chapitre pr�c�dent vous connaissez maintenant
  les �l�ments de base qui constituent un document \LaTeX{}. Dans ce
  chapitre, nous allons compl�ter vos connaissances afin de vous
  rendre capables de cr�er des documents r�alistes.
\end{intro}

\section{La structure du document et le langage}
\secby{Hanspeter Schmid}{hanspi@schmid-werren.ch}

La principale raison d'�tre d'un texte (� l'exception de certains textes
de la litt�rature contemporaine
\footnote{
NdT. \emph{Diff�rente � tout prix}, traduction du \emph{Different at
  all cost} du texte original, lui-m�me une traduction du
suisse-allemand UVA (\emph{Um's Verrecken Anders}).
}%
) est de diffuser des id�es, de
l'information ou de la connaissance au lecteur. Celui-ci comprendra
mieux le texte si ces id�es sont bien structur�es et il
ressentira d'autant mieux cette structure si la typographie utilis�e
refl�te la structure logique et s�mantique du contenu.

Ce qui distingue \LaTeX{} des autres logiciels de traitement de texte
c'est qu'il suffit d'indiquer � \LaTeX{} la structure logique et
s�mantique d'un texte. Il en d�duit la forme typographique en fonction
des \og r�gles \fg{} d�finies dans la classe de document et les diff�rents
fichiers de style.

L'�l�ment de texte le plus important pour \LaTeX{} (et en typographie)
est le \wi{paragraphe}.  Le paragraphe est la forme typographique qui
contient une pens�e coh�rente ou qui d�veloppe une id�e. Vous allez
apprendre dans les pages suivantes la diff�rence entre un retour � la
ligne (obtenu avec la commande \texttt{\bs\bs}) et un changement de
paragraphe (obtenu en laissant une ligne vide dans le document
source). Une nouvelle r�flexion doit d�buter sur un nouveau
paragraphe, mais si vous poursuivez une r�flexion d�j� entam�e, un
simple retour � la ligne suffit.

En g�n�ral, on sous-estime compl�tement l'importance du d�coupage en
paragraphes. Certains ignorent m�me la signification d'un changement
de paragraphe ou bien, notamment avec \LaTeX{}, coupent des paragraphes
sans le savoir. Cette erreur est particuli�rement fr�quente lorsque
des �quations sont pr�sentes au milieu du texte. �tudiez les exemples
suivants et essayez de comprendre pourquoi des lignes vides
(changements de paragraphe) sont parfois utilis�es avant et apr�s
l'�quation et parfois non. (Si vous ne comprenez pas suffisamment les
commandes utilis�es, lisez d'abord la suite du chapitre puis revenez �
cette section.)

\begin{code}
\begin{verbatim}
% Exemple 1
\dots{} lorsqu'Einstein introduit sa formule
\begin{equation}
  e = m \cdot c^2 \; ,
\end{equation}
qui est en m�me temps la formule la plus connue et la 
moins comprise de la physique. 

% Exemple 2
\dots{} d'o� vient la loi des courants de Kirchhoff :
\begin{equation}
  \sum_{k=1}{n} I_k = 0 \; .
\end{equation}

La loi des tensions de Kirchhof s'en d�duit\dots

% Exemple 3
\dots{} qui a plusieurs avantages.

\begin{equation}
  I_D = I_F - I_R
\end{equation} 
est le c\oe{}ur d'un mod�le de transistor tr�s 
diff�rent\dots
\end{verbatim}
\end{code}

L'unit� de texte imm�diatement inf�rieure est la phrase. Dans les
documents anglo-saxons, l'espace apr�s le point terminant une phrase
est plus grande que celle qui suit un point apr�s une
abr�viation. (Ceci n'est pas vrai dans les r�gles de la typographie
fran�aise). En g�n�ral, \LaTeX{} se d�brouille pour d�terminer la
bonne largeur des espaces. S'il n'y arrive pas, vous verrez dans la
suite comment le forcer � faire quelque chose de correct.

La structure du texte s'�tend m�me aux morceaux d'une phrase. Les
r�gles grammaticales de chaque langue g�rent la ponctuation de
mani�re tr�s pr�cise. Dans la plupart des langues, la virgule
repr�sente une courte respiration dans le flux du langage. Si vous ne
savez pas trop o� placer une virgule, lisez la phrase � voix haute en
respirant � chaque virgule. Si cela ne sonne pas naturellement �
certains endroits, supprimez la virgule; au contraire, si vous
ressentez le besoin de respirer (ou de marquer une courte pause),
ins�rez un virgule � cet endroit. 

Enfin, les paragraphes d'un texte sont �galement structur�s au niveau
sup�rieur, en les regroupant en sections, chapitres, etc. L'effet
typographique d'une commande telle que 

\begin{center}
\verb|\section{La structure du texte et du langage}| 
\end{center}

\noindent est suffisament �vident pour 
comprendre comment utiliser ces structures de haut niveau.

\section{Sauts de ligne et de page}
 
\subsection{Paragraphes justifi�s}

\index{justification}
Les livres sont souvent compos�s de lignes qui ont toutes la m�me
longueur ; on dit qu'elles sont justifi�es � droite. \LaTeX{} ins�re
des retours � la ligne et des espacements entre les mots de mani�re �
optimiser la pr�sentation de l'ensemble d'un paragraphe. En cas de
besoin, il coupe les mots qui ne tiennent pas en entier sur une
ligne. La pr�sentation exacte d'un paragraphe d�pend de la classe de
document\,\footnote{et des r�gles typographiques propres de chaque pays
\emph{NdT.}}. Normalement la premi�re ligne d'un paragraphe est
en retrait par rapport � la marge gauche  
et il n'y a pas d'espace vertical particuli�re entre deux
paragraphes (cf. section~\ref{parsp}).

Dans certains cas particuliers, il peut �tre n�cessaire de demander �
\LaTeX{} de couper une ligne :
\begin{lscommand}
\ci{\bs} ou \ci{newline} 
\end{lscommand}
\noindent commence une nouvelle ligne sans commencer un nouveau
paragraphe.

\begin{lscommand}
\ci{\bs*}
\end{lscommand}
\noindent emp�che un saut de page apr�s le saut de ligne demand�.

\begin{lscommand}
\ci{newpage}
\end{lscommand}
\noindent provoque un saut de page.

\begin{lscommand}
\ci{linebreak}\verb|[|\emph{n}\verb|]|,
\ci{nolinebreak}\verb|[|\emph{n}\verb|]|, 
\ci{pagebreak}\verb|[|\emph{n}\verb|]|,
\ci{nopagebreak}\verb|[|\emph{n}\verb|]|
\end{lscommand}
\noindent sugg�rent (en anglais) o� un saut de ligne ou de page peut
appara�tre ou non. Ces commandes permettent � l'auteur de param�trer
leur action par l'interm�diaire du param�tre optionnel \emph{n}
pouvant prendre une valeur entre z�ro et quatre. En donnant � \emph{n}
une valeur inf�rieure � quatre, vous laissez � \LaTeX{} la possibilit�
de ne pas tenir compte de votre commande si cela devait rendre le
r�sultat r�ellement laid.  Ne confondez pas ces commandes \og break
\fg{} avec les commandes \og new \fg{}.  M�me lorsque vous utilisez
une commande \og break \fg{}, \LaTeX{} essaye de justifier le bord
droit du texte et d'ajuster la longueur totale de la page, comme
expliqu� plus loin ; cela peut mener � des blancs inesth�tiques dans
votre texte.  Si vous voulez r�ellement commencer une \og nouvelle
\fg{} ligne ou une \og nouvelle \fg{} page, utilisez la commande \og
new \fg{} correspondante.

\LaTeX{} essaye toujours de trouver les meilleurs endroits pour les
retours � la ligne.
S'il ne trouve pas de solution pour couper les lignes de
mani�re conforme � ses normes de qualit�, il laisse d�passer un bout
de ligne sur la marge droite du paragraphe. \LaTeX{} �met alors le
message d'erreur \og \wi{overfull hbox}\,\footnote{d�bordement
horizontal} \fg{}. Cela se produit surtout quand \LaTeX{} ne trouve pas
de point de c�sure dans un mot.\footnote{Bien que \LaTeX{} signale un
  avertissement lorsque cela arrive (Overfull hbox) et affiche la
  ligne qui pose probl�me, celle-ci n'est pas toujours facile �
  retrouver dans le texte. En utilisant l'option \texttt{draft} dans
  la commande \ci{documentclass}, ces lignes probl�matiques seront
  marqu�es d'une �paisse marque noire dans la marge de droite.}
En utilisant alors la commande \ci{sloppy},
vous pouvez demander � \LaTeX{} d'�tre moins exigeant. Il ne produira
 plus de lignes trop longues en ajoutant de l'espace entre les
mots du paragraphe, m�me si ceux-ci finissent trop espac�s selon ses
crit�res. Dans ce cas le message \og \wi{underfull hbox}\,\footnote{bo�te
horizontale pas assez pleine} \fg{} est produit. Souvent, malgr� tout, le
r�sultat est acceptable. La commande \ci{fussy} agit dans l'autre
sens, au cas o� vous voudriez voir \LaTeX{} revenir � ses exigences
normales. 

\subsection{C�sure} \label{hyph}
\index{cesure@c�sure}
\LaTeX{} coupe les mots en fin de ligne si n�cessaire. Si l'algorithme de
c�sure\footnote{\emph{\wi{Hyphenation}} en anglais} ne trouve pas
l'endroit correct pour couper un mot\,\footnote{Ce qui est normalement
plut�t rare. Si vous observez de nombreuses erreurs de c�sure, c'est
probablement un probl�me de sp�cification de la langue du document ou
du codage de sortie. Voir le paragraphe sur le support multilingue,
page~\pageref{international}.}, vous pouvez utiliser les 
commandes suivantes pour informer \TeX{} de l'exception.

La commande :
\begin{lscommand}
\ci{hyphenation}\verb|{|\emph{liste de mots}\verb|}|
\end{lscommand}
\noindent permet de ne couper les mots cit�s en argument qu'aux
endroits indiqu�s par \og \verb|-| \fg{}. Cette commande doit �tre
plac�e dans le pr�ambule et ne doit contenir que des mots compos�s de
lettres ou signes consid�r�s comme normaux par \LaTeX{}. La casse des
caract�res n'est pas prise en compte. Les informations de c�sure sont
associ�s au langage actif lors de l'invocation de la commande de
c�sure. Cela signifie que si vous placez une commande de c�sure dans
le pr�ambule, cela influencera la c�sure de l'anglais (NdT. par d�faut
les documents sont suppos�s �tre en anglais). Si vous placez la
commande apr�s \verb|\begin{document}| et que vous utilisez une
extension comme \pai{babel} pour le support d'une autre langue,
alors les suggestions de c�sure seront actives pour le langage
activ� via \pai{babel}.

L'exemple ci-dessous permet �
\og anticonstitutionnellement \fg{} d'�tre coup�, ainsi qu'� \og
Anticonstitutionnellement \fg{}. Mais il emp�che toute c�sure de \og
FORTRAN \fg{}, \og Fortran \fg{} ou \og fortran \fg{}. Ni les
caract�res sp�ciaux ni les symboles ne sont autoris�s dans cette
commande.

\begin{code}
\verb|\hyphenation{FORTRAN}|\\
\verb|\hyphenation{Anti-cons-ti-tu-tion-nel-le-ment}|
\end{code}

La commande \verb|\hyphenation{|\emph{liste de mots}\verb|}| a un effet
\emph{global} sur toutes les occurrences des mots de la liste.
Si l'exception ne concerne qu'une occurrence d'un mot on utilise
la commande \ci{-} qui ins�re un point de  c�sure potentiel dans un
mot. Ces positions deviennent alors les \emph{seuls} points de c�sure
possibles pour cette occurrence du mot. 

\begin{example}
I think this is: su\-per\-cal\-%
i\-frag\-i\-lis\-tic\-ex\-pi\-%
al\-i\-do\-cious
\end{example}

Normalement, en fran�ais, on ne coupe pas  
la derni�re syllabe d'un mot si elle est muette, mais il arrive qu'on
soit oblig� de le faire, par exemple si on travaille sur des textes
�troits (cas de colonnes multiples).

Exemple: on pourra coder \verb+ils ex\-pri\-ment+
pour autoriser \emph{exceptionnellement} le rejet � la ligne suivante de
la syllabe muette \texttt{ment}.

Plusieurs mots peuvent �tre maintenus ensemble sur une ligne avec la
commande :
\begin{lscommand}
\ci{mbox}\verb|{|\emph{texte}\verb|}|
\end{lscommand}
Elle a pour effet d'interdire toute coupure de ligne dans \emph{texte}.

\begin{example}
Mon num\'ero de t\'el\'ephone va  
changer. \`A partir du 18 octobre,
ce sera le  \mbox{0561 336 330}.

Le param\`etre 
\mbox{\emph{nom du fichier}}
de la commande \texttt{input} 
contient le nom du fichier
\`a lire.
\end{example}

\ci{fbox} est similaire � \ci{mbox}, � la diff�rence qu'un liser�
visible sera dessin� autour du contenu.

\section{Cha�nes pr�tes � l'emploi}

Dans les exemples pr�c�dents, vous avez d�couvert certaines commandes
permettant de produire le logo \LaTeX{} et quelques autres cha�nes de
caract�res sp�cifiques. Voici une liste de quelques-unes de ces
commandes :

\vspace{2ex}

\noindent
\begin{tabular}{@{}lll@{}}
Commande&R�sultat&Description\\
\hline
\ci{today} & \today   & Date du jour\\
\ci{TeX} & \TeX       & Logo TeX\\
\ci{LaTeX} & \LaTeX   & Logo LaTeX\\
\ci{LaTeXe} & \LaTeXe & Sa version actuelle\\
\end{tabular}

\section{Caract�res sp�ciaux et symboles}
 
\subsection{Guillemets}

Pour ins�rer des \wi{guillemets} n'utilisez pas le caract�re \verb|"|
\index{""@\texttt{""}} comme sur une machine � �crire. En typographie,
il y a des guillemets ouvrants et fermants sp�cifiques. En anglais,
utilisez deux~\textasciigrave pour les guillemets ouvrants et deux~\textquotesingle
pour les guillemets fermants.
En fran�ais, avec l'option \pai{francais} de l'extension \pai{babel},
utilisez \ci{og} et \ci{fg} ou bien utilisez directement 
\texttt{�} et \texttt{�} si vous disposez d'un moyen de saisir ces
caract�res. 
\begin{example}
``Please press the `x' key.''

�~Appuyez sur la touche `x'.~�
\end{example}
%SC: Hmmm, I do not like much this translation
Je suis conscient que le rendu n'est pas id�al, mais il s'agit
effectivement d'un accent grave (\textasciigrave) pour l'ouverture et
d'une quote (\textquotesingle) (i.e. pas une apostrophe au sens
typographique du terme) pour la fermeture, et ce malgr� ce que la
police choisie semble indiquer.
 
\subsection{Tirets}

\LaTeX{} conna�t quatre types de \wi{tiret}s. Trois d'entre eux sont
obtenus en juxtaposant un nombre variable de tirets simples. Le
quatri�me n'est  pas r�ellement un tiret~---~il s'agit du signe
math�matique moins. \index{-} \index{--} \index{---} \index{-@$-$}
\index{moins (signe)}

\begin{example}
belle-fille, pages 13-67\\
il parle ---~en vain~--- 
du pass�.\\
oui~---~ou non ? \\
$0$, $1$ et $-1$
\end{example}

Notez que les exemples ci-dessus respectent les r�gles de la
typographie fran�aise concernant l'usage des tirets, qui diff�rent
des habitudes anglo-saxonnes, en particulier le double tiret n'est pas
utilis� en fran�ais.

\subsection{Tilde ($\sim$)}
\index{www}\index{URL}\index{tilde}
Un caract�re souvent utilis� dans les adresses sur le web est le
tilde. Pour produire ce caract�re avec \LaTeX{}, on peut utiliser 
\verb|\~|, mais le r�sultat : \~{} n'est pas tout � fait le symbole
attendu. Essayez ceci � la place :
\cih{sim}

\begin{example}
http://www.rich.edu/\~{}bush \\
http://www.clever.edu/$\sim$demo
\end{example}  

Voir aussi l'extension \pai{hyperref} qui inclut une commande
\ci{url}. 

\subsection{Symbole degr� (\texorpdfstring{\degres}{\string�})}

L'exemple suivent montre comment obtenir un symbole \wi{degr�} avec
\LaTeX{} :

\begin{example}
Il fait $-30\,^{\circ}\mathrm{C}$.
Je vais bient\^ot devenir 
supra-conducteur.
\end{example}

L'extension \pai{babel} avec l'option \pai{francais} introduit la commande
\ci{degres} qui donne un meilleur r�sultat :
\begin{example}
Il fait -30~\degres C.
Je vais bient\^ot devenir 
supra-conducteur.
\end{example}

L'extension \pai{textcomp} met aussi � disposition le symbole degr�
via les commandes \ci{textcelsius}.

\subsection{Le symbole de l'euro \texorpdfstring{(\officialeuro)}{}}

�crire sur tout sujet �conomique de nos jours requiert l'utilisation
du symbole de l'Euro. De nombreuses polices de caract�res contiennent
un symbole Euro. Apr�s avoir charg� l'extension \pai{textcomp} dans le
pr�ambule
\begin{lscommand}
\ci{usepackage}\verb|{textcomp}| 
\end{lscommand}
vous pouvez utiliser la commande
\begin{lscommand}
\ci{texteuro}
\end{lscommand}
pour y acc�der.

Si votre police ne fournit pas son propre symbole de l'Euro ou si vous
ne l'aimez pas, il vous reste deux alternatives :

Tout d'abord l'extension \pai{eurosym} qui fournit un symbol officiel
de l'Euro :
\begin{lscommand}
\ci{usepackage}\verb|[|\emph{official}\verb|]{eurosym}|
\end{lscommand}
Si vous pr�f�rez un symbole qui correspond � votre police, utilisez
plut�t \texttt{gen} � la place de \texttt{option}.

% Si les polices Adobe Eurofonts sont install�es sur votre syst�me (vous
% pouvez les obtenir gratuitement sur
% \url{ftp://ftp.adobe.com/pub/adobe/type/win/all}), vous pouvez
% utiliser l'extension \pai{europs} et la commande \ci{EUR} (pour
% un symbole de l'Euro qui correspond � la police courante).
% ne fonctionne pas
% soit
% l'extension \pai{eurosans} et la commande \ci{euro} (pour l'\og Euro
% officiel \fg{}).

% L'extension \pai{marvosym} fournit �galement des symboles vari�s, y
% compris celui de l'Euro, sous le nom \ci{EURtm}. Son d�faut est de ne
% pas proposer des versions italiques et grasses de ce symbole.

\begin{table}[!htbp]
\caption{Un sac plein d'Euros} \label{eurosymb}
\begin{lined}{10cm}
\begin{tabular}{llccc}
LM+textcomp  &\verb+\texteuro+ & \huge\texteuro &\huge\sffamily\texteuro
                                                &\huge\ttfamily\texteuro\\
eurosym      &\verb+\euro+ & \huge\officialeuro &\huge\sffamily\officialeuro
                                                &\huge\ttfamily\officialeuro\\
$[$gen$]$eurosym &\verb+\euro+ & \huge\geneuro  &\huge\sffamily\geneuro
                                                &\huge\ttfamily\geneuro\\
%europs       &\verb+\EUR + & \huge\EURtm        &\huge\EURhv
%                                                &\huge\EURcr\\
% eurosans     &\verb+\euro+ & \huge\EUROSANS  &\huge\sffamily\EUROSANS
%                                             & \huge\ttfamily\EUROSANS \\
% marvosym     &\verb+\EURtm+  & \huge\mvchr101  &\huge\mvchr101
%                                                &\huge\mvchr101
\end{tabular}
\medskip
\end{lined}
\end{table}

\subsection{Points de suspension (\texorpdfstring{\dots}{...})}

Sur une machine � �crire, une \wi{virgule} ou un \wi{point} occupent la
m�me largeur que les autres lettres. En typographie professionnelle,
le point occupe tr�s peu de place et il est plac� tout pr�s du caract�re
qui le pr�c�de. Il n'est donc pas possible d'utiliser trois points
cons�cutifs pour cr�er des \wi{points de suspension}. � la place on
utilise la commande sp�cifique : 
\begin{lscommand}
\ci{dots}
\end{lscommand}
\index{...@\dots}
\nonfrenchspacing
\begin{example}
Non pas comme \c{c}a... 
mais ainsi :\\
New York, Tokyo, Budapest\dots
\end{example}
\frenchspacing

\subsection{Ligatures}

Certaines s�quences de lettres ne sont pas compos�es simplement en
juxtaposant les diff�rentes lettres les unes � la suite des autres,
mais en utilisant des symboles sp�ciaux.
\begin{code}
{\large ff fi fl ffi\dots}\quad
\`a la place de\quad {\large f{}f f{}i f{}l f{}f{}i\dots}
\end{code}
Ces \wi{ligature}s peuvent �tre d�sactiv�es en ins�rant un
\ci{mbox}\verb|{}| entre les lettres en question. Cela peut s'av�rer
utile pour certains mots compos�s\,\footnote{Il n'existe pas d'exemple en
fran�ais. \emph{NdT.}}.
\begin{example}
\Large Not shelfful\\
but shelf\mbox{}ful
\end{example}
 
\subsection{Accents et caract�res sp�ciaux}

\LaTeX{} permet l'utilisation d'\wi{accent}s et de \wi{caract�res
sp�ciaux} issus de nombreuses langues. Le tableau~\ref{accents} montre
tous les accents que l'on peut ajouter � la lettre o. Ils s'appliquent
naturellement aux autres lettres.

Pour placer un accent sur un i ou un j, il faut supprimer leur
point. Ceci s'obtient en tapant \verb|\i| et \verb|\j|.

\begin{example}
H\^otel, na\"\i ve, \'el\`eve,\\ 
sm\o rrebr\o d, !`Se\~norita!,\\
Sch\"onbrunner Schlo\ss{} 
Stra\ss e
\end{example}

\begin{table}[!hbp]
\caption{Accents et caract�res sp�ciaux} \label{accents}
\begin{lined}{10cm}
\begin{tabular}{*4{cl}}
\A{\`o} & \A{\'o} & \A{\^o} & \A{\~o} \\
\A{\=o} & \A{\.o} & \A{\"o} & \B{\c}{c}\\[6pt]
\B{\u}{o} & \B{\v}{o} & \B{\H}{o} & \B{\c}{o} \\
\B{\d}{o} & \B{\b}{o} & \B{\t}{oo} \\[6pt]
\A{\oe}  &  \A{\OE} & \A{\ae} & \A{\AE} \\
\A{\aa} &  \A{\AA} \\[6pt]
\A{\o}  & \A{\O} & \A{\l} & \A{\L} \\
\A{\i}  & \A{\j} & !` & \verb|!|\verb|`| & ?` & \verb|?|\verb|`| 
\end{tabular}
\index{i et j@\i{} et \j{} sans points}\index{scandinaves (caract�res)}
\index{ae@\ae}\index{umlaut}\index{accent!grave}\index{accent!aigu}
\index{accent!circonflexe}
\index{oe@\oe}\index{aa@\aa}
\index{c�dille}

\bigskip
\end{lined}
\end{table}
 
\section{Support multilingue\label{international}}
\index{international} 

Pour composer des documents dans des langues autres que l'anglais,
\LaTeX{} doit pouvoir s'adapter aux r�gles typographiques et aux
r�gles de c�sure propres � chaque langue. Il y a plusieurs
domaines pour lesquels il faut configurer \LaTeX{} pour chaque
langue : 
\begin{enumerate}
\item Toutes les cha�nes de caract�res g�n�r�es automatiquement
  \footnote{\og Table des mati�res \fg{}, \og Liste des figures \fg{},...} 
  doivent �tre traduites. Pour de nombreuses langues, ces adaptations
  peuvent �tre r�alis�es par l'extension \pai{babel} de Johannes
  Braams.
\item \LaTeX{} doit conna�tre les r�gles de c�sure de la nouvelle
      langue. D�finir les r�gles de c�sure utilis�es par \LaTeX{} est
      une t�che assez complexe, qui impose la construction de formats
      sp�cifiques. Votre \guide{} devrait vous indiquer 
      quelles sont les langues support�es par votre installation de
      \LaTeX{} et comment configurer les r�gles de c�sure.
    \item Certaines r�gles typographiques changent en fonction de la
      langue ou de la r�gion g�ographique. Par exemple le fran�ais
      impose un espace avant chaque deux-points (:). Ces changements
      peuvent �tre pris en compte par l'extension \pai{babel} ou par des
      extensions sp�cifiques (telles que \pai{french}~\cite{french}
      pour le fran�ais).
\end{enumerate}

Si votre syst�me est configur� correctement, vous pouvez s�lectionner
la langue utilis�e par l'extension \pai{babel} 
avec la commande : 
\begin{lscommand}
\ci{usepackage}\verb|[|\emph{langue}\verb|]{babel}|
\end{lscommand}
\noindent apr�s la commande \verb|\documentclass|. Une liste des
\emph{langue}s prises en compte par votre syst�me \LaTeX{} est
affich�e � chaque ex�cution du compilateur. Babel activera les r�gles
de c�sure appropri�es pour le langage que vous avez choisi.
Lorsque la c�sure d'une langue n'est pas prise en compte par votre moteur
\LaTeX{}, \pai{babel} continuera � fonctionner mais � cause des
c�sures incorrectes le r�sultat ne sera pas satisfaisant.

\textsf{Babel} d�finit �galement pour certaines langues de nouvelles
commandes qui simplifient la saisie des caract�res
sp�ciaux. L'\wi{Allemand} contient par exemple de nombreux umlauts
(\"a\"o\"u). Avec\textsf{babel}, vous pouvez entrer un
\"o en tapant \verb|"o| au lieu de~\verb|\"o|.

Si vous appelez babel avec plusieurs langues,
\begin{lscommand}
\ci{usepackage}\verb|[|\emph{langueA}\verb|,|\emph{langueB}\verb|]{babel}| 
\end{lscommand}
\noindent alors la derni�re langue dans la liste d'options sera active
(i.e. langueB). Vous pouvez utiliser la commande
\begin{lscommand}
\ci{selectlanguage}\verb|{|\emph{langueA}\verb|}|
\end{lscommand}
\noindent pour changer la langue active.

%Input Encoding
\newcommand{\ieih}[1]{%
\index{encodings!input!#1@\texttt{#1}}%
\index{input encodings!#1@\texttt{#1}}%
\index{#1@\texttt{#1}}}
\newcommand{\iei}[1]{%
\ieih{#1}\texttt{#1}}
%Font Encoding
\newcommand{\feih}[1]{%
\index{encodings!font!#1@\texttt{#1}}%
\index{font encodings!#1@\texttt{#1}}%
\index{#1@\texttt{#1}}}
\newcommand{\fei}[1]{%
\feih{#1}\texttt{#1}}

La plupart des syst�mes informatiques modernes vous permettent de
saisir directement depuis le clavier les caract�res accentu�s ou les
symboles sp�cifiques de l'alphabet d'une langue. \LaTeXe{} g�re la
vari�t� des codages d'entr�e des diff�rents langages et plateformes
gr�ce � l'extension \pai{inputenc} :
\label{inputenc}
\begin{lscommand}
\ci{usepackage}\verb|[|\emph{codage}\verb|]{inputenc}|
\end{lscommand}

%SC: is this still true ?
Pour faciliter la lecture, cette extension est utilis�e dans la suite
du document pour repr�senter les caract�res accentu�s dans les
exemples.

En utilisant cette extension, il faut prendre garde au fait que les
syst�mes informatiques n'utilisent pas tous le m�me codage des
caract�res sp�ciaux. Par exemple, l'umlaut allemand \"a est cod� 132
sous OS/2, sous les syst�mes Unix utilisant ISO-LATIN~1 il est cod�
228 et sous le codage cyrillique cp1251 pour Windows cette lettre
n'existe m�me pas; c'est pourquoi vous devez exploiter cette
fonctionnalit� avec soin. Les codages suivants peuvent s'av�rer utiles
en fonction du type de syst�me sous lequel vous travaillez.
\footnote{Pour en savoir plus sur les codages d'entr�e reconnus
  pour les alphabets latins et cyrilliques, lisez la
  documentation de \texttt{inputenc.dtx} et \texttt{cyinpenc.dtx},
  respectivement. La section~\ref{sec:Packages} indique comment
  produire cette documentation.}

\begin{center}
\begin{tabular}{l | r | r }
Syst�me & \multicolumn{2}{c}{encodings}\\
d'exploitation & Alphabet latin & Alphabet cyrillique\\
 \hline
Mac     &  \iei{applemac} & \iei{macukr}  \\
Unix    &  \iei{latin1}   & \iei{koi8-ru}  \\ 
Windows &  \iei{ansinew}  & \iei{cp1251}    \\
DOS, OS/2  &  \iei{cp850} & \iei{cp866nav}
\end{tabular}
\end{center}

Si vous avez un document multilingue qui cause des conflits entre
codages d'entr�e, vous devriez consid�rer l'utilisation d'unicode �
l'aide de l'extension \pai{ucs}.

\begin{lscommand}
\ci{usepackage}\verb|{ucs}|\\ 
\ci{usepackage}\verb|[|\iei{utf8x}\verb|]{inputenc}| 
\end{lscommand}
\noindent vous permettra de cr�er des fichiers d'entr�e \LaTeX{} en
\iei{utf8x}, un codage multi-octets o� chaque caract�re peut n�cessiter
entre un et quatre octets pour �tre cod�.
\footnote{NdT. D'anciennes versions de ce document pr�conisent la
  conversion des caract�res accentu�s et sp�ciaux en s�quences du type
  {\bs' e} pour l'�change de documents \LaTeX{}. Cette
  pratique tend � �tre de moins en moins n�cessaire gr�ce �
  Unicode. Il existe cependant des utilitaires pour faire la
  conversion automatique dans les deux sens  : \texttt{recode} sous Unix,
\texttt{Tower of Babel} sur Macintosh, etc. y compris de et vers
Unicode.}

Le codage des polices de caract�res est une autre histoire.
Celui-ci d�finit � quelle position dans une police de \TeX{} se
trouve chaque caract�re. Des codages multiples pourraient �tre mis en
correspondance avec un codage unique, ce qui r�duit le nombre
d'ensembles de polices requis. Les codages de polices sont manipul�s
via l'extension \pai{fontenc}: \label{fontenc}
\begin{lscommand}
\ci{usepackage}\verb|[|\emph{codage}\verb|]{fontenc}| \index{codages des polices}
\end{lscommand}
\noindent o� \emph{codage} est le codage de la police. Il est possible
d'en charger plusieurs simultan�ment.

Le codage de police par d�faut de \LaTeX{} est \label{OT1} \fei{OT1},
celui de la police \TeX{} Computer Modern originelle. Celui-ci
contient seulement les 128 caract�res de l'ensemble ASCII sur 7
bits. Au besoin, \TeX{} fabrique des caract�res accentu�s en combinant
un caract�re normal avec un accent. Bien que le r�sultat soit
visuellement parfait, cette approche emp�che le syst�me de c�sure de
fonctionner pour les mots contenant des caract�res accentu�s. De plus,
certaines lettres latines ne peuvent �tre fabriqu�es de cette fa�on,
sans parler des alphabets non-latins, comme le grec ou le cyrillique.

Pour surpasser ces limitations, plusieurs ensembles de polices 8 bits
proches de CM furent propos�es. Les polices \emph{Extended Cork} (EC)
dans le codage \fei{T1} contiennent les lettres et caract�res de
ponctuation pour la plupart des langues europ�ennes bas�es sur un
alphabet latin. L'ensemble de polices LH contient les lettres pour
formatter les langages � l'alphabet cyrillique. � cause du nombre
imposant de glyphes cyrilliques, ils sont organis�s en quatre codages
de police : \fei{T2A}, \fei{T2B}, \fei{T2C} et \fei{X2}.
\footnote{La liste des langages pris en compte par chacun de ces
  codages peut �tre consult�e dans \cite{cyrguide}.}
Le paquet CB contient des polices dans le codage \fei{LGR} pour la
composition de textes en grec.

L'usage de ces polices activera ou am�liorera les c�sures dans les
documents non-anglophones. L'autre avantage des polices proches de CM
est qu'elles fournissent les polices des familles CM dans tous les
poids, formes et tailles optiquement �chelonnables de polices.
%SC: I hate the translation of scalable

\subsection{Support de la langue portugaise}

\secby{Demerson Andre Polli}{polli@linux.ime.usp.br}
Pour activer la c�sure et faire passer tous les textes automatis�s en
portugais\index{Portugu\^es} utilisez la commande :
\begin{lscommand}
\verb|\usepackage[portuguese]{babel}|
\end{lscommand}
Si vous �tes au Br�sil, remplacez le langage par \texttt{\wi{brazilian}}.

Au vu du nombre cons�quent d'accents en portugais, vous devriez
utiliser
\begin{lscommand}
\verb|\usepackage[latin1]{inputenc}|
\end{lscommand}
pour les saisir facilement ainsi que
\begin{lscommand}
\verb|\usepackage[T1]{fontenc}|
\end{lscommand}
pour obtenir des c�sures correctes.

Voyez le tableau~\ref{portuguese} qui r�capitule ce
qu'il y a � ajouter en pr�ambule pour �crire en langue
portugaise. Remarquez que nous indiquons ici le codage latin1 : cela
ne fonctionnera donc pas comme attendu sous Mac ou DOS. Utilisez
simplement le codage ad�quat.

\begin{table}[btp]
\caption{Pr�ambule pour les documents portugais.} \label{portuguese}
\begin{lined}{5cm}
\begin{verbatim}
\usepackage[portuguese]{babel}
\usepackage[latin1]{inputenc}
\usepackage[T1]{fontenc}
\end{verbatim}
\bigskip
\end{lined}
\end{table}


\subsection{Support de la langue fran�aise} 

% Pour le fran�ais, l'extension \pai{french} a �t� 
% d�velopp�e par Bernard Gaulle~\cite{french}. 
% Mais le traducteur boycotte les versions r�centes (frenchpro) 

\secby{Daniel Flipo}{daniel.flipo@univ-lille1.fr}
Voici quelques conseils pour cr�er des documents en
\wi{fran�ais}\index{fran�ais} � l'aide de \LaTeX{}. Le support de la
langue fran�aise est activ� par la commande suivante :

\begin{lscommand}
\verb|\usepackage[francais]{babel}|
\end{lscommand}

Remarquez que, pour des raisons historiques, le nom de l'option
\textsf{babel} pour le fran�ais est soit \emph{francais}, soit \emph{frenchb}
mais pas \emph{french}.

Cette commande active les r�gles de c�sure sp�cifiques du fran�ais et
adaptent \LaTeX{} � la plupart des r�gles sp�cifiques de la
typographie fran�aise~\cite{ftypo} : pr�sentation des listes,
insertion automatique de l'espacement avant les signes de ponctuation
doubles, etc.  Les mots g�n�r�s automatiquement par \LaTeX{} sont
traduits en fran�ais et certaines commandes suppl�mentaires (cf.
table~\ref{cmd-french}) sont disponibles.

\begin{table}[!htbp]
\caption{Commandes de saisie en fran�ais.} \label{cmd-french}
\begin{lined}{9cm}
\begin{tabular}{ll}
\verb+\og guillemets \fg{}+         \quad &\og guillemets \fg \\[1ex]
\verb+M\up{me}, D\up{r}+            \quad &M\up{me}, D\up{r}  \\[1ex]
\verb+1\ier{}, 1\iere{}, 1\ieres{}+ \quad &1\ier{}, 1\iere{}, 1\ieres{}\\[1ex]
\verb+2\ieme{} 4\iemes{}+           \quad &2\ieme{} 4\iemes{}\\[1ex]
\verb+\No 1, \no 2+                 \quad &\No 1, \no 2   \\[1ex]
\verb+20~\degres C, 45\degres+      \quad &20~\degres C, 45\degres \\[1ex]
\verb+\bsc{M. Durand}+              \quad &\bsc{M.~Durand} \\[1ex]
\verb+\nombre{1234,56789}+          \quad &\nombre{1234,56789}
\end{tabular}
\bigskip
\end{lined}
\end{table}


Vous remarquerez �galement que la mise en page des listes est chang�e
lors du passage � la langue fran�aise. Pour obtenir toutes les
informations sur l'option \pai{frenchb} de \pai{babel} et comment
modifier son comportement, compilez le fichier \texttt{frenchb.dtx}
avec \LaTeX{} et consultez le fichier \texttt{frenchb.dvi} ainsi
produit.

Dans cette traduction, un certain nombre d'ajouts pr�sentent les
sp�cificit�s de la typographie fran�aise tout au long du texte.

\subsection{Support de la langue allemande}

Voici quelques conseils pour cr�er des documents en
\wi{allemand}\index{deutsch} � l'aide de \LaTeX{}. Le support de la
langue allemande est activ� par la commande suivante :

\begin{lscommand}
\verb|\usepackage[german]{babel}|
\end{lscommand}

La c�sure allemande est alors activ�e, si votre syst�me a �t�
configur� pour cela. Le texte produit automatiquement par \LaTeX{} est
traduit en allemand (par ex. ``Kapitel'' pour un chapitre). De
nouvelles commandes (list�es dans la table~\ref{german}) permettent la
saisie simplifi�e des caract�res sp�ciaux m�me sans utiliser
l'extension inputenc. Avec inputenc cette capacit� devient un peu vaine
mais votre texte est alors limit� au codage utilis�.

\begin{table}[!htbp]
\caption{Caract�res sp�ciaux en allemand.} \label{german}
\begin{lined}{8cm}
\begin{tabular}{*2{ll}}
\verb|"a| & "a \hspace*{1ex} & \verb|"s| & "s \\[1ex]
\verb|"`| & "` & \verb|"'| & "' \\[1ex]
\verb|"<| or \ci{flqq} & "<  & \verb|">| or \ci{frqq} & "> \\[1ex]
\ci{flq} & \flq & \ci{frq} & \frq \\[1ex]
\ci{dq} & " \\
\end{tabular}
\bigskip
\end{lined}
\end{table}

Les livres allemands contiennent souvent des marques de citation
fran�aises (\flqq guil\-le\-mets\frqq). Les typographes allemands les
utilisent diff�remment, cependant. Une citation dans un livre allemand
ressemblerait plut�t � \frqq ceci\flqq. En Suisse allemande, les
typographes utilisent les \flqq guillemets\frqq~comme les fran�ais le font.

Un probl�me majeur d�coule de l'utilisation de commandes comme
\verb+flq+ : si vous utilisez la police OT1 (la police par d�faut) les
guillements ressembleront au symbole math�matique \og $\ll$ \fg{}, de
quoi causer des maux d'estomac � un typographe. Les polices cod�es T1,
d'un autre c�t�, ne contient pas les symboles requis. Ainsi si vous
utilisez ce type de marque de citation, assurez-vous d'utiliser le
codage T1 (\verb|\usepackage[T1]{fontenc}|).

\subsection{Support de la langue cor�enne\protect\footnotemark}
\label{support_korean}%
\footnotetext{%
En r�ponse aux probl�mes usuellement rencontr�s par les utilisateurs
cor�ens de \LaTeX{}.
Cette section fut �crite par Karnes KIM au nom de l'�quipe de
traduction cor�enne de lshort. Elle fut ensuite traduite en anglais
par SHIN Jungshik, raccourcie par Tobi Oetiker puis traduite en
fran�ais par Samuel Colin.}

Pour utiliser \LaTeX{} avec la langue cor�enne, nous devons r�soudre
trois probl�mes :

\begin{enumerate}
\item Nous devons �tre capables d'�diter des \wi{fichiers d'entr�e en
    cor�en}. Ces fichiers doivent �tre au format text brut, mais
  puisque le cor�en dispose d'un ensemble de caract�res situ� en
  dehors du r�pertoire US-ASCII, ils appara�tront �tranges dans tout
  �diteur ASCII usuel. Les codages cor�ens les plus r�pandus sont
  EUC-KR et son extension (avec compatibilit� ascendante) utilis�e
  dans la version cor�enne de Windows, CP949/Windows-949/UHC. Dans ces
  codages, chaque caract�re US-ASCII repr�sente son caract�re usuel,
  comme dans d'autres codages compatibles avec US-ASCII
  (ISO-8859-\textit{x}, EUC-JP, Big5 ou Shift\_JIS). En revanche
  les caract�res hangul, hanjas (des caract�res chinois utilis�es en
  Cor�e), hangul jamos, hirakanas, katakanas, grecs et cyrilliques,
  les symboles et les lettres d�riv�e de KS~X~1001 sont repr�sent�s
  par deux octets cons�cutifs. Le premier poss�de son propre ensemble
  MSB. Jusqu'au milieu des ann�es 1990, il fallait beaucoup de temps
  et d'effort pour installer un environnement capable de g�rer la
  langue cor�enne sous un syst�me non localis�. Vous pouvez lire en
  diagonale la ressource d�sormais d�suette \url{http://jshin.net/faq}
  pour vous faire une id�e de la difficult� d'utiliser du cor�en sous
  un syst�me non-cor�en dans les ann�es 1990. De nos jours les trois
  syst�mes majeurs (Mac OS, Unix, Windows) sont �quip�s d'un bon
  support multilingue et de fonctionnalit�s d'internationalisation, de
  mani�re qu'�diter un fichier texte cor�en n'est plus autant un
  probl�me, m�me sur un syst�me non-cor�en.

\item \TeX{} and \LaTeX{} ont �t� d�velopp�s � l'origine pour des
  documents n'ayant pas plus de 256 caract�res dans leur
  alphabet. Pour les faire fonctionner avec des langages ayant
  consid�rablement plus de caract�res comme le cor�en
\footnote{Le hangul cor�en est un script alphab�tique avec 14
  consonnes de base et 10 voyelles (jamos) de base. � la diff�rence
  des alphabets latin ou cyrillique, les caract�res individuels
  doivent �tre arrang�s sous forme de groupes rectangulaires de la
  taille de caract�res chinois. Chaque groupe repr�sente une
  syllabe. Un nombre illimit� de syllabes peut �tre fabriqu� avec cet
  ensemble fini de consonnes et de voyelles. Les standards
  orthographiques cor�ens modernes (tant en Cor�e du Nord qu'en Cor�e
  du Sud) imposent cependant des restrictions sur la formation de ces
  groupes. C'est pourquoi il existe seulement un nombre fini de
  syllabes syntaxiquement correctes. Le codage cor�en d�finit des
  points de code individuels pour chacune des syllabes (KS~X~1001:1998
  et KS~X~1002:1992). Ainsi l'hangul, bien qu'alphab�tique, est trait�
  comme les syst�mes d'�criture chinois et japonais avec des dizaines
  de milliers de caract�res
  id�ographiques/logographiques. l'ISO~10646/Unicode propose les deux
  fa�ons de repr�senter l'hangul utilis� pour le cor�en \emph{moderne}
  via l'hangul jamos conjoignant (alphabets:
  \url{http://www.unicode.org/charts/PDF/U1100.pdf}) en plus de toutes
  les syllabes hangul orthographiquement correctes en cor�en
  \emph{moderne}
  (\url{http://www.unicode.org/charts/PDF/UAC00.pdf}). L'un des plus
  �minents d�fis en typographie cor�enne avec \LaTeX{} et autres
  syst�mes simiaires est la prise en compte de syllabes du cor�en
  m�dian~---~et futur~---~via la conjonction jamos en
  Unicode. L'espoir r�side dans des moteurs \TeX{} futurs tels que
  $\Omega$ et $\Lambda$ pour proposer des solutions de mani�re �
  �loigner les linguistes et historiens cor�ens de MS Word, qui
  poss�de un support relativement bon du cor�en m�dian.
%SC: many technical things, I most probably got some wrong (Middle
%Korean?)
}
ou le chinois, un m�canisme de \og sous-police \fg{} a �t�
d�velopp�. Il divise une police CJK avec des (dizaines de) milliers de
glyphes en un ensemble de sous-polices ayant chacune 256 glyphes. Pour
le cor�en, il y a d�j� 3 extensions couramment utilis�es : \wi{H\LaTeX}
par UN~Koaunghi, \wi{h\LaTeX{}p} par CHA~Jaechoon et l'extension \wi{CJK}
par Werner~Lemberg.\footnote{%
  Elles peuvent �tre obtenues depuis
  \CTANref|language/korean/HLaTeX/|,
  \CTANref|language/korean/CJK/| et
  \texttt{http://knot.kaist.ac.kr/htex/}
}
  H\LaTeX{} et h\LaTeX{}p sont sp�cifique au cor�en et fournissent une
  localisation cor�enne au-del� du support de police. Elles peuvent
  toutes deux compiler des fichiers textes cod�s en EUC-KR. H\LaTeX{}
  peut m�me compiler des fichiers d'entr�e cod�s en
  CP949/Windows-949/UHC et UTF-8 si elle est utilis�e avec $\Lambda$ ou
  $\Omega$.

  L'extension CJK n'est pas sp�cifique au cor�en. Elle peut compiler des
  fichiers d'entr�e en UTF-8 ainsi que diff�rents codages CJK comprenant
  EUC-KR et CP949/Windows-949/UHC. Elle peut �tre utilis�e pour saisir
  des documents avec support multilingue (en particulier chinois,
  japonais et cor�en). L'extension CJK n'a pas de localisation cor�enne
  comme celle offerte par H\LaTeX{} et n'est pas livr�e avec autant de
  polices sp�ciales cor�ennes qu'H\LaTeX{}.

\item L'id�e principale derri�re l'utilisation d'un syst�me de
  traitement comme \TeX{} et \LaTeX{} est d'obtenir des documents
  typographi�s d'une mani�re \og esth�tiquement \fg{}
  satisfaisante. En cons�quence l'�l�ment le plus important en
  typographie est un ensemble de polices bien con�ues. La distribution
  H\LaTeX{} inclut les polices \PSi{} UHC
  \index{Police cor�enne!police UHC} de 10 familles diff�rentes et des
  polices TrueType Munhwabu\footnote{Minist�re cor�en de la culture.}
  de 5 diff�rentes familles. L'extension CJK fonctionne avec un
  ensemble de polices utilis� par des versions ant�rieures d'H\LaTeX{}
  et elle peut utiliser la police TrueType Bitstream cyberbit.
\end{enumerate}

Pour utiliser l'extension H\LaTeX{} pour votre document cor�en,
ins�rer la d�claration suivante en pr�ambule :
\begin{lscommand}
\verb+\usepackage{hangul}+
\end{lscommand}

Cette commande active la localisation cor�enne. Les textes
automatiques (chapitres, sections,\dots) sont traduits en cor�en et le
format du document est modifi� afin de suivre les conventions
cor�ennes. L' extension propose �galement une \og s�lection de
particules \fg{} automatique. En cor�en, il y a des paires de
particules postfix�es �quivalentes grammaticalement mais diff�rentes
d'apparence. La correction d'une paire d�pend du fait que la syllabe
qui pr�c�de est une consonne ou une voyelle (c'est en fait plus
complexe, mais donne une id�e g�n�rale). Les cor�ens natifs n'ont
aucun probl�me pour choisir la bonne particule, mais il n'est pas
possible de d�terminer quelle particule utiliser pour les r�f�rences
et pour les textes automatiques qui changeront au fur et � mesure de
l'�dition du document. Placer les particules appropri�es manuellement
requiert un douloureux effort � chaque ajout/retrait de r�f�rence ou �
chaque r�ordonnancement de parties du document.
H\LaTeX{} soulage ses utilisateurs de ce processus ennuyeux et risqu�.

Au cas o� vous n'avez pas besoin de fonctionnalit�s de localisation
mais souhaitez simplement saisir du texte cor�en, ins�rez plut�t la
ligne suivante en pr�ambule :
\begin{lscommand}
\verb+\usepackage{hfont}+
\end{lscommand}

Pour plus de d�tails sur la typographie cor�enne avec H\LaTeX{},
r�f�rez-vous au \emph{H\LaTeX{} Guide}. Consultez �galement le site
web des utilisateurs cor�ens de \TeX{} (KTUG) sur
\url{http://www.ktug.or.kr/}. Il existe aussi une traduction cor�enne
de ce manuel.

\subsection{Support du grec}
\secby{Nikolaos Pothitos}{pothitos@di.uoa.gr}
Les commandes � ins�rer en pr�ambule pour �crire en \wi{grec}
\index{grec} se trouvent dans le tableau~\ref{preamble-greek}. Ce
pr�ambule active la c�sure et change le texte automatique en grec.%
\,\footnote{Si vous ajoutez l'option \texttt{utf8x} �
  \texttt{inputenc}, vous pourrez saisir du grec et des caract�res
  unicodes grecs polytoniques.}

\begin{table}[btp]
\caption{Pr�ambule pour les documents grecs.} \label{preamble-greek}
\begin{lined}{7cm}
\begin{verbatim}
\usepackage[english,greek]{babel}
\usepackage[iso-8859-7]{inputenc}
\end{verbatim}
\bigskip
\end{lined}
\end{table}

De nouvelles commandes pour une saisie simplifi�e du grec devient
aussi disponibles. Pour passer temporairement en alphabet latin et
vice-versa, vous pouvez utiliser les commandes
\verb|\textlatin{|\emph{texte latin}\verb|}| and
\verb|\textgreek{|\emph{texte grec}\verb|}|. Elles prennent un
argument qui sera format� avec la police la plus pertinente.
Sinon vous pouvez aussi utiliser la commande
\verb|\selectlanguage{...}| pr�sent�e pr�c�demment. Le
tableau~\ref{sym-greek} pr�sente quelques caract�res de ponctuation
grecs. Vous pouvez utiliser \verb|\euro| pour obtenir le symbole de
l'Euro.

\begin{table}[!htbp]
\caption{Caract�res sp�ciaux grecs.} \label{sym-greek}
\begin{lined}{4cm}
%SC: Nope, already in french :-P
%\selectlanguage{french}
\begin{tabular}{*2{ll}}
\verb|;| \hspace*{1ex}  &  $\cdot$ \hspace*{1ex}  &  \verb|?| \hspace*{1ex}&  ;   \\[1ex]
\verb|((|               &  \og                    &  \verb|))|&  \fg \\[1ex]
\verb|``|               &  `                      &  \verb|''| &  '   \\
\end{tabular}
%\selectlanguage{english}
\bigskip
\end{lined}
\end{table}


\subsection{Support du cyrillique}

\secby{Maksym Polyakov}{polyama@myrealbox.com}
La version~3.7h de \pai{babel} comprend un support pour les codages
\fei{T2*} et pour le formatage des textes bulgares, russes et
ukrainiens � base de letters cyrilliques.

Le support du cyrillique se base sur des m�canismes \LaTeX{} standards
via les extensions \pai{fontenc} et \pai{inputenc}. Si cependant vous
voulez utiliser du cyrillique en mode math�matique, vous devrez
charger \pai{mathtext} avant \pai{fontenc} :
\footnote{
Si vous utilisez les extensions \AmS-\LaTeX{}, chargez-les aussi avant
\pai{fontenc} et \pai{babel}.
}
\begin{lscommand}
\verb+\usepackage{mathtext}+\\
\verb+\usepackage[+\fei{T1}\verb+,+\fei{T2A}\verb+]{fontenc}+\\
\verb+\usepackage[+\iei{koi8-ru}\verb+]{inputenc}+\\
\verb+\usepackage[english,bulgarian,russian,ukranian]{babel}+
\end{lscommand}

En g�n�ral, \pai{babel} choisira lui-m�me le codage de police par
d�faut : pour les trois langues pr�cit�es il s'agit de \fei{T2A}. Les
documents ne sont cependant pas limit�s � un seule codage de
police. Pour les documents multilingues avec des alphabets cyrillique
et latin, il est raisonnable d'inclure les codages de polices latines
explicitement. \pai{babel} prendra en charge le changement vers un
codage de fonte appropri� lorsqu'un autre langage est s�lectionn� dans
le document.

En plus d'activer les c�sures, de traduire les textes automatiques et
d'activer certaines r�gles typographiques (comme \ci{frenchspacing}),
\pai{babel} fournit des commandes additionnelles pour permettre un
formatage conforme aux conventions bulgare, russe ou ukrainienne.


Pour ces trois langues, une ponctuation sp�ciale est fournie. Le
tiret cyrillique pour le texte (plus �troit que le tiret latin et
entour� d'espaces fines), un tiret pour le dialogue, des marques de
citation et des commandes pour faciliter les c�sures, voyez le
tableau~\ref{Cyrillic}.


% Table borrowed from Ukrainian.dtx
%SC: please, some french specialist of cyrillic typography could help ?
\begin{table}[htb]
  \begin{center}
  \index{""-@\texttt{""}\texttt{-}} 
  \index{""---@\texttt{""}\texttt{-}\texttt{-}\texttt{-}} 
  \index{""=@\texttt{""}\texttt{=}} 
  \index{""`@\texttt{""}\texttt{`}} 
  \index{""'@\texttt{""}\texttt{'}} 
  \index{"">@\texttt{""}\texttt{>}} 
  \index{""<@\texttt{""}\texttt{<}} 
  \caption[Bulgare, russe et ukrainien]{Les d�finitions additionnelles
    de \pai{babel} pour le bulgare, le russe et l'ukrainien}\label{Cyrillic}
  \begin{tabular}{@{}p{.1\hsize}@{}p{.9\hsize}@{}}
   \hline
   \verb="|= & D�sactiver la ligature ici.               \\
   \verb|"-| & Un tiret explicit autorisant la c�sure dans
               le reste du mot.                             \\
   \verb|"---| & Tiret long cyrillique pour le texte.        \\
   \verb|"--~| & Tiret long cyrillique pour les noms compos�s.       \\
   \verb|"--*| & Tiret long cyrillique pour le dialogue.         \\
   \verb|""| & comme \verb|"-|, mais n'affiche pas le tiret
               (pour les mots compos�s avec tiret, p.e. \verb|x-""y|
                ou d'autres signes comme \og disable/enable \fg{}). \\
   \verb|"~| & pour une marque de mot compos� sans rupture.        \\
   \verb|"=| & pour une marque de mot compos� avec rupture, pour
               autoriser les c�sures dans les mots le composant. \\
   \verb|",| & espace fine pour les initiales avec rupture possible
               pour le nom qui suit.                            \\
   \verb|"`| & pour les marques doubles allemandes de citation
               (ressemble � ,\kern-0.08em,).                     \\
   \verb|"'| & Pour les marques doubles allemandes de citation,
               � droite (ressemble � ``).                        \\%''
   \verb|"<| & Pour les guillemets � gauche (comme $<\!\!<$).  \\
   \verb|">| & Pour les guillemets � droite (comme $>\!\!>$). \\
   \hline
  \end{tabular}
  \end{center}
\end{table}


Les options \pai{babel} pour le russe et l'ukrainien d�finissent les
commandes \ci{Asbuk} et \ci{asbuk} qui agissent comme \ci{Alph} et
\ci{alph}, mais produisent des majuscules et des minuscules des
alphabets russe et ukrainien (en fonction duquel est la langue active
du document). L'option bulgare de \pai{babel} fournit les commandes
\ci{enumBul} et \ci{enumLat} (\ci{enumEng}) qui font produire �
\ci{Alph} et \ci{alph} des lettres des alphabets bulgare ou latin
(anglais, resp.). Le comportement par d�faut de \ci{Alph} et \ci{alph}
pour le bulgare est de produire des lettres de l'alphabet bulgare.

%Finally, math alphabets are redefined and  as well as the commands for math
%operators according to Cyrillic typesetting traditions. 

\section{L'espace entre les mots}

Pour obtenir une marge droite align�e, \LaTeX{} ins�re des espaces
plus ou moins larges entre les mots. Apr�s la ponctuation finale
d'une phrase, les r�gles de la typographie anglo-saxonne veulent que
l'on ins�re une espace plus large.  Mais si un point suit une lettre
majuscule, \LaTeX{} consid�re qu'il s'agit d'une abr�viation et ins�re
alors une espace normale.

Toute exception � ces r�gles doit �tre sp�cifi�e par l'auteur du
document. Un antislash qui pr�c�de une espace g�n�re une espace qui ne
sera pas �largie par \LaTeX{}.  Un tilde \og \verb|~| \fg{} produit
une espace interdisant tout saut de ligne (dit espace
\emph{ins�cable}).  \verb|~| est � utiliser pour �viter les coupures
ind�sirables : on code \verb|M.~Dupont|, \verb|cf.~Fig.~5|, etc.  La
commande \verb|\@| avant un point indique que celui-ci termine une
phrase, m�me lorsqu'il suit une majuscule.
\cih{"@}
\index{~@$\sim$} \index{tilde@tilde ( \verb.~.)}
\index{., espace apr�s}
\index{espace ins�cable}

\nonfrenchspacing
\begin{example}
M.~Dupont vous remercie\\
R.E.M.\\
I like BASIC\@. Do you?
\end{example}
\frenchspacing

La commande \verb|\,| permet d'ins�rer une demi-espace ins�cable.

\begin{example}
1\,234\,567
\end{example}

L'ajout d'espace suppl�mentaire � la fin d'une phrase peut �tre
supprim� par la commande :
\begin{lscommand}
\ci{frenchspacing}
\end{lscommand}
\noindent qui est active par d�faut avec l'option \pai{francais} de
l'extension \pai{babel}. Dans ce cas, la commande \verb|\@| n'est pas
n�cessaire. 



\section{Titres, chapitres et sections}

Pour aider le lecteur � suivre votre pens�e, vous souhaitez s�parer
vos documents en chapitres, sections ou sous-sections. \LaTeX{}
utilise pour cela des commandes qui prennent en argument le titre de
chaque �l�ment. C'est � vous de les utiliser dans l'ordre.

Dans la classe de document \texttt{article}, les commandes de
sectionnement suivantes sont disponibles : \nopagebreak
\begin{lscommand}
\ci{section}\verb|{...}|\\
\ci{subsection}\verb|{...}|\\
\ci{subsubsection}\verb|{...}|\\
\ci{paragraph}\verb|{...}|\\
\ci{subparagraph}\verb|{...}|
\end{lscommand}

Si vous souhaitez d�couper votre document sans influencer la
num�rotation des chapitres ou des sections vous pouvez utiliser la
commande :
\begin{lscommand}
\ci{part}\verb|{...}|
\end{lscommand}

Dans les classes \texttt{report} et \texttt{book}, une commande de
sectionnemnent sup�rieur est disponible (elle s'intercale
entre \verb|\part| et \verb|\section|) :
\begin{lscommand}
\ci{chapter}\verb|{...}|
\end{lscommand}

Puisque la classe \texttt{article} ne conna�t pas les chapitres, il
est facile par exemple de regrouper des articles en tant que chapitres
d'un livre en remplacant le \texttt{\bs title} de chaque article par
\texttt{\bs chapter}.

L'espacement entre les sections, la num�rotation et le
choix de la police et de la taille des titres sont g�r�s
automatiquement par \LaTeX{}.

Deux commandes de sectionnement ont un comportement sp�cial :
\begin{itemize}
\item la commande \ci{part} ne change pas la num�rotation des
      chapitres ;
\item la commande \ci{appendix} ne prend pas d'argument. Elle bascule
      simplement la num�rotation des chapitres\,\footnote{Pour la classe
      article, elle change la num�rotation des sections} en lettres.
\end{itemize}

\LaTeX{} peut ensuite cr�er la table des mati�res en r�cup�rant la
liste des titres et de leur num�ro de page d'une ex�cution pr�c�dente
(fichier \texttt{.toc}). La commande :
\begin{lscommand}
\ci{tableofcontents}
\end{lscommand}
\noindent imprime la table des mati�res � l'endroit o� la commande est
invoqu�e. Un document doit �tre trait� (on dit aussi \og compil�
\fg{}) deux fois par \LaTeX{} pour avoir une table des mati�res
correcte. Dans certains cas, un troisi�me passage est m�me
n�cessaire. \LaTeX{} vous indique quand cela est n�cessaire.

Toutes les commandes cit�es ci-dessus existent dans une forme
\og �toil�e \fg{} obtenue en ajoutant une �toile \verb|*| au nom de la
commande. Ces commandes produisent des titres de sections qui
n'apparaissent pas dans la table des mati�res et qui ne sont pas
num�rot�s. On peut ainsi remplacer la commande
\verb|\section{Introduction}| par
\verb|\section*{Introduction}|.

Par d�faut, les titres de section apparaissent dans la table des
mati�res exactement comme ils sont dans le texte. Parfois il n'est pas
possible de faire tenir un titre trop long dans la table des
mati�res. On peut donner un titre sp�cifique pour la table des
mati�res en argument optionnel avant le titre principal :
\begin{code}
\verb|\chapter[Le LAAS du CNRS]{Le Laboratoire|\\
\verb|         d'Analyse et d'Architecture|\\
\verb|        des Syst�mes du Centre National|\\
\verb|        de la Recherche Scientifique}|
\end{code} 

Le \wi{titre du document} est obtenu par la commande :
\begin{lscommand}
\ci{maketitle}
\end{lscommand}
Les �l�ments de ce titre sont d�finis par les commandes :
\begin{lscommand}
\ci{title}\verb|{...}|, \ci{author}\verb|{...}| 
et �ventuellement \ci{date}\verb|{...}| 
\end{lscommand}
\noindent qui doivent �tre appel�es avant \verb|\maketitle|. Dans
l'argument de la commande \ci{author}, vous pouvez citer plusieurs
auteurs en s�parant leurs noms par des commandes \ci{and}.

Vous trouverez un exemple des commandes cit�es ci-dessus sur la
figure~\ref{document}, page~\pageref{document}.

En plus des commandes de sectionnement expliqu�es ci-dessus, \LaTeXe{}
a introduit trois nouvelles commandes destin�es � �tre utilis�es avec
la classe \texttt{book} :
\begin{description}
\item[\ci{frontmatter}] doit �tre la premi�re commande apr�s le
  d�but du corps du document (\verb|\begin{document}|),
    elle introduit le prologue du document.
    Les num�ros de pages sont alors en romain (i, ii, iii, etc.) et
    les sections non-num�rot�es, comme si vous utilisiez les variantes
    �toil�es des commandes de sectionnement
    (p.e. \verb|\chapter*{Preface}|) mais que les sections
    apparaissaient tout de m�me en table des mati�res ;

\item[\ci{mainmatter}] se place juste avant le d�but du premier
  (vrai) chapitre du livre,  la num�rotation des pages se fait alors 
  en chiffres arabes et le compteur de pages est remis �~1 ;

\item[\ci{appendix}] indique le d�but des appendices, les num�ros
  des chapitres sont alors remplac�s par des lettres majuscules (A, B,
  etc.) ;

\item[\ci{backmatter}] se place juste avant la bibliographie et les
  index. Avec les classes standard de document, cette commmande n'a
  aucun effet visible.
\end{description}

\section{R�f�rences crois�es}

Dans les livres, rapports ou articles, on trouve souvent des
\wi{r�f�rences crois�es} vers des figures, des tableaux ou des passages
particuliers du texte. \LaTeX{} dispose des commandes suivantes pour
faire des r�f�rences crois�es :

\begin{lscommand}
\ci{label}\verb|{|\emph{marque}\verb|}|, \ci{ref}\verb|{|\emph{marque}\verb|}| 
et \ci{pageref}\verb|{|\emph{marque}\verb|}|
\end{lscommand}
\noindent o� \emph{marque} est un identificateur choisi par
l'utilisateur. \LaTeX{} remplace \verb|\ref| par le num�ro de la
section, de la sous-section, de la figure, du tableau, ou du th�or�me
o� la commande \verb|\label| correspondante a �t�
plac�e. \verb|\pageref| affichera la page de la commande \verb|\label|
correspondante.
L'utilisation de r�f�rences crois�es rend n�cessaire de compiler deux fois le
document : � la premi�re compilation les num�ros correspondant aux �tiquettes
\verb|\label{}| sont inscrits dans le fichier \texttt{.aux} et, � la
compilation suivante, \verb|\ref{}| et \verb|\pageref{}| peuvent imprimer
ces num�ros%
\,\footnote{Ces commandes ne connaissent pas le type du num�ro auquel
elles se r�f�rent, elles utilisent le dernier num�ro g�n�r�
automatiquement.}. 

\begin{example}
Une r�f�rence � cette 
section\label{ma-section}
ressemble � :
�~voir section~\ref{ma-section},
page~\pageref{ma-section}.~�
\end{example}
 
\section{Notes de bas de page}
La commande :
\begin{lscommand}
\ci{footnote}\verb|{|\emph{texte}\verb|}|
\end{lscommand}
\noindent imprime une note de bas de page en bas de la page en cours.
Les notes de bas de page doivent �tre plac�es apr�s le mot o� la
phrase auquel elles se r�f�rent%
\,\footnote{La typographie fran�aise demande une espace fine avant la
marque de renvoi � la note.}
%SC: hmmm, is it french good practice ?
Les notes qui se r�f�rent � une (partie de) phrase devraient �tre
plac�es apr�s une virgule ou un point.\footnote{Remarquez que les
  notes de bas de page d�tournent l'attention du lecteur du corps du
  document. Apr�s tout, tout le monde lit les notes de bas de
  page~---~nous sommes une esp�ce curieuse, alors pourquoi ne pas plus
  simplement int�grer tout ce que vous souhaitez dire dans le corps du
  document ?\footnotemark}
\footnotetext{Un guide ne va pas forc�ment dans la direction qu'il
  indique :-).}
\nopagebreak[2]

\begin{example}
Les notes de bas de page%
\,\footnote{ceci est une note
	  de bas de page.}
sont tr�s pris�es par les 
utilisateurs de \LaTeX{}.
\end{example}

\section{Souligner l'importance d'un mot}

Dans un manuscrit r�alis� sur une machine � �crire, les mots
importants sont \texttt{valoris�s en les \underline{soulignant}},
via la commande :
\begin{lscommand}
\ci{underline}\verb|{|\textit{texte}\verb|}|
\end{lscommand}

Dans un ouvrage
imprim�, on pr�f�re les \emph{mettre en valeur}%
\,\footnote{\emph{Emphasized} en anglais.}. 
La commande de mise en valeur est :
\begin{lscommand}
\ci{emph}\verb|{|\emph{texte}\verb|}|
\end{lscommand}

Son argument est le texte � mettre en valeur. En g�n�ral, la police
\emph{italique} est utilis�e pour la mise en valeur, sauf si le texte
est d�ja en italique, auquel cas on utilise une police romaine (droite).

\begin{example}
\emph{Pour \emph{insister}
dans un passage d�j� 
mis en valeur, \LaTeX{} 
utilise une police droite.}
\end{example}
 
Remarquez la diff�rence entre demander � \LaTeX{} de \emph{mettre en
valeur} un mot et lui demander de changer de \emph{police de
caract�res} :

\begin{example}
\textit{Vous pouvez aussi
  \emph{mettre en valeur} du
 texte en italique,}
\textsf{ou dans une police
  \emph{sans-serif},} 
\texttt{ou dans une police
  \emph{machine � �crire}.}
\end{example}

%\clearpage
\section{Environnements} \label{env}

Pour composer du texte dans des contextes sp�cifiques, \LaTeX{}
d�finit des \wi{environnement}s diff�rents pour divers types de mise
en page :

\begin{lscommand}
\ci{begin}\verb|{|\emph{nom}\verb|}|\quad
   \emph{texte}\quad
\ci{end}\verb|{|\emph{nom}\verb|}|
\end{lscommand}
\noindent 
\emph{nom} est le nom de l'environnement. Les environnements peuvent
�tre imbriqu�s, � condition que l'ordre d'imbrication est pr�serv�.
\begin{code}
\verb|\begin{aaa}...\begin{bbb}...\end{bbb}...\end{aaa}|
\end{code}

\noindent Dans les sections suivantes tous les environnements
importants sont pr�sent�s.

\subsection{Listes, �num�rations et descriptions}

L'environnement \ei{itemize} est utilis� pour des listes simples,
\ei{enumerate} est utilis� pour des �num�rations (listes
num�rot�es) et \ei{description} est utilis� pour des descriptions. 
\cih{item}

% \begin{example}
% Les diff�rents types de liste :
% \begin{itemize}
% \item \texttt{itemize}
% \item \texttt{enumerate}
% \item \texttt{description}
% \end{itemize}
% \end{example}

\begin{otherlanguage}{english}
\begin{example}
\begin{enumerate}
\item Il est possible d'imbriquer
les environnements � sa guise :
\begin{itemize}
\item mais cela peut ne pas 
  �tre  tr�s beau,
\item ni facile � suivre.
\end{itemize}
\item Souvenez-vous :
\begin{description}
\item[Clart� :] les faits ne 
vont pas devenir plus sens�s 
parce  qu'ils sont dans une liste,
\item[Synth�se :] cependant,
une liste peut tr�s bien 
r�sumer des faits.
\end{description}
\end{enumerate}
\end{example}
\end{otherlanguage}

Notez que l'option \pai{francais} de l'extension \pai{babel} utilise
une pr�sentation des 
listes simples qui respecte les r�gles typographiques fran�aises :

\begin{example}
Une liste simple fran�aise :
\begin{itemize}
\item voici un �l�ment ;
\item puis un autre.
\end{itemize}
\end{example}

\begin{otherlanguage}{english}
\begin{example}
An english list:
\begin{itemize}
\item one item
\item an other one
\end{itemize}
\end{example}
\end{otherlanguage}

\subsection{Alignements � gauche, � droite et centrage}

Les environnements \ei{flushleft} et \ei{flushright} produisent des
textes \wi{align�}s � gauche ou � droite. L'environnement \ei{center}
produit un texte centr�. Si vous n'utilisez pas la commande \ci{\bs}
pour indiquer les sauts de ligne, ceux-ci sont calcul�s
automatiquement par \LaTeX{}.

\begin{example}
\begin{flushleft}
Ce texte est\\
align� � gauche.
\LaTeX{} n'essaye pas 
d'aligner la marge droite. 
\end{flushleft}
\end{example}

\begin{example}
\begin{flushright}
Ce texte est\\
align� � droite.
\LaTeX{} n'essaye pas 
d'aligner la marge gauche. 
\end{flushright}
\end{example}

\begin{example}
\begin{center}
Au centre de la terre.
\end{center}
\end{example}

\subsection{Citations et vers}

L'environnement \ei{quote} est utile pour les citations, les phrases
importantes ou les exemples.

\begin{example}
Une r�gle typographique 
simple pour la longueur
des lignes :
\begin{quote}
Une ligne ne devrait pas comporter
plus de 66~caract�res.
\end{quote}
C'est pourquoi les pages 
compos�es par \LaTeX{} ont des 
marges importantes et
pourquoi les journaux utilisent 
souvent plusieurs colonnes.
\end{example}

Il existe deux autres environnements comparables : \ei{quotation} et
\ei{verse}. L'environnement \ei{quotation} est utile pour des
citations plus longues, couvrant plusieurs
paragraphes parce qu'il indente ceux-ci.
L'environnement \ei{verse} est utilis� pour la po�sie, l�
o� les retours � la ligne sont importants. Les vers sont s�par�s par
des commandes \ci{\bs} et les strophes par une ligne vide\footnote{Les
puristes constateront que l'environnement \ei{verse} ne respecte pas
les r�gles de la typographie fran�aise : les rejets devraient �tre
pr�fix�s par \emph{[} et align�s � droite sur le vers pr�c�dent.}.


\begin{example}
Voici le d�but d'une 
fugue de Boris Vian :
\begin{flushleft}
\begin{verse}
Les gens qui n'ont plus 
  rien � faire\\
Se suivent dans la rue comme\\
Des wagons de chemin de fer.

Fer fer fer\\
Fer fer fer\\
Fer quoi faire\\
Fer coiffeur.\\
\end{verse}
\end{flushleft}
\end{example}

\subsection{Abstract}

Lors d'une publication scientifique il est usuel de d�marrer celle-ci
avec un \emph{abstract}, ou r�sum�, cens� donner au lecteur une vue
d'ensemble de ce qu'il doit attendre du document. \LaTeX{} fournit un
environnement \ei{abstract} � cette fin. Normalement \ei{abstract} est
utilis� dans les documents utilisant la classe article.

\newenvironment{abstract}%
        {\begin{center}\begin{small}\begin{minipage}{0.8\textwidth}}%
        {\end{minipage}\end{small}\end{center}}
\begin{example}
\begin{abstract}
L'abstrait abstract r�sum�.
\end{abstract}
\end{example}

\subsection{Impression \emph{verbatim}}

Tout texte inclus entre \verb|\begin{|\ei{verbatim}\verb|}| et
\verb|\end{verbatim}| est imprim� tel quel, comme s'il avait �t� tap�
� la machine, avec tous les retours � la ligne et les espaces, sans
qu'aucune commande \LaTeX{} ne soit ex�cut�e.

� l'int�rieur d'un paragraphe, une fonctionnalit� �quivalente peut
�tre obtenue par
\begin{lscommand}
\ci{verb}\verb|+|\emph{texte}\verb|+|
\end{lscommand}
\noindent Le caract�re \verb|+| est seulement un exemple de caract�re
s�parateur. Vous pouvez utiliser n'importe quel caract�re, sauf les
lettres, \verb|*| ou l'espace. La plupart des exemples de commandes
\LaTeX{} dans ce document sont r�alis�s avec cette commande.

\begin{example}
La commande \verb|\dots| \dots

\begin{verbatim}
10 PRINT "HELLO WORLD ";
20 GOTO 10
\end{verbatim}
\end{example}

\begin{example}
\begin{verbatim*}
La version �toil�e de 
l'environnement  verbatim
met    les   espaces   en 
�vidence
\end{verbatim*}
\end{example}

La commande \ci{verb} peut �galement �tre utilis�e avec une �toile :
\begin{example}
\verb*|comme ceci :-) |
\end{example}

L'environnement \texttt{verbatim} et la commande \verb|\verb| ne
peuvent �tre utilis�s � l'int�rieur d'autres commandes comme
\verb|\footnote{}|.


\subsection{Tableaux}

\newcommand{\mfr}[1]{\framebox{\rule{0pt}{0.7em}\texttt{#1}}}

L'environnement \ei{tabular} permet de r�aliser des tableaux avec ou
sans lignes de s�paration horizontales ou verticales. \LaTeX{}
ajuste automatiquement la largeur des colonnes.

L'argument \emph{description} de la commande :
\begin{lscommand}
\verb|\begin{tabular}[|\emph{position}\verb|]{|\emph{description}\verb|}|
\end{lscommand} 
\noindent d�finit le format des colonnes du tableau. Utilisez un
\mfr{l} pour une colonne align�e � gauche, \mfr{r} pour
une colonne align�e � droite et \mfr{c} pour une colonne
centr�e. \mfr{p{\{\emph{largeur}\}}} permet de r�aliser une colonne
justifi�e � droite sur plusieurs lignes et enfin
\mfr{|} permet d'obtenir une ligne verticale.
\index{"|@ \verb."|.} 

Si le texte d'une colonne est trop large pour la page, \LaTeX{}
n'ins�rera pas automatiquement de saut de ligne. Gr�ce �
\mfr{p\{\emph{largeur}\}} vous pouvez d�finir un type sp�cial de
colonne qui fera passer le texte � la ligne comme pour un paragraph
usuel.

L'argument \emph{position} d�finit la position verticale du tableau
par rapport au texte environnant. Utilisez les lettres \mfr{t},
\mfr{b} et \mfr{c} pour l'aligner en haut (\emph{top}), en bas
(\emph{bottom}) ou au centre (\emph{center}) respectivement.

� l'int�rieur de l'environnement \texttt{tabular}, le caract�re
\texttt{\&} est le s�parateur de colonnes, \ci{\bs} commence une nouvelle
ligne et \ci{hline} ins�re une ligne horizontale. Vous pouvez ajoutez
des lignes partielles via la commande
\ci{cline}\texttt{\{}\emph{j}\texttt{-}\emph{i}\texttt{\}}, o� j et i
sont les num�ros de colonnes de d�but et de fin de ligne.
\index{\&}

\begin{example}
\begin{tabular}{|r|l|}
\hline
7C0 & hexad�cimal \\
3700 & octal \\
11111000000 & binaire \\
\hline \hline
1984 & d�cimal \\
\hline
\end{tabular}
\end{example}

\begin{example}
\begin{tabular}{|p{4.7cm}|}
\hline
Bienvenue dans ce 
cadre.\\
Merci de votre visite.\\
\hline 
\end{tabular}

\end{example}

La construction \mfr{@\{...\}} permet d'imposer le s�parateur de
colonnes. Cette commande supprime l'espacement inter-colonnes et le
remplace par ce qui est indiqu� entre les crochets. Une utilisation
courante de cette commande est pr�sent�e plus loin comme solution au
probl�me de l'alignement des nombres d�cimaux. Une autre utilisation
possible est de supprimer l'espacement dans un tableau avec
\mfr{@\{\}}.

\begin{example}
\begin{tabular}{@{} l @{}}
\hline 
sans espace\\\hline
\end{tabular}
\end{example}
\begin{example}
\begin{tabular}{l}
\hline
avec espaces\\
\hline
\end{tabular}
\end{example}

%
% This part by Mike Ressler
%

\index{alignement d�cimal} S'il n'y a pas de commande pr�vue%
\,\footnote{Si les extensions de l'ensemble \og tools \fg{} sont install�es
          sur votre syst�me, jetez un \oe il sur l'extension
          \pai{dcolumn} faite pour r�soudre ce probl�me.} 
pour aligner les nombres sur le point d�cimal (ou la virgule si on
respecte les r�gles fran�aises) nous pouvons \og tricher \fg{} et
r�aliser cet alignement en utilisant deux colonnes : la premi�re
align�e � droite contient la partie enti�re et la seconde align�e �
gauche contient la partie d�cimale. La commande \verb|\@{,}| dans la
description du tableau remplace l'espace normale entre les colonnes par
une simple virgule, donnant l'impression d'une seule colonne align�e
sur le s�parateur d�cimal. N'oubliez pas de remplacer dans votre
tableau le point ou la virgule 
par un s�parateur de colonnes (\verb|&|) ! Un label peut �tre
plac� au-dessus de cette colonne en utilisant la commande
\ci{multicolumn}. 

\begin{example}
\begin{tabular}{c r @{,} l}
Expression       &
\multicolumn{2}{c}{Valeur} \\
\hline
$\pi$               & 3&1416  \\
$\pi^{\pi}$         & 36&46   \\
$(\pi^{\pi})^{\pi}$ & 80662&7 \\
\end{tabular}
\end{example}
Autre exemple d'utilisation de \verb+\multicolumn+ :
\begin{example}
\begin{tabular}{|l|l|}
\hline
\multicolumn{2}{|c|}{\textbf{Nom}}\\
\hline
Dupont & Jules \\
\hline
\end{tabular}
\end{example}

\LaTeX{} traite le contenu d'un environnement \texttt{tabular} comme
une bo�te indivisible, en particulier il ne peut y avoir de coupure de
page. Pour r�aliser de longs tableaux s'�tendant sur plusieurs pages
il faut avoir recours aux extensions \pai{supertabular} ou
\pai{longtable}.

Parfois les tableaux par d�faut de \LaTeX{} donnent une impression
d'�troitesse. Si vous voulez leur donner plus d'extension, vous pouvez
le faire � l'aide des valeurs \ci{arraystretch} et \ci{tabcolsep}.

\begin{example}
\begin{tabular}{|l|}
\hline
Ces lignes sont\\\hline
� l'�troit\\\hline
\end{tabular}

{\renewcommand{\arraystretch}{1.5}
\renewcommand{\tabcolsep}{0.2cm}
\begin{tabular}{|l|}
\hline
Un tableau\\\hline
moins �troit\\\hline
\end{tabular}}

\end{example}

Si vous voulez seulement augmenter la hauteur d'un ligne dans un
tableau, vous pouvez utiliser un filet de largeur nulle%
\footnote{En typographie professionnelle ceci est appel� un
\wi{montant}.}. Donnez � ce filet \ci{rule} la hauteur voulue.

\begin{example}
\begin{tabular}{|c|}
\hline
\rule{1pt}{4ex}\'Etai\dots\\
\hline
\rule{0pt}{4ex} montant \\
\hline
\end{tabular}
\end{example}

\section{Objets flottants}

De nos jours, la plupart des publications contiennent un nombre
important de figures et de tableaux. Ces �l�ments n�cessitent un
traitement particulier car ils ne peuvent �tre coup�s par un
changement de page. On pourrait imaginer de commencer une nouvelle
page chaque fois qu'une figure ou un tableau ne rentrerait pas dans la
page en cours. Cette fa�on de faire laisserait de nombreuses pages � moiti�
blanches, ce qui ne serait r�ellement pas beau.
\index{tableau}
\index{figure}

La solution est de laisser \og flotter \fg{} les figures et les tableaux
qui ne rentrent pas sur la page en cours, vers une page suivante et de
compl�ter la page avec le texte qui suit l'objet \og flottant \fg{}. 
\LaTeX{} fournit deux
environnements pour les \wi{objets flottants} adapt�s respectivement
aux figures (\ei{figure}) et aux tableaux (\ei{table}). Pour
faire le meilleur usage de ces deux environnements, il est important
de comprendre comment \LaTeX{} traite ces objets flottants de mani�re
interne. Dans le cas contraire ces objets deviendront une cause de
frustration intense 
car \LaTeX{} ne les placera jamais � l'endroit o� vous souhaitiez les
voir. 

\bigskip
Commen�ons par regarder les commandes que \LaTeX{} propose pour les
objets flottants :

Tout objet inclus dans un environnement \ei{figure} ou \ei{table} est
trait� comme un objet flottant. Les deux environnements flottants ont
un param�tre optionnel :
\begin{lscommand}
\verb|\begin{figure}[|\emph{placement}\verb|]| ou
\verb|\begin{table}[|\ldots\verb|]|
\end{lscommand}
\noindent 
appel� \emph{placement}. Ce param�tre permet de dire � \LaTeX{} o�
vous autorisez l'objet � flotter. Un \emph{placement} est compos�
d'une cha�ne de caract�res repr�sentant des \emph{placements
possibles}. Reportez-vous au tableau~\ref{tab:permiss}.


\begin{table}[!htbp]
\caption{Placements possibles}\label{tab:permiss}
\noindent \begin{minipage}{\textwidth}
\medskip
\begin{center}
\begin{tabular}{@{}cp{8cm}@{}}
Caract�re & Emplacement pour l'objet flottant\dots\\
\hline
\rule{0pt}{1.05em}%
\texttt{h} & \emph{here}, ici, � l'emplacement dans
	     le texte o� la commande se trouve. Utile pour les petits
	     objets.\\[0.3ex] 
\texttt{t} & \emph{top}, en haut d'une page\\[0.3ex]
\texttt{b} & \emph{bottom}, en bas d'une page\\[0.3ex]
\texttt{p} & \emph{page}, sur une page � part ne contenant que des
             objets flottants.\\[0.3ex]
\texttt{!} & ici, sans prendre en compte les param�tres
             internes\,\footnote{tels que le nombre maximum d'objets
             flottants sur une page} qui pourraient emp�cher ce
             placement. 
\end{tabular}
\end{center}
\end{minipage}
\end{table}

Un tableau peut commencer par exemple par la ligne suivante :
\begin{code}
\verb|\begin{table}[!hbp]|
\end{code}
\noindent L'\wi{emplacement} \verb|[!hbp]| permet � \LaTeX{} de placer
le tableau soit sur place (\texttt{h}), soit en bas de page
(\texttt{b}) soit enfin sur une page � part (\texttt{p}), et
tout cela m�me si les r�gles internes de \LaTeX{} ne sont pas toutes
respect�es (\texttt{!}). Si aucun placement n'est indiqu�, les
classes standard utilisent \verb|[tbp]| par d�faut.

\LaTeX{} place tous les objets flottants qu'il rencontre
en suivant les indications fournies par l'auteur. Si un objet ne peut
�tre plac� sur la page en cours, il est plac� soit dans la file des
figures soit dans la file des tableaux\,\footnote{Il s'agit de files
FIFO (\emph{First In, First Out}) : premier arriv�, premier servi.}. 
Quand une nouvelle page est
entam�e, \LaTeX{} essaye d'abord de voir si les objets en t�te des
deux files pourraient �tre plac�s sur une page sp�ciale, � part.  Si
cela n'est pas possible, les objets en t�te des deux files sont
trait�s comme s'ils venaient d'�tre trouv�s dans le texte : \LaTeX{}
essaye de les placer selon leurs sp�cifications de placement (sauf
\texttt{h}, qui n'est plus possible). Tous les
nouveaux objets flottants rencontr�s dans la suite du texte sont
ajout�s � la queue des files. \LaTeX{} respecte scrupuleusement
l'ordre d'apparition des objets flottants. C'est pourquoi un objet
flottant qui ne peut �tre plac� dans le texte repousse tous les
autres � la fin du document.

D'o� la r�gle :
\begin{quote}
Si \LaTeX{} ne place pas les objets flottants comme vous le souhaitez,
c'est souvent � cause d'un seul objet trop grand qui bouche l'une des
deux files d'objets flottants.
\end{quote}                 

Essayer d'imposer � \LaTeX{} un emplacement particulier pose souvent
probl�me : si l'objet flottant ne tient pas � l'emplacement demand�,
alors il est coinc� et bloque le reste des objets flottants. En
particulier, l'utilisation de la seule option \verb+[h]+ pour un
flottant est une id�e \emph{� proscrire}, les versions modernes de
\LaTeX{} changent d'ailleurs automatiquement l'option \verb+[h]+ en
\verb+[ht]+.

Voici quelques �l�ments suppl�mentaires qu'il est bon de conna�tre sur
les environnements \ei{table} et \ei{figure}.

Avec la commande :
\begin{lscommand}
\ci{caption}\verb|{|\emph{texte de la l�gende}\verb|}|
\end{lscommand}
\noindent
vous d�finissez une l�gende pour l'objet. Un num�ro (incr�ment�
 automatiquement) et le mot \og Figure \fg{} ou
 \og Table \fg{}\footnote{Avec l'extension \texttt{babel}, la
 pr�sentation des l�gendes est modifi�e pour ob�ir aux r�gles
 fran�aises.} sont ajout�s par \LaTeX.

Les deux commandes :
\begin{lscommand}
\ci{listoffigures} et \ci{listoftables} 
\end{lscommand}
\noindent fonctionnent de la m�me mani�re que la commande
\verb|\tableofcontents| ; elles impriment respectivement la liste des
figures et des tableaux. Dans ces listes, la l�gende est reprise en
entier. Si vous d�sirez utiliser des l�gendes longues, vous pouvez
en donner une version courte entre crochets qui sera utilis�e pour la
table :
\begin{code}
\verb|\caption[courte]{LLLLLoooooonnnnnggggguuuueee}| 
\end{code}

Avec \ci{label} et \ci{ref} vous pouvez faire r�f�rence � votre objet
� l'int�rieur de votre texte. La commande \ci{label} doit appara�tre
\emph{apr�s} la commande \ci{caption} d'une l�gende si vous voulez
r�f�rencer le num�ro de cette l�gende.

L'exemple suivant dessine un carr� et l'ins�re dans le document. Vous
pouvez utiliser cette commande pour r�server de la place pour une
illustration que vous allez coller sur le document termin�.

\begin{code}
\begin{verbatim}
La figure~\ref{blanche} est un exemple de Pop-Art.
\begin{figure}[!htbp]
\makebox[\textwidth]{\framebox[5cm]{\rule{0pt}{5cm}}}
\caption{Cinq centim�tres sur cinq.\label{blanche}}
\end{figure}
\end{verbatim}
\end{code}

Dans l'exemple ci-dessus\,\footnote{En supposant que la file
des figures soit vide.} \LaTeX{} va s'acharner~(\texttt{!}) �
placer la figure l� o� se trouve la commande~(\texttt{h}) dans le
texte. S'il n'y arrive pas, il essayera de la placer en
bas~(\texttt{b}) de la page. Enfin s'il ne peut la placer sur la
page courante, il essayera de cr�er une page � part avec d'autres
objets flottants. S'il n'y a pas suffisamment de tableaux en attente
pour remplir une page sp�cifique, \LaTeX{} continue et, au d�but de la
page suivante, r�essayera de placer la figure comme si elle venait
d'appara�tre dans le texte.

Dans certains cas il peut s'av�rer n�cessaire d'utiliser la commande :
\begin{lscommand}
\ci{clearpage} ou m�me \ci{cleardoublepage} 
\end{lscommand}
Elle ordonne � \LaTeX{} de placer tous les objets en attente
imm�diatement puis de commencer une nouvelle
page. \ci{cleardoublepage} commence une nouvelle page de droite.

En section~\ref{eps}, vous apprendrez � inclure une figure
\PSi{} dans vos documents.

\section{Protection des commandes \texorpdfstring{\og fragiles \fg}{\string�fragiles \string�}}

Les arguments de commandes telles que \ci{section} ou \ci{caption}
etc., peuvent appara�tre plusieurs fois dans le document (par exemple
dans la table des mati�res, les hauts de pages\dots), on dit qu'il
s'agit d'arguments \og mobiles \fg{} (\emph{moving arguments}).
Certaines commandes, entre autres \ci{footnote}, \ci{phantom} etc., ne
produisent pas le r�sultat escompt� quand elles sont ex�cut�es comme
argument de commandes de type \ci{section}. La compilation du document
�chouera alors. On dit de ces commandes qu'elles sont \og fragiles
\fg, ce qui signifie qu'elles ont besoin de la protection (comme nous
tous ?). Il faut alors les faire pr�c�der de \ci{protect}.

La commande \ci{protect} n'a d'effet que sur la commande qui la suit
imm�diatement, mais \emph{pas ses arguments} �ventuels.
La plupart du temps un \ci{protect} de trop ne produira aucun effet pervers.

Voici un exemple d'utilisation de \verb+\protect+ :
\begin{center}
\begin{verbatim}
\section{Je suis prudent
   \protect\footnote{Je prot�ge ma note de bas de page.}}
\end{verbatim}
\end{center}

\endinput

% Local Variables:
% TeX-master: "lshort2e"
% mode: latex
% mode: flyspell
% End:

%%%%%%%%%%%%%%%%%%%%%%%%%%%%%%%%%%%%%%%%%%%%%%%%%%%%%%%%%%%%%%%%
% Contents: Math typesetting with LaTeX
% $Id: math.tex,v 1.2 2003/03/19 20:57:46 oetiker Exp $
%%%%%%%%%%%%%%%%%%%%%%%%%%%%%%%%%%%%%%%%%%%%%%%%%%%%%%%%%%%%%%%%%
 
\chapter{Typesetting Mathematical Formulae}

\begin{intro}
  Now you are ready! In this chapter, we will attack the main strength
  of \TeX{}: mathematical typesetting. But be warned, this chapter
  only scratches the surface. While the things explained here are
  sufficient for many people, don't despair if you can't find a
  solution to your mathematical typesetting needs here. It is highly likely
  that your problem is addressed in \AmS-\LaTeX{}.
\end{intro}
  


\section{The \texorpdfstring{\AmS}{AMS}-\LaTeX{} bundle}

If you want to typeset (advanced) \wi{mathematics}, you should
use \AmS-\LaTeX{}. The \AmS-\LaTeX{} bundle is a collection of packages and classes for
mathematical typesetting. We will mostly deal with the \pai{amsmath} package
which is a part of the bundle. \AmS-\LaTeX{} is produced by The \emph{\wi{American Mathematical Society}} 
and it is used extensively for mathematical typesetting. \LaTeX{} itself does provide
some basic features and environments for mathematics, but they are limited (or
maybe it's the other way around: \AmS-\LaTeX{} is \emph{unlimited}!) and
in some cases inconsistent. 

\AmS-\LaTeX{} is a part of the required distribution and is provided
with all recent \LaTeX{} distributions.\footnote{If yours is missing it, go to
  \texttt{CTAN:macros/latex/required/amslatex}.} In this chapter, we assume
  \pai{amsmath} is loaded in the preamble; \verb|\usepackage{amsmath}|.

\section{Single Equations}
  
There two ways to typeset mathematical \wi{formulae}: in-line within a paragraph
(\emph{\wi{text style}}), or the paragraph can be broken to 
typeset it separately (\textit{\wi{display style}}). Mathematical \wi{equation}s 
\emph{within} a paragraph is entered between \index{$@\texttt{\$}} %$
between \texttt{\$} and \texttt{\$}:
\begin{example}
Add $a$ squared and $b$ squared
to get $c$ squared. Or, using 
a more mathematical approach:
$a^2 + b^2 = c^2$
\end{example}
\begin{example}
\TeX{} is pronounced as 
$\tau\epsilon\chi$\\[5pt]
100~m$^{3}$ of water\\[5pt]
This comes from my $\heartsuit$
\end{example}

If you want your larger equations to be set apart
from the rest of the paragraph, it is preferable to \emph{display} them
rather than to break the paragraph apart.
To do this, you enclose them between \verb|\begin{|\ei{equation}\verb|}| and
\verb|\end{equation}|.\footnote{This is an \textsf{amsmath} command. If you don't
have access to the package for some obscure reason, you can use \LaTeX's own
\ei{displaymath} environment instead.} You can then \ci{label} an equation number and refer to
it somewhere else in the text by using the \ci{eqref} command. If you want to
name the equation something specific, you \ci{tag} it instead. You can't use
\ci{eqref} with \ci{tag}.
\begin{example}
Add $a$ squared and $b$ squared
to get $c$ squared. Or, using
a more mathematical approach
 \begin{equation}
   a^2 + b^2 = c^2
 \end{equation}
Einstein says
 \begin{equation}
   E = mc^2 \label{clever}
 \end{equation}
He didn't say
 \begin{equation}
  1 + 1 = 3 \tag{dumb}
 \end{equation}
This is a reference to 
\eqref{clever}. 
\end{example}

If you don't want \LaTeX{} to number the equations, use the starred
version of \texttt{equation} using an asterisk, \ei{equation*}, or even easier, enclose the
equation in \ci{[} and \ci{]}:\footnote{\index{equation!\textsf{amsmath}}
  \index{equation!\LaTeX{}}This is again from \textsf{amsmath}. If you 
didn't load the package, use \LaTeX{}'s own \texttt{equation} environment
instead. The naming of the \textsf{amsmath}/\LaTeX{} commands may seem a bit
confusing, but it's really not a problem since everybody uses \textsf{amsmath} anyway.
In general, it is best to load the package from the beginning because you might use it
later on, and then \LaTeX's unnumbered \texttt{equation} clashes with
\AmS-\LaTeX's numbered \texttt{equation}.}
\begin{example}
Add $a$ squared and $b$ squared
to get $c$ squared. Or, using
a more mathematical approach
 \begin{equation*}
   a^2 + b^2 = c^2
 \end{equation*}
or you can type less for the
same effect:
 \[ a^2 + b^2 = c^2 \]
\end{example}


Note the difference in typesetting style between \wi{text style} and \wi{display style}
equations: 
\begin{example}
This is text style: 
$\lim_{n \to \infty} 
 \sum_{k=1}^n \frac{1}{k^2} 
 = \frac{\pi^2}{6}$.
And this is display style:
 \begin{equation}
  \lim_{n \to \infty} 
  \sum_{k=1}^n \frac{1}{k^2} 
  = \frac{\pi^2}{6}
 \end{equation}
\end{example}

In text style, enclose tall or deep math expressions or sub
expressions in \ci{smash}. This makes \LaTeX{} ignore the height of
these expressions. This keeps the line spacing even.

\begin{example}
A $d_{e_{e_p}}$ mathematical
expression  followed by a
$h^{i^{g^h}}$ expression. As
opposed to a smashed 
\smash{$d_{e_{e_p}}$} expression 
followed by a
\smash{$h^{i^{g^h}}$} expression.
\end{example}

\subsection{Math Mode}

There are also differences between \emph{\wi{math mode}} and \emph{text mode}. For
example, in \emph{math mode}: 

\begin{enumerate}

\item \index{spacing!math mode} Most spaces and line breaks do not have any significance, as all spaces
are either derived logically from the mathematical expressions, or
have to be specified with special commands such as \ci{,}, \ci{quad} or
\ci{qquad} (we'll get back to that later, see section~\ref{sec:math-spacing}).
 
\item Empty lines are not allowed. Only one paragraph per formula.

\item Each letter is considered to be the name of a variable and will be
typeset as such. If you want to typeset normal text within a formula
(normal upright font and normal spacing) then you have to enter the
text using the \verb|\text{...}| command (see also section \ref{sec:fontsz} on
page \pageref{sec:fontsz}).

\end{enumerate}
\begin{example}
$\forall x \in \mathbf{R}:
 \qquad x^{2} \geq 0$
\end{example}
\begin{example}
$x^{2} \geq 0\qquad
 \text{for all }x\in\mathbf{R}$
\end{example}
 
Mathematicians can be very fussy about which symbols are used:
it would be conventional here to use the `\wi{blackboard bold}' font,
\index{bold symbols} which is obtained using \ci{mathbb} from the
package \pai{amssymb}.\footnote{\pai{amssymb} is not a part
  of the \AmS-\LaTeX{} bundle, but it is perhaps still a part of your \LaTeX{}
  distribution. Check your distribution
  or go to \texttt{CTAN:/fonts/amsfonts/latex/} to obtain it.}
\ifx\mathbb\undefined\else
The last example becomes
\begin{example}
$x^{2} \geq 0\qquad
 \text{for all } x 
 \in \mathbb{R}$
\end{example}
\fi
See Table~\ref{mathalpha} on \pageref{mathalpha} and
Table~\ref{mathfonts} on \pageref{mathfonts} for more math fonts.



\section{Building Blocks of a Mathematical Formula}

In this section, we describe the most important commands used in mathematical
typesetting. Most of the commands in this section will not require
\textsf{amsmath} (if they do, it will be stated clearly), but load it anyway.


\textbf{Lowercase \wi{Greek letters}} are entered as \verb|\alpha|,
 \verb|\beta|, \verb|\gamma|, \ldots, uppercase letters
are entered as \verb|\Gamma|, \verb|\Delta|, \ldots\footnote{There is no
  uppercase Alpha, Beta etc. defined in \LaTeXe{} because it looks the same as a 
  normal roman A, B\ldots{} Once the new math coding is done, things will
  change.} 

Take a look at Table~\ref{greekletters} on page~\pageref{greekletters} for a
list of Greek letters.
\begin{example}
$\lambda,\xi,\pi,\theta,
 \mu,\Phi,\Omega,\Delta$
\end{example}


\textbf{Exponents and Subscripts} can be specified using\index{exponent}\index{subscript}
the \verb|^|\index{^@\verb"|^"|} and the \verb|_|\index{_@\verb"|_"|} character.
Most math mode commands act only on the next character, so if you
want a command to affect several characters, you have to group them
together using curly braces: \verb|{...}|.

Table~\ref{binaryrel} on page \pageref{binaryrel} lists a lot of other binary
relations like $\subseteq$ and $\perp$.

\begin{example}
$p^3_{ij} \qquad
 m_\text{Knuth} \\[5pt]
 a^x+y \neq a^{x+y}\qquad 
 e^{x^2} \neq {e^x}^2$
\end{example}


The \textbf{\wi{square root}} is entered as \ci{sqrt}; the
$n^\text{th}$ root is generated with \verb|\sqrt[|$n$\verb|]|. The size of
the root sign is determined automatically by \LaTeX. If just the sign
is needed, use \verb|\surd|.

See other kinds of arrows like $\hookrightarrow$ and $\rightleftharpoons$ on
Table~\ref{tab:arrows} on page \pageref{tab:arrows}. 
\begin{example}
$\sqrt{x} \Leftrightarrow x^{1/2}
 \quad \sqrt[3]{2}
 \quad \sqrt{x^{2} + \sqrt{y}}
 \quad \surd[x^2 + y^2]$
\end{example}


\index{dots!three}
\index{vertical!dots}
\index{horizontal!dots}
Usually you don't typeset an explicit \textbf{\wi{dot}} sign to indicate
the multiplication operation when handling symbols; however sometimes it is written
to help the reader's eyes in grouping a formula.
You should use \ci{cdot} which typesets a single dot centered. \ci{cdots} is
three centered \textbf{\wi{dots}} while \ci{ldots} sets the dots on the
baseline. Besides that, there are \ci{vdots} for 
vertical and \ci{ddots} for \wi{diagonal dots}. You can find another example in
section~\ref{sec:arraymat}.
\begin{example}
$\Psi = v_1 \cdot v_2
 \cdot \ldots \qquad 
 n! = 1 \cdot 2 
 \cdots (n-1) \cdot n$
\end{example}

The commands \ci{overline} and \ci{underline} create
\textbf{horizontal lines} directly over or under an expression:
\index{horizontal!line} \index{line!horizontal}
\begin{example}
$0.\overline{3} = 
 \underline{\underline{1/3}}$
\end{example}

The commands \ci{overbrace} and \ci{underbrace} create
long \textbf{horizontal braces} over or under an expression:
\index{horizontal!brace} \index{brace!horizontal} 
\begin{example}
$\underbrace{\overbrace{a+b+c}^6 
 \cdot \overbrace{d+e+f}^9}
 _\text{meaning of life} = 42$
\end{example}

\index{mathematical!accents} To add mathematical accents such as \textbf{small
arrows} or \textbf{\wi{tilde}} signs to variables, the commands
given in Table~\ref{mathacc} on page~\pageref{mathacc} might be useful.  Wide hats and
tildes covering several characters are generated with \ci{widetilde}
and \ci{widehat}. Notice the difference between \ci{hat} and \ci{widehat} and the placement of
\ci{bar} for a variable with subscript. The \wi{apostrophe} mark
\verb|'|\index{'@\verb"|'"|} gives a \wi{prime}:
% a dash is --
\begin{example}
$f(x) = x^2 \qquad f'(x) 
 = 2x \qquad f''(x) = 2\\[5pt]
 \hat{XY} \quad \widehat{XY}
 \quad \bar{x_0} \quad \bar{x}_0$
\end{example}


\textbf{Vectors}\index{vectors} are often specified by adding small
\wi{arrow symbols} on top of a variable. This is done with the
\ci{vec} command. The two commands \ci{overrightarrow} and
\ci{overleftarrow} are useful to denote the vector from $A$ to $B$:
\begin{example}
$\vec{a} \qquad
 \vec{AB} \qquad
 \overrightarrow{AB}$
\end{example}


Names of log-like functions are often typeset in an upright
font, and not in italics as variables are, so \LaTeX{} supplies the
following commands to typeset the most important function names:
\index{mathematical!functions}

\begin{tabular}{llllll}
\ci{arccos} &  \ci{cos}  &  \ci{csc} &  \ci{exp} &  \ci{ker}    & \ci{limsup} \\
\ci{arcsin} &  \ci{cosh} &  \ci{deg} &  \ci{gcd} &  \ci{lg}     & \ci{ln}     \\
\ci{arctan} &  \ci{cot}  &  \ci{det} &  \ci{hom} &  \ci{lim}    & \ci{log}    \\
\ci{arg}    &  \ci{coth} &  \ci{dim} &  \ci{inf} &  \ci{liminf} & \ci{max}    \\
\ci{sinh}   & \ci{sup}   &  \ci{tan}  & \ci{tanh}&  \ci{min}    & \ci{Pr}     \\
\ci{sec}    & \ci{sin} \\
\end{tabular}

\begin{example}
\[\lim_{x \rightarrow 0}
 \frac{\sin x}{x}=1\]
\end{example}

For functions missing from the list, use the \ci{DeclareMathOperator}
command. There is even a starred version for functions with limits.
This command works only in the preamble so the commented lines in the
example below must be put into the preamble.

\begin{example}
%\DeclareMathOperator{\argh}{argh}
%\DeclareMathOperator*{\nut}{Nut}
\[3\argh = 2\nut_{x=1}\]
\end{example}

For the \wi{modulo function}, there are two commands: \ci{bmod} for the
binary operator ``$a \bmod b$'' and \ci{pmod}
for expressions
such as ``$x\equiv a \pmod{b}$:''
\begin{example}
$a\bmod b \\
 x\equiv a \pmod{b}$
\end{example}

A built-up \textbf{\wi{fraction}} is typeset with the
\ci{frac}\verb|{...}{...}| command. In in-line equations, the fraction is shrunk to
fit the line. This style is obtainable in display style with \ci{tfrac}. The
reverse, i.e.\ display style fraction in text, is made with \ci{dfrac}.
Often the slashed form $1/2$ is preferable, because it looks better
for small amounts of `fraction material:'
\begin{example}
In display style:
\[3/8 \qquad \frac{3}{8} 
 \qquad \tfrac{3}{8} \]
\end{example}

\begin{example}
In text style:
$1\frac{1}{2}$~hours \qquad
$1\dfrac{1}{2}$~hours
\end{example}
 
Here the \ci{partial} command for \wi{partial derivative}s is used:
\begin{example}
\[\sqrt{\frac{x^2}{k+1}}\qquad
  x^\frac{2}{k+1}\qquad
  \frac{\partial^2f}
  {\partial x^2} \]
\end{example}

To typeset \wi{binomial coefficient}s or similar structures, use
the command \ci{binom} from \pai{amsmath}:
\begin{example}
Pascal's rule is
\begin{equation*}
 \binom{n}{k} =\binom{n-1}{k}
 + \binom{n-1}{k-1}
\end{equation*}
\end{example}

For \wi{binary relations} it may be useful to stack symbols over each other.
\ci{stackrel}\verb|{#1}{#2}| puts the symbol given
in \verb|#1| in superscript-like size over \verb|#2| which
is set in its usual position.
\begin{example}
\begin{equation*}
 f_n(x) \stackrel{*}{\approx} 1
\end{equation*}
\end{example}

The \textbf{\wi{integral operator}} is generated with \ci{int}, the
\textbf{\wi{sum operator}} with \ci{sum}, and the \textbf{\wi{product operator}}
with \ci{prod}. The upper and lower limits are specified with~\verb|^|
and~\verb|_| like subscripts and superscripts:
\begin{example}
\begin{equation*}
\sum_{i=1}^n \qquad
\int_0^{\frac{\pi}{2}} \qquad
\prod_\epsilon
\end{equation*}
\end{example}

To get more control over the placement of indices in complex
expressions, \pai{amsmath} provides the \ci{substack} command:
\begin{example}
\begin{equation*}
\sum^n_{\substack{0<i<n \\ 
        j\subseteq i}}
   P(i,j) = Q(i,j)
\end{equation*}
\end{example}



\LaTeX{} provides all sorts of symbols for \textbf{\wi{braces}} and other
\textbf{\wi{delimiters}} (e.g.~$[\;\langle\;\|\;\updownarrow$).
Round and square braces can be entered with the corresponding keys and
curly braces with \verb|\{|, but all other delimiters are generated with
special commands (e.g.~\verb|\updownarrow|).
\begin{example}
\begin{equation*}
{a,b,c} \neq \{a,b,c\}
\end{equation*}
\end{example}

If you put \ci{left} in front of an opening delimiter and
\ci{right} in front of a closing delimiter, \LaTeX{} will automatically
determine the correct size of the delimiter. Note that you must close
every \ci{left} with a corresponding \ci{right}. If you
don't want anything on the right, use the invisible ``\ci{right.}'':
\begin{example}
\begin{equation*}
1 + \left(\frac{1}{1-x^{2}}
    \right)^3 \qquad 
\left. \ddagger \frac{~}{~}\right)
\end{equation*}
\end{example}

In some cases it is necessary to specify the correct size of a
mathematical delimiter\index{mathematical!delimiter} by hand,
which can be done using the commands \ci{big}, \ci{Big}, \ci{bigg} and
\ci{Bigg} as prefixes to most delimiter commands:
\begin{example}
$\Big((x+1)(x-1)\Big)^{2}$\\
$\big( \Big( \bigg( \Bigg( \quad
\big\} \Big\} \bigg\} \Bigg\} \quad
\big\| \Big\| \bigg\| \Bigg\| \quad
\big\Downarrow \Big\Downarrow 
\bigg\Downarrow \Bigg\Downarrow$
\end{example}
 For a list of all delimiters available, see Table~\ref{tab:delimiters} on page
\pageref{tab:delimiters}. 

\section{Vertically Aligned Material} 

\subsection{Multiple Equations}
\index{equation!multiple}

For formulae running over several lines or for \wi{equation system}s,
you can use the environments \ei{align} and \verb|align*|
instead of \texttt{equation} and \texttt{equation*}.\footnote{The \ei{align}
  environment is from \textsf{amsmath}. A similar environment without \textsf{amsmath}
  from \LaTeX{} is \ei{eqnarray}, but it is generally not advised to use that
because of spacing and label inconsistencies.} With \ei{align} each line gets an
equation number. The \verb|align*| does not number anything.

The \ei{align} environments center the single equation around the \verb|&|
sign. The \verb|\\| command breaks the lines. If you only want to enumerate some
of equations, use \ci{nonumber} to remove the number. It has to be placed
\emph{before} \verb|\\|:
\begin{example}
\begin{align}
f(x) &= (a+b)(a-b) \label{1}\\
     &= a^2-ab+ba-b^2  \\ 
     &= a^2+b^2 \tag{wrong}
\end{align}
This is a reference to \eqref{1}.
\end{example}

\index{long equations} \textbf{Long equations} will not be
automatically divided into neat bits.  The author has to specify
where to break them and correct the indent:
\begin{example}
\begin{align}
f(x) &= 3x^5 + x^4 + 2x^3 
                \nonumber \\
     &\qquad + 9x^2 + 12x + 23 \\
     &= g(x) - h(x)
\end{align}
\end{example}
The \pai{amsmath} package provides a couple of other useful environments: \verb|flalign|,
\verb|gather|, \verb|multline| and \verb|split|. See the
documentation for the package for a wide range of commands, environments and more.

\subsection{Arrays and Matrices} \label{sec:arraymat}

To typeset \textbf{arrays}, use the \ei{array} environment. It works
somewhat similar to the \texttt{tabular} environment. The \verb|\\| command is
used to break the lines:
\begin{example}
\begin{equation*}
 \mathbf{X} = \left( 
  \begin{array}{ccc}
   x_1 & x_2 & \ldots \\
   x_3 & x_4 & \ldots \\
   \vdots & \vdots & \ddots
  \end{array} \right)
\end{equation*}
\end{example}


The \ei{array} environment can also be used to typeset \wi{piecewise function}s by
using a ``\verb|.|'' as an invisible \ci{right} delimiter:\footnote{If you want
  to typeset a lot of constructions like these, the \ei{cases} environment from
  \textsf{amsmath} simplifies the syntax, so it is worth a look.}  
\begin{example}
\begin{equation*}
|x| = \left\{
 \begin{array}{rl}
  -x & \text{if } x < 0\\
   0 & \text{if } x = 0\\
   x & \text{if } x > 0
 \end{array} \right.
\end{equation*}
\end{example}



\ei{array} can be used to typeset matrices\index{matrix} as well, but
\pai{amsmath} provides a better solution using the different \ei{matrix}
environments. There are six versions with different delimiters: \ei{matrix}
(none), \ei{pmatrix} $($, \ei{bmatrix} $[$, \ei{Bmatrix} $\{$, \ei{vmatrix} $\vert$ and
\ei{Vmatrix} $\Vert$. You don't have to specify the number of columns as with
\ei{array}. The maximum number is 10, but it is customisable (though it is not
very often you need 10 columns!):
\begin{example}
\begin{equation*}
 \begin{matrix} 
   1 & 2 \\
   3 & 4 
 \end{matrix} \qquad
 \begin{bmatrix} 
   1 & 2 & 3 \\
   4 & 5 & 6 \\ 
   7 & 8 & 9
 \end{bmatrix}
\end{equation*}
\end{example}



\section{Spacing in Math Mode} \label{sec:math-spacing}

\index{math spacing} If the spacing within formulae chosen by \LaTeX{}
is not satisfactory, it can be adjusted by inserting special
spacing commands: \ci{,} for
$\frac{3}{18}\:\textrm{quad}$ (\demowidth{0.166em}), \ci{:} for $\frac{4}{18}\:
\textrm{quad}$ (\demowidth{0.222em}) and \ci{;} for $\frac{5}{18}\:
\textrm{quad}$ (\demowidth{0.277em}).  The escaped space character
\verb*|\ | generates a medium sized space comparable to the interword spacing and \ci{quad}
(\demowidth{1em}) and \ci{qquad} (\demowidth{2em}) produce large
spaces. The size of a \ci{quad} corresponds to the width of the
character `M' of the current font. \verb|\!|\cih{"!} produces a
negative space of $-\frac{3}{18}\:\textrm{quad}$ ($-$\demowidth{0.166em}).

Note that `d' in the differential is conventionally set in roman:
\begin{example}
\begin{equation*}
 \int_1^2 \ln x \mathrm{d}x \qquad
 \int_1^2 \ln x \,\mathrm{d}x
\end{equation*}
\end{example}


In the next example, we define a new command \ci{ud} which produces
``$\,\mathrm{d}$'' (notice the spacing \demowidth{0.166em} before the
$\text{d}$), so we don't have to write it every time. The \ci{newcommand} is
placed in the preamble. %  More on
% \ci{newcommand} in section~\ref{} on page \pageref{}. To Do: Add label and
% reference to "Customising LaTeX" -> "New Commands, Environments and Packages"
% -> "New Commands".
\begin{example}
\newcommand{\ud}{\,\mathrm{d}}

\begin{equation*}
 \int_a^b f(x)\ud x 
\end{equation*}
\end{example}

If you want to typeset multiple integrals, you'll discover that the spacing
between the integrals is too wide. You can correct it using \ci{!}, but
\pai{amsmath} provides an easier way for fine-tuning
the spacing, namely the \ci{iint}, \ci{iiint}, \ci{iiiint}, and \ci{idotsint}
commands.

\begin{example}
\newcommand{\ud}{\,\mathrm{d}}

\[ \int\int f(x)g(y) 
                  \ud x \ud y \]
\[ \int\!\!\!\int 
         f(x)g(y) \ud x \ud y \]
\[ \iint f(x)g(y) \ud x \ud y \]
\end{example}

See the electronic document \texttt{testmath.tex} (distributed with
\AmS-\LaTeX) or Chapter 8 of \companion{} for further details.

\subsection{Phantoms}

When vertically aligning text using \verb|^| and \verb|_| \LaTeX{} is sometimes
just a little too helpful. Using the \ci{phantom} command you can
reserve space for characters that do not show up in the final output.
The easiest way to understand this is to look at an example:
\begin{example}
\begin{equation*}
{}^{14}_{6}\text{C}
\qquad \text{versus} \qquad
{}^{14}_{\phantom{1}6}\text{C}
\end{equation*}
\end{example}
If you want to typeset a lot of isotopes as in the example, the \pai{mhchem}
package is very useful for typesetting isotopes and chemical formulae too.


\section{Fiddling with the Math Fonts}\label{sec:fontsz}
Different math fonts are listed on Table~\ref{mathalpha} on page
\pageref{mathalpha}.
\begin{example}
 $\Re \qquad
  \mathcal{R} \qquad
  \mathfrak{R} \qquad
  \mathbb{R} \qquad $  
\end{example}
The last two require \pai{amssymb} or \pai{amsfonts}.

Sometimes you need to tell \LaTeX{} the correct font
size. In math mode, this is set with the following four commands:
\begin{flushleft}
\ci{displaystyle}~($\displaystyle 123$),
 \ci{textstyle}~($\textstyle 123$), 
\ci{scriptstyle}~($\scriptstyle 123$) and
\ci{scriptscriptstyle}~($\scriptscriptstyle 123$).
\end{flushleft}

If $\sum$ is placed in a fraction, it'll be typeset in text style unless you tell
\LaTeX{} otherwise:
\begin{example}
\begin{equation*}
 P = \frac{\displaystyle{ 
   \sum_{i=1}^n (x_i- x)
   (y_i- y)}} 
   {\displaystyle{\left[
   \sum_{i=1}^n(x_i-x)^2
   \sum_{i=1}^n(y_i- y)^2
   \right]^{1/2}}}
\end{equation*}    
\end{example}
Changing styles generally affects the way big operators and limits are displayed.

% This is not a math accent, and no maths book would be set this way.
% mathop gets the spacing right.


\subsection{Bold Symbols}
\index{bold symbols}

It is quite difficult to get bold symbols in \LaTeX{}; this is
probably intentional as amateur typesetters tend to overuse them.  The
font change command \verb|\mathbf| gives bold letters, but these are
roman (upright) whereas mathematical symbols are normally italic, and
furthermore it doesn't work on lower case Greek letters.
There is a \ci{boldmath} command, but \emph{this can only be used
outside math mode}. It works for symbols too, though:
\begin{example}
$\mu, M \qquad 
\mathbf{\mu}, \mathbf{M}$
\qquad \boldmath{$\mu, M$}
\end{example}

The package \pai{amsbsy} (included by \pai{amsmath}) as well as the
\pai{bm} from the \texttt{tools} bundle make this much easier as they include
a \ci{boldsymbol} command:

\begin{example}
$\mu, M \qquad
\boldsymbol{\mu}, \boldsymbol{M}$
\end{example}


\section{Theorems, Lemmas, \ldots}

When writing mathematical documents, you probably need a way to
typeset ``Lemmas'', ``Definitions'', ``Axioms'' and similar
structures.
\begin{lscommand}
\ci{newtheorem}\verb|{|\emph{name}\verb|}[|\emph{counter}\verb|]{|%
         \emph{text}\verb|}[|\emph{section}\verb|]|
\end{lscommand}
The \emph{name} argument is a short keyword used to identify the
``theorem''. With the \emph{text} argument you define the actual name
of the ``theorem'', which will be printed in the final document.

The arguments in square brackets are optional. They are both used to
specify the numbering used on the ``theorem''. Use  the \emph{counter}
argument to specify the \emph{name} of a previously declared
``theorem''. The new ``theorem'' will then be numbered in the same
sequence.  The \emph{section} argument allows you to specify the
sectional unit within which the ``theorem'' should get its numbers.

After executing the \ci{newtheorem} command in the preamble of your
document, you can use the following command within the document.
\begin{code}
\verb|\begin{|\emph{name}\verb|}[|\emph{text}\verb|]|\\
This is my interesting theorem\\
\verb|\end{|\emph{name}\verb|}|     
\end{code}

The \pai{amsthm} package (part of \AmS-\LaTeX) provides the 
\ci{theoremstyle}\verb|{|\emph{style}\verb|}|
command which lets you define what the theorem is all about by picking
from three predefined styles: \texttt{definition} (fat title, roman body),
\texttt{plain} (fat title, italic body) or \texttt{remark} (italic
title, roman body).

This should be enough theory. The following examples should
remove any remaining doubt, and make it clear that the
\verb|\newtheorem| environment is way too complex to understand.

% actually define things
\theoremstyle{definition} \newtheorem{law}{Law}
\theoremstyle{plain}      \newtheorem{jury}[law]{Jury}
\theoremstyle{remark}     \newtheorem*{marg}{Margaret}

First define the theorems:

\begin{verbatim}
\theoremstyle{definition} \newtheorem{law}{Law}
\theoremstyle{plain}      \newtheorem{jury}[law]{Jury}
\theoremstyle{remark}     \newtheorem*{marg}{Margaret}
\end{verbatim}

\begin{example}
\begin{law} \label{law:box}
Don't hide in the witness box
\end{law}
\begin{jury}[The Twelve]
It could be you! So beware and
see law~\ref{law:box}.\end{jury}
\begin{marg}No, No, No\end{marg}
\end{example}

The ``Jury'' theorem uses the same counter as the ``Law''
theorem, so it gets a number that is in sequence with
the other ``Laws''. The argument in square brackets is used to specify 
a title or something similar for the theorem.
\begin{example}
\newtheorem{mur}{Murphy}[section]

\begin{mur} If there are two or 
more ways to do something, and 
one of those ways can result in
a catastrophe, then someone 
will do it.\end{mur}
\end{example}

The ``Murphy'' theorem gets a number that is linked to the number of
the current section. You could also use another unit, for example chapter or
subsection.

The \pai{amsthm} package also provides the \ei{proof} environment.

\begin{example}
\begin{proof}
 Trivial, use
\[E=mc^2\]
\end{proof}
\end{example}

With the command \ci{qedhere} you can move the `end of proof' symbol
around for situations where it would end up alone on a line.

\begin{example}
\begin{proof}
 Trivial, use
\[E=mc^2 \qedhere\]
\end{proof}
\end{example}

If you want to customize your theorems down to the last dot, the
\pai{ntheorem} package offers a plethora of options.


%

% Local Variables:
% TeX-master: "lshort"
% mode: latex
% mode: flyspell
% End:
 
%%%%%%%%%%%%%%%%%%%%%%%%%%%%%%%%%%%%%%%%%%%%%%%%%%%%%%%%%%%%%%%%%
% Contents: TeX and LaTeX and AMS symbols for Maths
% $Id: lssym.tex,v 1.2 2003/03/19 20:57:46 oetiker Exp $
%%%%%%%%%%%%%%%%%%%%%%%%%%%%%%%%%%%%%%%%%%%%%%%%%%%%%%%%%%%%%%%%%


\section{Liste des symboles math�matiques}  \label{symbols}

\index{symboles!math�matiques}
 
Les tableaux suivants montrent tous les symboles accessibles en mode
\emph{math�matique}.

%
% Conditional Text in case the AMS Fonts are installed
%
Pour utiliser des symboles pr�sents dans les tables~\ref{AMSD}
�~\ref{AMSNBR}\,\footnote{Ces tables sont d�riv�es du
fichier \texttt{symbols.tex} de David~Carlisle et modifi�es selon les
suggestions de Josef~Tkadlec}, l'extension \pai{amssymb} doit �tre
charg�e dans le pr�ambule du document et les polices math�matiques de
l'AMS doivent �tre install�es sur votre syst�me. Si les extensions et
les polices de l'AMS ne sont pas install�es sur votre syst�me, vous
pouvez les r�cup�rer sur\\
+\CTANref|macros/latex/required/amslatex|. Il existe une
liste beaucoup plus compl�te de symboles sur
 \CTANref|info/symbols/comprehensive|.
 
\begin{table}[!h]
\caption{Accents en mode math�matique}  \label{mathacc}
\begin{symbols}{*4{cl}}
\W{\hat}{a}     & \W{\check}{a} & \W{\tilde}{a} & \W{\acute}{a} \\
\W{\grave}{a} & \W{\dot}{a} & \W{\ddot}{a}    & \W{\breve}{a} \\
\W{\bar}{a} &\W{\vec}{a} &\W{\widehat}{A}&\W{\widetilde}{A}\\  
\end{symbols}
\end{table}
 
\begin{table}[!h]
\caption{Alphabet grec minuscule}
\begin{symbols}{*4{cl}}
 \X{\alpha}     & \X{\theta}     & \X{o}          & \X{\upsilon}  \\
 \X{\beta}      & \X{\vartheta}  & \X{\pi}        & \X{\phi}      \\
 \X{\gamma}     & \X{\iota}      & \X{\varpi}     & \X{\varphi}   \\
 \X{\delta}     & \X{\kappa}     & \X{\rho}       & \X{\chi}      \\
 \X{\epsilon}   & \X{\lambda}    & \X{\varrho}    & \X{\psi}      \\
 \X{\varepsilon}& \X{\mu}        & \X{\sigma}     & \X{\omega}    \\
 \X{\zeta}      & \X{\nu}        & \X{\varsigma}  & &             \\
 \X{\eta}       & \X{\xi}        & \X{\tau} 
\end{symbols}
\end{table}

\begin{table}[!h]
\caption{Alphabet grec majuscule}
\begin{symbols}{*4{cl}}
 \X{\Gamma}     & \X{\Lambda}    & \X{\Sigma}     & \X{\Psi}      \\
 \X{\Delta}     & \X{\Xi}        & \X{\Upsilon}   & \X{\Omega}    \\
 \X{\Theta}     & \X{\Pi}        & \X{\Phi} 
\end{symbols}
\end{table}
\clearpage 

\begin{table}[!tbp]
\caption{Relations binaires}
\bigskip
Vous pouvez produire la n�gation de ces symboles en les pr�fixant par
la commande \ci{not}.
\begin{symbols}{*3{cl}}
 \X{<}           & \X{>}           & \X{=}          \\
 \X{\leq}ou \verb|\le|   & \X{\geq}ou \verb|\ge|   & \X{\equiv}     \\
 \X{\ll}         & \X{\gg}         & \X{\doteq}     \\
 \X{\prec}       & \X{\succ}       & \X{\sim}       \\
 \X{\preceq}     & \X{\succeq}     & \X{\simeq}     \\
 \X{\subset}     & \X{\supset}     & \X{\approx}    \\
 \X{\subseteq}   & \X{\supseteq}   & \X{\cong}      \\
 \X{\sqsubset}$^a$ & \X{\sqsupset}$^a$ & \X{\Join}$^a$    \\
 \X{\sqsubseteq} & \X{\sqsupseteq} & \X{\bowtie}    \\
 \X{\in}         & \X{\ni}, \verb|\owns|  & \X{\propto}    \\
 \X{\vdash}      & \X{\dashv}      & \X{\models}    \\
 \X{\mid}        & \X{\parallel}   & \X{\perp}      \\
 \X{\smile}      & \X{\frown}      & \X{\asymp}     \\
 \X{:}           & \X{\notin}      & \X{\neq}ou \verb|\ne|
\end{symbols}
\end{table}

\begin{table}[!tbp]
\caption{Op�rateurs binaires}
\begin{symbols}{*3{cl}}
 \X{+}              & \X{-}              & &                 \\
 \X{\pm}            & \X{\mp}            & \X{\triangleleft} \\
 \X{\cdot}          & \X{\div}           & \X{\triangleright}\\
 \X{\times}         & \X{\setminus}      & \X{\star}         \\
 \X{\cup}           & \X{\cap}           & \X{\ast}          \\
 \X{\sqcup}         & \X{\sqcap}         & \X{\circ}         \\
 \X{\vee}, \verb|\lou|     & \X{\wedge}, \verb|\land|  & \X{\bullet}       \\
 \X{\oplus}         & \X{\ominus}        & \X{\diamond}      \\
 \X{\odot}          & \X{\oslash}        & \X{\uplus}        \\
 \X{\otimes}        & \X{\bigcirc}       & \X{\amalg}        \\
 \X{\bigtriangleup} &\X{\bigtriangledown}& \X{\dagger}       \\
 \X{\lhd}$^a$         & \X{\rhd}$^a$         & \X{\ddagger}      \\
 \X{\unlhd}$^a$       & \X{\unrhd}$^a$       & \X{\wr}
\end{symbols}
\centerline{\footnotesize $^a$Utilisez l'extension \textsf{latexsym}
pour avoir acc�s � ces symboles}
\end{table}

\begin{table}[!tbp]
\caption{Op�rateurs n-aires}
\begin{symbols}{*4{cl}}
 \X{\sum}      & \X{\bigcup}   & \X{\bigvee}   & \X{\bigoplus}\\
 \X{\prod}     & \X{\bigcap}   & \X{\bigwedge} &\X{\bigotimes}\\
 \X{\coprod}   & \X{\bigsqcup} & &             & \X{\bigodot} \\
 \X{\int}      & \X{\oint}     & &             & \X{\biguplus}
\end{symbols}
 
\end{table}


\begin{table}[!tbp]
\caption{Fl�ches}
\begin{symbols}{*3{cl}}
 \X{\leftarrow}ou \verb|\gets|& \X{\longleftarrow} & \X{\uparrow}          \\
 \X{\rightarrow}ou \verb|\to|& \X{\longrightarrow} & \X{\downarrow}        \\
 \X{\leftrightarrow}    & \X{\longleftrightarrow}& \X{\updownarrow}      \\
 \X{\Leftarrow}         & \X{\Longleftarrow}     & \X{\Uparrow}          \\
 \X{\Rightarrow}        & \X{\Longrightarrow}    & \X{\Downarrow}        \\
 \X{\Leftrightarrow}    & \X{\Longleftrightarrow}& \X{\Updownarrow}      \\
 \X{\mapsto}            & \X{\longmapsto}        & \X{\nearrow}          \\
 \X{\hookleftarrow}     & \X{\hookrightarrow}    & \X{\searrow}          \\
 \X{\leftharpoonup}     & \X{\rightharpoonup}    & \X{\swarrow}          \\
 \X{\leftharpoondown}   & \X{\rightharpoondown}  & \X{\nwarrow}          \\
 \X{\rightleftharpoons} & \X{\iff}(plus d'espace)& \X{\leadsto}$^a$

\end{symbols}
\centerline{\footnotesize $^a$Utilisez l'extension \textsf{latexsym}
pour obtenir ces symboles}
\end{table}

\begin{table}[!tbp]
\caption{D�limiteurs}\label{tab:delimiters}
\begin{symbols}{*4{cl}}
 \X{(}            & \X{)}            & \X{\uparrow} & \X{\Uparrow}    \\
 \X{[}ou \verb|\lbrack|   & \X{]}ou \verb|\rbrack|  & \X{\downarrow}   & \X{\Downarrow}  \\
 \X{\{}ou \verb|\lbrace|  & \X{\}}ou \verb|\rbrace|  & \X{\updownarrow} & \X{\Updownarrow}\\
 \X{\langle}      & \X{\rangle}  & \X{|}ou \verb|\vert| &\X{\|}ou \verb|\Vert|\\
 \X{\lfloor}      & \X{\rfloor}      & \X{\lceil}       & \X{\rceil}      \\
 \X{/}            & \X{\backslash}   & &
\end{symbols}
\end{table}

\begin{table}[!tbp]
\caption{Grands d�limiteurs}
\begin{symbols}{*4{cl}}
 \Y{\lgroup}      & \Y{\rgroup}      & \Y{\lmoustache}  & \Y{\rmoustache} \\
 \Y{\arrowvert}   & \Y{\Arrowvert}   & \Y{\bracevert} 
\end{symbols}
\end{table}


\begin{table}[!tbp]
\caption{Symboles divers}
\begin{symbols}{*4{cl}}
 \X{\dots}       & \X{\cdots}      & \X{\vdots}      & \X{\ddots}     \\
 \X{\hbar}       & \X{\imath}      & \X{\jmath}      & \X{\ell}       \\
 \X{\Re}         & \X{\Im}         & \X{\aleph}      & \X{\wp}        \\
 \X{\forall}     & \X{\exists}     & \X{\mho}$^a$      & \X{\partial}   \\
 \X{'}           & \X{\prime}      & \X{\emptyset}   & \X{\infty}     \\
 \X{\nabla}      & \X{\triangle}   & \X{\Box}$^a$     & \X{\Diamond}$^a$ \\
 \X{\bot}        & \X{\top}        & \X{\angle}      & \X{\surd}      \\
\X{\diamondsuit} & \X{\heartsuit}  & \X{\clubsuit}   & \X{\spadesuit} \\
 \X{\neg}ou \verb|\lnot| & \X{\flat}       & \X{\natural}    & \X{\sharp}

\end{symbols}
\centerline{\footnotesize $^a$Utilisez l'extension \textsf{latexsym}
pour obtenir ces symboles}
\end{table}

\begin{table}[!tbp]
\caption{Symboles non-math�matiques}
\bigskip
Ces symboles peuvent �galement �tre utilis�s en mode \emph{texte}.
\begin{symbols}{*4{cl}}
 \SC{\dag}  &  \SC{\S}  &  \SC{\copyright} &  \SC{\textregistered}  \\
 \SC{\ddag} &  \SC{\P}  &  \SC{\pounds}    &  \SC{\%}               \\
\end{symbols}
\end{table}

%
%
% If the AMS Stuff is not available, we drop out right here :-)
%

\begin{table}[!tbp]
\caption{D�limiteurs de l'AMS}\label{AMSD}
\bigskip
\begin{symbols}{*4{cl}}
\X{\ulcorner}&\X{\urcorner}&\X{\llcorner}&\X{\lrcorner}
\end{symbols}
\end{table}

\begin{table}[!tbp]
\caption{Caract�res grecs et h�breux de l'AMS}
\begin{symbols}{*5{cl}}
\X{\digamma}     &\X{\varkappa} & \X{\beth} &\X{\gimel} & \X{\daleth}
\end{symbols}
\end{table}

\begin{table}[!tbp]
\caption{Relations binaires de l'AMS}
\begin{symbols}{*3{cl}}
 \X{\lessdot}           & \X{\gtrdot}            & \X{\doteqdot}ou \verb|\Doteq| \\
 \X{\leqslant}          & \X{\geqslant}          & \X{\risingdotseq}     \\
 \X{\eqslantless}       & \X{\eqslantgtr}        & \X{\fallingdotseq}    \\
 \X{\leqq}              & \X{\geqq}              & \X{\eqcirc}           \\
 \X{\lll}ou \verb|\llless|      & \X{\ggg}ou \verb|\gggtr| & \X{\circeq}  \\
 \X{\lesssim}           & \X{\gtrsim}            & \X{\triangleq}        \\
 \X{\lessapprox}        & \X{\gtrapprox}         & \X{\bumpeq}           \\
 \X{\lessgtr}           & \X{\gtrless}           & \X{\Bumpeq}           \\
 \X{\lesseqgtr}         & \X{\gtreqless}         & \X{\thicksim}         \\
 \X{\lesseqqgtr}        & \X{\gtreqqless}        & \X{\thickapprox}      \\
 \X{\preccurlyeq}       & \X{\succcurlyeq}       & \X{\approxeq}         \\
 \X{\curlyeqprec}       & \X{\curlyeqsucc}       & \X{\backsim}          \\
 \X{\precsim}           & \X{\succsim}           & \X{\backsimeq}        \\
 \X{\precapprox}        & \X{\succapprox}        & \X{\vDash}            \\
 \X{\subseteqq}         & \X{\supseteqq}         & \X{\Vdash}            \\
 \X{\Subset}            & \X{\Supset}            & \X{\Vvdash}           \\
 \X{\sqsubset}          & \X{\sqsupset}          & \X{\backepsilon}      \\
 \X{\therefore}         & \X{\because}           & \X{\varpropto}        \\
 \X{\shortmid}          & \X{\shortparallel}     & \X{\between}          \\
 \X{\smallsmile}        & \X{\smallfrown}        & \X{\pitchfork}        \\
 \X{\vartriangleleft}   & \X{\vartriangleright}  & \X{\blacktriangleleft}\\
 \X{\trianglelefteq}    & \X{\trianglerighteq}   &\X{\blacktriangleright}
\end{symbols}
\end{table}

\begin{table}[!tbp]
\caption{Fl�ches de l'AMS}
\begin{symbols}{*3{cl}}
 \X{\dashleftarrow}      & \X{\dashrightarrow}     & \X{\multimap}          \\
 \X{\leftleftarrows}     & \X{\rightrightarrows}   & \X{\upuparrows}        \\
 \X{\leftrightarrows}    & \X{\rightleftarrows}    & \X{\downdownarrows}    \\
 \X{\Lleftarrow}         & \X{\Rrightarrow}        & \X{\upharpoonleft}     \\
 \X{\twoheadleftarrow}   & \X{\twoheadrightarrow}  & \X{\upharpoonright}    \\
 \X{\leftarrowtail}      & \X{\rightarrowtail}     & \X{\downharpoonleft}   \\
 \X{\leftrightharpoons}  & \X{\rightleftharpoons}  & \X{\downharpoonright}  \\
 \X{\Lsh}                & \X{\Rsh}                & \X{\rightsquigarrow}   \\
 \X{\looparrowleft}      & \X{\looparrowright}     &\X{\leftrightsquigarrow}\\
 \X{\curvearrowleft}     & \X{\curvearrowright}    & &                      \\
 \X{\circlearrowleft}    & \X{\circlearrowright}   & &
\end{symbols}
\end{table}

\begin{table}[!tbp]
\caption{N�gations des relations binaires et des fl�ches de l'AMS}\label{AMSNBR}
\begin{symbols}{*3{cl}}
 \X{\nless}           & \X{\ngtr}            & \X{\varsubsetneqq}  \\
 \X{\lneq}            & \X{\gneq}            & \X{\varsupsetneqq}  \\
 \X{\nleq}            & \X{\ngeq}            & \X{\nsubseteqq}     \\
 \X{\nleqslant}       & \X{\ngeqslant}       & \X{\nsupseteqq}     \\
 \X{\lneqq}           & \X{\gneqq}           & \X{\nmid}           \\
 \X{\lvertneqq}       & \X{\gvertneqq}       & \X{\nparallel}      \\
 \X{\nleqq}           & \X{\ngeqq}           & \X{\nshortmid}      \\
 \X{\lnsim}           & \X{\gnsim}           & \X{\nshortparallel} \\
 \X{\lnapprox}        & \X{\gnapprox}        & \X{\nsim}           \\
 \X{\nprec}           & \X{\nsucc}           & \X{\ncong}          \\
 \X{\npreceq}         & \X{\nsucceq}         & \X{\nvdash}         \\
 \X{\precneqq}        & \X{\succneqq}        & \X{\nvDash}         \\
 \X{\precnsim}        & \X{\succnsim}        & \X{\nVdash}         \\
 \X{\precnapprox}     & \X{\succnapprox}     & \X{\nVDash}         \\
 \X{\subsetneq}       & \X{\supsetneq}       & \X{\ntriangleleft}  \\
 \X{\varsubsetneq}    & \X{\varsupsetneq}    & \X{\ntriangleright} \\
 \X{\nsubseteq}       & \X{\nsupseteq}       & \X{\ntrianglelefteq}\\
 \X{\subsetneqq}      & \X{\supsetneqq}      &\X{\ntrianglerighteq}\\[0.5ex]
 \X{\nleftarrow}      & \X{\nrightarrow}     & \X{\nleftrightarrow}\\
 \X{\nLeftarrow}      & \X{\nRightarrow}     & \X{\nLeftrightarrow}

\end{symbols}
\end{table}

\begin{table}[!tbp]
\caption{Op�rateurs binaires de l'AMS}
\begin{symbols}{*3{cl}}
 \X{\dotplus}        & \X{\centerdot}      & \X{\intercal}      \\
 \X{\ltimes}         & \X{\rtimes}         & \X{\divideontimes} \\
 \X{\Cup}ou \verb|\doublecup|& \X{\Cap}ou \verb|\doublecap|& \X{\smallsetminus} \\
 \X{\veebar}         & \X{\barwedge}       & \X{\doublebarwedge}\\
 \X{\boxplus}        & \X{\boxminus}       & \X{\circleddash}   \\
 \X{\boxtimes}       & \X{\boxdot}         & \X{\circledcirc}   \\
 \X{\leftthreetimes} & \X{\rightthreetimes}& \X{\circledast}    \\
 \X{\curlyvee}       & \X{\curlywedge}  
\end{symbols}
\end{table}

\begin{table}[!tbp]
\caption{Symboles divers de l'AMS}
\begin{symbols}{*3{cl}}
 \X{\hbar}             & \X{\hslash}           & \X{\Bbbk}            \\
 \X{\square}           & \X{\blacksquare}      & \X{\circledS}        \\
 \X{\vartriangle}      & \X{\blacktriangle}    & \X{\complement}      \\
 \X{\triangledown}     &\X{\blacktriangledown} & \X{\Game}            \\
 \X{\lozenge}          & \X{\blacklozenge}     & \X{\bigstar}         \\
 \X{\angle}            & \X{\measuredangle}    & \X{\sphericalangle}  \\
 \X{\diagup}           & \X{\diagdown}         & \X{\backprime}       \\
 \X{\nexists}          & \X{\Finv}             & \X{\varnothing}      \\
 \X{\eth}              & \X{\mho}              
\end{symbols}
\end{table}



\begin{table}[!tbp]
\caption{Polices math�matiques}
\begin{symbols}{@{}*3l@{}}
Exemple& Commande & Extension � utiliser\\
\hline
\rule{0pt}{1.05em}$\mathrm{ABCDE abcde 1234}$
        & \verb|\mathrm{ABCDE abcde 1234}|
        &       \\
$\mathit{ABCDE abcde 1234}$
        & \verb|\mathit{ABCDE abcde 1234}|
        &       \\
$\mathnormal{ABCDE abcde 1234}$
        & \verb|\mathnormal{ABCDE abcde 1234}|
        &  \\
$\mathcal{ABCDE abcde 1234}$
        & \verb|\mathcal{ABCDE abcde 1234}|
%SC: to check
%        & \pai{euscript} avec l'option\,: \texttt{mathcal} \\
        &  \\
$\mathscr{ABCDE abcde 1234}$
        &\verb|\mathscr{ABCDE abcde 1234}|
        &\pai{mathrsfs}\\
$\mathfrak{ABCDE abcde 1234}$
        & \verb|\mathfrak{ABCDE abcde 1234}|
        &\pai{amsfonts} ou \textsf{amssymb}  \\
%SC: to check
%        &\pai{eufrak}                \\
$\mathbb{ABCDE abcde 1234}$
        & \verb|\mathbb{ABCDE abcde 1234}|
        &\pai{amsfonts} ou \pai{amssymb} \\
\end{symbols}
\end{table}


\endinput

%

% Local Variables:
% TeX-master: "lshort2e"
% mode: latex
% mode: flyspell
% End:

%%%%%%%%%%%%%%%%%%%%%%%%%%%%%%%%%%%%%%%%%%%%%%%%%%%%%%%%%%%%%%%%%
% Contents: Specialities of the LaTeX system
% $Id: spec.tex,v 1.1.1.1 2002/02/26 10:04:21 oetiker Exp $
%%%%%%%%%%%%%%%%%%%%%%%%%%%%%%%%%%%%%%%%%%%%%%%%%%%%%%%%%%%%%%%%%
 
\chapter{Specialities}
\begin{intro}
  When putting together a large document, \LaTeX{} will help you
  with some special features like index generation,
  bibliography management, and other things.
  A much more complete description of specialities and
  enhancements possible with \LaTeX{} can be found in the
  {\normalfont\manual{}} and {\normalfont \companion}.
\end{intro}

\section{Including EPS Graphics}\label{eps}
\LaTeX{} provides the basic facilities to work with floating bodies,
such as images or graphics, with the \texttt{figure} and
\texttt{table} environments.

There are also several ways to generate the actual
\wi{graphics} with basic \LaTeX{} or a \LaTeX{} extension package,
but most users find them quite difficult to understand, so
this manual will not explain them.
Please refer to \companion{} and the \manual{} for more information on
that subject.

A much easier way to get graphics into a document is to generate them
with a specialised software package\footnote{Such as XFig, CorelDraw!,
  Freehand, Gnuplot, \ldots} and then include the finished graphics
into the document. Here again, \LaTeX{} packages offer many ways to do
this, but this introduction will only discuss the use of \wi{Encapsulated
  PostScript} (EPS) graphics, because it is quite
easy to do and widely used.  In order to use pictures in the EPS
format, you must have a \wi{PostScript} printer\footnote{Another
  possibility to output PostScript is the \textsc{\wi{GhostScript}}
  program available from
  \texttt{CTAN:/tex-archive/support/ghostscript}. Windows and OS/2 users might
  want to look for \textsc{GSview}.} available for output.

A good set of commands for inclusion of graphics is provided in the
\pai{graphicx} package by D.~P.~Carlisle. It is part of a whole family
of packages called the ``graphics''
bundle.\footnote{\texttt{CTAN:/tex-archive/macros/latex/required/graphics}}

\newpage
Assuming you are working on a system with a
PostScript printer available for output and with the \textsf{graphicx}
package installed, you can use the following step by step guide to
include a picture into your document:

\begin{enumerate}
\item Export the picture from your graphics program in EPS
  format.\footnote{If your software can not export into EPS format, you
    can try to install a PostScript printer driver (such as an Apple
    LaserWriter, for example) and then print to a file with this
    driver. With some luck this file will be in EPS format. Note that
    an EPS must not contain more than one page. Some printer drivers
    can be explicitly configured to produce EPS format.}
\item Load the \textsf{graphicx} package in the preamble of the input
  file with
\begin{lscommand}
\verb|\usepackage[|\emph{driver}\verb|]{graphicx}|
\end{lscommand}
where \emph{driver} is the name of your ``dvi to postscript''
converter program. The most
widely used program is called \texttt{dvips}. The name of the driver is
required, because there is no standard on how graphics are included in
\TeX{}. Knowing the name of the \emph{driver}, the
\textsf{graphicx} package can choose the correct method to insert
information about the graphics into the \eei{.dvi}~file, so that the
printer understands it and can correctly include the \eei{.eps} file.
\item Use the command 
\begin{lscommand}
\ci{includegraphics}\verb|[|\emph{key}=\emph{value}, \ldots\verb|]{|\emph{file}\verb|}|
\end{lscommand}
to include \emph{file} into your document. The optional parameter
accepts a comma separated list of \emph{keys} and associated
\emph{values}. The \emph{keys} can be used to alter the width, height
and rotation of the included graphic. Table~\ref{keyvals} lists the
most important keys.
\end{enumerate}

\begin{table}[htb]
\caption{Key Names for \textsf{graphicx} Package.}
\label{keyvals}
\begin{lined}{9cm}
\begin{tabular}{@{}ll}
\texttt{width}& scale graphic to the specified width\\
\texttt{height}& scale graphic to the specified height\\
\texttt{angle}& rotate graphic counterclockwise\\
\texttt{scale}& scale graphic \\
\end{tabular}

\bigskip
\end{lined}
\end{table}

\pagebreak
The following example code may help to clarify things:
\begin{code}
\begin{verbatim}
\begin{figure}
\begin{center}
\includegraphics[angle=90, width=0.5\textwidth]{test}
\end{center}
\end{figure}
\end{verbatim}
\end{code}
It includes the graphic stored in the file \texttt{test.eps}. The
graphic is \emph{first} rotated by an angle of 90 degrees and
\emph{then} scaled to the final width of 0.5 times the width of a
standard paragraph.  The aspect ratio is $1.0$, because no special
height is specified.  The width and height parameters can also be
specified in absolute dimensions. Refer to Table~\ref{units} on
page~\pageref{units} for more information. If you want to know more
about this topic, make sure to read \cite{graphics} and \cite{eps}.

\section{Bibliography}
 
You can produce a \wi{bibliography} with the \ei{thebibliography}
environment.  Each entry starts with
\begin{lscommand}
\ci{bibitem}\verb|[|\emph{label}\verb|]{|\emph{marker}\verb|}|
\end{lscommand}
The \emph{marker} is then used to cite the book, article or paper
within the document.
\begin{lscommand}
\ci{cite}\verb|{|\emph{marker}\verb|}|
\end{lscommand}
If you do not use the \emph{label} option, the entries will get enumerated
automatically.  The parameter after the \verb|\begin{thebibliography}|
command defines how much space to reserve for the number or labels. In the example below,
\verb|{99}| tells \LaTeX{} to expect that none of the bibliography item numbers will be wider
than the number 99.
\enlargethispage{2cm}
\begin{example}
Partl~\cite{pa} has 
proposed that \ldots 
\begin{thebibliography}{99}
\bibitem{pa} H.~Partl: 
\emph{German \TeX},
TUGboat Volume~9, Issue~1 (1988)
\end{thebibliography}
\end{example}

\chaptermark{Specialities} % w need to fix the damage done by the
                           %bibliography example.
\thispagestyle{fancyplain}

\newpage

For larger projects, you might want to check out the Bib\TeX{}
program. Bib\TeX{} is included with most \TeX{} distributions. It
allows you to maintain a bibliographic database and then extract the
references relevant to things you cited in your paper. The visual
presentation of Bib\TeX{} generated bibliographies is based on a style
sheets concept that allows you to create bibliographies following
a wide range of established designs.

\section{Indexing} \label{sec:indexing}
A very useful feature of many books is their \wi{index}. With \LaTeX{}
and the support program \texttt{makeindex},\footnote{On systems not
  necessarily supporting
  filenames longer than 8~characters, the program may be called
  \texttt{makeidx}.} an index can be generated quite easily.  This
introduction will only explain the basic index generation commands.
For a more in-depth view, please refer to \companion.  \index{makeindex
  program} \index{makeidx package}

To enable the indexing feature of \LaTeX{}, the \pai{makeidx} package
must be loaded in the preamble with:
\begin{lscommand}
\verb|\usepackage{makeidx}|
\end{lscommand}
\noindent and the special indexing commands must be enabled by putting 
the
\begin{lscommand}
  \ci{makeindex}
\end{lscommand}
\noindent command into the input file preamble.

The content of the index is specified with
\begin{lscommand}
  \ci{index}\verb|{|\emph{key}\verb|}|
\end{lscommand}
\noindent commands, where \emph{key} is the index entry. You enter the
index commands at the points in the text that you want the final index
entries to point to.  Table~\ref{index} explains the syntax of the
\emph{key} argument with several examples.

\begin{table}[!tp]
\caption{Index Key Syntax Examples.}
\label{index}
\begin{center}
\begin{tabular}{@{}lll@{}}
  \textbf{Example} &\textbf{Index Entry} &\textbf{Comment}\\\hline
  \rule{0pt}{1.05em}\verb|\index{hello}| &hello, 1 &Plain entry\\ 
\verb|\index{hello!Peter}|   &\hspace*{2ex}Peter, 3 &Subentry under `hello'\\ 
\verb|\index{Sam@\textsl{Sam}}|     &\textsl{Sam}, 2& Formatted entry\\ 
\verb|\index{Lin@\textbf{Lin}}|     &\textbf{Lin}, 7& Same as above\\ 
\verb.\index{Jenny|textbf}.     &Jenny, \textbf{3}& Formatted page number\\
\verb.\index{Joe|textit}.     &Joe, \textit{5}& Same as above\\
\verb.\index{eolienne@\'eolienne}.     &\'eolienne, 4& Handling of accents
\end{tabular}
\end{center}
\end{table}

When the input file is processed with \LaTeX{}, each \verb|\index|
command writes an appropriate index entry, together with the current
page number, to a special file. The file has the same name as the
\LaTeX{} input file, but a different extension (\verb|.idx|). This
\eei{.idx} file can then be processed with the \texttt{makeindex}
program.

\begin{lscommand}
  \texttt{makeindex} \emph{filename}
\end{lscommand}
The \texttt{makeindex} program generates a sorted index with the same base
file name, but this time with the extension \eei{.ind}. If now the
\LaTeX{} input file is processed again, this sorted index gets
included into the document at the point where \LaTeX{} finds
\begin{lscommand}
  \ci{printindex}
\end{lscommand}

The \pai{showidx} package that comes with \LaTeXe{} prints out all
index entries in the left margin of the text. This is quite useful for
proofreading a document and verifying the index.
   
\section{Fancy Headers}
\label{sec:fancy}

The \pai{fancyhdr} package,\footnote{Available from
  \texttt{CTAN:/tex-archive/macros/latex/contrib/supported/fancyhdr}.} written by
Piet van Oostrum, provides a few simple commands that allow you to
customize the header and footer lines of your document.  If you look
at the top of this page, you can see a possible application of this
package.

\begin{figure}[!htbp]
\begin{lined}{\textwidth}
\begin{verbatim}
\documentclass{book}
\usepackage{fancyhdr}
\pagestyle{fancy}
% with this we ensure that the chapter and section
% headings are in lowercase.
\renewcommand{\chaptermark}[1]{\markboth{#1}{}}
\renewcommand{\sectionmark}[1]{\markright{\thesection\ #1}}
\fancyhf{}  % delete current setting for header and footer
\fancyhead[LE,RO]{\bfseries\thepage}
\fancyhead[LO]{\bfseries\rightmark}
\fancyhead[RE]{\bfseries\leftmark}
\renewcommand{\headrulewidth}{0.5pt}
\renewcommand{\footrulewidth}{0pt}
\addtolength{\headheight}{0.5pt} % make space for the rule
\fancypagestyle{plain}{%
   \fancyhead{} % get rid of headers on plain pages
   \renewcommand{\headrulewidth}{0pt} % and the line
}
\end{verbatim}
\end{lined}
\caption{Example \pai{fancyhdr} Setup.} \label{fancyhdr}
\end{figure}

The tricky problem when customising headers and footers is to get
things like running section and chapter names in there. \LaTeX{}
accomplishes this with a two-stage approach. In the header and footer
definition, you use the commands \ci{rightmark} and \ci{leftmark} to
represent the current section and chapter heading, respectively.
The values of these two commands are overwritten whenever a chapter or
section command is processed. 

For ultimate flexibility, the \verb|\chapter| command and its friends
do not redefine \ci{rightmark} and \ci{leftmark} themselves. They call
yet another command (\ci{chaptermark}, \ci{sectionmark}, or
\ci{subsectionmark}) that is responsible for redefining \ci{rightmark}
and \ci{leftmark}.

If you want to change the look of the chapter
name in the header line, you need only ``renew'' the \ci{chaptermark}
command. \cih{sectionmark}\cih{subsectionmark}

 
Figure~\ref{fancyhdr} shows a possible setup for the \pai{fancyhdr}
package that makes the headers look about the same as they look in
this booklet. In any case, I suggest you fetch the documentation for
the package at the address mentioned in the footnote. 

\section{The Verbatim Package}

Earlier in this book, you got to know the \ei{verbatim}
\emph{environment}.  In this section, you are going to learn about the
\pai{verbatim} \emph{package}. The \pai{verbatim} package is basically
a re-implementation of the \ei{verbatim} environment that works around
some of the limitations of the original \ei{verbatim} environment.
This by itself is not spectacular, but the implementation of the
\pai{verbatim} package added new functionality, which is
why I am mentioning the package here. The \pai{verbatim}
package provides the

\begin{lscommand}
\ci{verbatiminput}\verb|{|\emph{filename}\verb|}|
\end{lscommand}

\noindent command, which allows you to include raw ASCII text into your
document as if it were inside a \ei{verbatim} environment.

As the \pai{verbatim} package is part of the `tools' bundle, you
should find it pre-installed on most systems. If you want to know more
about this package, make sure to read \cite{verbatim}.


\section{Downloading and Installing \LaTeX{} Packages}

Most \LaTeX{} installations come with a large set of pre-installed
style packages, but many more are available on the net. The main
place to look for style packages on the Internet is CTAN (\verb|http://www.ctan.org/|).

Packages such as \pai{geometry}, \pai{hyphenat}, and many
others are typically made up of two files: a file with the extension
\texttt{.ins} and another with the extension \texttt{.dtx}. There
will often be a \texttt{readme.txt} with a brief description of the
package. You should of course read this file first.

In any event, once you have copied the package files onto your
machine, you still have to process them in a way that (a) tells your
\TeX\ distribution about the new style package and (b) gives you
the documentation.  Here's how you do the first part:

\begin{enumerate}
\item Run \LaTeX{} on the \texttt{.ins} file. This will
  extract a \eei{.sty} file.
\item Move the \eei{.sty} file to a place where your distribution
  can find it. Usually this is in your \texttt{\ldots/\emph{localtexmf}/tex/latex}
  subdirectory (Windows or OS/2 users should feel free to change the
  direction of the slashes).
\item Refresh your distribution's file-name database. The command
  depends on the \LaTeX distribution you use:
  teTeX, fpTeX -- \texttt{texhash}; web2c -- \texttt{maktexlsr};
  MikTeX -- \texttt{initexmf -update-fndb} or use the GUI.
\end{enumerate}

\noindent Now you can extract the documentation from the
\texttt{.dtx} file:

\begin{enumerate}
\item Run \LaTeX\ on the \texttt{.dtx} file.  This will generate a
  \texttt{.dvi} file. Note that you may have to run \LaTeX\
  several times before it gets the cross-references right.
\item Check to see if \LaTeX\ has produced a \texttt{.idx} file
  among the various files you now have.
  If you do not see this file, then you may proceed to
  step~\ref{step:final}.
\item In order to generate the index, type the following:\\
        \fbox{\texttt{makeindex -s gind.ist \textit{name}}}\\
        (where \textit{name} stands for the main-file name without any
    extension).
 \item Run \LaTeX\ on the \texttt{.dtx} file once again. \label{step:next}
    
\item Last but not least, make a \texttt{.ps} or \texttt{.pdf}
  file to increase your reading pleasure.\label{step:final}
  
\end{enumerate}

Sometimes you will see that a \texttt{.glo}
(glossary) file has been produced. Run the following
command between
step~\ref{step:next} and~\ref{step:final}:

\noindent\texttt{makeindex -s gglo.ist -o \textit{name}.gls \textit{name}.glo}

\noindent Be sure to run \LaTeX\ on the \texttt{.dtx} one last
time before moving on to step~\ref{step:final}.


%%%%%%%%%%%%%%%%%%%%%%%%%%%%%%%%%%%%%%%%%%%%%%%%%%%%%%%%%%%%%%%%%
% Contents: Chapter on pdfLaTeX
% French original by Daniel Flipo 14/07/2001
%%%%%%%%%%%%%%%%%%%%%%%%%%%%%%%%%%%%%%%%%%%%%%%%%%%%%%%%%%%%%%%%%

\section{Working with pdf\LaTeX} \label{sec:pdftex}\index{PDF}
\secby{Daniel Flipo}{Daniel.Flipo@univ-lille1.fr}%
PDF is a \wi{hypertext} document format. Much like in a web page,
some words in the document are marked as hyperlinks. They link to other
places in the document or even to other documents. If you click
on such a hyperlink you get transported to the destination of the
link. In the context of \LaTeX{}, this means that all occurrences of
\ci{ref} and \ci{pageref} become hyperlinks. Additionally, the table
of contents, the index and all the other similar structures become
collections of hyperlinks.

Most web pages you find today are written in HTML \emph{(HyperText
  Markup Language)}. This format has two significant disadvantages
when writing scientific documents:
\begin{enumerate}
\item Including mathematical formulae into HTML documents is not
  generally supported. While there is a standard for it, most browsers
  used today do not support it, or lack the required fonts.
\item Printing HTML documents is possible, but the results vary widely
  between platforms and browsers. The results are miles removed from
  the quality we have come to expect in the \LaTeX{} world.
\end{enumerate}

There have been many attempts to create translators from \LaTeX{} to
HTML. Some were even quite successful in the sense that they are able
to produce legible web pages from a standard \LaTeX{} input file. But
all of them cut corners left and right to get the job done. As soon as
you start using more complex \LaTeX{} features and external packages
things tend to fall apart. Authors wishing to preserve the unique
typographic quality of their documents even when publishing on the web
turn to PDF \emph{(Portable Document Format)}, which preserves the 
layout of the document and permits hypertext
navigation. Most modern browsers come with plugins that allow the
direct display of PDF documents.

In contrast to the DVI and PS format, PDF documents can be displayed
and printed on most computer platforms (Unix, Mac, Windows), thanks to
the Adobe Acrobat Reader software, which can be downloaded freely from
Adobe's website. On many computers it even comes pre-installed with the base
OS distribution.

\subsection{PDF Documents for the Web}

The creation of a PDF file from \LaTeX{} source is very simple,
thanks to the pdf\TeX{} program developed by
H\`an~Th\'{\^e}~Th\`anh. pdf\TeX{} produces PDF output where normal
\TeX{} produces DVI. There is also a pdf\LaTeX{}, which produces
  PDF output from \LaTeX{} sources. \index{pdftex@pdf\TeX}\index{pdftex@pdf\LaTeX}

Both pdf\TeX{} and pdf\LaTeX{} are installed automatically by most
modern \TeX{} distributions, such as te\TeX{}, fp\TeX{},
Mik\TeX, \TeX{}Live and CMac\TeX{}.

To produce a PDF instead of DVI, it is sufficient to replace the
command \texttt{latex file.tex} by
\texttt{pdflatex file.tex}. On systems where \LaTeX{} is not called from the
command line, you may find a special button in the \TeX{}ControlCenter.

In \LaTeX{} you can define the the paper size with an 
optional documentclass argument such as \texttt{a4paper} or
\texttt{letterpaper}. This works in pdf\LaTeX{} too, but on top of this
pdf\TeX{} also needs to know the physical size of the paper and not just
the area to use for the layout.
\index{paper size}
If you use the
\pai{hyperref} package (see page \pageref{ssec:pdfhyperref}), the
papersize will be adjusted automatically. Otherwise you have to do this
manually by putting the following lines into the preamble of the document:
\begin{code}
\begin{verbatim}
\pdfpagewidth=\paperwidth
\pdfpageheight=\paperheight
\end{verbatim}
\end{code}

The following section will go into more detail regarding the differences
between normal \LaTeX{} and pdf\LaTeX{}. The main differences concern
three areas: the fonts to use, the format of images to include, and the
manual configuration of hyperlinks.

\subsection{The Fonts}

\wi{pdf\LaTeX} can deal with all sorts of fonts (PK bitmaps, TrueType,
PostScript type~1\dots) but prime \LaTeX{} font format, the bitmap PK
fonts produce very ugly results when the document is displayed
with Acrobat Reader. It is best to use PostScript Type 1 fonts
exclusively to produce documents that display well.

We have not yet talked about fonts in this book because \LaTeX{}
handles this on its own just fine, as it uses its own set of fonts well
adapted to the requirements of scientific publishing.
Actually, there are two sets of \TeX{} fonts: 
\emph{Computer Modern} (CM), consisting of 128 characters, which is the
default font set, and \emph{Extended Cork} (EC), made up from 256
characters. The speciality of the EC font set is that it contains
special characters for all the language-specific accented characters
used in European languages, such as
\"a or \`e. This allows hyphenation to work properly with words
that contain such special characters. You can enable the EC character
set by putting the command \verb+\usepackage[T1]{fontenc}+ into the
preamble of the document (see page \pageref{fontenc}).
Unfortunately, there is no free set of PostScript type~1 fonts
for the EC character set. Fortunately, there are two ways to cheat.
\begin{itemize}
\item You can put the line \verb+\usepackage{aeguill}+\paih{aeguill} into the
  preamble of your document, to use AE virtual fonts. 
\item You can use \verb+\usepackage{mltex}+, but this only works when
  your pdf\TeX{} has been compiled with the \wi{mltex} option.
\end{itemize}
The AE virtual fontset, like the {Ml\TeX} system, makes \TeX{} believe
it has a full 256 character fontset at its disposal by creating the
missing letters from characters available in the normal CM font, which
exists in a PostScript type~1 variant. This has the big advantage that
hyphenation works well in European languages. 
The only disadvantage of this approach is that the artificial AE
characters do not work with Acrobat Reader's 
\texttt{Find} function, so you cannot search for words with accented
characters in your final PDF file.

Another solution is not to use the CM fontset, but to switch to other
Postscript type~1 fonts. Actually, some of them are even included with
every copy of Acrobat Reader. Because these fonts have different
character sizes, the text layout on your pages will change. Generally
it will use more space than the CM fonts, which are very space-efficient. 
Also, the overall visual coherence of your document will
suffer because Times, Helvetica and Courier (the primary candidates
for such a replacement job) have not been designed to work in harmony in
a single document as has been done for the Computer Modern fonts.
 
Two ready-made font sets are available for this purpose:
\pai{pxfonts}, which is based on \emph{Palatino} as its main text body font,
and the \pai{txfonts} package, which is based on \emph{Times}. To use them it is
sufficient to put the following lines into the preamble of your
document:
\begin{code}
\begin{verbatim}
\usepackage[T1]{fontenc}
\usepackage{pxfonts}
\end{verbatim}
\end{code}

Note: you may find lines like
\begin{verbatim}
Warning: pdftex (file eurmo10): Font eurmo10 at ... not found
\end{verbatim}
in the  \texttt{.log} file after compiling your input file. They mean
that some font used in the document has not been found. You really have
to fix these problems, as the resulting PDF document may 
\emph{not display the pages with the missing characters at all}.

This whole font business, especially the lack of a good EC fontset 
equivalent in quality to the CM font in type~1 format,  is occupying the 
minds of many people, so new solutions are cropping up all the time.
\subsection{Using Graphics}
\label{ssec:pdfgraph}

Including graphics into a document works best with the 
\pai{graphicx} package (see page~\pageref{eps}). 
By using the special \emph{driver} option \texttt{pdftex} the
package will work with pdf\LaTeX{} as well:
\begin{code}
\begin{verbatim}
\usepackage[pdftex]{color,graphicx}
\end{verbatim}
\end{code}
In the sample above I have included the color option, as using color in
documents displayed on the web comes quite naturally.

So much for the good news. The bad news is that EPS (Encapsulated Postscript), your favorite format
for graphics in \LaTeX{}, does
\emph{not} work for PDF files. If you do not define a file extension
in the \ci{includegraphics} command, \pai{graphicx} will go
looking for a suitable file on its own, depending on the setting of
the \emph{driver} option. For \texttt{pdftex} this is
formats \texttt{.png}, \texttt{.pdf}, \texttt{.jpg}, \texttt{.mps} (MetaPost),
and \texttt{.tif}---but \emph{not} \texttt{.eps}.

The simple way out of this problem is to just convert your EPS files
into PDF format using the \texttt{epstopdf} utility found on many
systems. For vector graphics (drawings) this is a great solution. For
bitmaps (photos, scans) this is not ideal, because the PDF format
natively supports the inclusion of PNG and JPEG images. PNG is
good for screenshots and other images with few colors. JPEG is great
for photos, as it is very space-efficient.

It may even be desirable to not draw certain geometric figures, 
but rather describe the figure with a specialized command language, such as
\wi{MetaPost}, which can be found in most \TeX{} distributions, and
comes with its own extensive manual.

\subsection{Hypertext Links}
\label{ssec:pdfhyperref}

The \pai{hyperref} package will take care of turning all internal
references of your document into hyperlinks. For this to work
properly some magic is necessary, so you have to put
\verb+\usepackage[pdftex]{hyperref}+ as the \emph{last} command into
the preamble of your document.

Many options are available to customize the behaviour of the
\pai{hyperref} package:
\begin{itemize}
\item either as a comma separated list after the pdftex option\\
  \verb+\usepackage[pdftex]{hyperref}+
\item or on individual lines with the command
  \verb+\hypersetup{+\emph{options}\verb+}+.
\end{itemize}

The only required option is \texttt{pdftex}; the others are
optional and allow you to change the default behaviour of hyperref.\footnote{It
is worth noting that the \pai{hyperref} package is not limited to
work with pdf\TeX{}. It can also be configured to embed PDF-specific
information into the DVI output of normal \LaTeX{}, which then gets put
into the PS file by \texttt{dvips} and is finally picked up by Adobe
Distiller when it is used to turn the PS file PDF.} In the following
list the default values are written in an upright font.

\begin{description}
  \item[\texttt{bookmarks (=true,\textit{false})}] show or hide the
    bookmarks bar when displaying the document
  \item [\texttt{pdftoolbar (=true,\textit{false})}] show or hide
    Acrobat's toolbar
  \item [\texttt{pdfmenubar (=true,\textit{false})}] show or hide
    Acrobat's menu
  \item [\texttt{pdffitwindow (=true,\textit{false})}] adjust the
    initial magnification of the pdf when displayed
  \item [\texttt{pdftitle (=\{texte\})}] define the title that gets
    displayed in the \texttt{Document Info} window of Acrobat
  \item [\texttt{pdfauthor (=\{texte\})}] the name of the PDF's author
  \item [\texttt{pdfnewwindow (=true,\textit{false})}] define if a new
    window should get opened when a link leads out of the current
    document
  \item [\texttt{colorlinks (=true,\textit{false})}] show link
    ``zones'' in color. The color of these links can be configured
    using the following options:
    \begin{description}
    \item [\texttt{linkcolor (=color,\textit{red})}] color of internal
      links (sections, pages, etc.),
    \item [\texttt{citecolor (=color,\textit{green})}] color of
      citation links (bibliography)
    \item [\texttt{filecolor (=color,\textit{magenta})}] color of file
      links
    \item [\texttt{urlcolor (=color,\textit{cyan})}] color of url
      links (mail, web)
    \end{description}
\end{description}

\vspace{\baselineskip}

If you are happy with the defaults, use
\begin{code}
\begin{verbatim}
\usepackage[pdftex]{hyperref}
\end{verbatim}
\end{code}

To have the bookmark list open and links in color
(the \texttt{=true} values are optional):
\begin{code}
\begin{verbatim}
\usepackage[pdftex,bookmarks,colorlinks]{hyperref}
\end{verbatim}
\end{code}

When creating PDFs destined for printing, colored links are not a
  good thing as they end up in gray in the final output, making it
  difficult to read:
\begin{code}
\begin{verbatim}
\usepackage{hyperref}
\hypersetup{colorlinks,%
            citecolor=black,%
            filecolor=black,%
            linkcolor=black,%
            urlcolor=black,%
            pdftex}
\end{verbatim}
\end{code}

When you just want to provide information for the
  \texttt{Document Info} section of the PDF file:
\begin{code}
\begin{verbatim}
\usepackage[pdfauthor={Pierre Desproges}%
            pdftitle={Des femmes qui tombent},%
            pdftex]{hyperref}
\end{verbatim}
\end{code}

\vspace{\baselineskip}

In addition to the automatic hyperlinks for cross references, it is
possible to embed explicit links using
\begin{lscommand}
\ci{href}\verb|{|\emph{url}\verb|}{|\emph{text}\verb|}|
\end{lscommand}

The code
\begin{code}
\begin{verbatim}
The \href{http://www.ctan.org}{CTAN} website.
\end{verbatim}
\end{code}
produces the output ``\href{http://www.ctan.org}{CTAN}'';
a click on the word ``\textcolor{magenta}{CTAN}''
will take you to the CTAN website.

If the destination of the link is not a URL but a local file,
  you can use use the \ci{href} command: 
\begin{verbatim}
  The complete document is \href{manual.pdf}{here}
\end{verbatim}
Which produces the text ``The complete document is \textcolor{cyan}{here}''.
A click on the word
``\textcolor{cyan}{here}''
will open the file \texttt{manual.pdf}. (The filename is relative to
the location of the current document).

The author of an article might want her readers to easily send
  email messages by using the \ci{href} command inside the \ci{author}
  command on the title page of the document:
\begin{code}
\begin{verbatim}
\author{Mary Oetiker $<$\href{mailto:mary@oetiker.ch}%
       {mary@oetiker.ch}$>$
\end{verbatim}
\end{code}
Note that I have put the link so that my email address appears not only
in the link but also on the page itself. I did this because the
link\\
\verb+\href{mailto:mary@oetiker.ch}{Mary Oetiker}+\\
would
work well within Acrobat, but once the page is printed the email address
would not be visible anymore.


\subsection{Problems with Links}

Messages like the following:
\begin{verbatim}
! pdfTeX warning (ext4): destination with the same identifier
  (name{page.1}) has been already used, duplicate ignored
\end{verbatim}
appear when a counter gets reinitialized, for example by using
the command \ci{mainmatter} provided by the \texttt{book} document class. It
resets the page number counter to~1 prior to the first chapter of the
book. But as the preface of the book also has a page number~1 all
links to ``page 1'' would not be unique anymore, hence the notice
that ``\verb+duplicate+ has been \verb+ignored+.''

The counter measure consists of putting \texttt{plainpages=false} into
the hyperref options. This unfortunately only helps with the page
counter.
An even more radical solution is to use the option
\texttt{hypertexnames=false}, but this will cause the page links in
the index to stop working.

\subsection{Problems with Bookmarks}

The text displayed by bookmarks does not always look like you expect
it to look. Because bookmarks are ``just text,'' much fewer
characters are available for bookmarks than for normal \LaTeX{} text.
Hyperref will normally notice such problems and put up a warning:
\begin{code}
\begin{verbatim}
Package hyperref Warning: 
Token not allowed in a PDFDocEncoded string:
\end{verbatim}
\end{code}
You can now work around this problem by providing a text string for
the bookmarks, which replaces the offending text:
\begin{lscommand}
\ci{texorpdfstring}\verb|{|\emph{\TeX{} text}\verb|}{|\emph{Bookmark Text}\verb|}|
\end{lscommand}


Math expressions are a prime candidate for this kind of problem:
\begin{code}
\begin{verbatim}
\section{\texorpdfstring{$E=mc^2$}%
        {E\ =\ mc\texttwosuperior}}
\end{verbatim}
\end{code}
which turns \verb+\section{$E=mc^2$}+ to ``E=mc2'' in the bookmark area.

Color changes also do not travel well into bookmarks:
\begin{code}
\verb+\section{\textcolor{red}{Red !}}+
\end{code}
produces the string ``redRed!''. The command \verb+\textcolor+ gets ignored
but its argument (red) gets printed. 

If you use
\begin{code}
\verb+\section{\texorpdfstring{\textcolor{red}{Red !}}{Red\ !}}+
\end{code}
the result will be much more legible.


\subsubsection{Source Compatibility Between \LaTeX{} and pdf\LaTeX{}}
\label{sec:pdfcompat}


Ideally your document would compile equally well with \LaTeX{} and
pdf\LaTeX{}. The main problem in this respect is the inclusion of
graphics. The simple solution is to \emph{systematically drop} the
file extension from \ci{includegraphics} commands. They will then
automatically look for a file of a suitable format in the current
directory. All you have to do is create appropriate versions of the
graphics files. \LaTeX{} will look for \texttt{.eps}, and pdf\LaTeX{}
will try to include a file with the extension
\texttt{.png}, \texttt{.pdf}, \texttt{.jpg}, \texttt{.mps} or \texttt{.tif}
(in that order).


For the cases where you want to use different code for the
PDF version of your document, you can add:
\begin{code}
\begin{verbatim}
\newif\ifPDF
\ifx\pdfoutput\undefined\PDFfalse
\else\ifnum\pdfoutput > 0\PDFtrue
     \else\PDFfalse
     \fi
\fi
\end{verbatim}
\end{code}
as the very first few lines of your document. This defines a special
command that will allow you to easily write conditional code:
\begin{code}
\begin{verbatim}
\ifPDF
  \usepackage[T1]{fontenc}
  \usepackage{aeguill}
  \usepackage[pdftex]{graphicx,color}
  \usepackage[pdftex]{hyperref}
\else    
  \usepackage[T1]{fontenc}
  \usepackage[dvips]{graphicx}
  \usepackage[dvips]{hyperref}
\fi
\end{verbatim}
\end{code}
In the example above I have included the hyperref package even in the
non-PDF version. The effect of this is to make the \ci{href} command
work in all cases, which saves me from wrapping every occurrence into a
conditional statement.

Note that in recent \TeX{} distributions (\TeX{}Live for example), the
choice between \texttt{pdftex} and \texttt{dvips} when calling
\pai{graphicx} and  \pai{color} will happen automatically according to
the settings made automatically in the configuration files
\texttt{graphics.cfg} and \texttt{color.cfg}.

\section{Creating Presentations with \pai{pdfscreen}}
\label{sec:pdfscreen}
\secby{Daniel Flipo}{Daniel.Flipo@univ-lille1.fr}
You can present the results of your scientific work on a blackboard,
with transparencies, or directly from your laptop using some
presentation software. 

\wi{pdf\LaTeX} combined with the \pai{pdfscreen} package allows you to
create presentations in PDF, equally as colorful and lively as
is possible with \emph{PowerPoint}, but much more portable because
Acrobat Reader is available on many more systems.


The \pai{pdfscreen} class uses \pai{graphicx}, \pai{color} and
\pai{hyperref} with options adapted to screen presentations.
%La figure~\ref{fig:pdfscr} contient un exemple de fichier minimal �
%compiler avec \wi{pdf\LaTeX} et le 
%r�sultat produit.

% �cran captur� par ImageMagick (man ImageMagick) fonction � import �
% et convertie en jpg toujours par ImageMagick.
\begin{figure}[htbp]
\begin{verbatim}
\documentclass[pdftex,12pt]{article}
%%% misc extensions %%%%%%%%%%%%%%%%%%%%%%%%%%%%%%%%%%%%%
\usepackage[latin1]{inputenc}
\usepackage[english]{babel}
\usepackage[T1]{fontenc}
\usepackage{aeguill}
%%% pdfscreen %%%%%%%%%%%%%%%%%%%%%%%%%%%%%%%%%%%%%%%%%%%
\usepackage[screen,panelleft,chocolate]{pdfscreen}
% Screen Format
\panelwidth=25mm
%%          height width
\screensize{150mm}{200mm} 
%%          left right top  bottom
\marginsize{42mm}{8mm}{10mm}{10mm}
% Color or image for background
\overlayempty
\definecolor{mybg}{rgb}{1,0.9,0.7}
\backgroundcolor{mybg}
% Logo
\emblema{MyLogo}
%%% For PPower4 (post-processor) %%%%%%%%%%%%%%%%%%%%%%%%
\usepackage{pause}
%%%%%%%%%%%%%%%%%%%%%%%%%%%%%%%%%%%%%%%%%%%%%%%%%%%%%%%%%
\begin{document}
\begin{slide}
\begin{itemize}
\item Good News\dots \pause
\item Bad News
\end{itemize}
\end{slide}
\end{document}
\end{verbatim}
%  \begin{center}
%    \includegraphics[width=.8\textwidth]{pdfscr}
    \caption{Example pdfscreen input file}   
    \label{fig:pdfscr}   
%  \end{center}
\end{figure}

To create this type of document you normally work within the
\texttt{article} class. Figure \ref{fig:pdfscr} shows an example input file.
First you have to load the \pai{pdfscreen} package together with
appropriate options:
\begin{description}
\item[\texttt{screen}]: screen presentation. Use \texttt{print} to
  create a printable version.
\item[\texttt{panelright}] put a navigation panel on the right side of
  the screen. If the panel should be on the left side use
  \texttt{panelleft}.
\item[\texttt{french}] or some other supported language will render
  the text on the navigation buttons appropriately. This option is
  independent of the options set with the babel package.
\item[\texttt{chocolate}] color scheme for the navigation panel. Other
  choices are \texttt{gray}, \texttt{orange}, \texttt{palegreen},
 \texttt{bluelace} and \texttt{blue}, which is the default.
\end{description}
Then you configure the display format. Because the presentation will
always scale to the real size of the screen when displayed, this can
be used to configure the overall font size:
\begin{description}
\item[\ci{panelwidth}] defines the width of the navigation panel
\item[\ci{screensize}\texttt{\{}\emph{width}\texttt{\}\{}\emph{height}\texttt{\}}]
  define the width and height of the screen including the navigation
  panel.
\item[\ci{marginsize}\texttt{\{}\emph{left}\texttt{\}\{}\emph{right}\texttt{\}\{}\emph{top}\texttt{\}\{}\emph{bottom}\texttt{\}}]
  defines the margins of the document. In the example the document is
  not centered because the section numbers are kept in the left
  margin.
\end{description}

It is possible to use a background image in any of the image formats
supported by pdf\TeX{} using the command
\begin{lscommand}
\ci{overlay}\verb|{|\emph{image}\verb|}|
\end{lscommand}
or if you prefer a plain background you can define its color using
\begin{lscommand}
\ci{background}\verb|{|\emph{color}\verb|}|
\end{lscommand}
Finally if you want to place the logo of your organization into the
navigation panel use the command
\begin{lscommand}
\ci{emblema}\verb|{|\emph{logo}\verb|}|
\end{lscommand}

If you believe in the presentational power of successive exposure of
your bullet points you may want
to make use of the \pai{pause} package. It provides the command
\ci{pause}. You can place this command right into the flow of your
text wherever you want Acrobat to pause the display of your document.
The \pai{pause} package is part of the \texttt{ppower4} ($P^4$:
\emph{Pdf Presentation  Post-Processor}) system, which can post-process
pdf output from pdf\TeX{} and make it dance, sing and beg for
food. You can treat the output of pdf\TeX{} by passing it through the
post-processor. On the command line it looks like this:
\begin{lscommand}
\verb+ppower4 xy.pdf xyz.pdf+
\end{lscommand}

To control what goes onto a single slide, you can use the environment
\verb+\begin{slide}+ \ldots \verb+\end{slide}+. The content of each
slide will get displayed centered vertically on its page.

If you compile the example above you will get an error message:
\begin{verbatim}
! pdfTeX warning (dest): name{contents} has been
  referenced but does not exist, replaced by a fixed one
\end{verbatim}
This is because there is a button in the navigation panel that wants
to point to the table of contents, because this example does not
contain a \ci{tableofcontents} command the resolution of the link
fails.

If you want the table of contents to be displayed right inside the
navigation panel, you can use the option \texttt{paneltoc} when
calling \pai{pdfscreen}. This will only produce satisfactory results
if your presentation has very few and short entries in the table of
contents. You may want to provide short titles for your section
headings in square brackets.

This short introduction only scratches the surface of what is
possible with \pai{pdfscreen} and \texttt{PPower4}. Both come with
their own extensive documentation.


\section{\texorpdfstring{\Xy}{Xy}-pic}
\secby{Alberto Manuel Brand\~ao Sim\~oes}{albie@alfarrabio.di.uminho.pt}
\pai{xy} is a special package for drawing diagrams. To use it,
simply add the following line to the preamble of your document:
\begin{lscommand}
\verb|\usepackage[|\emph{options}\verb|]{xy}|
\end{lscommand}
\emph{options} is a list of functions from \Xy-pic you want to
load. These options are primarily useful when debugging the package.  I recommend
you pass the \verb!all! option, making \LaTeX{} load all the \Xy{} commands.

\Xy-pic diagrams are drawn over a matrix-oriented canvas, where
each diagram element is placed in a matrix slot:
\begin{example}
\begin{displaymath}
\xymatrix{A & B \\
          C & D }
\end{displaymath}
\end{example}
The \ci{xymatrix} command must be used in math mode. Here, we
specified two lines and two columns. To make this matrix a diagram we
just add directed arrows using the \ci{ar} command.
\begin{example}
\begin{displaymath}
\xymatrix{ A \ar[r] & B \ar[d] \\
           D \ar[u] & C \ar[l] }
\end{displaymath}
\end{example}
The arrow command is placed on the origin cell for the arrow. The
arguments are the direction the arrow should point to (\texttt{u}p,
\texttt{d}own, \texttt{r}ight and \texttt{l}eft).

\begin{example}
\begin{displaymath}
\xymatrix{
  A \ar[d] \ar[dr] \ar[r] & B \\
  D                       & C }
\end{displaymath}
\end{example}
To make diagonals, just use more than one direction. In
fact, you can repeat directions to make bigger arrows.
\begin{example}
\begin{displaymath}
\xymatrix{
  A \ar[d] \ar[dr] \ar[drr] & & \\
  B                      & C & D }
\end{displaymath}
\end{example}

We can draw even more interesting diagrams by adding
labels to the arrows. To do this, we use the common superscript and
subscript operators.
\begin{example}
\begin{displaymath}
\xymatrix{
  A \ar[r]^f \ar[d]_g &
             B \ar[d]^{g'} \\
  D \ar[r]_{f'}       & C }
\end{displaymath}
\end{example}

As shown, you use these operators as in math mode. The only
difference is that that superscript means ``on top of the arrow,''
and subscript means ``under the arrow.'' There is a third operator, the vertical bar: \verb+|+
It causes text to be placed \emph{in} the arrow.
\begin{example}
\begin{displaymath}
\xymatrix{
  A \ar[r]|f \ar[d]|g &
             B \ar[d]|{g'} \\
  D \ar[r]|{f'}       & C }
\end{displaymath}
\end{example}

To draw an arrow with an hole in it, use \verb!\ar[...]|\hole!.

In some situations, it is important to distinguish between different types of
arrows. This can be done by putting labels on them, or changing their appearance:

\begin{example}
\begin{displaymath}
\xymatrix{
 \bullet\ar@{->}[rr] && \bullet\\
 \bullet\ar@{.<}[rr] && \bullet\\
 \bullet\ar@{~)}[rr] && \bullet\\
 \bullet\ar@{=(}[rr] && \bullet\\
 \bullet\ar@{~/}[rr]  && \bullet\\
 \bullet\ar@{=+}[rr]   && \bullet
}
\end{displaymath}
\end{example}

Notice the difference between the following two diagrams:

\begin{example}
\begin{displaymath}
\xymatrix{
 \bullet \ar[r] 
         \ar@{.>}[r] & 
 \bullet
}
\end{displaymath}
\end{example}

\begin{example}
\begin{displaymath}
\xymatrix{
 \bullet \ar@/^/[r] 
         \ar@/_/@{.>}[r] &
 \bullet
}
\end{displaymath}
\end{example}

The modifiers between the slashes define how the curves are drawn.
\Xy-pic offers many ways to influence the drawing of curves;
for more information, check \Xy-pic documentation.
%

% Local Variables:
% TeX-master: "lshort2e"
% mode: latex
% mode: flyspell
% End:

3c3
< % $Id: custom.tex,v 1.2 1998/09/29 08:05:09 oetiker Exp oetiker $
---
> % $Id: custom.tex,v 1.2 2003/03/19 20:57:45 oetiker Exp $
8c8
< Documents produced by using the commands you have learned up to this
---
> Documents produced with the commands you have learned up to this
10c10
< looking fancy, they obey all the established rules of good
---
> fancy-looking, they obey all the established rules of good
13,14c13,14
< However there are situations where \LaTeX{} does not provide a
< command or environment which matches your needs, or the output
---
> However, there are situations where \LaTeX{} does not provide a
> command or environment that matches your needs, or the output
19c19
< which looks different than what is provided by default.
---
> that looks different from what is provided by default.
37,40c37,40
< In this example, I am using both a new environment called
< \ei{lscommand} which is responsible for drawing the box around the
< command and a new command named \ci{ci} which typesets the command
< name and also makes a corresponding entry in the index. You can check
---
> In this example, I am using both a new environment called\\
> \ei{lscommand}, which is responsible for drawing the box around the
> command, and a new command named \ci{ci}, which typesets the command
> name and makes a corresponding entry in the index. You can check
69c69
< short for ``The Not So Short Introduction to \LaTeXe''. Such a command
---
> short for ``The Not So Short Introduction to \LaTeXe.'' Such a command
82c82
< command which takes one argument.
---
> command that takes one argument.
98c98
< \LaTeX{} will not allow you to create a new command which would
---
> \LaTeX{} will not allow you to create a new command that would
112c112
< Similar to the \verb|\newcommand| command, there is also a command
---
> Just as with the \verb|\newcommand| command, there is a command
122,123c122,123
< Like the \verb|\newcommand| command, you can use \ci{newenvironment}
< with an optional argument or without. The material specified
---
> Again \ci{newenvironment} can have
> an optional argument. The material specified
144c144
< an environment which already exists. If you ever want to change an
---
> an environment that already exists. If you ever want to change an
148c148
< The commands used in this example will be explained later: For the
---
> The commands used in this example will be explained later. For the
153c153,212
< \subsection{Your own Package}
---
> \subsection{Extra Space}
> 
> When creating a new environment you may easily get bitten by extra spaces
> creeping in, which can potentially have fatal effects. For example when you
> want to create a title environment which supresses its own indentation as
> well as the one on the following paragraph. The \ci{ignorespaces} command in
> the begin block of the environment will make it ignore any space after
> executing the begin block. The end block is a bit more tricky as special
> processing occurs at the end of an environment. With the
> \ci{ignorespacesafterend} \LaTeX{} will issue an \ci{ignorespaces} after the
> special `end' processing has occured.
> 
> \begin{example}
> \newenvironment{simple}%
>  {\noindent}%
>  {\par\noindent}
> 
> \begin{simple}
> See the space\\to the left.
> \end{simple}
> Same\\here.
> \end{example}
> 
> \begin{example}
> \newenvironment{correct}%
>  {\noindent\ignorespaces}%
>  {\par\noindent%
>    \ignorespacesafterend}
> 
> \begin{correct}
> No space\\to the left.
> \end{correct}
> Same\\here.
> \end{example}
> 
> \subsection{Commandline \LaTeX}
> 
> If you work on a Unix like OS, you might be using Makefiles to build your
> \LaTeX{} projects. In that connection it might be interesting to produce
> different versions of the same document by calling \LaTeX{} with commandline
> parameters. If you add the following structure to your document:
> 
> \begin{verbatim}
> \usepackage{ifthen}
> \ifthenelse{\equal{\blackandwhite}{true}}{
>   % "black and white" mode; do something..
> }{
>   % "color" mode; do something different..
> }
> \end{verbatim}
> 
> Now you can call \LaTeX{} like this:
> \begin{verbatim}
> latex '\newcommand{\blackandwhite}{true}\input{test.tex}'
> \end{verbatim}
> 
> First the command \verb|\blackandwhite| gets defined and then the actual file is read with input.
> By setting \verb|\blackandwhite| to false the color version of the document would be produced.
> 
> \subsection{Your Own Package}
166c225,226
< \newcommand{\tnss}{The not so Short Introduction to \LaTeXe}
---
> \newcommand{\tnss}{The not so Short Introduction 
>                    to \LaTeXe}
175c235
< Writing a package consists  basically in copying the contents of
---
> Writing a package basically consists of copying the contents of
185c245
< package which contains the commands defined in the examples above.
---
> package that contains the commands defined in the examples above.
189c249
< \subsection{Font changing Commands}
---
> \subsection{Font Changing Commands}
206,208c266,268
< One important feature of \LaTeXe{} is, that the font attributes are
< independent. This means, that you can issue size or even font
< changing commands and still keep the bold or slant attribute set
---
> One important feature of \LaTeXe{} is that the font attributes are
> independent. This means that you can issue size or even font
> changing commands, and still keep the bold or slant attribute set
213,214c273,274
< switch to another font for math typesetting there exists another
< special set of commands. Refer to Table~\ref{mathfonts}.
---
> switch to another font for math typesetting you need another
> special set of commands; refer to Table~\ref{mathfonts}.
224,232c284,292
< \ci{textrm}\verb|{...}|        &      \textrm{\wi{roman}}&
< \ci{textsf}\verb|{...}|        &      \textsf{\wi{sans serif}}\\
< \ci{texttt}\verb|{...}|        &      \texttt{typewriter}\\[6pt]
< \ci{textmd}\verb|{...}|        &      \textmd{medium}&
< \ci{textbf}\verb|{...}|        &      \textbf{\wi{bold face}}\\[6pt]
< \ci{textup}\verb|{...}|        &       \textup{\wi{upright}}&
< \ci{textit}\verb|{...}|        &       \textit{\wi{italic}}\\
< \ci{textsl}\verb|{...}|        &       \textsl{\wi{slanted}}&
< \ci{textsc}\verb|{...}|        &       \textsc{\wi{small caps}}\\[6pt]
---
> \fni{textrm}\verb|{...}|        &      \textrm{\wi{roman}}&
> \fni{textsf}\verb|{...}|        &      \textsf{\wi{sans serif}}\\
> \fni{texttt}\verb|{...}|        &      \texttt{typewriter}\\[6pt]
> \fni{textmd}\verb|{...}|        &      \textmd{medium}&
> \fni{textbf}\verb|{...}|        &      \textbf{\wi{bold face}}\\[6pt]
> \fni{textup}\verb|{...}|        &       \textup{\wi{upright}}&
> \fni{textit}\verb|{...}|        &       \textit{\wi{italic}}\\
> \fni{textsl}\verb|{...}|        &       \textsl{\wi{slanted}}&
> \fni{textsc}\verb|{...}|        &       \textsc{\wi{Small Caps}}\\[6pt]
234c294
< \ci{textnormal}\verb|{...}|    &    \textnormal{document} font
---
> \fni{textnormal}\verb|{...}|    &    \textnormal{document} font
247,252c307,312
< \ci{tiny}      & \tiny        tiny font \\
< \ci{scriptsize}   & \scriptsize  very small font\\
< \ci{footnotesize} & \footnotesize  quite small font \\
< \ci{small}        &  \small            small font \\
< \ci{normalsize}   &  \normalsize  normal font \\
< \ci{large}        &  \large       large font
---
> \fni{tiny}      & \tiny        tiny font \\
> \fni{scriptsize}   & \scriptsize  very small font\\
> \fni{footnotesize} & \footnotesize  quite small font \\
> \fni{small}        &  \small            small font \\
> \fni{normalsize}   &  \normalsize  normal font \\
> \fni{large}        &  \large       large font
255,258c315,318
< \ci{Large}        &  \Large       larger font \\[5pt]
< \ci{LARGE}        &  \LARGE       very large font \\[5pt]
< \ci{huge}         &  \huge        huge \\[5pt]
< \ci{Huge}         &  \Huge        largest
---
> \fni{Large}        &  \Large       larger font \\[5pt]
> \fni{LARGE}        &  \LARGE       very large font \\[5pt]
> \fni{huge}         &  \huge        huge \\[5pt]
> \fni{Huge}         &  \Huge        largest
293,303c353,361
< \begin{lined}{\textwidth}
< \begin{tabular}{@{}lll@{}}
< \textit{Command}&\textit{Example}&    \textit{Output}\\[6pt]
< \ci{mathcal}\verb|{...}|&    \verb|$\mathcal{B}=c$|&     $\mathcal{B}=c$\\
< \ci{mathrm}\verb|{...}|&     \verb|$\mathrm{K}_2$|&      $\mathrm{K}_2$\\
< \ci{mathbf}\verb|{...}|&     \verb|$\sum x=\mathbf{v}$|& $\sum x=\mathbf{v}$\\
< \ci{mathsf}\verb|{...}|&     \verb|$\mathsf{G\times R}$|&        $\mathsf{G\times R}$\\
< \ci{mathtt}\verb|{...}|&     \verb|$\mathtt{L}(b,c)$|&   $\mathtt{L}(b,c)$\\
< \ci{mathnormal}\verb|{...}|& \verb|$\mathnormal{R_{19}}\neq R_{19}$|&
< $\mathnormal{R_{19}}\neq R_{19}$\\
< \ci{mathit}\verb|{...}|&     \verb|$\mathit{ffi}\neq ffi$|& $\mathit{ffi}\neq ffi$
---
> \begin{lined}{0.7\textwidth}
> \begin{tabular}{@{}ll@{}}
> \fni{mathrm}\verb|{...}|&     $\mathrm{Roman\ Font}$\\
> \fni{mathbf}\verb|{...}|&     $\mathbf{Boldface\ Font}$\\
> \fni{mathsf}\verb|{...}|&     $\mathsf{Sans\ Serif\ Font}$\\
> \fni{mathtt}\verb|{...}|&     $\mathtt{Typewriter\ Font}$\\
> \fni{mathit}\verb|{...}|&     $\mathit{Italic\ Font}$\\
> \fni{mathcal}\verb|{...}|&    $\mathcal{CALLIGRAPHIC\ FONT}$\\
> \fni{mathnormal}\verb|{...}|& $\mathnormal{Normal\ Font}$\\
305a364,376
> %\begin{tabular}{@{}lll@{}}
> %\textit{Command}&\textit{Example}&    \textit{Output}\\[6pt]
> %\fni{mathcal}\verb|{...}|&    \verb|$\mathcal{B}=c$|&     $\mathcal{B}=c$\\
> %\fni{mathscr}\verb|{...}|&    \verb|$\mathscr{B}=c$|&     $\mathscr{B}=c$\\
> %\fni{mathrm}\verb|{...}|&     \verb|$\mathrm{K}_2$|&      $\mathrm{K}_2$\\
> %\fni{mathbf}\verb|{...}|&     \verb|$\sum x=\mathbf{v}$|& $\sum x=\mathbf{v}$\\
> %\fni{mathsf}\verb|{...}|&     \verb|$\mathsf{G\times R}$|&        $\mathsf{G\times R}$\\
> %\fni{mathtt}\verb|{...}|&     \verb|$\mathtt{L}(b,c)$|&   $\mathtt{L}(b,c)$\\
> %\fni{mathnormal}\verb|{...}|& \verb|$\mathnormal{R_{19}}\neq R_{19}$|&
> %$\mathnormal{R_{19}}\neq R_{19}$\\
> %\fni{mathit}\verb|{...}|&     \verb|$\mathit{ffi}\neq ffi$|& $\mathit{ffi}\neq ffi$
> %\end{tabular}
> 
327,328c398,400
< {\Large Don't read this! It is not
< true. You can believe me!\par}
---
> {\Large Don't read this! 
>  It is not true.
>  You can believe me!\par}
362c434,435
< \newcommand{\oops}[1]{\textbf{#1}}
---
> \newcommand{\oops}[1]{%
>  \textbf{#1}}
364c437
< it's occupied by a \oops{machine}
---
> it's occupied by \oops{machines}
369,370c442,443
< stage whether you want to use some other visual representation of danger
< than \verb|\textbf| without having to wade through your document,
---
> stage that you want to use some visual representation of danger other
> than \verb|\textbf|, without having to wade through your document,
401c474
< the lines are not spread, therefore the default line spread factor
---
> the lines are not spread, so the default line spread factor
403a477,497
> Note that the effect of the \ci{linespread} command is rather drastic and     
> not appropriate for published work. So if you have a good reason for
> changing the line spacing you might want to use the command:
> \begin{lscommand}
> \verb|\setlength{\baselineskip}{1.5\baselineskip}|
> \end{lscommand}
> 
> \begin{example}
> {\setlength{\baselineskip}%
>            {1.5\baselineskip}
> This paragraph is typeset with
> the baseline skip set to 1.5 of
> what it was before. Note the par
> command at the end of the
> paragraph.\par}
> 
> This paragraph has a clear
> purpose, it shows that after the
> curly brace has been closed,
> everything is back to normal.
> \end{example}
418,419c512,513
< \TeX{}, that it can compress and expand the inter paragraph skip by the
< amount specified if this is necessary to properly fit the paragraphs
---
> \TeX{} that it can compress and expand the inter paragraph skip by the
> amount specified, if this is necessary to properly fit the paragraphs
427c521
< place after the \verb|\tableofcontents| or to not use them at all,
---
> place below the command \verb|\tableofcontents| or to not use them at all,
432c526
< If you want to indent a paragraph which is not indented, you can use 
---
> If you want to indent a paragraph that is not indented, you can use 
458c552
< \emph{length} in the simplest case just is a number plus a unit.  The
---
> \emph{length} in the simplest case is just a number plus a unit.  The
489,491c583,587
< remaining space on a line is filled up. If two
< \verb|\hspace{\stretch{|\emph{n}\verb|}}| commands are issued on the
< same line, they grow according to the stretch factor.
---
> remaining space on a line is filled up. If multiple
> \verb|\hspace{\stretch{|\emph{n}\verb|}}| commands are issued on the same
> line, they occupy all available space in proportion of their respective
> stretch factors.
> 
497a594,603
> When using horizontal space together with text, it may make sense to make
> the space adjust its size relative to the size of the current font.
> This can be done by using the text-relative units \texttt{em} and
> \texttt{ex}:
> 
> \begin{example}
> {\Large{}big\hspace{1em}y}\\
> {\tiny{}tin\hspace{1em}y}
> \end{example}
>  
508c614
< the starred version of the command \verb|\vspace*| instead of \verb|\vspace|.
---
> the starred version of the command, \verb|\vspace*|, instead of \verb|\vspace|.
511c617
< The \verb|\stretch| command in connection with \verb|\pagebreak| can
---
> The \verb|\stretch| command, in connection with \verb|\pagebreak|, can
538c644
< \makeatletter\@layout\makeatother
---
> \makeatletter\@mylayout\makeatother
542a649,660
> \cih{footskip}
> \cih{headheight}
> \cih{headsep}
> \cih{marginparpush}
> \cih{marginparsep}
> \cih{marginparwidth}
> \cih{oddsidemargin}
> \cih{paperheight}
> \cih{paperwidth}
> \cih{textheight}
> \cih{textwidth}
> \cih{topmargin}
548c665
< text \wi{margins}. But sometimes you may not be happy with 
---
> text \wi{margins}, but sometimes you may not be happy with 
552,554c669,671
< Figure~\ref{fig:layout} shows all the parameters which can be changed.
< The figure was produced with the \pai{layout} package from the tools bundle%
< \footnote{\texttt{CTAN:/tex-archive/macros/latex/required/tools}}.
---
> Figure~\ref{fig:layout} shows all the parameters that can be changed.
> The figure was produced with the \pai{layout} package from the tools bundle.%
> \footnote{\CTANref|macros/latex/required/tools|}
570c687
< This is also the reason why newspapers are typeset in multiple columns.
---
> This is also why newspapers are typeset in multiple columns.
584c701
< The second command adds a length to any of the parameters. 
---
> The second command adds a length to any of the parameters:
598,599c715,716
< In this context, you might want to look at the \pai{calc} package,
< it allows you to use arithmetic operations in the argument of setlength
---
> In this context, you might want to look at the \pai{calc} package.
> It allows you to use arithmetic operations in the argument of \ci{setlength}
603c720
< \section{More fun with lengths}
---
> \section{More Fun With Lengths}
614,616c731,733
< \ci{settoheight}\verb|{|\emph{lscommand}\verb|}{|\emph{text}\verb|}|\\
< \ci{settodepth}\verb|{|\emph{lscommand}\verb|}{|\emph{text}\verb|}|\\
< \ci{settowidth}\verb|{|\emph{lscommand}\verb|}{|\emph{text}\verb|}|
---
> \ci{settoheight}\verb|{|\emph{variable}\verb|}{|\emph{text}\verb|}|\\
> \ci{settodepth}\verb|{|\emph{variable}\verb|}{|\emph{text}\verb|}|\\
> \ci{settowidth}\verb|{|\emph{variable}\verb|}{|\emph{text}\verb|}|
632c749
< $b$ -- are adjunct to the right 
---
> $b$ -- are adjoin to the right 
651,653c768,770
< the point is that \TeX{} operates on glue and boxes. Not only a letter
< can be a box. You can put virtually everything into a box including
< other boxes. Each box will then be handled by \LaTeX{} as if it was a
---
> the point is that \TeX{} operates on glue and boxes. Letters are not the only things that
> can be boxes. You can put virtually everything into a box, including
> other boxes. Each box will then be handled by \LaTeX{} as if it were a
680,681c797,798
< between a \ei{minipage} and a \ei{parbox} is that you cannot use all commands
< and environments inside a \ei{parbox} while almost anything is possible in
---
> between a \ei{minipage} and a \ci{parbox} is that you cannot use all commands
> and environments inside a \ei{parbox}, while almost anything is possible in
685,687c802,804
< everything, there is also a class of boxing commands which operates
< only on horizontally aligned material. We already know one of them.
< It's called \ci{mbox}, it simply packs up a series of boxes into
---
> everything, there is also a class of boxing commands that operates
> only on horizontally aligned material. We already know one of them;
> it's called \ci{mbox}. It simply packs up a series of boxes into
698,706c815,822
<   material inside the box. You can even set the
<   width to 0pt so that the text inside the box will be typeset without
<   influencing the surrounding boxes.}  Apart from the length
<   expressions you can also use \ci{width}, \ci{height}, \ci{depth} and
<   \ci{totalheight} in the width parameter. They are set from values
<   obtained by measuring the typeset \emph{text}. The \emph{pos} parameter takes
< a one letter value: \textbf{c}enter, \textbf{l}eft flush,
< \textbf{r}ight flush or \textbf{s} which spreads the text inside the
< box to fill it.
---
> material inside the box. You can even set the
> width to 0pt so that the text inside the box will be typeset without
> influencing the surrounding boxes.}  Besides the length
> expressions, you can also use \ci{width}, \ci{height}, \ci{depth}, and
> \ci{totalheight} in the width parameter. They are set from values
> obtained by measuring the typeset \emph{text}. The \emph{pos} parameter takes
> a one letter value: \textbf{c}enter, flush\textbf{l}eft,
> flush\textbf{r}ight, or \textbf{s}pread the text to fill the box.
735c851
< \ci{raisebox}\verb|{|\emph{lift}\verb|}[|\emph{depth}\verb|][|\emph{height}\verb|]{|\emph{text}\verb|}|
---
> \ci{raisebox}\verb|{|\emph{lift}\verb|}[|\emph{extend-above-baseline}\verb|][|\emph{extend-below-baseline}\verb|]{|\emph{text}\verb|}|
739c855
< box. You can use \ci{width}, \ci{height}, \ci{depth} and
---
> box. You can use \ci{width}, \ci{height}, \ci{depth}, and
766c882
< \newpage
---
> 
776c892
< lines. The line on the title page for example, has been created with a
---
> lines. The line on the title page, for example, has been created with a
795,798c911,914
< %%% Local Variables: 
< %%% mode: latex
< %%% TeX-master: "lshort2e"
< %%% End: 
---
> \bigskip
> {\flushright The End.\par}
> 
> %
799a916,920
> % Local Variables:
> % TeX-master: "lshort2e"
> % mode: latex
> % mode: flyspell
> % End:

\backmatter
%%%%%%%%%%%%%%%%%%%%%%%%%%%%%%%%%%%%%%%%%%%%%%%%%%%%%%%%%%%%%%%%%
% Contents: The Bibliography
% File: biblio.tex (lshort2e.tex)
% $Id: biblio.tex,v 1.1.1.1 2002/02/26 10:04:20 oetiker Exp $
%%%%%%%%%%%%%%%%%%%%%%%%%%%%%%%%%%%%%%%%%%%%%%%%%%%%%%%%%%%%%%%%%
\begin{thebibliography}{99}
\addcontentsline{toc}{chapter}{\bibname} 
\bibitem{manual} Leslie Lamport.  \newblock \emph{{\LaTeX:} A Document
    Preparation System}.  \newblock Addison-Wesley, Reading,
  Massachusetts, second edition, 1994, ISBN~0-201-52983-1.
  
\bibitem{texbook} Donald~E. Knuth.  \newblock \textit{The \TeX{}book,}
  Volume~A of \textit{Computers and Typesetting}, Addison-Wesley,
  Reading, Massachusetts, second edition, 1984, ISBN~0-201-13448-9.

\bibitem{companion} Michel Goossens, Frank Mittelbach and Alexander
  Samarin.  \newblock \emph{The {\LaTeX} Companion}.  \newblock
  Addison-Wesley, Reading, Massachusetts, 1994, ISBN~0-201-54199-8.
 
\bibitem{local} Each \LaTeX{} installation should provide a so-called
  \emph{\LaTeX{} Local Guide}, which explains the things that are
  special to the local system.  It should be contained in a file called
  \texttt{local.tex}. Unfortunately, some lazy sysops do not provide such a
  document. In this case, go and ask your local \LaTeX{} guru for help.
 
\bibitem{usrguide} \LaTeX3 Project Team.  \newblock \emph{\LaTeXe~for
    authors}.  \newblock Comes with the \LaTeXe{} distribution as
  \texttt{usrguide.tex}.

\bibitem{clsguide} \LaTeX3 Project Team.  \newblock \emph{\LaTeXe~for
    Class and Package writers}.  \newblock Comes with the \LaTeXe{}
  distribution as \texttt{clsguide.tex}.

\bibitem{fntguide} \LaTeX3 Project Team.  \newblock \emph{\LaTeXe~Font
    selection}.  \newblock Comes with the \LaTeXe{} distribution as
  \texttt{fntguide.tex}.

\bibitem{graphics} D.~P.~Carlisle.  \newblock \emph{Packages in the
    `graphics' bundle}.  \newblock Comes with the `graphics' bundle as
  \texttt{grfguide.tex}, available from the same source your \LaTeX{}
  distribution came from.

\bibitem{verbatim} Rainer~Sch\"opf, Bernd~Raichle, Chris~Rowley.  
\newblock \emph{A New Implementation of \LaTeX's verbatim
  Environments}.
 \newblock Comes with the `tools' bundle as
  \texttt{verbatim.dtx}, available from the same source your \LaTeX{}
  distribution came from. 

\bibitem{catalogue} Graham~Williams.  \newblock \emph{The TeX
    Catalogue} is a very complete listing of many \TeX{} and \LaTeX{}
    related packages.
  \newblock Available online from \texttt{CTAN:/tex-archive/help/Catalogue/catalogue.html}
  
\bibitem{eps} Keith~Reckdahl.  \newblock \emph{Using EPS Graphics in
    \LaTeXe{} Documents}, which explains everything and much more than
  you ever wanted to know about EPS files and their use in \LaTeX{}
  documents.  \newblock Available online from
  \texttt{CTAN:/tex-archive/info/epslatex.ps}

\bibitem{xy-pic} Kristoffer H. Rose
  \newblock \emph{\Xy-pic User's Guide}.  \newblock
  Downloadable from CTAN with \Xy-pic distribution 

\end{thebibliography}


%

% Local Variables:
% TeX-master: "lshort2e"
% mode: latex
% mode: flyspell
% End:

%\cleardoublepage
\addcontentsline{toc}{chapter}{\numberline{}Index} 
\printindex
\end{document}
